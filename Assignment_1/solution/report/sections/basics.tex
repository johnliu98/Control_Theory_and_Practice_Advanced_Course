The process to be controlled is modelled by the transfer function $G(s)$:

\begin{align*}
  G(s) = \dfrac{3(1-s)}{(5s + 1)(10s + 1)}
\end{align*}


\begin{figure}[H]\centering
  \scalebox{1}{% Generated with LaTeXDraw 2.0.8
% Tue Apr 05 05:24:32 CEST 2016
% \usepackage[usenames,dvipsnames]{pstricks}
% \usepackage{epsfig}
% \usepackage{pst-grad} % For gradients
% \usepackage{pst-plot} % For axes
\scalebox{1} % Change this value to rescale the drawing.
{
\begin{pspicture}(0,-2.245)(12.999063,2.265)
\psframe[linewidth=0.04,dimen=outer](5.6,1.315)(3.48,0.055)
\psline[linewidth=0.04cm,arrowsize=0.05291667cm 2.0,arrowlength=1.4,arrowinset=0.4]{->}(5.6,0.655)(6.96,0.655)
\psline[linewidth=0.04cm,arrowsize=0.05291667cm 2.0,arrowlength=1.4,arrowinset=0.4]{->}(2.06,0.675)(3.52,0.675)
\usefont{T1}{ptm}{m}{n}
\rput(4.514531,0.685){$F(s)$}
\usefont{T1}{ptm}{m}{n}
\rput(7.974531,0.645){$G(s)$}
\usefont{T1}{ptm}{m}{n}
\rput(6.1445312,0.985){$u$}
\pscircle[linewidth=0.04,dimen=outer](10.14,0.675){0.5}
\psline[linewidth=0.04cm,arrowsize=0.05291667cm 2.0,arrowlength=1.4,arrowinset=0.4]{->}(9.02,0.695)(9.66,0.695)
\psline[linewidth=0.04cm,arrowsize=0.05291667cm 2.0,arrowlength=1.4,arrowinset=0.4]{->}(10.13,2.245)(10.13,1.145)
\usefont{T1}{ptm}{m}{n}
\rput(10.1545315,0.685){$\Sigma$}
\usefont{T1}{ptm}{m}{n}
\rput(10.504531,1.785){$d$}
\psline[linewidth=0.04cm,arrowsize=0.05291667cm 2.0,arrowlength=1.4,arrowinset=0.4]{->}(10.6,0.675)(11.7,0.675)
\pscircle[linewidth=0.04,dimen=outer](1.56,0.735){0.5}
\usefont{T1}{ptm}{m}{n}
\rput(1.5745312,0.745){$\Sigma$}
\psline[linewidth=0.04cm,arrowsize=0.05291667cm 2.0,arrowlength=1.4,arrowinset=0.4]{->}(0.0,0.695)(1.1,0.695)
\pscircle[linewidth=0.04,dimen=outer](11.32,-1.745){0.5}
\usefont{T1}{ptm}{m}{n}
\rput(11.334531,-1.735){$\Sigma$}
\usefont{T1}{ptm}{m}{n}
\rput(11.524531,1.065){$y$}
\psline[linewidth=0.04cm,arrowsize=0.05291667cm 2.0,arrowlength=1.4,arrowinset=0.4]{->}(12.92,-1.745)(11.82,-1.745)
\usefont{T1}{ptm}{m}{n}
\rput(12.624531,-1.395){$n$}
\usefont{T1}{ptm}{m}{n}
\rput(1.2645313,-0.035){$-$}
\usefont{T1}{ptm}{m}{n}
\rput(0.3,1){$r$}
\psline[linewidth=0.04cm,arrowsize=0.05291667cm 2.0,arrowlength=1.4,arrowinset=0.4]{<-}(1.52,0.275)(1.54,-1.745)
\psline[linewidth=0.04cm](1.54,-1.745)(10.86,-1.745)
\psline[linewidth=0.04cm,arrowsize=0.05291667cm 2.0,arrowlength=1.4,arrowinset=0.4]{->}(11.32,0.675)(11.34,-1.285)
\psframe[linewidth=0.04,dimen=outer](9.0,1.315)(6.88,0.055)
\end{pspicture}
}

}
  \caption{Closed loop block diagram, where $F-$controller, $G-$system,
    $r-$reference signal, $u-$control signal, $d-$disturbance signal, $y-$output
    signal, $n-$measurement noise.}
  \label{fig:block_1}
\end{figure}


\subsection{Exercise 1}

In order for the closed-loop system to have

\begin{itemize}
  \item a phase margin of $30^{\circ}$,
  \item a crossover frequency of $0.4$ rad/s, and
  \item zero steady-state error for a step response in the reference signal
\end{itemize}
we consider a lead-lag controller of the form

\begin{align*}
  F(s) = K \cdot \dfrac{\tau_D s + 1}{\beta \tau_d s + 1} \cdot \dfrac{\tau_I s + 1}{\tau_I s + \gamma}
\end{align*}
whose $K, \tau_D, \beta, \tau_I$ and $\gamma$ coefficients shall be configured
in such a way that the closed-loop system fulfills the above requirements.

We first consider the third requirement: the error $E(s)$ is given by

\begin{align*}
  E(s) = R(s) - Y(s) = \dfrac{1}{1 + F(s)G(s)} R(s) = \dfrac{1}{1 + F(s)G(s)} \cdot \dfrac{1}{s}
\end{align*}
for a step reference. The steady state error will thus be

\begin{align*}
  e(\infty) = \displaystyle\lim_{t\to \infty} e(t) = \displaystyle\lim_{s\to 0} sE(s) = \dfrac{1}{1 + F(0)G(0)} = \dfrac{\gamma}{\gamma + 3K}
\end{align*}
For $e(\infty)$ to be zero, either $\gamma = 0$ or $K \to \infty$. Sensibly, we
choose $\gamma = 0$.

If we set $\tau_I = 1$, then the lag component becomes $F_g(s) = \dfrac{s+1}{s}$,
and the phase margin of $F_g(s)G(s)$ is equal to
$\phi_m^0 = 130.6013^{\circ} - 180^{\circ} = -49.3987^{\circ}$. Thus,
$\beta, \tau_d$ and $K$ can be obtained by the following equations:

\begin{itemize}
  \item $\beta = \dfrac{1 - sin(30 - \phi_m^0)}{1 + sin(30 - \phi_m^0)} = 0.0086$
  \item $\tau_D = \dfrac{1}{\omega_c\sqrt\beta} = \dfrac{1}{0.4\sqrt\beta} = 26.9459$
  \item $K = \dfrac{\sqrt\beta}{|F_g(j\omega_c)G(j\omega_c)|} =
    \dfrac{\sqrt\beta}{0.9436} = 0.0983$
\end{itemize}

Finally, now that the value of each coefficient has been identified, the
controller is identified as

\begin{align*}
  F(s) = 0.0983 \cdot \dfrac{26.9459 s + 1}{0.2317 s + 1} \cdot \dfrac{s + 1}{s}
\end{align*}

We can verify that all three requirements have been met by plotting the bode
diagram (figure \ref{fig:bode_1.1}) and the step response
(figure \ref{fig:step_response_1.1}) for the initial, uncontrolled process $G(s)$
and the final, controlled process $F(s)G(s)$.


\begin{figure}[H]\centering
  \scalebox{1}{% This file was created by matlab2tikz.
%
%The latest updates can be retrieved from
%  http://www.mathworks.com/matlabcentral/fileexchange/22022-matlab2tikz-matlab2tikz
%where you can also make suggestions and rate matlab2tikz.
%
\definecolor{mycolor1}{rgb}{0.00000,0.44700,0.74100}%
\definecolor{mycolor2}{rgb}{0.85000,0.32500,0.09800}%
%
\begin{tikzpicture}

\begin{axis}[%
width=4.008in,
height=1.551in,
at={(0.818in,1.941in)},
scale only axis,
separate axis lines,
every outer x axis line/.append style={white!40!black},
every x tick label/.append style={font=\color{white!40!black}},
xmode=log,
xmin=0.001,
xmax=100,
xtick={0.001,0.01,0.1,1,10,100},
xticklabels={\empty},
xminorticks=true,
every outer y axis line/.append style={white!40!black},
every y tick label/.append style={font=\color{white!40!black}},
ymin=-100,
ymax=50,
ylabel={Magnitude [dB]},
axis background/.style={fill=white}
]
\addplot [color=mycolor1,solid,forget plot]
  table[row sep=crcr]{%
1e-20	389.395184693267\\
7.42228722820556e-19	351.984429559325\\
7.42228722820556e-14	251.984429559325\\
7.42228722820556e-10	171.984429559325\\
7.42228722820556e-07	111.984429560768\\
7.42228722820556e-05	71.9844439869546\\
0.000742228722820557	51.985871985355\\
0.000866927441358339	50.6374973330137\\
0.00101257626587676	49.2893138387565\\
0.00118269493535748	47.941391004338\\
0.00138139452528949	46.5938235390263\\
0.00161347679562269	45.2467404487036\\
0.00188455022975229	43.9003173563815\\
0.00220116556872376	42.5547931635748\\
0.00257097411596817	41.2104925088622\\
0.00300291263814868	39.8678559053744\\
0.00350741933041878	38.5274799325054\\
0.00409668606509297	37.1901703743702\\
0.0047849530195533	35.8570116261041\\
0.00558885280334817	34.529455791107\\
0.00652781240063433	33.2094342126773\\
0.00762452264127392	31.8994919203729\\
0.00890548654579129	30.6029403309014\\
0.0104016597954543	29.3240136705068\\
0.0121491988050336	28.0679978167204\\
0.0141903344761125	26.8412751751989\\
0.0165743927460073	25.6511976799964\\
0.0193589865948076	24.5056715051295\\
0.022611408316496	23.4123330977935\\
0.0264102556996675	22.3772488066035\\
0.0308473314160163	21.4032057610526\\
0.0358539970228913	20.5158810584628\\
0.0360298615170745	20.487865557815\\
0.0420830866512346	19.6222421423993\\
0.04915328862022	18.7900157109728\\
0.0574113254145479	17.9680307097373\\
0.0670567601553975	17.1280052802005\\
0.0783226837225929	16.2391561486473\\
0.0914813476119835	15.2712453664504\\
0.106850743145443	14.1975191974545\\
0.124802286026205	12.9971157572751\\
0.145769791943941	11.6567275072687\\
0.170259960133408	10.1715316011419\\
0.198864618231587	8.54552404453135\\
0.232275024341645	6.79136385462149\\
0.271298571926367	4.92974520531357\\
0.316878301220307	2.98830524387246\\
0.370115688672039	1.00015758497665\\
0.432297265144511	-0.997801080327817\\
0.504925706127038	-2.96673029727747\\
0.589756145282716	-4.86790208525688\\
0.688838588881866	-6.66513685460721\\
0.804567456105603	-8.32767877306328\\
0.939739442406957	-9.83325844568467\\
1.09762110425136	-11.1707257490607\\
1.28202779848443	-12.3414841437476\\
1.49741588397014	-13.359204928319\\
1.74899041363751	-14.2478529337003\\
2.04283091941404	-15.0385749094166\\
2.3860383297555	-15.7661866740867\\
2.78690657017049	-16.4658271888818\\
3.25512299362572	-17.1700549823824\\
3.80200248441866	-17.9064849952487\\
4.44076089285483	-18.6960709205294\\
5.18683440853253	-19.5522145609504\\
6.05825259018656	-20.4808580074838\\
7.07607406670345	-21.4815208878963\\
8.26489543842728	-22.5489868091129\\
9.65344567682841	-23.6751924182053\\
11.2752803867546	-24.850911990748\\
13.1695927087562	-26.067003089325\\
15.3821604576918	-27.3151676525038\\
17.9664523860985	-28.5883076334239\\
20.9849203062065	-29.8806004279363\\
24.5105082959265	-31.1874117449028\\
28.6284154601723	-32.5051333056041\\
33.4381548462803	-33.8310005204839\\
39.0559582691306	-35.162920048107\\
45.6175851607957	-36.4993204673791\\
53.2816033231793	-37.8390296139836\\
62.2332208660713	-39.1811772021362\\
72.6887619329644	-40.5251192827204\\
84.9008943746914	-41.8703805720463\\
99.1647357024747	-43.2166109185069\\
115.824984876402	-44.5635527053762\\
135.28425227562	-45.9110165837281\\
158.012789151699	-47.2588634824281\\
184.55985168644	-48.6069913121167\\
215.566974277122	-49.9553251601966\\
2155.66974277122	-69.9537862488486\\
215566.974277122	-109.953770702391\\
215566974.277122	-169.953770700836\\
2155669742771.22	-249.953770700836\\
2.15566974277122e+17	-349.953770700836\\
1e+20	-403.282126181659\\
};
\addplot [color=black,solid,forget plot]
  table[row sep=crcr]{%
1.07011595011444	0\\
1.07011595011444	-10.9637498946078\\
};
\addplot [color=black,dotted,forget plot]
  table[row sep=crcr]{%
1.07011595011444	-100\\
1.07011595011444	0\\
};
\addplot [color=black,dotted,forget plot]
  table[row sep=crcr]{%
0.400002397146949	-100\\
0.400002397146949	0\\
};
\addplot [color=black,dotted,forget plot]
  table[row sep=crcr]{%
0.001	0\\
100	0\\
};
\addplot [color=mycolor2,solid,forget plot]
  table[row sep=crcr]{%
1e-20	9.54242509439325\\
2e-18	9.54242509439325\\
2e-13	9.54242509439325\\
2e-09	9.54242509439325\\
2e-06	9.54242509223915\\
0.0002	9.54240355342386\\
0.002	9.5402713627845\\
0.00233717165820117	9.53948415809935\\
0.00273118567994939	9.53840931622052\\
0.00319162488223131	9.53694181661812\\
0.00372968760918032	9.53493836280318\\
0.00435846008706016	9.53220349137111\\
0.00509323469443899	9.52847069079282\\
0.00595188188820484	9.523376781155\\
0.00695528483103661	9.5164272310713\\
0.00812784729090762	9.50694935691696\\
0.00949808716524821	9.49402944976971\\
0.0110993300648712	9.47642882468319\\
0.0129705198263185	9.4524726693549\\
0.015157165665104	9.41990459926948\\
0.0177124490055704	9.3756994527308\\
0.0206985169065763	9.31582791766569\\
0.0241879935404239	9.23497048019702\\
0.0282657464857158	9.12618702942878\\
0.0330309507921571	8.98056487405763\\
0.0385995010174335	8.78689410208395\\
0.0451068298993264	8.53145439220796\\
0.0527112022160033	8.19803359634517\\
0.0615975639444767	7.76831613571356\\
0.0719820403326324	7.22274903422338\\
0.0841171922824609	6.54189245208831\\
0.0982981588850128	5.7081056096305\\
0.114869835499704	4.70728255494117\\
0.134235261956069	3.53033132881272\\
0.156865424887466	2.17420785880134\\
0.183310712599335	0.64247195019924\\
0.214214301065913	-1.05460482459853\\
0.250327796616312	-2.90039069917478\\
0.292529515755795	-4.87273889497134\\
0.341845846705877	-6.94527456121148\\
0.399476212197379	-9.08893859687721\\
0.466822240636634	-11.2735590547231\\
0.545521855116954	-13.469314728112\\
0.637489109354333	-15.6481051398104\\
0.744960739397426	-17.7849167612729\\
0.870550563296124	-19.8592047122183\\
1.01731305178338	-21.8561288781271\\
1.18881761607313	-23.7673253463181\\
1.38923541952819	-25.5909044763286\\
1.62344082454525	-27.3305851581909\\
1.89712994194695	-28.9941784484001\\
2.21695916612161	-30.591832166926\\
2.59070706522436	-32.1344421718128\\
3.02746356377195	-33.6324772160582\\
3.53785101874225	-35.0952750398668\\
4.13428256597126	-36.5307370480754\\
4.83126402009161	-37.9452981430786\\
5.64574667052256	-39.3440537449169\\
6.59753955386447	-40.7309560125116\\
7.70979122957661	-42.1090239892333\\
9.00955277620718	-43.4805381722373\\
10.5284357008095	-44.8472070148147\\
12.3033807625627	-46.2103026729137\\
14.3775564091595	-47.5707682110338\\
16.801408676838	-48.9293004504898\\
19.6338880886805	-50.2864130316176\\
22.9438833905788	-51.6424839028691\\
26.8118969947666	-52.9977908089921\\
31.3320028793888	-54.352537664743\\
36.6141345621924	-55.7068740767075\\
42.7867587941599	-57.06090975141\\
50	-58.4147251067686\\
500	-78.4163585758919\\
50000	-118.416375077397\\
50000000	-178.416375079047\\
500000000000	-258.416375079047\\
5e+16	-358.416375079047\\
1e+20	-424.436974992327\\
};
\addplot [color=black,solid,forget plot]
  table[row sep=crcr]{%
0.565697298192371	0\\
0.565697298192371	-13.9796947246107\\
};
\addplot [color=black,dotted,forget plot]
  table[row sep=crcr]{%
0.565697298192371	-100\\
0.565697298192371	0\\
};
\addplot [color=black,dotted,forget plot]
  table[row sep=crcr]{%
0.19468266671785	-100\\
0.19468266671785	0\\
};
\addplot [color=black,dotted,forget plot]
  table[row sep=crcr]{%
0.001	0\\
100	0\\
};
\end{axis}

\begin{axis}[%
width=4.008in,
height=1.376in,
at={(0.818in,0.44in)},
scale only axis,
separate axis lines,
every outer x axis line/.append style={white!40!black},
every x tick label/.append style={font=\color{white!40!black}},
xmode=log,
xmin=0.001,
xmax=100,
xminorticks=true,
xlabel={Frequency [rad/s]},
every outer y axis line/.append style={white!40!black},
every y tick label/.append style={font=\color{white!40!black}},
ymin=87.3,
ymax=362.7,
ytick={ 90, 180, 270, 360},
ylabel={Phase [deg]},
axis background/.style={fill=white},
legend style={legend cell align=left,align=left,draw=white!15!black,font=\footnotesize}
]
\addplot [color=mycolor1,solid]
  table[row sep=crcr]{%
1e-20	270\\
7.42228722820556e-19	270\\
7.42228722820556e-14	270.00000000005\\
7.42228722820556e-10	270.000000498153\\
7.42228722820556e-07	270.000498153103\\
7.42228722820556e-05	270.049815166267\\
0.000742228722820557	270.498009135958\\
0.000866927441358339	270.581616324444\\
0.00101257626587676	270.679233691098\\
0.00118269493535748	270.793193502241\\
0.00138139452528949	270.926207114083\\
0.00161347679562269	271.081421133214\\
0.00188455022975229	271.262478584491\\
0.00220116556872376	271.47358334649\\
0.00257097411596817	271.719564344844\\
0.00300291263814868	272.00593311206\\
0.00350741933041878	272.338923729747\\
0.00409668606509297	272.725497000256\\
0.0047849530195533	273.173279759678\\
0.00558885280334817	273.690394037736\\
0.00652781240063433	274.285107676896\\
0.00762452264127392	274.965206982973\\
0.00890548654579129	275.736954077867\\
0.0104016597954543	276.603453279483\\
0.0121491988050336	277.562229162789\\
0.0141903344761125	278.601848749684\\
0.0165743927460073	279.697558311344\\
0.0193589865948076	280.80622044001\\
0.022611408316496	281.861367332833\\
0.0264102556996675	282.769851331282\\
0.0308473314160163	283.41208179375\\
0.0358539970228913	283.64798616792\\
0.0360298615170745	283.647711520023\\
0.0420830866512346	283.32750094829\\
0.04915328862022	282.310121388357\\
0.0574113254145479	280.480734002758\\
0.0670567601553975	277.767327263544\\
0.0783226837225929	274.151385269464\\
0.0914813476119835	269.671039228471\\
0.106850743145443	264.416710347831\\
0.124802286026205	258.520851653062\\
0.145769791943941	252.14430330074\\
0.170259960133408	245.461661173159\\
0.198864618231587	238.64714283365\\
0.232275024341645	231.861598459296\\
0.271298571926367	225.241284253943\\
0.316878301220307	218.889522817943\\
0.370115688672039	212.872481405581\\
0.432297265144511	207.219524078839\\
0.504925706127038	201.927396530488\\
0.589756145282716	196.966680134544\\
0.688838588881866	192.288875983497\\
0.804567456105603	187.832965186944\\
0.939739442406957	183.530946126407\\
1.09762110425136	179.312390045011\\
1.28202779848443	175.108385516231\\
1.49741588397014	170.855371421533\\
1.74899041363751	166.499315567464\\
2.04283091941404	162.000479681282\\
2.3860383297555	157.338595263732\\
2.78690657017049	152.517671485825\\
3.25512299362572	147.569020575045\\
3.80200248441866	142.550781687103\\
4.44076089285483	137.54270510727\\
5.18683440853253	132.636383149874\\
6.05825259018656	127.922950340148\\
7.07607406670345	123.481444116035\\
8.26489543842728	119.37071478422\\
9.65344567682841	115.626210266353\\
11.2752803867546	112.261160469049\\
13.1695927087562	109.270584363125\\
15.3821604576918	106.636382665101\\
17.9664523860985	104.332229975199\\
20.9849203062065	102.327589150766\\
24.5105082959265	100.590655020432\\
28.6284154601723	99.0903156744338\\
33.4381548462803	97.7973339501577\\
39.0559582691306	96.6849658471426\\
45.6175851607957	95.7292009441936\\
53.2816033231793	94.9087654224656\\
62.2332208660713	94.2049868124482\\
72.6887619329644	93.6015865776328\\
84.9008943746914	93.0844425944518\\
99.1647357024747	92.6413470270285\\
115.824984876402	92.2617741564018\\
135.28425227562	91.9366657427537\\
158.012789151699	91.6582371886735\\
184.55985168644	91.4198052081494\\
215.566974277122	91.2156362571621\\
2155.66974277122	90.1215787498256\\
215566.974277122	90.0012157890261\\
215566974.277122	90.000001215789\\
2155669742771.22	90.0000000001215\\
2.15566974277122e+17	90\\
1e+20	90\\
};
\addlegendentry{compensated};

\addplot [color=black,solid,forget plot]
  table[row sep=crcr]{%
0.400002397146949	180\\
0.400002397146949	209.999781862523\\
};
\addplot [color=black,dotted,forget plot]
  table[row sep=crcr]{%
0.400002397146949	180\\
0.400002397146949	362.7\\
};
\addplot [color=black,dotted,forget plot]
  table[row sep=crcr]{%
1.07011595011444	180\\
1.07011595011444	362.7\\
};
\addplot [color=black,dotted,forget plot]
  table[row sep=crcr]{%
0.001	180\\
100	180\\
};
\addplot [color=mycolor2,solid]
  table[row sep=crcr]{%
1e-20	360\\
2e-18	360\\
2e-13	359.999999999817\\
2e-09	359.999998166535\\
2e-06	359.998166535056\\
0.0002	359.816653677598\\
0.002	358.166707057904\\
0.00233717165820117	357.857713309198\\
0.00273118567994939	357.496671340435\\
0.00319162488223131	357.074832598203\\
0.00372968760918032	356.581989130365\\
0.00435846008706016	356.006236740965\\
0.00509323469443899	355.333703648514\\
0.00595188188820484	354.548242018487\\
0.00695528483103661	353.631081137728\\
0.00812784729090762	352.560443659562\\
0.00949808716524821	351.311131022292\\
0.0110993300648712	349.854091937563\\
0.0129705198263185	348.156000329509\\
0.015157165665104	346.178888370186\\
0.0177124490055704	343.879908783462\\
0.0206985169065763	341.211340669741\\
0.0241879935404239	338.121005299058\\
0.0282657464857158	334.553318224698\\
0.0330309507921571	330.45125668309\\
0.0385995010174335	325.759533418817\\
0.0451068298993264	320.429183255855\\
0.0527112022160033	314.423515661962\\
0.0615975639444767	307.724920694494\\
0.0719820403326324	300.341402826855\\
0.0841171922824609	292.311215861625\\
0.0982981588850128	283.703987688419\\
0.114869835499704	274.617513269482\\
0.134235261956069	265.170676374278\\
0.156865424887466	255.493937938071\\
0.183310712599335	245.718902476226\\
0.214214301065913	235.967959986227\\
0.250327796616312	226.344805701653\\
0.292529515755795	216.92711616751\\
0.341845846705877	207.763129632883\\
0.399476212197379	198.87347874817\\
0.466822240636634	190.258308900665\\
0.545521855116954	181.908190496998\\
0.637489109354333	173.816260230269\\
0.744960739397426	165.988612867988\\
0.870550563296124	158.450214088947\\
1.01731305178338	151.244614722344\\
1.18881761607313	144.42754139993\\
1.38923541952819	138.056522823193\\
1.62344082454525	132.180057608636\\
1.89712994194695	126.829620534566\\
2.21695916612161	122.016214540679\\
2.59070706522436	117.731251952524\\
3.02746356377195	113.950302973245\\
3.53785101874225	110.637989893703\\
4.13428256597126	107.752716289267\\
4.83126402009161	105.250528216065\\
5.64574667052256	103.087901236868\\
6.59753955386447	101.223539630327\\
7.70979122957661	99.6193945177002\\
9.00955277620718	98.2411232672412\\
10.5284357008095	97.058179996095\\
12.3033807625627	96.0436808596074\\
14.3775564091595	95.1741447388524\\
16.801408676838	94.4291757200772\\
19.6338880886805	93.7911289091959\\
22.9438833905788	93.2447841050906\\
26.8118969947666	92.7770406996795\\
31.3320028793888	92.3766401312885\\
36.6141345621924	92.0339179406489\\
42.7867587941599	91.7405849994885\\
50	91.489536140167\\
500	90.1489688725705\\
50000	90.0014896902672\\
50000000	90.0000014896903\\
500000000000	90.000000000149\\
5e+16	90\\
1e+20	90\\
};
\addlegendentry{uncompensated};

\addplot [color=black,solid,forget plot]
  table[row sep=crcr]{%
0.19468266671785	180\\
0.19468266671785	241.942764038024\\
};
\addplot [color=black,dotted,forget plot]
  table[row sep=crcr]{%
0.19468266671785	180\\
0.19468266671785	362.7\\
};
\addplot [color=black,dotted,forget plot]
  table[row sep=crcr]{%
0.565697298192371	180\\
0.565697298192371	362.7\\
};
\addplot [color=black,dotted,forget plot]
  table[row sep=crcr]{%
0.001	180\\
100	180\\
};
\end{axis}
\end{tikzpicture}%
}
  \caption{The frequency response of the initial, uncontrolled process
    (\texttt{red}) and final, controlled process (\texttt{blue}).}
  \label{fig:bode_1.1}
\end{figure}

\begin{figure}[H]\centering
  \scalebox{0.8}{% This file was created by matlab2tikz.
%
%The latest updates can be retrieved from
%  http://www.mathworks.com/matlabcentral/fileexchange/22022-matlab2tikz-matlab2tikz
%where you can also make suggestions and rate matlab2tikz.
%
\definecolor{mycolor1}{rgb}{0.00000,0.44700,0.74100}%
\definecolor{mycolor2}{rgb}{0.85000,0.32500,0.09800}%
%
\begin{tikzpicture}

\begin{axis}[%
width=4.008in,
height=3.052in,
at={(0.818in,0.44in)},
scale only axis,
unbounded coords=jump,
separate axis lines,
every outer x axis line/.append style={white!40!black},
every x tick label/.append style={font=\color{white!40!black}},
xmin=0,
xmax=70,
xmajorgrids,
xlabel={Time [sec]},
every outer y axis line/.append style={white!40!black},
every y tick label/.append style={font=\color{white!40!black}},
ymin=-0.5,
ymax=3.2,
ymajorgrids,
ylabel={Amplitude},
axis background/.style={fill=white},
legend style={at={(0.971,0.852)},legend cell align=left,align=left,draw=white!15!black,font=\footnotesize}
]
\addplot [color=mycolor1,solid]
  table[row sep=crcr]{%
0	0\\
0.0253786164908058	-0.0165578982412938\\
0.0507572329816116	-0.0315371880884875\\
0.0761358494724174	-0.0450647822638625\\
0.101514465963223	-0.0572565226768271\\
0.126893082454029	-0.0682181554373126\\
0.152271698944835	-0.0780462200600244\\
0.177650315435641	-0.0868288604098345\\
0.203028931926446	-0.0946465642742535\\
0.228407548417252	-0.101572837843043\\
0.253786164908058	-0.107674820822448\\
0.279164781398864	-0.113013847407572\\
0.30454339788967	-0.117645957876825\\
0.329922014380476	-0.121622365153176\\
0.355300630871281	-0.124989880294669\\
0.380679247362087	-0.127791300528018\\
0.406057863852893	-0.130065763121106\\
0.431436480343699	-0.131849068100224\\
0.456815096834505	-0.133173972553412\\
0.48219371332531	-0.134070459020054\\
0.507572329816116	-0.13456598024689\\
0.532950946306922	-0.134685682389982\\
0.558329562797728	-0.134452608559206\\
0.583708179288534	-0.133887884434949\\
0.60908679577934	-0.133010887534518\\
0.634465412270145	-0.131839401566947\\
0.659844028760951	-0.13038975718832\\
0.685222645251757	-0.128676960354268\\
0.710601261742563	-0.126714809360998\\
0.735979878233369	-0.124516001570214\\
0.761358494724174	-0.122092230725675\\
0.78673711121498	-0.119454275689301\\
0.812115727705786	-0.116612081351859\\
0.837494344196592	-0.113574832406859\\
0.862872960687398	-0.110351020615676\\
0.888251577178203	-0.106948506136667\\
0.913630193669009	-0.103374573440665\\
0.939008810159815	-0.0996359822892491\\
0.964387426650621	-0.0957390142102963\\
0.989766043141427	-0.0916895148670676\\
1.01514465963223	-0.0874929326822413\\
1.04052327612304	-0.0831543540464865\\
1.06590189261384	-0.0786785354121838\\
1.09128050910465	-0.0740699325464452\\
1.11665912559546	-0.0693327271934688\\
1.14203774208626	-0.0644708513742611\\
1.16741635857707	-0.0594880095317005\\
1.19279497506787	-0.054387698710614\\
1.21817359155868	-0.0491732269458562\\
1.24355220804948	-0.0438477300161553\\
1.26893082454029	-0.0384141867076137\\
1.2943094410311	-0.0328754327180909\\
1.3196880575219	-0.027234173322152\\
1.34506667401271	-0.0214929949057333\\
1.37044529050351	-0.0156543754700769\\
1.39582390699432	-0.0097206941957266\\
1.42120252348513	-0.0036942401493891\\
1.44658113997593	0.00242277979081805\\
1.47195975646674	0.00862823372285534\\
1.49733837295754	0.0149200581581694\\
1.52271698944835	0.0212962520517285\\
1.54809560593915	0.0277548713724332\\
1.57347422242996	0.0342940241662475\\
1.59885283892077	0.0409118660685861\\
1.62423145541157	0.0476065962263175\\
1.64961007190238	0.0543764535932288\\
1.67498868839318	0.0612197135659802\\
1.70036730488399	0.068134684930473\\
1.7257459213748	0.0751197070912045\\
1.7511245378656	0.0821731475585937\\
1.77650315435641	0.0892933996714603\\
1.80188177084721	0.0964788805338486\\
1.82726038733802	0.103728029147217\\
1.85263900382882	0.11103930472068\\
1.87801762031963	0.11841118514352\\
1.90339623681044	0.125842165605561\\
1.92877485330124	0.133330757352277\\
1.95415346979205	0.140875486562651\\
1.97953208628285	0.148474893338863\\
2.00491070277366	0.156127530797836\\
2.03028931926446	0.163831964255557\\
2.05566793575527	0.171586770495876\\
2.08104655224608	0.179390537116223\\
2.10642516873688	0.187241861943352\\
2.13180378522769	0.195139352512806\\
2.15718240171849	0.203081625606386\\
2.1825610182093	0.211067306842368\\
2.20793963470011	0.219095030313704\\
2.23331825119091	0.227163438269861\\
2.25869686768172	0.235271180838304\\
2.28407548417252	0.243416915782023\\
2.30945410066333	0.251599308289776\\
2.33483271715413	0.259817030796057\\
2.36021133364494	0.268068762828016\\
2.38558995013575	0.276353190876838\\
2.41096856662655	0.284669008291287\\
2.43634718311736	0.293014915191327\\
2.46172579960816	0.301389618399909\\
2.48710441609897	0.309791831391208\\
2.51248303258978	0.318220274253696\\
2.53786164908058	0.326673673666628\\
2.56324026557139	0.335150762888611\\
2.58861888206219	0.343650281757055\\
2.613997498553	0.35217097669741\\
2.6393761150438	0.360711600741183\\
2.66475473153461	0.369270913551831\\
2.69013334802542	0.377847681457677\\
2.71551196451622	0.386440677491115\\
2.74089058100703	0.395048681433388\\
2.76626919749783	0.403670479864315\\
2.79164781398864	0.412304866216386\\
2.81702643047945	0.420950640832711\\
2.84240504697025	0.42960661102832\\
2.86778366346106	0.438271591154391\\
2.89316227995186	0.446944402665004\\
2.91854089644267	0.455623874186057\\
2.94391951293347	0.464308841585995\\
2.96929812942428	0.47299814804808\\
2.99467674591509	0.481690644143892\\
3.02005536240589	0.490385187907825\\
3.0454339788967	0.499080644912352\\
3.0708125953875	0.507775888343833\\
3.09619121187831	0.516469799078688\\
3.12156982836912	0.525161265759752\\
3.14694844485992	0.533849184872661\\
3.17232706135073	0.542532460822112\\
3.19770567784153	0.551210006007876\\
3.22308429433234	0.559880740900434\\
3.24846291082314	0.568543594116139\\
3.27384152731395	0.57719750249178\\
3.29922014380476	0.585841411158492\\
3.32459876029556	0.594474273614893\\
3.34997737678637	0.6030950517994\\
3.37535599327717	0.611702716161637\\
3.40073460976798	0.620296245732884\\
3.42611322625878	0.628874628195507\\
3.45149184274959	0.637436859951316\\
3.4768704592404	0.645981946188805\\
3.5022490757312	0.654508900949236\\
3.52762769222201	0.663016747191521\\
3.55300630871281	0.671504516855877\\
3.57838492520362	0.679971250926212\\
3.60376354169443	0.688415999491229\\
3.62914215818523	0.696837821804204\\
3.65452077467604	0.705235786341431\\
3.67989939116684	0.713608970859312\\
3.70527800765765	0.72195646245006\\
3.73065662414845	0.730277357596022\\
3.75603524063926	0.738570762222583\\
3.78141385713007	0.746835791749663\\
3.80679247362087	0.755071571141774\\
3.83217109011168	0.763277234956648\\
3.85754970660248	0.771451927392409\\
3.88292832309329	0.779594802333296\\
3.9083069395841	0.787705023393927\\
3.9336855560749	0.795781763962096\\
3.95906417256571	0.803824207240108\\
3.98444278905651	0.811831546284636\\
4.00982140554732	0.819802984045113\\
4.03520002203812	0.827737733400645\\
4.06057863852893	0.835635017195456\\
4.08595725501974	0.843494068272856\\
4.11133587151054	0.851314129507734\\
4.13671448800135	0.859094453837589\\
4.16209310449215	0.866834304292079\\
4.18747172098296	0.87453295402112\\
4.21285033747376	0.882189686321502\\
4.23822895396457	0.889803794662062\\
4.26360757045538	0.897374582707395\\
4.28898618694618	0.904901364340104\\
4.31436480343699	0.91238346368162\\
4.33974341992779	0.919820215111565\\
4.3651220364186	0.927210963285684\\
4.39050065290941	0.934555063152342\\
4.41587926940021	0.941851879967606\\
4.44125788589102	0.949100789308889\\
4.46663650238182	0.956301177087197\\
4.49201511887263	0.963452439557964\\
4.51739373536343	0.970553983330482\\
4.54277235185424	0.977605225375951\\
4.56815096834505	0.98460559303413\\
4.59352958483585	0.991554524018619\\
4.61890820132666	0.998451466420772\\
4.64428681781746	1.00529587871224\\
4.66966543430827	1.01208722974617\\
4.69504405079908	1.01882499875704\\
4.72042266728988	1.02550867535921\\
4.74580128378069	1.03213775954404\\
4.77117990027149	1.03871176167583\\
4.7965585167623	1.0452302024863\\
4.8219371332531	1.05169261306789\\
4.84731574974391	1.05809853486571\\
4.87269436623472	1.0644475196682\\
4.89807298272552	1.07073912959653\\
4.92345159921633	1.07697293709277\\
4.94883021570713	1.08314852490673\\
4.97420883219794	1.08926548608162\\
4.99958744868874	1.09532342393847\\
5.02496606517955	1.10132195205927\\
5.05034468167036	1.10726069426896\\
5.07572329816116	1.11313928461618\\
5.10110191465197	1.11895736735285\\
5.12648053114277	1.12471459691254\\
5.15185914763358	1.13041063788767\\
5.17723776412439	1.13604516500557\\
5.20261638061519	1.1416178631034\\
5.227994997106	1.14712842710185\\
5.2533736135968	1.1525765619778\\
5.27875223008761	1.15796198273582\\
5.30413084657842	1.16328441437857\\
5.32950946306922	1.1685435918761\\
5.35488807956003	1.17373926013407\\
5.38026669605083	1.17887117396085\\
5.40564531254164	1.18393909803368\\
5.43102392903244	1.18894280686362\\
5.45640254552325	1.19388208475957\\
5.48178116201406	1.19875672579123\\
5.50715977850486	1.20356653375103\\
5.53253839499567	1.20831132211507\\
5.55791701148647	1.21299091400306\\
5.58329562797728	1.21760514213728\\
5.60867424446808	1.22215384880057\\
5.63405286095889	1.22663688579338\\
5.6594314774497	1.2310541143898\\
5.6848100939405	1.23540540529279\\
5.71018871043131	1.23969063858832\\
5.73556732692211	1.24390970369877\\
5.76094594341292	1.24806249933528\\
5.78632455990373	1.25214893344935\\
5.81170317639453	1.25616892318343\\
5.83708179288534	1.26012239482076\\
5.86246040937614	1.26400928373431\\
5.88783902586695	1.26782953433489\\
5.91321764235775	1.27158310001841\\
5.93859625884856	1.2752699431124\\
5.96397487533937	1.27889003482161\\
5.98935349183017	1.28244335517298\\
6.01473210832098	1.28592989295966\\
6.04011072481178	1.28934964568442\\
6.06548934130259	1.29270261950218\\
6.0908679577934	1.29598882916192\\
6.1162465742842	1.29920829794777\\
6.14162519077501	1.30236105761944\\
6.16700380726581	1.30544714835191\\
6.19238242375662	1.30846661867447\\
6.21776104024742	1.31141952540905\\
6.24313965673823	1.31430593360793\\
6.26851827322904	1.31712591649071\\
6.29389688971984	1.31987955538077\\
6.31927550621065	1.32256693964097\\
6.34465412270145	1.32518816660886\\
6.37003273919226	1.32774334153116\\
6.39541135568306	1.33023257749776\\
6.42078997217387	1.3326559953751\\
6.44616858866468	1.33501372373897\\
6.47154720515548	1.33730589880676\\
6.49692582164629	1.33953266436922\\
6.52230443813709	1.34169417172165\\
6.5476830546279	1.34379057959454\\
6.57306167111871	1.3458220540838\\
6.59844028760951	1.34778876858041\\
6.62381890410032	1.34969090369968\\
6.64919752059112	1.35152864720994\\
6.67457613708193	1.35330219396085\\
6.69995475357273	1.35501174581129\\
6.72533337006354	1.35665751155673\\
6.75071198655435	1.35823970685626\\
6.77609060304515	1.35975855415921\\
6.80146921953596	1.36121428263131\\
6.82684783602676	1.3626071280806\\
6.85222645251757	1.3639373328828\\
6.87760506900837	1.36520514590649\\
6.90298368549918	1.36641082243785\\
6.92836230198999	1.36755462410509\\
6.95374091848079	1.36863681880254\\
6.9791195349716	1.36965768061451\\
7.0044981514624	1.37061748973876\\
7.02987676795321	1.37151653240971\\
7.05525538444402	1.37235510082145\\
7.08063400093482	1.37313349305038\\
7.10601261742563	1.37385201297767\\
7.13139123391643	1.37451097021148\\
7.15676985040724	1.37511068000893\\
7.18214846689805	1.37565146319788\\
7.20752708338885	1.37613364609848\\
7.23290569987966	1.37655756044454\\
7.25828431637046	1.37692354330477\\
7.28366293286127	1.37723193700376\\
7.30904154935207	1.37748308904286\\
7.33442016584288	1.37767735202093\\
7.35979878233369	1.37781508355493\\
7.38517739882449	1.37789664620037\\
7.4105560153153	1.37792240737166\\
7.4359346318061	1.37789273926241\\
7.46131324829691	1.37780801876557\\
7.48669186478771	1.3776686273935\\
7.51207048127852	1.37747495119803\\
7.53744909776933	1.37722738069042\\
7.56282771426013	1.37692631076123\\
7.58820633075094	1.37657214060026\\
7.61358494724174	1.37616527361637\\
7.63896356373255	1.37570611735734\\
7.66434218022335	1.3751950834297\\
7.68972079671416	1.37463258741856\\
7.71509941320497	1.37401904880752\\
7.74047802969577	1.37335489089849\\
7.76585664618658	1.37264054073166\\
7.79123526267738	1.37187642900547\\
7.81661387916819	1.37106298999661\\
7.841992495659	1.37020066148014\\
7.8673711121498	1.36928988464966\\
7.89274972864061	1.36833110403756\\
7.91812834513141	1.36732476743541\\
7.94350696162222	1.36627132581436\\
7.96888557811303	1.36517123324577\\
7.99426419460383	1.36402494682191\\
8.01964281109464	1.36283292657677\\
8.04502142758544	1.36159563540707\\
8.07040004407625	1.36031353899339\\
8.09577866056705	1.35898710572146\\
8.12115727705786	1.35761680660365\\
8.14653589354867	1.35620311520059\\
8.17191451003947	1.35474650754304\\
8.19729312653028	1.35324746205391\\
8.22267174302108	1.35170645947054\\
8.24805035951189	1.35012398276711\\
8.27342897600269	1.34850051707738\\
8.2988075924935	1.34683654961757\\
8.32418620898431	1.34513256960955\\
8.34956482547511	1.34338906820425\\
8.37494344196592	1.34160653840529\\
8.40032205845672	1.33978547499297\\
8.42570067494753	1.33792637444847\\
8.45107929143833	1.33602973487831\\
8.47645790792914	1.33409605593917\\
8.50183652441995	1.33212583876294\\
8.52721514091075	1.33011958588214\\
8.55259375740156	1.3280778011556\\
8.57797237389236	1.32600098969448\\
8.60335099038317	1.32388965778862\\
8.62872960687398	1.32174431283325\\
8.65410822336478	1.31956546325599\\
8.67948683985559	1.31735361844423\\
8.70486545634639	1.31510928867289\\
8.7302440728372	1.31283298503248\\
8.75562268932801	1.31052521935763\\
8.78100130581881	1.30818650415588\\
8.80637992230962	1.30581735253693\\
8.83175853880042	1.30341827814231\\
8.85713715529123	1.30098979507533\\
8.88251577178203	1.29853241783155\\
8.90789438827284	1.29604666122963\\
8.93327300476365	1.29353304034254\\
8.95865162125445	1.29099207042928\\
8.98403023774526	1.28842426686694\\
9.00940885423606	1.28583014508329\\
9.03478747072687	1.28321022048973\\
9.06016608721768	1.28056500841473\\
9.08554470370848	1.27789502403768\\
9.11092332019929	1.27520078232328\\
9.13630193669009	1.2724827979563\\
9.1616805531809	1.26974158527686\\
9.1870591696717	1.26697765821617\\
9.21243778616251	1.26419153023274\\
9.23781640265332	1.26138371424913\\
9.26319501914412	1.25855472258908\\
9.28857363563493	1.25570506691526\\
9.31395225212573	1.25283525816738\\
9.33933086861654	1.24994580650097\\
9.36470948510734	1.24703722122652\\
9.39008810159815	1.24411001074918\\
9.41546671808896	1.24116468250905\\
9.44084533457976	1.23820174292185\\
9.46622395107057	1.23522169732024\\
9.49160256756137	1.23222504989564\\
9.51698118405218	1.2292123036405\\
9.54235980054299	1.22618396029123\\
9.56773841703379	1.22314052027159\\
9.5931170335246	1.22008248263668\\
9.6184956500154	1.2170103450174\\
9.64387426650621	1.21392460356554\\
9.66925288299701	1.21082575289938\\
9.69463149948782	1.2077142860499\\
9.72001011597863	1.20459069440748\\
9.74538873246943	1.20145546766921\\
9.77076734896024	1.19830909378676\\
9.79614596545104	1.19515205891484\\
9.82152458194185	1.19198484736019\\
9.84690319843265	1.1888079415312\\
9.87228181492346	1.18562182188807\\
9.89766043141427	1.18242696689355\\
9.92303904790507	1.1792238529643\\
9.94841766439588	1.17601295442283\\
9.97379628088668	1.17279474344997\\
9.99917489737749	1.16956969003803\\
10.0245535138683	1.16633826194449\\
10.0499321303591	1.16310092464627\\
10.0753107468499	1.15985814129466\\
10.1006893633407	1.1566103726708\\
10.1260679798315	1.1533580771418\\
10.1514465963223	1.15010171061741\\
10.1768252128131	1.14684172650735\\
10.2022038293039	1.14357857567922\\
10.2275824457947	1.14031270641701\\
10.2529610622855	1.13704456438025\\
10.2783396787764	1.13377459256373\\
10.3037182952672	1.13050323125787\\
10.329096911758	1.12723091800968\\
10.3544755282488	1.12395808758433\\
10.3798541447396	1.12068517192735\\
10.4052327612304	1.11741260012746\\
10.4306113777212	1.11414079837996\\
10.455989994212	1.11087018995078\\
10.4813686107028	1.10760119514116\\
10.5067472271936	1.10433423125293\\
10.5321258436844	1.10106971255436\\
10.5575044601752	1.09780805024673\\
10.582883076666	1.09454965243142\\
10.6082616931568	1.09129492407769\\
10.6336403096476	1.08804426699104\\
10.6590189261384	1.0847980797822\\
10.6843975426292	1.08155675783675\\
10.7097761591201	1.07832069328532\\
10.7351547756109	1.07509027497446\\
10.7605333921017	1.07186588843811\\
10.7859120085925	1.06864791586964\\
10.8112906250833	1.06543673609462\\
10.8366692415741	1.06223272454406\\
10.8620478580649	1.0590362532284\\
10.8874264745557	1.05584769071202\\
10.9128050910465	1.05266740208844\\
10.9381837075373	1.04949574895603\\
10.9635623240281	1.04633308939449\\
10.9889409405189	1.04317977794177\\
11.0143195570097	1.04003616557171\\
11.0396981735005	1.0369025996723\\
11.0650767899913	1.03377942402443\\
11.0904554064821	1.03066697878139\\
11.1158340229729	1.02756560044888\\
11.1412126394638	1.02447562186567\\
11.1665912559546	1.02139737218485\\
11.1919698724454	1.01833117685566\\
11.2173484889362	1.01527735760596\\
11.242727105427	1.01223623242528\\
11.2681057219178	1.00920811554844\\
11.2934843384086	1.00619331743981\\
11.3188629548994	1.00319214477811\\
11.3442415713902	1.00020490044187\\
11.369620187881	0.997231883495416\\
11.3949988043718	0.994273389175467\\
11.4203774208626	0.991329708878309\\
11.4457560373534	0.988401130147564\\
11.4711346538442	0.985487936662525\\
11.496513270335	0.982590408227077\\
11.5218918868258	0.979708820759185\\
11.5472705033166	0.976843446280962\\
11.5726491198075	0.973994552909304\\
11.5980277362983	0.97116240484709\\
11.6234063527891	0.968347262374949\\
11.6487849692799	0.96554938184359\\
11.6741635857707	0.962769015666687\\
11.6995422022615	0.960006412314329\\
11.7249208187523	0.957261816307013\\
11.7502994352431	0.954535468210198\\
11.7756780517339	0.951827604629394\\
11.8010566682247	0.949138458205813\\
11.8264352847155	0.946468257612549\\
11.8518139012063	0.943817227551296\\
11.8771925176971	0.941185588749614\\
11.9025711341879	0.93857355795871\\
11.9279497506787	0.935981347951765\\
11.9533283671695	0.933409167522776\\
11.9787069836603	0.930857221485923\\
12.0040856001512	0.928325710675458\\
12.029464216642	0.925814831946101\\
12.0548428331328	0.923324778173967\\
12.0802214496236	0.920855738257974\\
12.1056000661144	0.918407897121786\\
12.1309786826052	0.915981435716231\\
12.156357299096	0.913576531022237\\
12.1817359155868	0.91119335605425\\
12.2071145320776	0.908832079864147\\
12.2324931485684	0.90649286754564\\
12.2578717650592	0.904175880239149\\
12.28325038155	0.901881275137173\\
12.3086289980408	0.899609205490122\\
12.3340076145316	0.897359820612623\\
12.3593862310224	0.895133265890302\\
12.3847648475132	0.892929682787015\\
12.410143464004	0.890749208852553\\
12.4355220804948	0.888591977730792\\
12.4609006969857	0.886458119168299\\
12.4862793134765	0.884347759023383\\
12.5116579299673	0.882261019275592\\
12.5370365464581	0.880198018035646\\
12.5624151629489	0.878158869555806\\
12.5877937794397	0.876143684240675\\
12.6131723959305	0.874152568658418\\
12.6385510124213	0.872185625552418\\
12.6639296289121	0.870242953853333\\
12.6893082454029	0.86832464869158\\
12.7146868618937	0.866430801410219\\
12.7400654783845	0.864561499578247\\
12.7654440948753	0.862716827004289\\
12.7908227113661	0.860896863750687\\
12.8162013278569	0.859101686147978\\
12.8415799443477	0.857331366809761\\
12.8669585608385	0.855585974647944\\
12.8923371773294	0.853865574888374\\
12.9177157938202	0.852170229086832\\
12.943094410311	0.850499995145402\\
12.9684730268018	0.848854927329208\\
12.9938516432926	0.847235076283501\\
13.0192302597834	0.84564048905111\\
13.0446088762742	0.844071209090237\\
13.069987492765	0.842527276292601\\
13.0953661092558	0.841008727001926\\
13.1207447257466	0.839515594032756\\
13.1461233422374	0.838047906689613\\
13.1715019587282	0.836605690786476\\
13.196880575219	0.835188968666584\\
13.2222591917098	0.833797759222555\\
13.2476378082006	0.832432077916818\\
13.2730164246914	0.831091936802361\\
13.2983950411822	0.82977734454377\\
13.3237736576731	0.828488306438576\\
13.3491522741639	0.827224824438896\\
13.3745308906547	0.825986897173356\\
13.3999095071455	0.824774519969305\\
13.4252881236363	0.823587684875311\\
13.4506667401271	0.822426380683925\\
13.4760453566179	0.821290592954717\\
13.5014239731087	0.820180304037587\\
13.5268025895995	0.819095493096319\\
13.5521812060903	0.818036136132411\\
13.5775598225811	0.81700220600914\\
13.6029384390719	0.815993672475882\\
13.6283170555627	0.815010502192671\\
13.6536956720535	0.814052658754999\\
13.6790742885443	0.813120102718838\\
13.7044529050351	0.812212791625902\\
13.7298315215259	0.811330680029123\\
13.7552101380167	0.810473719518347\\
13.7805887545076	0.809641858746244\\
13.8059673709984	0.808835043454426\\
13.8313459874892	0.808053216499767\\
13.85672460398	0.807296317880925\\
13.8821032204708	0.806564284765052\\
13.9074818369616	0.805857051514703\\
13.9328604534524	0.805174549714916\\
13.9582390699432	0.804516708200487\\
13.983617686434	0.803883453083408\\
14.0089963029248	0.80327470778048\\
14.0343749194156	0.802690393041094\\
14.0597535359064	0.802130426975169\\
14.0851321523972	0.801594725081248\\
14.110510768888	0.801083200274743\\
14.1358893853788	0.800595762916335\\
14.1612680018696	0.800132320840504\\
14.1866466183604	0.79969277938421\\
14.2120252348513	0.799277041415695\\
14.2374038513421	0.798885007363421\\
14.2627824678329	0.79851657524513\\
14.2881610843237	0.798171640697021\\
14.3135397008145	0.797850097003049\\
14.3389183173053	0.79755183512432\\
14.3642969337961	0.797276743728609\\
14.3896755502869	0.797024709219965\\
14.4150541667777	0.796795615768421\\
14.4404327832685	0.796589345339791\\
14.4658113997593	0.796405777725558\\
14.4911900162501	0.796244790572847\\
14.5165686327409	0.796106259414466\\
14.5419472492317	0.795990057699038\\
14.5673258657225	0.795896056821187\\
14.5927044822133	0.795824126151803\\
14.6180830987041	0.795774133068357\\
14.643461715195	0.795745942985279\\
14.6688403316858	0.79573941938439\\
14.6942189481766	0.795754423845377\\
14.7195975646674	0.795790816076317\\
14.7449761811582	0.795848453944236\\
14.770354797649	0.79592719350571\\
14.7957334141398	0.79602688903749\\
14.8211120306306	0.796147393067161\\
14.8464906471214	0.796288556403819\\
14.8718692636122	0.79645022816877\\
14.897247880103	0.796632255826241\\
14.9226264965938	0.79683448521411\\
14.9480051130846	0.797056760574626\\
14.9733837295754	0.797298924585151\\
14.9987623460662	0.797560818388889\\
15.024140962557	0.797842281625611\\
15.0495195790478	0.798143152462375\\
15.0748981955387	0.798463267624229\\
15.1002768120295	0.798802462424898\\
15.1256554285203	0.799160570797454\\
15.1510340450111	0.799537425324952\\
15.1764126615019	0.79993285727105\\
15.2017912779927	0.800346696610583\\
15.2271698944835	0.800778772060115\\
15.2525485109743	0.801228911108444\\
15.2779271274651	0.801696940047057\\
15.3033057439559	0.802182684000554\\
15.3286843604467	0.802685966957007\\
15.3540629769375	0.803206611798274\\
15.3794415934283	0.803744440330247\\
15.4048202099191	0.804299273313048\\
15.4301988264099	0.80487093049115\\
15.4555774429007	0.805459230623442\\
15.4809560593915	0.806063991513209\\
15.5063346758824	0.806685030038044\\
15.5317132923732	0.807322162179684\\
15.557091908864	0.807975203053758\\
15.5824705253548	0.808643966939455\\
15.6078491418456	0.809328267309095\\
15.6332277583364	0.810027916857623\\
15.6586063748272	0.810742727531995\\
15.683984991318	0.811472510560468\\
15.7093636078088	0.812217076481797\\
15.7347422242996	0.812976235174319\\
15.7601208407904	0.813749795884932\\
15.7854994572812	0.814537567257966\\
15.810878073772	0.815339357363937\\
15.8362566902628	0.816154973728188\\
15.8616353067536	0.81698422335941\\
15.8870139232444	0.817826912778036\\
15.9123925397352	0.818682848044516\\
15.9377711562261	0.819551834787466\\
15.9631497727169	0.820433678231673\\
15.9885283892077	0.821328183225986\\
16.0139070056985	0.822235154271053\\
16.0392856221893	0.823154395546928\\
16.0646642386801	0.82408571094054\\
16.0900428551709	0.825028904073007\\
16.1154214716617	0.825983778326811\\
16.1408000881525	0.826950136872827\\
16.1661787046433	0.827927782697186\\
16.1915573211341	0.828916518628001\\
16.2169359376249	0.829916147361925\\
16.2423145541157	0.830926471490553\\
16.2676931706065	0.831947293526664\\
16.2930717870973	0.832978415930299\\
16.3184504035881	0.834019641134669\\
16.3438290200789	0.835070771571901\\
16.3692076365697	0.836131609698606\\
16.3945862530606	0.837201958021285\\
16.4199648695514	0.838281619121547\\
16.4453434860422	0.839370395681163\\
16.470722102533	0.840468090506928\\
16.4961007190238	0.841574506555354\\
16.5214793355146	0.842689446957175\\
16.5468579520054	0.843812715041661\\
16.5722365684962	0.844944114360758\\
16.597615184987	0.846083448713027\\
16.6229938014778	0.847230522167399\\
16.6483724179686	0.848385139086737\\
16.6737510344594	0.8495471041512\\
16.6991296509502	0.850716222381417\\
16.724508267441	0.85189229916146\\
16.7498868839318	0.853075140261612\\
16.7752655004226	0.854264551860949\\
16.8006441169134	0.855460340569702\\
16.8260227334043	0.85666231345143\\
16.8514013498951	0.857870278044976\\
16.8767799663859	0.859084042386221\\
16.9021585828767	0.860303415029634\\
16.9275371993675	0.861528205069599\\
16.9529158158583	0.862758222161545\\
16.9782944323491	0.863993276542856\\
17.0036730488399	0.865233179053568\\
17.0290516653307	0.86647774115685\\
17.0544302818215	0.86772677495927\\
17.0798088983123	0.868980093230846\\
17.1051875148031	0.870237509424871\\
17.1305661312939	0.871498837697526\\
17.1559447477847	0.872763892927267\\
17.1813233642755	0.874032490733992\\
17.2067019807663	0.875304447497984\\
17.2320805972571	0.876579580378632\\
17.257459213748	0.877857707332926\\
17.2828378302388	0.879138647133723\\
17.3082164467296	0.880422219387792\\
17.3335950632204	0.88170824455363\\
17.3589736797112	0.882996543959044\\
17.384352296202	0.884286939818512\\
17.4097309126928	0.885579255250311\\
17.4351095291836	0.886873314293412\\
17.4604881456744	0.888168941924148\\
17.4858667621652	0.889465964072645\\
17.511245378656	0.890764207639029\\
17.5366239951468	0.89206350050939\\
17.5620026116376	0.893363671571522\\
17.5873812281284	0.894664550730422\\
17.6127598446192	0.895965968923558\\
17.63813846111	0.8972677581359\\
17.6635170776008	0.898569751414721\\
17.6888956940917	0.899871782884155\\
17.7142743105825	0.901173687759526\\
17.7396529270733	0.902475302361434\\
17.7650315435641	0.903776464129615\\
17.7904101600549	0.905077011636548\\
17.8157887765457	0.906376784600844\\
17.8411673930365	0.907675623900385\\
17.8665460095273	0.90897337158523\\
17.8919246260181	0.910269870890283\\
17.9173032425089	0.911564966247725\\
17.9426818589997	0.912858503299208\\
17.9680604754905	0.914150328907815\\
17.9934390919813	0.915440291169771\\
18.0188177084721	0.916728239425933\\
18.0441963249629	0.918014024273034\\
18.0695749414537	0.919297497574685\\
18.0949535579445	0.920578512472154\\
18.1203321744354	0.921856923394896\\
18.1457107909262	0.923132586070849\\
18.171089407417	0.924405357536502\\
18.1964680239078	0.925675096146714\\
18.2218466403986	0.926941661584307\\
18.2472252568894	0.928204914869422\\
18.2726038733802	0.929464718368633\\
18.297982489871	0.930720935803839\\
18.3233611063618	0.931973432260909\\
18.3487397228526	0.9332220741981\\
18.3741183393434	0.934466729454242\\
18.3994969558342	0.935707267256688\\
18.424875572325	0.936943558229028\\
18.4502541888158	0.938175474398584\\
18.4756328053066	0.939402889203657\\
18.5010114217974	0.940625677500552\\
18.5263900382882	0.94184371557038\\
18.551768654779	0.943056881125613\\
18.5771472712699	0.944265053316427\\
18.6025258877607	0.945468112736806\\
18.6279045042515	0.94666594143043\\
18.6532831207423	0.947858422896318\\
18.6786617372331	0.949045442094268\\
18.7040403537239	0.950226885450052\\
18.7294189702147	0.951402640860403\\
18.7547975867055	0.952572597697764\\
18.7801762031963	0.953736646814827\\
18.8055548196871	0.954894680548842\\
18.8309334361779	0.956046592725709\\
18.8563120526687	0.957192278663851\\
18.8816906691595	0.958331635177867\\
18.9070692856503	0.959464560581963\\
18.9324479021411	0.960590954693177\\
18.9578265186319	0.961710718834377\\
18.9832051351227	0.962823755837047\\
19.0085837516136	0.96392997004387\\
19.0339623681044	0.965029267311077\\
19.0593409845952	0.966121555010609\\
19.084719601086	0.967206742032048\\
19.1100982175768	0.968284738784352\\
19.1354768340676	0.969355457197376\\
19.1608554505584	0.970418810723188\\
19.1862340670492	0.97147471433718\\
19.21161268354	0.972523084538969\\
19.2369913000308	0.973563839353105\\
19.2623699165216	0.974596898329569\\
19.2877485330124	0.97562218254407\\
19.3131271495032	0.976639614598149\\
19.338505765994	0.97764911861908\\
19.3638843824848	0.978650620259573\\
19.3892629989756	0.979644046697288\\
19.4146416154664	0.980629326634147\\
19.4400202319573	0.98160639029546\\
19.4653988484481	0.982575169428855\\
19.4907774649389	0.983535597303023\\
19.5161560814297	0.984487608706266\\
19.5415346979205	0.985431139944871\\
19.5669133144113	0.986366128841283\\
19.5922919309021	0.987292514732109\\
19.6176705473929	0.988210238465927\\
19.6430491638837	0.989119242400922\\
19.6684277803745	0.990019470402339\\
19.6938063968653	0.990910867839758\\
19.7191850133561	0.991793381584191\\
19.7445636298469	0.992666960005009\\
19.7699422463377	0.993531552966691\\
19.7953208628285	0.994387111825403\\
19.8206994793193	0.995233589425402\\
19.8460780958101	0.996070940095281\\
19.871456712301	0.99689911964404\\
19.8968353287918	0.997718085356987\\
19.9222139452826	0.998527795991491\\
19.9475925617734	0.999328211772553\\
19.9729711782642	1.00011929438823\\
19.998349794755	1.0009010069849\\
20.0237284112458	1.00167331416236\\
20.0491070277366	1.00243618196876\\
20.0744856442274	1.00318957789545\\
20.0998642607182	1.00393347087155\\
20.125242877209	1.00466783125849\\
20.1506214936998	1.00539263084435\\
20.1760001101906	1.00610784283803\\
20.2013787266814	1.00681344186332\\
20.2267573431722	1.0075094039528\\
20.252135959663	1.00819570654162\\
20.2775145761538	1.0088723284611\\
20.3028931926446	1.00953924993223\\
20.3282718091355	1.01019645255902\\
20.3536504256263	1.01084391932171\\
20.3790290421171	1.01148163456986\\
20.4044076586079	1.01210958401529\\
20.4297862750987	1.0127277547249\\
20.4551648915895	1.01333613511339\\
20.4805435080803	1.01393471493579\\
20.5059221245711	1.01452348527992\\
20.5313007410619	1.01510243855871\\
20.5566793575527	1.0156715685024\\
20.5820579740435	1.01623087015063\\
20.6074365905343	1.0167803398444\\
20.6328152070251	1.01731997521789\\
20.6581938235159	1.01784977519024\\
20.6835724400067	1.01836973995717\\
20.7089510564975	1.01887987098245\\
20.7343296729883	1.01938017098937\\
20.7597082894792	1.01987064395199\\
20.78508690597	1.02035129508636\\
20.8104655224608	1.02082213084161\\
20.8358441389516	1.02128315889096\\
20.8612227554424	1.02173438812259\\
20.8866013719332	1.02217582863045\\
20.911979988424	1.02260749170497\\
20.9373586049148	1.02302938982365\\
20.9627372214056	1.02344153664162\\
20.9881158378964	1.02384394698201\\
21.0134944543872	1.02423663682635\\
21.038873070878	1.02461962330479\\
21.0642516873688	1.02499292468628\\
21.0896303038596	1.02535656036868\\
21.1150089203504	1.02571055086874\\
21.1403875368412	1.02605491781204\\
21.165766153332	1.02638968392285\\
21.1911447698229	1.02671487301391\\
21.2165233863137	1.02703050997613\\
21.2419020028045	1.02733662076824\\
21.2672806192953	1.02763323240633\\
21.2926592357861	1.02792037295337\\
21.3180378522769	1.02819807150862\\
21.3434164687677	1.02846635819702\\
21.3687950852585	1.0287252641585\\
21.3941737017493	1.02897482153721\\
21.4195523182401	1.02921506347068\\
21.4449309347309	1.02944602407902\\
21.4703095512217	1.02966773845392\\
21.4956881677125	1.02988024264775\\
21.5210667842033	1.03008357366243\\
21.5464454006941	1.03027776943846\\
21.5718240171849	1.03046286884369\\
21.5972026336757	1.03063891166225\\
21.6225812501666	1.03080593858322\\
21.6479598666574	1.03096399118946\\
21.6733384831482	1.03111311194627\\
21.698717099639	1.03125334419004\\
21.7240957161298	1.0313847321169\\
21.7494743326206	1.03150732077126\\
21.7748529491114	1.03162115603443\\
21.8002315656022	1.03172628461306\\
21.825610182093	1.03182275402768\\
21.8509887985838	1.03191061260117\\
21.8763674150746	1.03198990944712\\
21.9017460315654	1.03206069445833\\
21.9271246480562	1.03212301829512\\
21.952503264547	1.03217693237371\\
21.9778818810378	1.0322224888546\\
22.0032604975286	1.03225974063082\\
22.0286391140194	1.03228874131628\\
22.0540177305102	1.03230954523406\\
22.0793963470011	1.03232220740464\\
22.1047749634919	1.03232678353423\\
22.1301535799827	1.03232333000294\\
22.1555321964735	1.03231190385308\\
22.1809108129643	1.03229256277737\\
22.2062894294551	1.03226536510718\\
22.2316680459459	1.03223036980072\\
22.2570466624367	1.03218763643129\\
22.2824252789275	1.03213722517549\\
22.3078038954183	1.03207919680143\\
22.3331825119091	1.03201361265697\\
22.3585611283999	1.03194053465791\\
22.3839397448907	1.03186002527623\\
22.4093183613815	1.03177214752835\\
22.4346969778723	1.03167696496331\\
22.4600755943631	1.03157454165109\\
22.4854542108539	1.03146494217082\\
22.5108328273448	1.03134823159908\\
22.5362114438356	1.03122447549815\\
22.5615900603264	1.03109373990435\\
22.5869686768172	1.03095609131633\\
22.612347293308	1.03081159668344\\
22.6377259097988	1.03066032339402\\
22.6631045262896	1.03050233926384\\
22.6884831427804	1.03033771252444\\
22.7138617592712	1.03016651181158\\
22.739240375762	1.0299888061537\\
22.7646189922528	1.02980466496032\\
22.7899976087436	1.02961415801063\\
22.8153762252344	1.02941735544193\\
22.8407548417252	1.02921432773822\\
22.866133458216	1.0290051457188\\
22.8915120747068	1.02878988052687\\
22.9168906911976	1.02856860361815\\
22.9422693076885	1.02834138674964\\
22.9676479241793	1.02810830196827\\
22.9930265406701	1.0278694215997\\
23.0184051571609	1.02762481823709\\
23.0437837736517	1.02737456472995\\
23.0691623901425	1.02711873417305\\
23.0945410066333	1.02685739989528\\
23.1199196231241	1.02659063544864\\
23.1452982396149	1.02631851459725\\
23.1706768561057	1.02604111130643\\
23.1960554725965	1.02575849973173\\
23.2214340890873	1.02547075420814\\
23.2468127055781	1.02517794923926\\
23.2721913220689	1.02488015948654\\
23.2975699385597	1.02457745975859\\
23.3229485550505	1.02426992500052\\
23.3483271715413	1.02395763028335\\
23.3737057880322	1.02364065079344\\
23.399084404523	1.02331906182204\\
23.4244630210138	1.02299293875483\\
23.4498416375046	1.02266235706153\\
23.4752202539954	1.02232739228561\\
23.5005988704862	1.02198812003403\\
23.525977486977	1.021644615967\\
23.5513561034678	1.02129695578789\\
23.5767347199586	1.02094521523311\\
23.6021133364494	1.02058947006214\\
23.6274919529402	1.02022979604752\\
23.652870569431	1.01986626896501\\
23.6782491859218	1.01949896458376\\
23.7036278024126	1.01912795865652\\
23.7290064189034	1.01875332691\\
23.7543850353942	1.01837514503519\\
23.779763651885	1.01799348867786\\
23.8051422683759	1.01760843342906\\
23.8305208848667	1.01722005481567\\
23.8558995013575	1.01682842829109\\
23.8812781178483	1.01643362922599\\
23.9066567343391	1.01603573289904\\
23.9320353508299	1.01563481448785\\
23.9574139673207	1.01523094905986\\
23.9827925838115	1.01482421156343\\
24.0081712003023	1.01441467681886\\
24.0335498167931	1.01400241950959\\
24.0589284332839	1.01358751417349\\
24.0843070497747	1.01317003519409\\
24.1096856662655	1.0127500567921\\
24.1350642827563	1.01232765301676\\
24.1604428992471	1.01190289773751\\
24.1858215157379	1.01147586463556\\
24.2112001322287	1.01104662719564\\
24.2365787487196	1.01061525869777\\
24.2619573652104	1.01018183220917\\
24.2873359817012	1.00974642057619\\
24.312714598192	1.00930909641636\\
24.3380932146828	1.00886993211055\\
24.3634718311736	1.00842899979509\\
24.3888504476644	1.00798637135418\\
24.4142290641552	1.00754211841214\\
24.439607680646	1.00709631232596\\
24.4649862971368	1.0066490241778\\
24.4903649136276	1.00620032476762\\
24.5157435301184	1.00575028460592\\
24.5411221466092	1.00529897390649\\
24.5665007631	1.00484646257934\\
24.5918793795908	1.00439282022365\\
24.6172579960816	1.00393811612082\\
24.6426366125724	1.00348241922765\\
24.6680152290633	1.00302579816952\\
24.6933938455541	1.00256832123376\\
24.7187724620449	1.00211005636304\\
24.7441510785357	1.00165107114885\\
24.7695296950265	1.0011914328251\\
24.7949083115173	1.00073120826181\\
24.8202869280081	1.00027046395883\\
24.8456655444989	0.999809266039744\\
24.8710441609897	0.999347680245772\\
24.8964227774805	0.998885771929818\\
24.9218013939713	0.998423606050588\\
24.9471800104621	0.9979612471668\\
24.9725586269529	0.997498759431489\\
24.9979372434437	0.997036206586394\\
25.0233158599345	0.996573651956441\\
25.0486944764253	0.996111158444316\\
25.0740730929161	0.99564878852513\\
25.0994517094069	0.995186604241167\\
25.1248303258978	0.994724667196734\\
25.1502089423886	0.994263038553092\\
25.1755875588794	0.993801779023487\\
25.2009661753702	0.99334094886826\\
25.226344791861	0.992880607890063\\
25.2517234083518	0.992420815429151\\
25.2771020248426	0.991961630358778\\
25.3024806413334	0.991503111080672\\
25.3278592578242	0.991045315520612\\
25.353237874315	0.990588301124089\\
25.3786164908058	0.990132124852058\\
25.4039951072966	0.989676843176786\\
25.4293737237874	0.989222512077784\\
25.4547523402782	0.988769187037835\\
25.480130956769	0.98831692303911\\
25.5055095732598	0.987865774559378\\
25.5308881897506	0.987415795568297\\
25.5562668062415	0.986967039523808\\
25.5816454227323	0.986519559368611\\
25.6070240392231	0.986073407526733\\
25.6324026557139	0.985628635900184\\
25.6577812722047	0.985185295865709\\
25.6831598886955	0.984743438271621\\
25.7085385051863	0.984303113434728\\
25.7339171216771	0.983864371137351\\
25.7592957381679	0.983427260624426\\
25.7846743546587	0.982991830600699\\
25.8100529711495	0.982558129228005\\
25.8354315876403	0.982126204122642\\
25.8608102041311	0.981696102352826\\
25.8861888206219	0.981267870436236\\
25.9115674371127	0.980841554337652\\
25.9369460536035	0.980417199466669\\
25.9623246700943	0.97999485067551\\
25.9877032865852	0.979574552256916\\
26.013081903076	0.97915634794213\\
26.0384605195668	0.978740280898959\\
26.0638391360576	0.978326393729932\\
26.0892177525484	0.977914728470532\\
26.1145963690392	0.977505326587521\\
26.13997498553	0.977098228977348\\
26.1653536020208	0.976693475964638\\
26.1907322185116	0.976291107300773\\
26.2161108350024	0.975891162162543\\
26.2414894514932	0.975493679150897\\
26.266868067984	0.975098696289762\\
26.2922466844748	0.974706251024953\\
26.3176253009656	0.974316380223164\\
26.3430039174564	0.973929120171037\\
26.3683825339472	0.973544506574317\\
26.393761150438	0.973162574557084\\
26.4191397669289	0.972783358661066\\
26.4445183834197	0.972406892845034\\
26.4698969999105	0.972033210484273\\
26.4952756164013	0.971662344370139\\
26.5206542328921	0.971294326709682\\
26.5460328493829	0.970929189125361\\
26.5714114658737	0.970566962654827\\
26.5967900823645	0.970207677750788\\
26.6221686988553	0.969851364280945\\
26.6475473153461	0.969498051528015\\
26.6729259318369	0.969147768189814\\
26.6983045483277	0.96880054237943\\
26.7236831648185	0.968456401625462\\
26.7490617813093	0.968115372872334\\
26.7744403978001	0.967777482480685\\
26.7998190142909	0.96744275622783\\
26.8251976307817	0.967111219308293\\
26.8505762472725	0.966782896334416\\
26.8759548637634	0.966457811337033\\
26.9013334802542	0.966135987766218\\
26.926712096745	0.965817448492106\\
26.9520907132358	0.965502215805778\\
26.9774693297266	0.965190311420218\\
27.0028479462174	0.964881756471338\\
27.0282265627082	0.964576571519073\\
27.053605179199	0.964274776548539\\
27.0789837956898	0.96397639097126\\
27.1043624121806	0.963681433626464\\
27.1297410286714	0.963389922782438\\
27.1551196451622	0.963101876137954\\
27.180498261653	0.962817310823753\\
27.2058768781438	0.962536243404102\\
27.2312554946346	0.962258689878404\\
27.2566341111254	0.961984665682873\\
27.2820127276162	0.961714185692279\\
27.3073913441071	0.961447264221742\\
27.3327699605979	0.961183915028592\\
27.3581485770887	0.960924151314292\\
27.3835271935795	0.960667985726416\\
27.4089058100703	0.960415430360686\\
27.4342844265611	0.960166496763065\\
27.4596630430519	0.959921195931916\\
27.4850416595427	0.959679538320203\\
27.5104202760335	0.959441533837765\\
27.5357988925243	0.959207191853629\\
27.5611775090151	0.958976521198389\\
27.5865561255059	0.958749530166636\\
27.6119347419967	0.958526226519434\\
27.6373133584875	0.958306617486863\\
27.6626919749783	0.9580907097706\\
27.6880705914691	0.957878509546559\\
27.7134492079599	0.957670022467577\\
27.7388278244508	0.957465253666158\\
27.7642064409416	0.957264207757253\\
27.7895850574324	0.957066888841103\\
27.8149636739232	0.956873300506115\\
27.840342290414	0.956683445831802\\
27.8657209069048	0.956497327391751\\
27.8910995233956	0.956314947256654\\
27.9164781398864	0.956136306997368\\
27.9418567563772	0.955961407688037\\
27.967235372868	0.955790249909237\\
27.9926139893588	0.955622833751184\\
28.0179926058496	0.955459158816971\\
28.0433712223404	0.95529922422585\\
28.0687498388312	0.955143028616557\\
28.094128455322	0.954990570150674\\
28.1195070718128	0.954841846516038\\
28.1448856883036	0.954696854930172\\
28.1702643047945	0.954555592143774\\
28.1956429212853	0.954418054444232\\
28.2210215377761	0.954284237659174\\
28.2464001542669	0.954154137160063\\
28.2717787707577	0.954027747865819\\
28.2971573872485	0.953905064246477\\
28.3225360037393	0.953786080326885\\
28.3479146202301	0.953670789690426\\
28.3732932367209	0.953559185482777\\
28.3986718532117	0.953451260415705\\
28.4240504697025	0.95334700677088\\
28.4494290861933	0.953246416403733\\
28.4748077026841	0.953149480747336\\
28.5001863191749	0.953056190816312\\
28.5255649356657	0.952966537210773\\
28.5509435521565	0.952880510120286\\
28.5763221686473	0.952798099327868\\
28.6017007851381	0.952719294214002\\
28.627079401629	0.952644083760686\\
28.6524580181198	0.952572456555495\\
28.6778366346106	0.952504400795682\\
28.7032152511014	0.952439904292291\\
28.7285938675922	0.952378954474294\\
28.753972484083	0.952321538392758\\
28.7793511005738	0.952267642725022\\
28.8047297170646	0.952217253778904\\
28.8301083335554	0.952170357496923\\
28.8554869500462	0.95212693946054\\
28.880865566537	0.952086984894423\\
28.9062441830278	0.952050478670721\\
28.9316227995186	0.952017405313363\\
28.9570014160094	0.951987749002371\\
28.9823800325002	0.951961493578188\\
29.007758648991	0.951938622546022\\
29.0331372654818	0.951919119080202\\
29.0585158819727	0.951902966028558\\
29.0838944984635	0.951890145916796\\
29.1092731149543	0.951880640952905\\
29.1346517314451	0.951874433031563\\
29.1600303479359	0.95187150373856\\
29.1854089644267	0.951871834355229\\
29.2107875809175	0.951875405862889\\
29.2361661974083	0.951882198947297\\
29.2615448138991	0.951892194003107\\
29.2869234303899	0.951905371138338\\
29.3123020468807	0.951921710178854\\
29.3376806633715	0.95194119067284\\
29.3630592798623	0.951963791895297\\
29.3884378963531	0.951989492852533\\
29.4138165128439	0.952018272286662\\
29.4391951293347	0.95205010868011\\
29.4645737458255	0.952084980260122\\
29.4899523623164	0.952122865003273\\
29.5153309788072	0.952163740639978\\
29.540709595298	0.952207584659012\\
29.5660882117888	0.952254374312022\\
29.5914668282796	0.952304086618047\\
29.6168454447704	0.952356698368036\\
29.6422240612612	0.95241218612936\\
29.667602677752	0.952470526250333\\
29.6929812942428	0.952531694864727\\
29.7183599107336	0.952595667896278\\
29.7437385272244	0.952662421063201\\
29.7691171437152	0.952731929882693\\
29.794495760206	0.952804169675437\\
29.8198743766968	0.9528791155701\\
29.8452529931876	0.952956742507823\\
29.8706316096784	0.953037025246712\\
29.8960102261692	0.953119938366316\\
29.9213888426601	0.953205456272105\\
29.9467674591509	0.953293553199933\\
29.9721460756417	0.953384203220503\\
29.9975246921325	0.953477380243814\\
30.0229033086233	0.953573058023606\\
30.0482819251141	0.953671210161795\\
30.0736605416049	0.953771810112893\\
30.0990391580957	0.953874831188427\\
30.1244177745865	0.953980246561337\\
30.1497963910773	0.954088029270369\\
30.1751750075681	0.954198152224457\\
30.2005536240589	0.954310588207089\\
30.2259322405497	0.954425309880661\\
30.2513108570405	0.95454228979082\\
30.2766894735313	0.954661500370792\\
30.3020680900221	0.954782913945696\\
30.3274467065129	0.954906502736841\\
30.3528253230038	0.955032238866014\\
30.3782039394946	0.955160094359747\\
30.4035825559854	0.955290041153568\\
30.4289611724762	0.955422051096239\\
30.454339788967	0.955556095953976\\
30.4797184054578	0.95569214741465\\
30.5050970219486	0.95583017709197\\
30.5304756384394	0.955970156529653\\
30.5558542549302	0.95611205720557\\
30.581232871421	0.956255850535873\\
30.6066114879118	0.956401507879106\\
30.6319901044026	0.956549000540294\\
30.6573687208934	0.956698299775014\\
30.6827473373842	0.956849376793441\\
30.708125953875	0.957002202764376\\
30.7335045703658	0.957156748819258\\
30.7588831868566	0.957312986056141\\
30.7842618033475	0.95747088554366\\
30.8096404198383	0.957630418324972\\
30.8350190363291	0.957791555421674\\
30.8603976528199	0.957954267837692\\
30.8857762693107	0.95811852656316\\
30.9111548858015	0.958284302578258\\
30.9365335022923	0.958451566857043\\
30.9619121187831	0.958620290371238\\
30.9872907352739	0.958790444094013\\
31.0126693517647	0.958961999003729\\
31.0380479682555	0.959134926087661\\
31.0634265847463	0.959309196345696\\
31.0888052012371	0.959484780793999\\
31.1141838177279	0.959661650468663\\
31.1395624342187	0.959839776429323\\
31.1649410507095	0.960019129762745\\
31.1903196672003	0.960199681586394\\
31.2156982836912	0.960381403051961\\
31.241076900182	0.96056426534888\\
31.2664555166728	0.9607482397078\\
31.2918341331636	0.960933297404041\\
31.3172127496544	0.96111940976101\\
31.3425913661452	0.961306548153602\\
31.367969982636	0.961494684011558\\
31.3933485991268	0.961683788822805\\
31.4187272156176	0.961873834136755\\
31.4441058321084	0.962064791567587\\
31.4694844485992	0.962256632797487\\
31.49486306509	0.962449329579865\\
31.5202416815808	0.962642853742539\\
31.5456202980716	0.962837177190891\\
31.5709989145624	0.963032271910984\\
31.5963775310532	0.963228109972659\\
31.621756147544	0.963424663532595\\
31.6471347640348	0.963621904837333\\
31.6725133805257	0.963819806226277\\
31.6978919970165	0.964018340134658\\
31.7232706135073	0.964217479096467\\
31.7486492299981	0.964417195747357\\
31.7740278464889	0.964617462827508\\
31.7994064629797	0.964818253184468\\
31.8247850794705	0.965019539775952\\
31.8501636959613	0.965221295672612\\
31.8755423124521	0.96542349406078\\
31.9009209289429	0.965626108245166\\
31.9262995454337	0.965829111651531\\
31.9516781619245	0.966032477829326\\
31.9770567784153	0.966236180454296\\
32.0024353949061	0.966440193331048\\
32.0278140113969	0.966644490395592\\
32.0531926278877	0.96684904571784\\
32.0785712443785	0.967053833504077\\
32.1039498608693	0.967258828099397\\
32.1293284773602	0.967464003990101\\
32.154707093851	0.967669335806067\\
32.1800857103418	0.967874798323082\\
32.2054643268326	0.968080366465137\\
32.2308429433234	0.968286015306695\\
32.2562215598142	0.968491720074919\\
32.281600176305	0.968697456151865\\
32.3069787927958	0.968903199076644\\
32.3323574092866	0.969108924547546\\
32.3577360257774	0.969314608424133\\
32.3831146422682	0.969520226729294\\
32.408493258759	0.969725755651266\\
32.4338718752498	0.969931171545621\\
32.4592504917406	0.970136450937216\\
32.4846291082314	0.970341570522115\\
32.5100077247222	0.970546507169464\\
32.535386341213	0.970751237923343\\
32.5607649577039	0.970955740004579\\
32.5861435741947	0.971159990812517\\
32.6115221906855	0.971363967926772\\
32.6369008071763	0.971567649108928\\
32.6622794236671	0.971771012304215\\
32.6876580401579	0.971974035643148\\
32.7130366566487	0.972176697443125\\
32.7384152731395	0.972378976210003\\
32.7637938896303	0.972580850639624\\
32.7891725061211	0.972782299619318\\
32.8145511226119	0.972983302229365\\
32.8399297391027	0.973183837744428\\
32.8653083555935	0.973383885634943\\
32.8906869720843	0.973583425568481\\
32.9160655885751	0.973782437411074\\
32.9414442050659	0.973980901228508\\
32.9668228215567	0.974178797287576\\
32.9922014380476	0.974376106057302\\
33.0175800545384	0.974572808210133\\
33.0429586710292	0.974768884623084\\
33.06833728752	0.974964316378868\\
33.0937159040108	0.975159084766975\\
33.1190945205016	0.975353171284727\\
33.1444731369924	0.975546557638292\\
33.1698517534832	0.975739225743676\\
33.195230369974	0.975931157727665\\
33.2206089864648	0.976122335928748\\
33.2459876029556	0.976312742897996\\
33.2713662194464	0.976502361399916\\
33.2967448359372	0.976691174413266\\
33.322123452428	0.97687916513184\\
33.3475020689188	0.977066316965214\\
33.3728806854096	0.977252613539473\\
33.3982593019005	0.977438038697885\\
33.4236379183913	0.977622576501564\\
33.4490165348821	0.977806211230083\\
33.4743951513729	0.977988927382062\\
33.4997737678637	0.978170709675727\\
33.5251523843545	0.97835154304943\\
33.5505310008453	0.978531412662141\\
33.5759096173361	0.978710303893908\\
33.6012882338269	0.978888202346286\\
33.6266668503177	0.979065093842729\\
33.6520454668085	0.979240964428959\\
33.6774240832993	0.9794158003733\\
33.7028026997901	0.979589588166977\\
33.7281813162809	0.979762314524391\\
33.7535599327717	0.97993396638336\\
33.7789385492625	0.980104530905329\\
33.8043171657533	0.98027399547555\\
33.8296957822442	0.980442347703233\\
33.855074398735	0.980609575421667\\
33.8804530152258	0.980775666688309\\
33.9058316317166	0.980940609784845\\
33.9312102482074	0.981104393217223\\
33.9565888646982	0.981267005715654\\
33.981967481189	0.981428436234583\\
34.0073460976798	0.981588673952638\\
34.0327247141706	0.981747708272546\\
34.0581033306614	0.981905528821011\\
34.0834819471522	0.982062125448587\\
34.108860563643	0.982217488229496\\
34.1342391801338	0.982371607461442\\
34.1596177966246	0.982524473665378\\
34.1849964131154	0.982676077585259\\
34.2103750296062	0.982826410187762\\
34.235753646097	0.982975462661981\\
34.2611322625879	0.983123226419091\\
34.2865108790787	0.983269693091993\\
34.3118894955695	0.983414854534925\\
34.3372681120603	0.983558702823053\\
34.3626467285511	0.983701230252032\\
34.3880253450419	0.983842429337542\\
34.4134039615327	0.983982292814803\\
34.4387825780235	0.984120813638056\\
34.4641611945143	0.98425798498003\\
34.4895398110051	0.984393800231376\\
34.5149184274959	0.984528253000078\\
34.5402970439867	0.984661337110842\\
34.5656756604775	0.984793046604464\\
34.5910542769683	0.984923375737161\\
34.6164328934591	0.985052318979898\\
34.6418115099499	0.985179871017675\\
34.6671901264407	0.9853060267488\\
34.6925687429315	0.985430781284136\\
34.7179473594224	0.985554129946326\\
34.7433259759132	0.985676068268997\\
34.768704592404	0.98579659199594\\
34.7940832088948	0.985915697080269\\
34.8194618253856	0.98603337968356\\
34.8448404418764	0.986149636174967\\
34.8702190583672	0.986264463130315\\
34.895597674858	0.986377857331179\\
34.9209762913488	0.986489815763934\\
34.9463549078396	0.986600335618793\\
34.9717335243304	0.986709414288816\\
34.9971121408212	0.986817049368909\\
35.022490757312	0.986923238654796\\
35.0478693738028	0.987027980141977\\
35.0732479902936	0.98713127202466\\
35.0986266067844	0.987233112694685\\
35.1240052232752	0.987333500740422\\
35.1493838397661	0.987432434945647\\
35.1747624562569	0.987529914288415\\
35.2001410727477	0.987625937939898\\
35.2255196892385	0.987720505263217\\
35.2508983057293	0.987813615812254\\
35.2762769222201	0.987905269330447\\
35.3016555387109	0.987995465749567\\
35.3270341552017	0.98808420518848\\
35.3524127716925	0.988171487951894\\
35.3777913881833	0.988257314529089\\
35.4031700046741	0.988341685592632\\
35.4285486211649	0.988424601997074\\
35.4539272376557	0.988506064777641\\
35.4793058541465	0.988586075148896\\
35.5046844706373	0.988664634503402\\
35.5300630871281	0.988741744410358\\
35.5554417036189	0.988817406614229\\
35.5808203201098	0.988891623033362\\
35.6061989366006	0.988964395758581\\
35.6315775530914	0.989035727051783\\
35.6569561695822	0.989105619344505\\
35.682334786073	0.989174075236492\\
35.7077134025638	0.989241097494243\\
35.7330920190546	0.989306689049554\\
35.7584706355454	0.989370852998041\\
35.7838492520362	0.989433592597655\\
35.809227868527	0.989494911267188\\
35.8346064850178	0.989554812584766\\
35.8599851015086	0.989613300286329\\
35.8853637179994	0.989670378264106\\
35.9107423344902	0.989726050565076\\
35.936120950981	0.989780321389418\\
35.9614995674718	0.989833195088959\\
35.9868781839626	0.989884676165599\\
36.0122568004535	0.989934769269745\\
36.0376354169443	0.989983479198719\\
36.0630140334351	0.990030810895168\\
36.0883926499259	0.990076769445462\\
36.1137712664167	0.990121360078086\\
36.1391498829075	0.990164588162022\\
36.1645284993983	0.990206459205125\\
36.1899071158891	0.990246978852492\\
36.2152857323799	0.99028615288482\\
36.2406643488707	0.990323987216766\\
36.2660429653615	0.990360487895292\\
36.2914215818523	0.990395661098008\\
36.3168001983431	0.990429513131508\\
36.3421788148339	0.990462050429701\\
36.3675574313247	0.990493279552137\\
36.3929360478155	0.990523207182325\\
36.4183146643063	0.990551840126048\\
36.4436932807971	0.990579185309677\\
36.469071897288	0.990605249778474\\
36.4944505137788	0.990630040694893\\
36.5198291302696	0.990653565336883\\
36.5452077467604	0.990675831096176\\
36.5705863632512	0.990696845476582\\
36.595964979742	0.990716616092277\\
36.6213435962328	0.990735150666084\\
36.6467222127236	0.990752457027759\\
36.6721008292144	0.990768543112271\\
36.6974794457052	0.990783416958076\\
36.722858062196	0.990797086705393\\
36.7482366786868	0.990809560594482\\
36.7736152951776	0.990820846963914\\
36.7989939116684	0.990830954248837\\
36.8243725281592	0.990839890979256\\
36.84975114465	0.990847665778292\\
36.8751297611408	0.990854287360458\\
36.9005083776317	0.990859764529923\\
36.9258869941225	0.990864106178782\\
36.9512656106133	0.990867321285325\\
36.9766442271041	0.990869418912302\\
37.0020228435949	0.990870408205195\\
37.0274014600857	0.990870298390489\\
37.0527800765765	0.99086909877394\\
37.0781586930673	0.990866818738848\\
37.1035373095581	0.99086346774433\\
37.1289159260489	0.990859055323598\\
37.1542945425397	0.990853591082232\\
37.1796731590305	0.990847084696461\\
37.2050517755213	0.990839545911445\\
37.2304303920121	0.990830984539556\\
37.2558090085029	0.990821410458669\\
37.2811876249937	0.990810833610446\\
37.3065662414845	0.990799263998631\\
37.3319448579754	0.990786711687349\\
37.3573234744662	0.990773186799399\\
37.382702090957	0.990758699514559\\
37.4080807074478	0.990743260067895\\
37.4334593239386	0.990726878748071\\
37.4588379404294	0.990709565895662\\
37.4842165569202	0.990691331901476\\
37.509595173411	0.990672187204875\\
37.5349737899018	0.990652142292108\\
37.5603524063926	0.990631207694641\\
37.5857310228834	0.990609393987501\\
37.6111096393742	0.990586711787617\\
37.636488255865	0.990563171752167\\
37.6618668723558	0.990538784576944\\
37.6872454888466	0.990513560994706\\
37.7126241053374	0.990487511773552\\
37.7380027218282	0.990460647715294\\
37.763381338319	0.990432979653834\\
37.7887599548099	0.990404518453557\\
37.8141385713007	0.990375275007718\\
37.8395171877915	0.990345260236851\\
37.8648958042823	0.990314485087167\\
37.8902744207731	0.990282960528979\\
37.9156530372639	0.990250697555117\\
37.9410316537547	0.990217707179361\\
37.9664102702455	0.990184000434878\\
37.9917888867363	0.990149588372667\\
38.0171675032271	0.990114482060016\\
38.0425461197179	0.990078692578958\\
38.0679247362087	0.990042231024744\\
38.0933033526995	0.990005108504324\\
38.1186819691903	0.989967336134826\\
38.1440605856811	0.989928925042061\\
38.1694392021719	0.989889886359022\\
38.1948178186627	0.989850231224398\\
38.2201964351536	0.989809970781096\\
38.2455750516444	0.989769116174778\\
38.2709536681352	0.989727678552394\\
38.296332284626	0.989685669060738\\
38.3217109011168	0.98964309884501\\
38.3470895176076	0.989599979047381\\
38.3724681340984	0.989556320805574\\
38.3978467505892	0.989512135251458\\
38.42322536708	0.989467433509645\\
38.4486039835708	0.989422226696099\\
38.4739826000616	0.98937652591676\\
38.4993612165524	0.98933034226617\\
38.5247398330432	0.989283686826122\\
38.550118449534	0.989236570664305\\
38.5754970660248	0.989189004832972\\
38.6008756825156	0.98914100036761\\
38.6262542990064	0.989092568285628\\
38.6516329154973	0.989043719585051\\
38.6770115319881	0.988994465243227\\
38.7023901484789	0.988944816215548\\
38.7277687649697	0.988894783434174\\
38.7531473814605	0.98884437780678\\
38.7785259979513	0.988793610215307\\
38.8039046144421	0.988742491514721\\
38.8292832309329	0.988691032531797\\
38.8546618474237	0.988639244063899\\
38.8800404639145	0.988587136877787\\
38.9054190804053	0.988534721708421\\
38.9307976968961	0.988482009257789\\
38.9561763133869	0.98842901019374\\
38.9815549298777	0.988375735148834\\
39.0069335463685	0.988322194719199\\
39.0323121628593	0.988268399463404\\
39.0576907793501	0.988214359901346\\
39.083069395841	0.988160086513142\\
39.1084480123318	0.98810558973804\\
39.1338266288226	0.988050879973344\\
39.1592052453134	0.987995967573343\\
39.1845838618042	0.987940862848262\\
39.209962478295	0.987885576063219\\
39.2353410947858	0.9878301174372\\
39.2607197112766	0.98777449714204\\
39.2860983277674	0.987718725301425\\
39.3114769442582	0.987662811989902\\
39.336855560749	0.987606767231901\\
39.3622341772398	0.987550601000772\\
39.3876127937306	0.987494323217837\\
39.4129914102214	0.987437943751449\\
39.4383700267122	0.987381472416071\\
39.463748643203	0.987324918971364\\
39.4891272596938	0.987268293121288\\
39.5145058761846	0.987211604513218\\
39.5398844926755	0.987154862737073\\
39.5652631091663	0.987098077324453\\
39.5906417256571	0.987041257747803\\
39.6160203421479	0.986984413419571\\
39.6413989586387	0.986927553691396\\
39.6667775751295	0.9868706878533\\
39.6921561916203	0.986813825132897\\
39.7175348081111	0.986756974694611\\
39.7429134246019	0.986700145638917\\
39.7682920410927	0.986643347001582\\
39.7936706575835	0.986586587752931\\
39.8190492740743	0.986529876797119\\
39.8444278905651	0.986473222971421\\
39.8698065070559	0.986416635045531\\
39.8951851235467	0.98636012172088\\
39.9205637400375	0.986303691629961\\
39.9459423565283	0.986247353335672\\
39.9713209730192	0.986191115330671\\
39.99669958951	0.986134986036744\\
40.0220782060008	0.986078973804186\\
40.0474568224916	0.986023086911197\\
40.0728354389824	0.985967333563288\\
40.0982140554732	0.985911721892708\\
40.123592671964	0.985856259957872\\
40.1489712884548	0.985800955742813\\
40.1743499049456	0.985745817156643\\
40.1997285214364	0.985690852033029\\
40.2251071379272	0.985636068129679\\
40.250485754418	0.985581473127842\\
40.2758643709088	0.985527074631827\\
40.3012429873996	0.985472880168527\\
40.3266216038904	0.985418897186961\\
40.3520002203812	0.985365133057825\\
40.377378836872	0.985311595073065\\
40.4027574533629	0.985258290445452\\
40.4281360698537	0.985205226308176\\
40.4535146863445	0.985152409714452\\
40.4788933028353	0.985099847637142\\
40.5042719193261	0.985047546968381\\
40.5296505358169	0.984995514519227\\
40.5550291523077	0.984943757019317\\
40.5804077687985	0.984892281116538\\
40.6057863852893	0.984841093376706\\
40.6311650017801	0.984790200283268\\
40.6565436182709	0.984739608237009\\
40.6819222347617	0.984689323555768\\
40.7073008512525	0.98463935247418\\
40.7326794677433	0.984589701143416\\
40.7580580842341	0.984540375630945\\
40.7834367007249	0.984491381920302\\
40.8088153172157	0.984442725910877\\
40.8341939337066	0.984394413417706\\
40.8595725501974	0.98434645017128\\
40.8849511666882	0.984298841817369\\
40.910329783179	0.984251593916848\\
40.9357083996698	0.984204711945549\\
40.9610870161606	0.984158201294113\\
40.9864656326514	0.984112067267858\\
41.0118442491422	0.984066315086663\\
41.037222865633	0.984020949884859\\
41.0626014821238	0.983975976711132\\
41.0879800986146	0.983931400528439\\
41.1133587151054	0.983887226213938\\
41.1387373315962	0.983843458558925\\
41.164115948087	0.983800102268784\\
41.1894945645778	0.98375716196295\\
41.2148731810686	0.983714642174884\\
41.2402517975594	0.983672547352056\\
41.2656304140502	0.983630881855938\\
41.2910090305411	0.98358964996202\\
41.3163876470319	0.983548855859817\\
41.3417662635227	0.983508503652908\\
41.3671448800135	0.983468597358968\\
41.3925234965043	0.983429140909826\\
41.4179021129951	0.983390138151522\\
41.4432807294859	0.98335159284438\\
41.4686593459767	0.983313508663095\\
41.4940379624675	0.98327588919682\\
41.5194165789583	0.983238737949275\\
41.5447951954491	0.983202058338861\\
41.5701738119399	0.983165853698782\\
41.5955524284307	0.983130127277182\\
41.6209310449215	0.983094882237289\\
41.6463096614123	0.98306012165757\\
41.6716882779031	0.983025848531897\\
41.6970668943939	0.982992065769721\\
41.7224455108848	0.982958776196256\\
41.7478241273756	0.982925982552673\\
41.7732027438664	0.982893687496306\\
41.7985813603572	0.982861893600863\\
41.823959976848	0.982830603356649\\
41.8493385933388	0.9827998191708\\
41.8747172098296	0.98276954336752\\
41.9000958263204	0.982739778188337\\
41.9254744428112	0.982710525792358\\
41.950853059302	0.982681788256538\\
41.9762316757928	0.982653567575959\\
42.0016102922836	0.982625865664115\\
42.0269889087744	0.982598684353205\\
42.0523675252652	0.98257202539444\\
42.077746141756	0.98254589045835\\
42.1031247582468	0.982520281135107\\
42.1285033747376	0.982495198934853\\
42.1538819912285	0.982470645288035\\
42.1792606077193	0.982446621545751\\
42.2046392242101	0.982423128980102\\
42.2300178407009	0.982400168784553\\
42.2553964571917	0.982377742074299\\
42.2807750736825	0.982355849886644\\
42.3061536901733	0.982334493181381\\
42.3315323066641	0.982313672841187\\
42.3569109231549	0.982293389672018\\
42.3822895396457	0.982273644403516\\
42.4076681561365	0.98225443768942\\
42.4330467726273	0.98223577010799\\
42.4584253891181	0.982217642162428\\
42.4838040056089	0.982200054281319\\
42.5091826220997	0.982183006819064\\
42.5345612385905	0.982166500056332\\
42.5599398550813	0.982150534200511\\
42.5853184715722	0.982135109386173\\
42.610697088063	0.982120225675534\\
42.6360757045538	0.982105883058931\\
42.6614543210446	0.982092081455301\\
42.6868329375354	0.982078820712663\\
42.7122115540262	0.982066100608615\\
42.737590170517	0.982053920850822\\
42.7629687870078	0.98204228107753\\
42.7883474034986	0.982031180858062\\
42.8137260199894	0.982020619693343\\
42.8391046364802	0.982010597016411\\
42.864483252971	0.982001112192946\\
42.8898618694618	0.981992164521799\\
42.9152404859526	0.981983753235528\\
42.9406191024434	0.981975877500933\\
42.9659977189342	0.98196853641961\\
42.991376335425	0.981961729028491\\
43.0167549519158	0.981955454300405\\
43.0421335684067	0.981949711144636\\
43.0675121848975	0.981944498407485\\
43.0928908013883	0.981939814872838\\
43.1182694178791	0.98193565926274\\
43.1436480343699	0.981932030237973\\
43.1690266508607	0.98192892639863\\
43.1944052673515	0.981926346284708\\
43.2197838838423	0.981924288376691\\
43.2451625003331	0.981922751096143\\
43.2705411168239	0.981921732806308\\
43.2959197333147	0.981921231812705\\
43.3212983498055	0.981921246363732\\
43.3466769662963	0.981921774651275\\
43.3720555827871	0.981922814811317\\
43.3974341992779	0.981924364924547\\
43.4228128157687	0.981926423016984\\
43.4481914322595	0.981928987060586\\
43.4735700487504	0.981932054973882\\
43.4989486652412	0.981935624622591\\
43.524327281732	0.98193969382025\\
43.5497058982228	0.981944260328845\\
43.5750845147136	0.981949321859443\\
43.6004631312044	0.981954876072827\\
43.6258417476952	0.981960920580135\\
43.651220364186	0.981967452943496\\
43.6765989806768	0.981974470676675\\
43.7019775971676	0.981981971245714\\
43.7273562136584	0.981989952069578\\
43.7527348301492	0.981998410520803\\
43.77811344664	0.982007343926146\\
43.8034920631308	0.982016749567232\\
43.8288706796216	0.982026624681209\\
43.8542492961124	0.982036966461402\\
43.8796279126032	0.982047772057964\\
43.9050065290941	0.982059038578538\\
43.9303851455849	0.98207076308891\\
43.9557637620757	0.982082942613667\\
43.9811423785665	0.98209557413686\\
44.0065209950573	0.982108654602659\\
44.0318996115481	0.982122180916021\\
44.0572782280389	0.982136149943343\\
44.0826568445297	0.982150558513132\\
44.1080354610205	0.982165403416664\\
44.1334140775113	0.982180681408645\\
44.1587926940021	0.982196389207881\\
44.1841713104929	0.982212523497936\\
44.2095499269837	0.982229080927797\\
44.2349285434745	0.982246058112541\\
44.2603071599653	0.982263451633995\\
44.2856857764561	0.982281258041402\\
44.3110643929469	0.982299473852086\\
44.3364430094378	0.982318095552112\\
44.3618216259286	0.982337119596951\\
44.3872002424194	0.982356542412142\\
44.4125788589102	0.982376360393958\\
44.437957475401	0.982396569910059\\
44.4633360918918	0.982417167300161\\
44.4887147083826	0.982438148876693\\
44.5140933248734	0.982459510925455\\
44.5394719413642	0.982481249706278\\
44.564850557855	0.98250336145368\\
44.5902291743458	0.982525842377523\\
44.6156077908366	0.982548688663669\\
44.6409864073274	0.982571896474632\\
44.6663650238182	0.98259546195023\\
44.691743640309	0.982619381208239\\
44.7171222567998	0.98264365034504\\
44.7425008732906	0.982668265436268\\
44.7678794897814	0.982693222537458\\
44.7932581062723	0.982718517684691\\
44.8186367227631	0.982744146895236\\
44.8440153392539	0.98277010616819\\
44.8693939557447	0.982796391485122\\
44.8947725722355	0.982822998810703\\
44.9201511887263	0.982849924093347\\
44.9455298052171	0.982877163265844\\
44.9709084217079	0.982904712245987\\
44.9962870381987	0.982932566937204\\
45.0216656546895	0.982960723229184\\
45.0470442711803	0.982989176998502\\
45.0724228876711	0.983017924109236\\
45.0978015041619	0.983046960413592\\
45.1231801206527	0.983076281752518\\
45.1485587371435	0.983105883956316\\
45.1739373536343	0.983135762845257\\
45.1993159701251	0.983165914230187\\
45.224694586616	0.983196333913134\\
45.2500732031068	0.983227017687908\\
45.2754518195976	0.983257961340705\\
45.3008304360884	0.983289160650701\\
45.3262090525792	0.983320611390646\\
45.35158766907	0.983352309327456\\
45.3769662855608	0.9833842502228\\
45.4023449020516	0.983416429833683\\
45.4277235185424	0.98344884391303\\
45.4531021350332	0.983481488210258\\
45.478480751524	0.98351435847186\\
45.5038593680148	0.983547450441965\\
45.5292379845056	0.983580759862913\\
45.5546166009964	0.983614282475816\\
45.5799952174872	0.98364801402112\\
45.605373833978	0.983681950239158\\
45.6307524504688	0.983716086870708\\
45.6561310669597	0.983750419657537\\
45.6815096834505	0.983784944342954\\
45.7068882999413	0.983819656672341\\
45.7322669164321	0.983854552393701\\
45.7576455329229	0.983889627258189\\
45.7830241494137	0.983924877020635\\
45.8084027659045	0.983960297440082\\
45.8337813823953	0.983995884280299\\
45.8591599988861	0.984031633310301\\
45.8845386153769	0.984067540304866\\
45.9099172318677	0.984103601045039\\
45.9352958483585	0.984139811318643\\
45.9606744648493	0.984176166920777\\
45.9860530813401	0.984212663654313\\
46.0114316978309	0.984249297330391\\
46.0368103143217	0.984286063768904\\
46.0621889308125	0.984322958798984\\
46.0875675473034	0.984359978259483\\
46.1129461637942	0.984397117999443\\
46.138324780285	0.984434373878573\\
46.1637033967758	0.984471741767709\\
46.1890820132666	0.984509217549281\\
46.2144606297574	0.984546797117768\\
46.2398392462482	0.984584476380148\\
46.265217862739	0.98462225125635\\
46.2905964792298	0.984660117679692\\
46.3159750957206	0.984698071597325\\
46.3413537122114	0.984736108970666\\
46.3667323287022	0.984774225775822\\
46.392110945193	0.98481241800402\\
46.4174895616838	0.984850681662024\\
46.4428681781746	0.98488901277255\\
46.4682467946654	0.984927407374677\\
46.4936254111562	0.98496586152425\\
46.5190040276471	0.985004371294283\\
46.5443826441379	0.985042932775352\\
46.5697612606287	0.985081542075988\\
46.5951398771195	0.985120195323058\\
46.6205184936103	0.985158888662153\\
46.6458971101011	0.985197618257956\\
46.6712757265919	0.985236380294619\\
46.6966543430827	0.985275170976123\\
46.7220329595735	0.985313986526645\\
46.7474115760643	0.985352823190909\\
46.7727901925551	0.985391677234538\\
46.7981688090459	0.985430544944401\\
46.8235474255367	0.985469422628953\\
46.8489260420275	0.985508306618569\\
46.8743046585183	0.985547193265877\\
46.8996832750091	0.985586078946083\\
46.9250618914999	0.985624960057291\\
46.9504405079908	0.985663833020817\\
46.9758191244816	0.985702694281502\\
47.0011977409724	0.985741540308014\\
47.0265763574632	0.985780367593152\\
47.051954973954	0.985819172654136\\
47.0773335904448	0.985857952032902\\
47.1027122069356	0.985896702296383\\
47.1280908234264	0.985935420036789\\
47.1534694399172	0.985974101871884\\
47.178848056408	0.986012744445251\\
47.2042266728988	0.986051344426561\\
47.2296052893896	0.986089898511828\\
47.2549839058804	0.986128403423666\\
47.2803625223712	0.986166855911536\\
47.305741138862	0.986205252751989\\
47.3311197553528	0.986243590748908\\
47.3564983718436	0.986281866733737\\
47.3818769883345	0.986320077565713\\
47.4072556048253	0.986358220132087\\
47.4326342213161	0.986396291348342\\
47.4580128378069	0.986434288158408\\
47.4833914542977	0.98647220753487\\
47.5087700707885	0.986510046479166\\
47.5341486872793	0.986547802021791\\
47.5595273037701	0.986585471222487\\
47.5849059202609	0.98662305117043\\
47.6102845367517	0.986660538984413\\
47.6356631532425	0.986697931813024\\
47.6610417697333	0.986735226834819\\
47.6864203862241	0.986772421258488\\
47.7117990027149	0.986809512323017\\
47.7371776192057	0.986846497297848\\
47.7625562356965	0.986883373483031\\
47.7879348521873	0.986920138209369\\
47.8133134686782	0.986956788838563\\
47.838692085169	0.986993322763348\\
47.8640707016598	0.987029737407627\\
47.8894493181506	0.987066030226598\\
47.9148279346414	0.987102198706879\\
47.9402065511322	0.98713824036662\\
47.965585167623	0.987174152755624\\
47.9909637841138	0.987209933455449\\
48.0163424006046	0.987245580079516\\
48.0417210170954	0.987281090273205\\
48.0670996335862	0.987316461713947\\
48.092478250077	0.987351692111318\\
48.1178568665678	0.987386779207121\\
48.1432354830586	0.987421720775465\\
48.1686140995494	0.987456514622838\\
48.1939927160402	0.987491158588184\\
48.219371332531	0.98752565054296\\
48.2447499490219	0.987559988391202\\
48.2701285655127	0.987594170069583\\
48.2955071820035	0.987628193547458\\
48.3208857984943	0.987662056826917\\
48.3462644149851	0.987695757942826\\
48.3716430314759	0.987729294962864\\
48.3970216479667	0.987762665987556\\
48.4224002644575	0.987795869150306\\
48.4477788809483	0.987828902617416\\
48.4731574974391	0.987861764588111\\
48.4985361139299	0.987894453294552\\
48.5239147304207	0.987926967001849\\
48.5492933469115	0.987959304008066\\
48.5746719634023	0.987991462644226\\
48.6000505798931	0.988023441274311\\
48.6254291963839	0.988055238295249\\
48.6508078128747	0.988086852136913\\
48.6761864293656	0.988118281262101\\
48.7015650458564	0.98814952416652\\
48.7269436623472	0.988180579378763\\
48.752322278838	0.988211445460281\\
48.7777008953288	0.988242121005357\\
48.8030795118196	0.988272604641064\\
48.8284581283104	0.988302895027234\\
48.8538367448012	0.988332990856407\\
48.879215361292	0.988362890853794\\
48.9045939777828	0.988392593777218\\
48.9299725942736	0.988422098417064\\
48.9553512107644	0.988451403596219\\
48.9807298272552	0.988480508170011\\
49.006108443746	0.988509411026143\\
49.0314870602368	0.988538111084622\\
49.0568656767276	0.988566607297685\\
49.0822442932184	0.988594898649727\\
49.1076229097092	0.988622984157215\\
49.1330015262001	0.988650862868603\\
49.1583801426909	0.98867853386425\\
49.1837587591817	0.988705996256322\\
49.2091373756725	0.988733249188703\\
49.2345159921633	0.988760291836892\\
49.2598946086541	0.988787123407901\\
49.2852732251449	0.988813743140153\\
49.3106518416357	0.988840150303373\\
49.3360304581265	0.988866344198472\\
49.3614090746173	0.988892324157436\\
49.3867876911081	0.988918089543205\\
49.4121663075989	0.988943639749554\\
49.4375449240897	0.988968974200963\\
49.4629235405805	0.988994092352494\\
49.4883021570713	0.989018993689659\\
49.5136807735621	0.989043677728283\\
49.5390593900529	0.989068144014368\\
49.5644380065438	0.989092392123955\\
49.5898166230346	0.989116421662975\\
49.6151952395254	0.989140232267111\\
49.6405738560162	0.989163823601642\\
49.665952472507	0.989187195361295\\
49.6913310889978	0.989210347270087\\
49.7167097054886	0.989233279081171\\
49.7420883219794	0.989255990576674\\
49.7674669384702	0.989278481567535\\
49.792845554961	0.989300751893337\\
49.8182241714518	0.989322801422142\\
49.8436027879426	0.989344630050321\\
49.8689814044334	0.989366237702375\\
49.8943600209242	0.989387624330765\\
49.919738637415	0.989408789915732\\
49.9451172539058	0.989429734465116\\
49.9704958703966	0.989450458014174\\
49.9958744868875	0.989470960625393\\
50.0212531033783	0.989491242388304\\
50.0466317198691	0.989511303419291\\
50.0720103363599	0.9895311438614\\
50.0973889528507	0.989550763884143\\
50.1227675693415	0.989570163683306\\
50.1481461858323	0.989589343480743\\
50.1735248023231	0.989608303524182\\
50.1989034188139	0.989627044087022\\
50.2242820353047	0.989645565468123\\
50.2496606517955	0.989663867991605\\
50.2750392682863	0.989681952006639\\
50.3004178847771	0.989699817887233\\
50.3257965012679	0.989717466032025\\
50.3511751177587	0.989734896864064\\
50.3765537342495	0.989752110830599\\
50.4019323507403	0.98976910840286\\
50.4273109672312	0.989785890075837\\
50.452689583722	0.989802456368062\\
50.4780682002128	0.989818807821387\\
50.5034468167036	0.989834945000757\\
50.5288254331944	0.989850868493991\\
50.5542040496852	0.989866578911547\\
50.579582666176	0.989882076886301\\
50.6049612826668	0.989897363073315\\
50.6303398991576	0.989912438149605\\
50.6557185156484	0.989927302813909\\
50.6810971321392	0.989941957786456\\
50.70647574863	0.989956403808729\\
50.7318543651208	0.98997064164323\\
50.7572329816116	0.989984672073243\\
50.7826115981024	0.989998495902596\\
50.8079902145932	0.990012113955419\\
50.833368831084	0.990025527075911\\
50.8587474475748	0.990038736128088\\
50.8841260640657	0.990051741995553\\
50.9095046805565	0.990064545581241\\
50.9348832970473	0.990077147807184\\
50.9602619135381	0.990089549614262\\
50.9856405300289	0.990101751961956\\
51.0110191465197	0.990113755828106\\
51.0363977630105	0.990125562208659\\
51.0617763795013	0.990137172117426\\
51.0871549959921	0.990148586585827\\
51.1125336124829	0.990159806662648\\
51.1379122289737	0.990170833413788\\
51.1632908454645	0.990181667922008\\
51.1886694619553	0.990192311286683\\
51.2140480784461	0.990202764623547\\
51.2394266949369	0.990213029064444\\
51.2648053114277	0.990223105757074\\
51.2901839279185	0.990232995864742\\
51.3155625444094	0.990242700566103\\
51.3409411609002	0.990252221054911\\
51.366319777391	0.990261558539762\\
51.3916983938818	0.990270714243844\\
51.4170770103726	0.990279689404681\\
51.4424556268634	0.990288485273879\\
51.4678342433542	0.990297103116871\\
51.493212859845	0.990305544212663\\
51.5185914763358	0.990313809853582\\
51.5439700928266	0.990321901345018\\
51.5693487093174	0.990329820005169\\
51.5947273258082	0.990337567164791\\
51.620105942299	0.990345144166938\\
51.6454845587898	0.990352552366712\\
51.6708631752806	0.990359793131007\\
51.6962417917714	0.990366867838254\\
51.7216204082622	0.99037377787817\\
51.7469990247531	0.990380524651499\\
51.7723776412439	0.990387109569765\\
51.7977562577347	0.990393534055014\\
51.8231348742255	0.990399799539566\\
51.8485134907163	0.990405907465756\\
51.8738921072071	0.990411859285688\\
51.8992707236979	0.990417656460978\\
51.9246493401887	0.990423300462509\\
51.9500279566795	0.990428792770175\\
51.9754065731703	0.990434134872632\\
52.0007851896611	0.99043932826705\\
52.0261638061519	0.990444374458861\\
52.0515424226427	0.990449274961513\\
52.0769210391335	0.990454031296221\\
52.1022996556243	0.990458644991719\\
52.1276782721151	0.990463117584013\\
52.1530568886059	0.990467450616136\\
52.1784355050968	0.990471645637905\\
52.2038141215876	0.990475704205671\\
52.2291927380784	0.990479627882078\\
52.2545713545692	0.990483418235822\\
52.27994997106	0.990487076841407\\
52.3053285875508	0.990490605278901\\
52.3307072040416	0.990494005133701\\
52.3560858205324	0.99049727799629\\
52.3814644370232	0.990500425462\\
52.406843053514	0.99050344913077\\
52.4322216700048	0.990506350606918\\
52.4576002864956	0.990509131498897\\
52.4829789029864	0.990511793419064\\
52.5083575194772	0.990514337983447\\
52.533736135968	0.990516766811511\\
52.5591147524588	0.990519081525927\\
52.5844933689496	0.990521283752342\\
52.6098719854404	0.99052337511915\\
52.6352506019313	0.990525357257263\\
52.6606292184221	0.990527231799885\\
52.6860078349129	0.990529000382288\\
52.7113864514037	0.990530664641585\\
52.7367650678945	0.990532226216506\\
52.7621436843853	0.99053368674718\\
52.7875223008761	0.990535047874912\\
52.8129009173669	0.990536311241964\\
52.8382795338577	0.990537478491338\\
52.8636581503485	0.990538551266558\\
52.8890367668393	0.990539531211455\\
52.9144153833301	0.990540419969956\\
52.9397939998209	0.990541219185865\\
52.9651726163117	0.99054193050266\\
52.9905512328025	0.990542555563278\\
53.0159298492933	0.99054309600991\\
53.0413084657841	0.990543553483791\\
53.066687082275	0.990543929624998\\
53.0920656987658	0.990544226072246\\
53.1174443152566	0.990544444462683\\
53.1428229317474	0.990544586431694\\
53.1682015482382	0.990544653612698\\
53.193580164729	0.990544647636952\\
53.2189587812198	0.990544570133356\\
53.2443373977106	0.99054442272826\\
53.2697160142014	0.990544207045265\\
53.2950946306922	0.990543924705041\\
53.320473247183	0.99054357732513\\
53.3458518636738	0.990543166519763\\
53.3712304801646	0.990542693899672\\
53.3966090966554	0.990542161071907\\
53.4219877131462	0.990541569639652\\
53.447366329637	0.990540921202043\\
53.4727449461278	0.990540217353994\\
53.4981235626187	0.990539459686013\\
53.5235021791095	0.990538649784029\\
53.5488807956003	0.990537789229221\\
53.5742594120911	0.99053687959784\\
53.5996380285819	0.990535922461045\\
53.6250166450727	0.990534919384729\\
53.6503952615635	0.990533871929353\\
53.6757738780543	0.990532781649785\\
53.7011524945451	0.990531650095132\\
53.7265311110359	0.990530478808582\\
53.7519097275267	0.990529269327242\\
53.7772883440175	0.990528023181982\\
53.8026669605083	0.990526741897278\\
53.8280455769991	0.99052542699106\\
53.8534241934899	0.990524079974556\\
53.8788028099807	0.990522702352147\\
53.9041814264715	0.990521295621214\\
53.9295600429624	0.990519861271994\\
53.9549386594532	0.990518400787434\\
53.980317275944	0.99051691564305\\
54.0056958924348	0.990515407306785\\
54.0310745089256	0.990513877238871\\
54.0564531254164	0.99051232689169\\
54.0818317419072	0.990510757709643\\
54.107210358398	0.990509171129012\\
54.1325889748888	0.990507568577833\\
54.1579675913796	0.990505951475763\\
54.1833462078704	0.990504321233957\\
54.2087248243612	0.99050267925494\\
54.234103440852	0.990501026932482\\
54.2594820573428	0.990499365651481\\
54.2848606738336	0.990497696787839\\
54.3102392903244	0.990496021708349\\
54.3356179068152	0.990494341770574\\
54.360996523306	0.990492658322738\\
54.3863751397969	0.990490972703614\\
54.4117537562877	0.99048928624241\\
54.4371323727785	0.990487600258666\\
54.4625109892693	0.990485916062147\\
54.4878896057601	0.990484234952736\\
54.5132682222509	0.990482558220338\\
54.5386468387417	0.990480887144774\\
54.5640254552325	0.990479222995689\\
54.5894040717233	0.990477567032452\\
54.6147826882141	0.990475920504061\\
54.6401613047049	0.990474284649055\\
54.6655399211957	0.990472660695424\\
54.6909185376865	0.990471049860514\\
54.7162971541773	0.99046945335095\\
54.7416757706681	0.990467872362546\\
54.7670543871589	0.990466308080225\\
54.7924330036497	0.990464761677936\\
54.8178116201406	0.990463234318583\\
54.8431902366314	0.990461727153937\\
54.8685688531222	0.990460241324573\\
54.893947469613	0.990458777959789\\
54.9193260861038	0.990457338177543\\
54.9447047025946	0.990455923084377\\
54.9700833190854	0.990454533775354\\
54.9954619355762	0.990453171333996\\
55.020840552067	0.990451836832215\\
55.0462191685578	0.990450531330257\\
55.0715977850486	0.990449255876641\\
55.0969764015394	0.990448011508101\\
55.1223550180302	0.990446799249533\\
55.147733634521	0.990445620113942\\
55.1731122510118	0.990444475102386\\
55.1984908675026	0.990443365203933\\
55.2238694839934	0.990442291395609\\
55.2492481004843	0.990441254642354\\
55.2746267169751	0.990440255896975\\
55.3000053334659	0.990439296100109\\
55.3253839499567	0.990438376180181\\
55.3507625664475	0.990437497053362\\
55.3761411829383	0.990436659623539\\
55.4015197994291	0.990435864782273\\
55.4268984159199	0.990435113408772\\
55.4522770324107	0.990434406369859\\
55.4776556489015	0.99043374451994\\
55.5030342653923	0.990433128700978\\
55.5284128818831	0.990432559742469\\
55.5537914983739	0.990432038461417\\
55.5791701148647	0.99043156566231\\
55.6045487313555	0.990431142137102\\
55.6299273478463	0.990430768665196\\
55.6553059643371	0.990430446013422\\
55.680684580828	0.990430174936027\\
55.7060631973188	0.990429956174658\\
55.7314418138096	0.990429790458354\\
55.7568204303004	0.990429678503533\\
55.7821990467912	0.990429621013985\\
55.807577663282	0.990429618680866\\
55.8329562797728	0.99042967218269\\
55.8583348962636	0.990429782185332\\
55.8837135127544	0.99042994934202\\
55.9090921292452	0.990430174293339\\
55.934470745736	0.990430457667233\\
55.9598493622268	0.990430800079003\\
55.9852279787176	0.990431202131321\\
56.0106065952084	0.990431664414228\\
56.0359852116992	0.990432187505147\\
56.06136382819	0.990432771968892\\
56.0867424446808	0.990433418357678\\
56.1121210611716	0.990434127211134\\
56.1374996776625	0.99043489905632\\
56.1628782941533	0.99043573440774\\
56.1882569106441	0.990436633767362\\
56.2136355271349	0.990437597624632\\
56.2390141436257	0.990438626456501\\
56.2643927601165	0.990439720727444\\
56.2897713766073	0.990440880889481\\
56.3151499930981	0.990442107382207\\
56.3405286095889	0.990443400632813\\
56.3659072260797	0.990444761056117\\
56.3912858425705	0.990446189054593\\
56.4166644590613	0.990447685018401\\
56.4420430755521	0.990449249325418\\
56.4674216920429	0.990450882341273\\
56.4928003085337	0.99045258441938\\
56.5181789250245	0.990454355900976\\
56.5435575415153	0.990456197115157\\
56.5689361580062	0.990458108378916\\
56.594314774497	0.990460089997185\\
56.6196933909878	0.990462142262876\\
56.6450720074786	0.99046426545692\\
56.6704506239694	0.990466459848317\\
56.6958292404602	0.990468725694177\\
56.721207856951	0.990471063239766\\
56.7465864734418	0.990473472718555\\
56.7719650899326	0.99047595435227\\
56.7973437064234	0.990478508350937\\
56.8227223229142	0.990481134912942\\
56.848100939405	0.990483834225073\\
56.8734795558958	0.99048660646258\\
56.8988581723866	0.990489451789229\\
56.9242367888774	0.990492370357356\\
56.9496154053682	0.990495362307924\\
56.974994021859	0.990498427770582\\
57.0003726383499	0.990501566863724\\
57.0257512548407	0.990504779694549\\
57.0511298713315	0.990508066359119\\
57.0765084878223	0.990511426942426\\
57.1018871043131	0.990514861518454\\
57.1272657208039	0.99051837015024\\
57.1526443372947	0.990521952889943\\
57.1780229537855	0.990525609778907\\
57.2034015702763	0.990529340847731\\
57.2287801867671	0.990533146116337\\
57.2541588032579	0.990537025594033\\
57.2795374197487	0.990540979279593\\
57.3049160362395	0.990545007161319\\
57.3302946527303	0.990549109217117\\
57.3556732692211	0.990553285414569\\
57.3810518857119	0.990557535711006\\
57.4064305022027	0.990561860053582\\
57.4318091186936	0.99056625837935\\
57.4571877351844	0.990570730615335\\
57.4825663516752	0.990575276678616\\
57.507944968166	0.990579896476399\\
57.5333235846568	0.990584589906095\\
57.5587022011476	0.990589356855403\\
57.5840808176384	0.990594197202387\\
57.6094594341292	0.990599110815554\\
57.63483805062	0.990604097553941\\
57.6602166671108	0.990609157267192\\
57.6855952836016	0.990614289795641\\
57.7109739000924	0.990619494970399\\
57.7363525165832	0.990624772613432\\
57.761731133074	0.990630122537649\\
57.7871097495648	0.990635544546989\\
57.8124883660556	0.990641038436499\\
57.8378669825464	0.99064660399243\\
57.8632455990372	0.990652240992316\\
57.8886242155281	0.990657949205065\\
57.9140028320189	0.990663728391045\\
57.9393814485097	0.990669578302176\\
57.9647600650005	0.990675498682014\\
57.9901386814913	0.990681489265842\\
58.0155172979821	0.990687549780763\\
58.0408959144729	0.990693679945787\\
58.0662745309637	0.99069987947192\\
58.0916531474545	0.990706148062262\\
58.1170317639453	0.99071248541209\\
58.1424103804361	0.990718891208957\\
58.1677889969269	0.990725365132782\\
58.1931676134177	0.99073190685594\\
58.2185462299085	0.99073851604336\\
58.2439248463993	0.990745192352615\\
58.2693034628901	0.990751935434018\\
58.2946820793809	0.990758744930713\\
58.3200606958718	0.990765620478772\\
58.3454393123626	0.990772561707292\\
58.3708179288534	0.990779568238483\\
58.3961965453442	0.99078663968777\\
58.421575161835	0.990793775663885\\
58.4469537783258	0.990800975768964\\
58.4723323948166	0.990808239598643\\
58.4977110113074	0.990815566742152\\
58.5230896277982	0.990822956782415\\
58.548468244289	0.990830409296144\\
58.5738468607798	0.990837923853935\\
58.5992254772706	0.990845500020368\\
58.6246040937614	0.990853137354102\\
58.6499827102522	0.99086083540797\\
58.675361326743	0.99086859372908\\
58.7007399432338	0.990876411858911\\
58.7261185597246	0.990884289333409\\
58.7514971762155	0.990892225683086\\
58.7768757927063	0.990900220433116\\
58.8022544091971	0.990908273103436\\
58.8276330256879	0.990916383208839\\
58.8530116421787	0.990924550259074\\
58.8783902586695	0.990932773758944\\
58.9037688751603	0.990941053208401\\
58.9291474916511	0.990949388102648\\
58.9545261081419	0.990957777932233\\
58.9799047246327	0.990966222183147\\
59.0052833411235	0.990974720336922\\
59.0306619576143	0.990983271870729\\
59.0560405741051	0.990991876257475\\
59.0814191905959	0.991000532965897\\
59.1067978070867	0.991009241460665\\
59.1321764235775	0.991018001202475\\
59.1575550400683	0.991026811648146\\
59.1829336565592	0.991035672250718\\
59.20831227305	0.991044582459547\\
59.2336908895408	0.991053541720404\\
59.2590695060316	0.991062549475567\\
59.2844481225224	0.991071605163923\\
59.3098267390132	0.991080708221056\\
59.335205355504	0.991089858079352\\
59.3605839719948	0.991099054168084\\
59.3859625884856	0.991108295913517\\
59.4113412049764	0.991117582738997\\
59.4367198214672	0.991126914065045\\
59.462098437958	0.991136289309456\\
59.4874770544488	0.991145707887391\\
59.5128556709396	0.991155169211467\\
59.5382342874304	0.991164672691859\\
59.5636129039212	0.991174217736384\\
59.588991520412	0.991183803750601\\
59.6143701369029	0.991193430137899\\
59.6397487533937	0.991203096299593\\
59.6651273698845	0.991212801635011\\
59.6905059863753	0.991222545541592\\
59.7158846028661	0.991232327414973\\
59.7412632193569	0.991242146649078\\
59.7666418358477	0.991252002636215\\
59.7920204523385	0.99126189476716\\
59.8173990688293	0.991271822431248\\
59.8427776853201	0.991281785016467\\
59.8681563018109	0.991291781909539\\
59.8935349183017	0.991301812496014\\
59.9189135347925	0.991311876160356\\
59.9442921512833	0.991321972286032\\
59.9696707677741	0.991332100255597\\
59.9950493842649	0.991342259450782\\
60.0204280007557	0.991352449252579\\
60.0458066172466	0.99136266904133\\
60.0711852337374	0.991372918196807\\
60.0965638502282	0.9913831960983\\
60.121942466719	0.991393502124701\\
60.1473210832098	0.991403835654589\\
60.1726996997006	0.991414196066308\\
60.1980783161914	0.991424582738056\\
60.2234569326822	0.991434995047964\\
60.248835549173	0.991445432374177\\
60.2742141656638	0.991455894094935\\
60.2995927821546	0.991466379588657\\
60.3249713986454	0.991476888234016\\
60.3503500151362	0.991487419410024\\
60.375728631627	0.991497972496105\\
60.4011072481178	0.991508546872175\\
60.4264858646086	0.991519141918723\\
60.4518644810994	0.991529757016885\\
60.4772430975903	0.99154039154852\\
60.5026217140811	0.991551044896288\\
60.5280003305719	0.991561716443724\\
60.5533789470627	0.991572405575311\\
60.5787575635535	0.991583111676559\\
60.6041361800443	0.991593834134072\\
60.6295147965351	0.991604572335628\\
60.6548934130259	0.991615325670243\\
60.6802720295167	0.99162609352825\\
60.7056506460075	0.991636875301364\\
60.7310292624983	0.991647670382756\\
60.7564078789891	0.991658478167121\\
60.7817864954799	0.991669298050749\\
60.8071651119707	0.991680129431588\\
60.8325437284615	0.991690971709317\\
60.8579223449523	0.991701824285412\\
60.8833009614431	0.991712686563208\\
60.908679577934	0.991723557947972\\
60.9340581944248	0.991734437846958\\
60.9594368109156	0.991745325669482\\
60.9848154274064	0.991756220826977\\
61.0101940438972	0.991767122733059\\
61.035572660388	0.991778030803591\\
61.0609512768788	0.99178894445674\\
61.0863298933696	0.991799863113042\\
61.1117085098604	0.991810786195458\\
61.1370871263512	0.991821713129436\\
61.162465742842	0.991832643342971\\
61.1878443593328	0.991843576266656\\
61.2132229758236	0.991854511333749\\
61.2386015923144	0.991865447980219\\
61.2639802088052	0.99187638564481\\
61.289358825296	0.991887323769092\\
61.3147374417868	0.991898261797515\\
61.3401160582777	0.991909199177464\\
61.3654946747685	0.991920135359309\\
61.3908732912593	0.991931069796462\\
61.4162519077501	0.991942001945424\\
61.4416305242409	0.991952931265835\\
61.4670091407317	0.991963857220528\\
61.4923877572225	0.991974779275575\\
61.5177663737133	0.991985696900332\\
61.5431449902041	0.991996609567497\\
61.5685236066949	0.992007516753143\\
61.5939022231857	0.992018417936778\\
61.6192808396765	0.992029312601377\\
61.6446594561673	0.992040200233437\\
61.6700380726581	0.992051080323014\\
61.6954166891489	0.992061952363773\\
61.7207953056397	0.99207281585302\\
61.7461739221305	0.992083670291755\\
61.7715525386213	0.992094515184703\\
61.7969311551122	0.992105350040362\\
61.822309771603	0.992116174371035\\
61.8476883880938	0.992126987692876\\
61.8730670045846	0.99213778952592\\
61.8984456210754	0.992148579394126\\
61.9238242375662	0.992159356825411\\
61.949202854057	0.992170121351686\\
61.9745814705478	0.992180872508888\\
61.9999600870386	0.992191609837019\\
62.0253387035294	0.992202332880176\\
62.0507173200202	0.992213041186582\\
62.076095936511	0.992223734308623\\
62.1014745530018	0.992234411802874\\
62.1268531694926	0.992245073230131\\
62.1522317859834	0.992255718155441\\
62.1776104024742	0.99226634614813\\
62.202989018965	0.992276956781831\\
62.2283676354559	0.992287549634514\\
62.2537462519467	0.992298124288506\\
62.2791248684375	0.992308680330524\\
62.3045034849283	0.992319217351696\\
62.3298821014191	0.992329734947585\\
62.3552607179099	0.992340232718216\\
62.3806393344007	0.992350710268096\\
62.4060179508915	0.992361167206235\\
62.4313965673823	0.992371603146171\\
62.4567751838731	0.992382017705988\\
62.4821538003639	0.992392410508336\\
62.5075324168547	0.992402781180453\\
62.5329110333455	0.992413129354178\\
62.5582896498363	0.992423454665975\\
62.5836682663271	0.992433756756947\\
62.6090468828179	0.99244403527285\\
62.6344254993087	0.992454289864112\\
62.6598041157996	0.99246452018585\\
62.6851827322904	0.992474725897876\\
62.7105613487812	0.99248490666472\\
62.735939965272	0.992495062155636\\
62.7613185817628	0.992505192044618\\
62.7866971982536	0.992515296010408\\
62.8120758147444	0.992525373736511\\
62.8374544312352	0.992535424911201\\
62.862833047726	0.992545449227534\\
62.8882116642168	0.992555446383352\\
62.9135902807076	0.992565416081295\\
62.9389688971984	0.992575358028808\\
62.9643475136892	0.992585271938146\\
62.98972613018	0.992595157526379\\
63.0151047466708	0.992605014515401\\
63.0404833631616	0.992614842631932\\
63.0658619796524	0.992624641607522\\
63.0912405961433	0.992634411178555\\
63.1166192126341	0.992644151086254\\
63.1419978291249	0.992653861076676\\
63.1673764456157	0.992663540900722\\
63.1927550621065	0.992673190314133\\
63.2181336785973	0.99268280907749\\
63.2435122950881	0.992692396956214\\
63.2688909115789	0.992701953720568\\
63.2942695280697	0.992711479145651\\
63.3196481445605	0.992720973011396\\
63.3450267610513	0.992730435102571\\
63.3704053775421	0.992739865208771\\
63.3957839940329	0.992749263124416\\
63.4211626105237	0.992758628648747\\
63.4465412270145	0.992767961585818\\
63.4719198435053	0.992777261744492\\
63.4972984599961	0.992786528938435\\
63.5226770764869	0.992795762986107\\
63.5480556929778	0.992804963710755\\
63.5734343094686	0.992814130940407\\
63.5988129259594	0.992823264507858\\
63.6241915424502	0.992832364250666\\
63.649570158941	0.992841430011141\\
63.6749487754318	0.992850461636331\\
63.7003273919226	0.992859458978015\\
63.7257060084134	0.992868421892689\\
63.7510846249042	0.992877350241556\\
63.776463241395	0.99288624389051\\
63.8018418578858	0.992895102710128\\
63.8272204743766	0.992903926575652\\
63.8525990908674	0.992912715366974\\
63.8779777073582	0.992921468968628\\
63.903356323849	0.992930187269767\\
63.9287349403398	0.992938870164152\\
63.9541135568306	0.992947517550133\\
63.9794921733215	0.992956129330636\\
64.0048707898123	0.992964705413142\\
64.0302494063031	0.992973245709671\\
64.0556280227939	0.992981750136765\\
64.0810066392847	0.992990218615469\\
64.1063852557755	0.992998651071308\\
64.1317638722663	0.993007047434274\\
64.1571424887571	0.993015407638803\\
64.1825211052479	0.993023731623753\\
64.2078997217387	0.993032019332384\\
64.2332783382295	0.993040270712341\\
64.2586569547203	0.993048485715626\\
64.2840355712111	0.99305666429858\\
64.3094141877019	0.993064806421857\\
64.3347928041927	0.993072912050408\\
64.3601714206835	0.993080981153449\\
64.3855500371743	0.993089013704442\\
64.4109286536652	0.993097009681073\\
64.436307270156	0.99310496906522\\
64.4616858866468	0.993112891842938\\
64.4870645031376	0.993120778004425\\
64.5124431196284	0.993128627544002\\
64.5378217361192	0.993136440460083\\
64.56320035261	0.993144216755153\\
64.5885789691008	0.993151956435738\\
64.6139575855916	0.993159659512379\\
64.6393362020824	0.993167325999603\\
64.6647148185732	0.9931749559159\\
64.690093435064	0.99318254928369\\
64.7154720515548	0.993190106129296\\
64.7408506680456	0.993197626482917\\
64.7662292845364	0.993205110378597\\
64.7916079010272	0.993212557854196\\
64.816986517518	0.993219968951363\\
64.8423651340088	0.993227343715501\\
64.8677437504997	0.993234682195742\\
64.8931223669905	0.993241984444913\\
64.9185009834813	0.993249250519508\\
64.9438795999721	0.993256480479654\\
64.9692582164629	0.993263674389083\\
64.9946368329537	0.993270832315096\\
65.0200154494445	0.993277954328537\\
65.0453940659353	0.993285040503756\\
65.0707726824261	0.993292090918579\\
65.0961512989169	0.993299105654276\\
65.1215299154077	0.993306084795526\\
65.1469085318985	0.993313028430386\\
65.1722871483893	0.993319936650257\\
65.1976657648801	0.993326809549851\\
65.2230443813709	0.993333647227158\\
65.2484229978617	0.993340449783409\\
65.2738016143525	0.993347217323046\\
65.2991802308434	0.993353949953686\\
65.3245588473342	0.993360647786087\\
65.349937463825	0.993367310934113\\
65.3753160803158	0.993373939514698\\
65.4006946968066	0.993380533647815\\
65.4260733132974	0.993387093456435\\
65.4514519297882	0.9933936190665\\
65.476830546279	0.993400110606879\\
65.5022091627698	0.993406568209338\\
65.5275877792606	0.993412992008502\\
65.5529663957514	0.99341938214182\\
65.5783450122422	0.993425738749528\\
65.603723628733	0.993432061974617\\
65.6291022452238	0.993438351962791\\
65.6544808617146	0.993444608862434\\
65.6798594782054	0.993450832824574\\
65.7052380946962	0.993457024002844\\
65.7306167111871	0.993463182553447\\
65.7559953276779	0.993469308635123\\
65.7813739441687	0.993475402409103\\
65.8067525606595	0.993481464039083\\
65.8321311771503	0.993487493691178\\
65.8575097936411	0.99349349153389\\
65.8828884101319	0.993499457738071\\
65.9082670266227	0.993505392476885\\
65.9336456431135	0.993511295925767\\
65.9590242596043	0.993517168262393\\
65.9844028760951	0.993523009666638\\
66.0097814925859	0.993528820320539\\
66.0351601090767	0.993534600408261\\
66.0605387255675	0.993540350116053\\
66.0859173420583	0.993546069632219\\
66.1112959585491	0.993551759147074\\
66.1366745750399	0.99355741885291\\
66.1620531915308	0.993563048943957\\
66.1874318080216	0.993568649616345\\
66.2128104245124	0.993574221068071\\
66.2381890410032	0.993579763498956\\
66.263567657494	0.99358527711061\\
66.2889462739848	0.993590762106395\\
66.3143248904756	0.993596218691387\\
66.3397035069664	0.99360164707234\\
66.3650821234572	0.993607047457647\\
66.390460739948	0.993612420057302\\
66.4158393564388	0.993617765082865\\
66.4412179729296	0.993623082747426\\
66.4665965894204	0.993628373265561\\
66.4919752059112	0.993633636853305\\
66.517353822402	0.993638873728106\\
66.5427324388928	0.993644084108793\\
66.5681110553836	0.993649268215538\\
66.5934896718745	0.993654426269819\\
66.6188682883653	0.993659558494381\\
66.6442469048561	0.993664665113206\\
66.6696255213469	0.993669746351468\\
66.6950041378377	0.993674802435502\\
66.7203827543285	0.993679833592766\\
66.7457613708193	0.993684840051805\\
66.7711399873101	0.993689822042213\\
66.7965186038009	0.993694779794601\\
66.8218972202917	0.993699713540556\\
66.8472758367825	0.993704623512607\\
66.8726544532733	0.993709509944194\\
66.8980330697641	0.993714373069623\\
66.9234116862549	0.993719213124038\\
66.9487903027457	0.993724030343384\\
66.9741689192365	0.993728824964368\\
66.9995475357273	0.993733597224428\\
67.0249261522182	0.993738347361699\\
67.050304768709	0.993743075614971\\
67.0756833851998	0.993747782223664\\
67.1010620016906	0.993752467427786\\
67.1264406181814	0.993757131467901\\
67.1518192346722	0.993761774585095\\
67.177197851163	0.993766397020942\\
67.2025764676538	0.99377099901747\\
67.2279550841446	0.993775580817127\\
67.2533337006354	0.993780142662748\\
67.2787123171262	0.993784684797518\\
67.304090933617	0.993789207464947\\
67.3294695501078	0.993793710908826\\
67.3548481665986	0.993798195373204\\
67.3802267830894	0.993802661102348\\
67.4056053995802	0.993807108340716\\
67.430984016071	0.99381153733292\\
67.4563626325618	0.993815948323698\\
67.4817412490527	0.993820341557879\\
67.5071198655435	0.993824717280352\\
67.5324984820343	0.993829075736036\\
67.5578770985251	0.993833417169847\\
67.5832557150159	0.993837741826665\\
67.6086343315067	0.99384204995131\\
67.6340129479975	0.993846341788504\\
67.6593915644883	0.993850617582841\\
67.6847701809791	0.993854877578764\\
67.7101487974699	0.993859122020527\\
67.7355274139607	0.993863351152167\\
67.7609060304515	0.993867565217477\\
67.7862846469423	0.993871764459975\\
67.8116632634331	0.993875949122876\\
67.8370418799239	0.993880119449059\\
67.8624204964147	0.993884275681044\\
67.8877991129056	0.993888418060959\\
67.9131777293964	0.993892546830516\\
67.9385563458872	0.993896662230979\\
67.963934962378	0.993900764503138\\
67.9893135788688	0.993904853887281\\
68.0146921953596	0.99390893062317\\
68.0400708118504	0.993912994950007\\
68.0654494283412	0.993917047106416\\
68.090828044832	0.99392108733041\\
68.1162066613228	0.993925115859365\\
68.1415852778136	0.99392913293\\
68.1669638943044	0.993933138778343\\
68.1923425107952	0.993937133639713\\
68.217721127286	0.993941117748689\\
68.2430997437768	0.99394509133909\\
68.2684783602676	0.993949054643945\\
68.2938569767584	0.993953007895474\\
68.3192355932492	0.993956951325061\\
68.3446142097401	0.993960885163229\\
68.3699928262309	0.993964809639619\\
68.3953714427217	0.993968724982966\\
68.4207500592125	0.993972631421074\\
68.4461286757033	0.993976529180797\\
68.4715072921941	0.993980418488011\\
68.4968859086849	0.993984299567598\\
68.5222645251757	0.993988172643419\\
68.5476431416665	0.993992037938296\\
68.5730217581573	0.993995895673988\\
68.5984003746481	0.99399974607117\\
68.6237789911389	0.994003589349413\\
68.6491576076297	0.994007425727166\\
68.6745362241205	0.994011255421729\\
68.6999148406113	0.994015078649241\\
68.7252934571021	0.994018895624654\\
68.750672073593	0.994022706561715\\
68.7760506900837	0.994026511672951\\
68.8014293065746	0.994030311169643\\
68.8268079230654	0.994034105261816\\
68.8521865395562	0.994037894158211\\
68.877565156047	0.994041678066275\\
68.9029437725378	0.994045457192139\\
68.9283223890286	0.994049231740604\\
68.9537010055194	0.994053001915118\\
68.9790796220102	0.994056767917766\\
69.004458238501	0.994060529949249\\
69.0298368549	1\\
70 1\\
};
\addlegendentry{compensated};

\addplot [color=mycolor2,solid]
  table[row sep=crcr]{%
0	0\\
0.460517018597787	-0.0197120460601753\\
0.921034037195573	-0.0249219637793949\\
1.38155105579336	-0.0174804236051585\\
1.84206807439115	0.000950606030334783\\
2.30258509298893	0.0288800909480801\\
2.76310211158672	0.0649717792206224\\
3.22361913018451	0.108029520403529\\
3.68413614878229	0.156983924295601\\
4.14465316738008	0.210880239213343\\
4.60517018597787	0.268867340421988\\
5.06568720457565	0.330187729080071\\
5.52620422317344	0.394168450910358\\
5.98672124177123	0.460212851882473\\
6.44723826036901	0.527793095551121\\
6.9077552789668	0.596443373401153\\
7.36827229756459	0.665753745664516\\
7.82878931616237	0.73536455564665\\
8.28930633476016	0.804961365678897\\
8.74982335335795	0.874270367442648\\
9.21034037195573	0.94305422362988\\
9.67085739055352	1.01110830174968\\
10.1313744091513	1.07825726439438\\
10.5918914277491	1.14435198347221\\
11.0524084463469	1.20926674882302\\
11.5129254649447	1.27289674428538\\
11.9734424835425	1.33515576669924\\
12.4339595021402	1.3959741655298\\
12.894476520738	1.45529698280368\\
13.3549935393358	1.5130822748762\\
13.8155105579336	1.56929959921286\\
14.2760275765314	1.62392865088481\\
14.7365445951292	1.67695803485948\\
15.197061613727	1.72838416142574\\
15.6575786323247	1.77821025323909\\
16.1180956509225	1.82644545351611\\
16.5786126695203	1.87310402585783\\
17.0391296881181	1.9182046370475\\
17.4996467067159	1.96176971495628\\
17.9601637253137	2.00382487440781\\
18.4206807439115	2.04439840450647\\
18.8811977625093	2.08352081152825\\
19.341714781107	2.12122441201499\\
19.8022317997048	2.15754297120537\\
20.2627488183026	2.1925113823844\\
20.7232658369004	2.22616538314115\\
21.1837828554982	2.25854130489612\\
21.644299874096	2.28967585239691\\
22.1048168926938	2.3196059101886\\
22.5653339112915	2.3483683733445\\
23.0258509298893	2.375999999997\\
23.4863679484871	2.40253728343972\\
23.9468849670849	2.42801634178192\\
24.4074019856827	2.45247282332722\\
24.8679190042805	2.47594182602257\\
25.3284360228783	2.49845782948057\\
25.7889530414761	2.52005463822225\\
26.2494700600738	2.54076533491694\\
26.7099870786716	2.56062224251462\\
27.1705040972694	2.5796568942732\\
27.6310211158672	2.59790001078062\\
28.091538134465	2.61538148316033\\
28.5520551530628	2.6321303617286\\
29.0125721716606	2.64817484944522\\
29.4730891902583	2.66354229956503\\
29.9336062088561	2.67825921695736\\
30.3941232274539	2.69235126261501\\
30.8546402460517	2.7058432609233\\
31.3151572646495	2.71875920930426\\
31.7756742832473	2.73112228989123\\
32.2361913018451	2.74295488292576\\
32.6967083204429	2.75427858160137\\
33.1572253390406	2.76511420810885\\
33.6177423576384	2.77548183066431\\
34.0782593762362	2.78540078132602\\
34.538776394834	2.79488967442735\\
34.9992934134318	2.80396642547366\\
35.4598104320296	2.81264827036817\\
35.9203274506274	2.82095178484839\\
36.3808444692251	2.82889290402886\\
36.8413614878229	2.83648694195918\\
37.3018785064207	2.84374861111786\\
37.7623955250185	2.850692041773\\
38.2229125436163	2.8573308011505\\
38.6834295622141	2.86367791235855\\
39.1439465808119	2.86974587302493\\
39.6044635994097	2.87554667361052\\
40.0649806180074	2.88109181536804\\
40.5254976366052	2.88639232792088\\
40.986014655203	2.89145878644135\\
41.4465316738008	2.89630132841197\\
41.9070486923986	2.9009296699573\\
42.3675657109964	2.905353121737\\
42.8280827295942	2.90958060439383\\
43.2885997481919	2.91362066355281\\
43.7491167667897	2.91748148437008\\
44.2096337853875	2.92117090563196\\
44.6701508039853	2.92469643340635\\
45.1306678225831	2.9280652542503\\
45.5911848411809	2.93128424797857\\
46.0517018597787	2.93435999999934\\
46.5122188783765	2.93729881322412\\
46.9727358969742	2.94010671955949\\
47.433252915572	2.94278949098932\\
47.8937699341698	2.94535265025618\\
48.3542869527676	2.94780148115166\\
48.8148039713654	2.95014103842498\\
49.2753209899632	2.95237615732023\\
49.735838008561	2.95451146275209\\
50.1963550271587	2.95655137813063\\
50.6568720457565	2.95850013384525\\
51.1173890643543	2.96036177541837\\
51.5779060829521	2.96214017133906\\
52.0384231015499	2.96383902058692\\
52.4989401201477	2.96546185985644\\
52.9594571387455	2.96701207049175\\
53.4199741573433	2.96849288514184\\
53.880491175941	2.96990739414578\\
54.3410081945388	2.97125855165766\\
54.8015252131366	2.97254918152047\\
55.2620422317344	2.97378198289808\\
55.7225592503322	2.97495953567429\\
56.18307626893	2.97608430562747\\
56.6435932875278	2.9771586493895\\
57.1041103061255	2.97818481919694\\
57.5646273247233	2.97916496744263\\
58.0251443433211	2.98010115103531\\
58.4856613619189	2.98099533557488\\
58.9461783805167	2.98184939935044\\
59.4066953991145	2.98266513716822\\
59.8672124177123	2.98344426401622\\
60.3277294363101	2.98418841857201\\
60.7882464549078	2.9848991665601\\
61.2487634735056	2.98557800396507\\
61.7092804921034	2.98622636010614\\
62.1697975107012	2.98684560057917\\
62.630314529299	2.98743703007125\\
63.0908315478968	2.98800189505339\\
63.5513485664946	2.98854138635629\\
64.0118655850924	2.98905664163409\\
64.4723826036901	2.98954874772077\\
64.9328996222879	2.99001874288373\\
65.3934166408857	2.99046761897902\\
65.8539336594835	2.99089632351208\\
66.3144506780813	2.99130576160839\\
66.7749676966791	2.99169679789753\\
67.2354847152769	2.99207025831459\\
67.6960017338746	2.99242693182227\\
68.1565187524724	2.99276757205732\\
68.6170357710702	2.99309289890427\\
69.077552789668	2.99340359999991\\
69.5380698082658	2.99370033217119\\
69.9985868268636	2.99398372280974\\
70.4591038454614	2.99425437118548\\
70.9196208640591	2.99451284970214\\
71.3801378826569	2.99475970509718\\
71.8406549012547	2.99499545958851\\
72.3011719198525	2.99522061197029\\
72.7616889384503	2.9954356386602\\
73.2222059570481	2.99564099470008\\
73.6827229756459	2.99583711471218\\
74.1432399942437	2.99602441381285\\
74.6037570128414	2.99620328848556\\
75.0642740314392	2.99637411741511\\
75.524791050037	2.99653726228463\\
75.9853080686348	2.99669306853713\\
76.4458250872326	2.99684186610298\\
76.9063421058304	2.9969839700951\\
77.3668591244282	2.99711968147303\\
77.8273761430259	2.99724928767735\\
78.2878931616	3\\
};
\addlegendentry{uncompensated};


\end{axis}
\end{tikzpicture}%
}
  \caption{The step response of the initial, uncontrolled process
    (\texttt{red}) and final, controlled process (\texttt{blue}).}
  \label{fig:step_response_1.1}
\end{figure}


\subsection{Exercise 2}

The bandwidth and the resonance peak of the closed-loop system, along with the
rise time and the overshoot of a step response are found in table \ref{tbl:ex1.2}.

\begin{table}[H] \centering
    \begin{tabular}{|c|c|c|c|} \hline
      $\omega_B$ $[rad/s]$ & $M_T$ $[dB]$ & $T_r$ $[sec]$ & $M$ $\%$  \\ \hline
      $0.7474$             & $5.8247$     & $2.4583$      & $37.7922$ \\ \hline
    \end{tabular}
    \caption{Closed loop system characteristics for a phase margin of
      $30^{\circ}$.}
    \label{tbl:ex1.2}
\end{table}

\subsection{Exercise 3}

If we now require a phase margin of $50^{\circ}$, and the crossover frequency
to remain unchanged at $\omega_c = 0.4$ rad/s, then the $\tau_I$ coefficient
needs to be increased. For $\tau_I = 2$, the lag component becomes
$F_g(s) = \dfrac{2s + 1}{2s}$. In this case, the phase margin of $F_g(s)G(s)$
is equal to $\phi_m^0 = 147.4597^{\circ} - 180^{\circ} = -32.5403^{\circ}$. Thus,
$\beta, \tau_d$ and $K$, obtained the same way as before, become:

\begin{itemize}
  \item $\beta = \dfrac{1 - sin(50 - \phi_m^0)}{1 + sin(50 - \phi_m^0)} = 0.0042$
  \item $\tau_D = \dfrac{1}{\omega_c\sqrt\beta} = \dfrac{1}{0.4\sqrt\beta} = 38.3493$
  \item $K = \dfrac{\sqrt\beta}{|F_g(j\omega_c)G(j\omega_c)|} =
    \dfrac{\sqrt\beta}{0.5610} = 0.1162$
\end{itemize}

Finally, now that the value of each coefficient has been identified, the
controller is identified as

\begin{align*}
  F(s) = 0.1162 \cdot \dfrac{38.3493 s + 1}{0.1611 s + 1} \cdot \dfrac{2s + 1}{2s}
\end{align*}

We can verify that all three requirements have been met by plotting the bode
diagram (figure \ref{fig:bode_1.3}) and the step response
(figure \ref{fig:step_response_1.3}) for the initial, uncontrolled process $G(s)$
and the final, controlled process $F(s)G(s)$.


\begin{figure}[H]\centering
  \scalebox{1}{% This file was created by matlab2tikz.
%
%The latest updates can be retrieved from
%  http://www.mathworks.com/matlabcentral/fileexchange/22022-matlab2tikz-matlab2tikz
%where you can also make suggestions and rate matlab2tikz.
%
\definecolor{mycolor1}{rgb}{0.00000,0.44700,0.74100}%
\definecolor{mycolor2}{rgb}{0.85000,0.32500,0.09800}%
%
\begin{tikzpicture}

\begin{axis}[%
width=4.008in,
height=1.551in,
at={(0.818in,1.941in)},
scale only axis,
separate axis lines,
every outer x axis line/.append style={white!40!black},
every x tick label/.append style={font=\color{white!40!black}},
xmode=log,
xmin=0.001,
xmax=1000,
xtick={0.01,1,100},
xticklabels={\empty},
xminorticks=true,
every outer y axis line/.append style={white!40!black},
every y tick label/.append style={font=\color{white!40!black}},
ymin=-100,
ymax=50,
ylabel={Magnitude [dB]},
axis background/.style={fill=white}
]
\addplot [color=mycolor1,solid,forget plot]
  table[row sep=crcr]{%
1e-20	384.826019451753\\
5.21521748431671e-19	350.480570971117\\
5.21521748431673e-14	250.480570971117\\
5.21521748431673e-10	170.480570971117\\
5.21521748431673e-07	110.480570972713\\
5.21521748431673e-05	70.4805869250894\\
0.000521521748431672	50.4821660261642\\
0.000608640022522472	49.1409816018221\\
0.000710311081235165	47.8000059892123\\
0.000828965913274034	46.4593146531961\\
0.000967441594991455	45.1190102705196\\
0.0011290491258236	43.7792324899109\\
0.00131765259538411	42.440171148293\\
0.0015377615751272	41.1020841289582\\
0.00179463894369543	39.7653214219306\\
0.00209442672409202	38.4303574110492\\
0.00244429294148611	37.0978339621558\\
0.00285260301306982	35.7686174875544\\
0.00332911976795528	34.4438737177603\\
0.00388523687965389	33.1251642118554\\
0.00453425129258534	31.8145682715881\\
0.00529168115642499	30.5148321577586\\
0.00617563687020467	29.229543140424\\
0.00720725410795487	27.9633171959973\\
0.00841119917967438	26.7219739766835\\
0.00981625881097599	25.5126494186208\\
0.0114560284432349	24.3437661385936\\
0.0133697155117244	23.2247522927902\\
0.0156030769083854	22.1653893379448\\
0.0182095130442748	21.174706972641\\
0.0212513446710894	20.259454807886\\
0.0248013029909898	19.4223564641253\\
0.0289442686837448	18.6605226470242\\
0.0337793014319113	17.9644526794788\\
0.0368072464521512	17.6003378887955\\
0.0394220084706697	17.3179196862536\\
0.0460073087951251	16.6987718734412\\
0.0536926591181881	16.0804507190154\\
0.0626618187127568	15.4339486859549\\
0.0731292431568232	14.7299857736278\\
0.0853452120374067	13.941258290876\\
0.0996018132184145	13.0446179439438\\
0.116239926758255	12.022998660714\\
0.135657375464993	10.8669131840822\\
0.158318436971512	9.5754282388142\\
0.184764944767569	8.15661845775845\\
0.215629243618073	6.62749240413889\\
0.251649309135958	5.01329261021518\\
0.293686393023628	3.34602920336439\\
0.342745615886551	1.66221608581402\\
0.4	-7.7146197324263e-15\\
0.466818516660356	-1.60392452040725\\
0.544798818742438	-3.11714443983305\\
0.635805441109148	-4.51371589079191\\
0.74201438225789	-5.77553187965388\\
0.865965133165696	-6.89291403276041\\
1.01062139735999	-7.86458159121255\\
1.17944195405202	-8.69711470612952\\
1.37646335869389	-9.40395578769605\\
1.60639645835704	-10.0040117934284\\
1.8747390296467	-10.5200319720915\\
2.18790723236237	-10.9770180995358\\
2.55338902200467	-11.4008780380489\\
2.97992318927264	-11.8173763731983\\
3.47770830744514	-12.2512677805608\\
4.05864658364734	-12.7254164059176\\
4.73662844456718	-13.2597544435629\\
5.52786466116025	-13.8701018838919\\
6.45127395355508	-14.5670801958131\\
7.52893534392043	-15.3554793526255\\
8.78661607320166	-16.2343727821957\\
10.2543877043901	-17.1980265854559\\
11.9673451435589	-18.237361607291\\
13.9664457706968	-19.3415705232294\\
16.2994887442349	-20.4995292690085\\
19.0222578947648	-21.7008032471367\\
22.1998555349121	-22.9362195496727\\
25.9082590772047	-24.1980851681687\\
30.2361376791822	-25.4801680664243\\
35.2869723523354	-26.7775476581525\\
41.1815302273805	-28.086412031708\\
48.0607521363735	-29.4038494539625\\
56.0891225547072	-30.7276588405088\\
65.4586024794234	-32.0561890607126\\
76.393219280261	-33.3882085345508\\
89.1544232683017	-34.722802456279\\
104.047339059545	-36.059293404174\\
121.428061205585	-37.3971808754069\\
141.712168532335	-38.7360956958048\\
165.384660767468	-40.0757658929009\\
193.011555044614	-41.4159912831905\\
225.253419560589	-42.7566246201958\\
262.881167979867	-44.0975576481758\\
306.794492235759	-45.4387108057553\\
3067.94492235759	-65.4370461935004\\
306794.492235759	-105.437029377446\\
306794492.235759	-165.437029375764\\
3067944922357.58	-245.437029375764\\
3.0679449223576e+17	-345.437029375764\\
1e+20	-395.700078203282\\
};
\addplot [color=black,solid,forget plot]
  table[row sep=crcr]{%
2.13881627287479	0\\
2.13881627287479	-10.9125847499973\\
};
\addplot [color=black,dotted,forget plot]
  table[row sep=crcr]{%
2.13881627287479	-100\\
2.13881627287479	0\\
};
\addplot [color=black,dotted,forget plot]
  table[row sep=crcr]{%
0.400001047387089	-100\\
0.400001047387089	0\\
};
\addplot [color=black,dotted,forget plot]
  table[row sep=crcr]{%
0.001	0\\
1000	0\\
};
\addplot [color=mycolor2,solid,forget plot]
  table[row sep=crcr]{%
1e-20	9.54242509439325\\
2e-18	9.54242509439325\\
2e-13	9.54242509439325\\
2e-09	9.54242509439325\\
2e-06	9.54242509223915\\
0.0002	9.54240355342386\\
0.002	9.5402713627845\\
0.00233717165820117	9.53948415809935\\
0.00273118567994939	9.53840931622052\\
0.00319162488223131	9.53694181661812\\
0.00372968760918032	9.53493836280318\\
0.00435846008706016	9.53220349137111\\
0.00509323469443899	9.52847069079282\\
0.00595188188820484	9.523376781155\\
0.00695528483103661	9.5164272310713\\
0.00812784729090762	9.50694935691696\\
0.00949808716524821	9.49402944976971\\
0.0110993300648712	9.47642882468319\\
0.0129705198263185	9.4524726693549\\
0.015157165665104	9.41990459926948\\
0.0177124490055704	9.3756994527308\\
0.0206985169065763	9.31582791766569\\
0.0241879935404239	9.23497048019702\\
0.0282657464857158	9.12618702942878\\
0.0330309507921571	8.98056487405763\\
0.0385995010174335	8.78689410208395\\
0.0451068298993264	8.53145439220796\\
0.0527112022160033	8.19803359634517\\
0.0615975639444767	7.76831613571356\\
0.0719820403326324	7.22274903422338\\
0.0841171922824609	6.54189245208831\\
0.0982981588850128	5.7081056096305\\
0.114869835499704	4.70728255494117\\
0.134235261956069	3.53033132881272\\
0.156865424887466	2.17420785880134\\
0.183310712599335	0.64247195019924\\
0.214214301065913	-1.05460482459853\\
0.250327796616312	-2.90039069917478\\
0.292529515755795	-4.87273889497134\\
0.341845846705877	-6.94527456121148\\
0.399476212197379	-9.08893859687721\\
0.466822240636634	-11.2735590547231\\
0.545521855116954	-13.469314728112\\
0.637489109354333	-15.6481051398104\\
0.744960739397426	-17.7849167612729\\
0.870550563296124	-19.8592047122183\\
1.01731305178338	-21.8561288781271\\
1.18881761607313	-23.7673253463181\\
1.38923541952819	-25.5909044763286\\
1.62344082454525	-27.3305851581909\\
1.89712994194695	-28.9941784484001\\
2.21695916612161	-30.591832166926\\
2.59070706522436	-32.1344421718128\\
3.02746356377195	-33.6324772160582\\
3.53785101874225	-35.0952750398668\\
4.13428256597126	-36.5307370480754\\
4.83126402009161	-37.9452981430786\\
5.64574667052256	-39.3440537449169\\
6.59753955386447	-40.7309560125116\\
7.70979122957661	-42.1090239892333\\
9.00955277620718	-43.4805381722373\\
10.5284357008095	-44.8472070148147\\
12.3033807625627	-46.2103026729137\\
14.3775564091595	-47.5707682110338\\
16.801408676838	-48.9293004504898\\
19.6338880886805	-50.2864130316176\\
22.9438833905788	-51.6424839028691\\
26.8118969947666	-52.9977908089921\\
31.3320028793888	-54.352537664743\\
36.6141345621924	-55.7068740767075\\
42.7867587941599	-57.06090975141\\
50	-58.4147251067686\\
500	-78.4163585758919\\
50000	-118.416375077397\\
50000000	-178.416375079047\\
500000000000	-258.416375079047\\
5e+16	-358.416375079047\\
1e+20	-424.436974992327\\
};
\addplot [color=black,solid,forget plot]
  table[row sep=crcr]{%
0.565697298192371	0\\
0.565697298192371	-13.9796947246107\\
};
\addplot [color=black,dotted,forget plot]
  table[row sep=crcr]{%
0.565697298192371	-100\\
0.565697298192371	0\\
};
\addplot [color=black,dotted,forget plot]
  table[row sep=crcr]{%
0.19468266671785	-100\\
0.19468266671785	0\\
};
\addplot [color=black,dotted,forget plot]
  table[row sep=crcr]{%
0.001	0\\
1000	0\\
};
\end{axis}

\begin{axis}[%
width=4.008in,
height=1.376in,
at={(0.818in,0.44in)},
scale only axis,
separate axis lines,
every outer x axis line/.append style={white!40!black},
every x tick label/.append style={font=\color{white!40!black}},
xmode=log,
xmin=0.001,
xmax=1000,
xminorticks=true,
xlabel={Frequency [rad/s]},
every outer y axis line/.append style={white!40!black},
every y tick label/.append style={font=\color{white!40!black}},
ymin=87.3,
ymax=362.7,
ytick={ 90, 180, 270, 360},
ylabel={Phase [deg]},
axis background/.style={fill=white},
legend style={legend cell align=left,align=left,draw=white!15!black, font=\footnotesize}
]
\addplot [color=mycolor1,solid]
  table[row sep=crcr]{%
1e-20	270\\
5.21521748431671e-19	270\\
5.21521748431673e-14	270.000000000072\\
5.21521748431673e-10	270.000000722712\\
5.21521748431673e-07	270.000722711787\\
5.21521748431673e-05	270.072271028963\\
0.000521521748431672	270.722562063834\\
0.000608640022522472	270.843200144906\\
0.000710311081235165	270.983953114826\\
0.000828965913274034	271.148159153166\\
0.000967441594991455	271.339701165699\\
0.0011290491258236	271.56309020467\\
0.00131765259538411	271.823558397378\\
0.0015377615751272	272.127160406156\\
0.00179463894369543	272.480880818629\\
0.00209442672409202	272.892742168774\\
0.00244429294148611	273.371903930461\\
0.00285260301306982	273.928735954614\\
0.00332911976795528	274.574839258662\\
0.00388523687965389	275.322971303992\\
0.00453425129258534	276.186810217242\\
0.00529168115642499	277.180461516409\\
0.00617563687020467	278.317572265027\\
0.00720725410795487	279.609876287122\\
0.00841119917967438	281.064964706002\\
0.00981625881097599	282.683089200298\\
0.0114560284432349	284.452912247664\\
0.0133697155117244	286.346384018343\\
0.0156030769083854	288.313391724052\\
0.0182095130442748	290.277436923246\\
0.0212513446710894	292.134104734615\\
0.0248013029909898	293.754073874637\\
0.0289442686837448	294.99152285552\\
0.0337793014319113	295.69714785486\\
0.0368072464521512	295.807816172377\\
0.0394220084706697	295.733414736657\\
0.0460073087951251	294.989080862958\\
0.0536926591181881	293.390712432004\\
0.0626618187127568	290.910198369422\\
0.0731292431568232	287.568156533572\\
0.0853452120374067	283.433296322036\\
0.0996018132184145	278.617690624606\\
0.116239926758255	273.268136020661\\
0.135657375464993	267.5543978503\\
0.158318436971512	261.65557526461\\
0.184764944767569	255.745626049202\\
0.215629243618073	249.978622390958\\
0.251649309135958	244.474409815792\\
0.293686393023628	239.306277494844\\
0.342745615886551	234.493283131357\\
0.4	230\\
0.466818516660356	225.745254372497\\
0.544798818742438	221.619183987364\\
0.635805441109148	217.505346084058\\
0.74201438225789	213.302579845573\\
0.865965133165696	208.940925000191\\
1.01062139735999	204.387664338663\\
1.17944195405202	199.64296698157\\
1.37646335869389	194.728169558086\\
1.60639645835704	189.671811958677\\
1.8747390296467	184.498354807259\\
2.18790723236237	179.222543823673\\
2.55338902200467	173.849946606836\\
2.97992318927264	168.382368309563\\
3.47770830744514	162.825957735473\\
4.05864658364734	157.199543626153\\
4.73662844456718	151.540764023816\\
5.52786466116025	145.907896334964\\
6.45127395355508	140.376253931101\\
7.52893534392043	135.02966904582\\
8.78661607320166	129.949382236703\\
10.2543877043901	125.203614445293\\
11.9673451435589	120.840585966855\\
13.9664457706968	116.886123382965\\
16.2994887442349	113.345284049272\\
19.0222578947648	110.206457640526\\
22.1998555349121	107.446323916023\\
25.9082590772047	105.034504540693\\
30.2361376791822	102.937321110489\\
35.2869723523354	101.120513728384\\
41.1815302273805	99.5510242769942\\
48.0607521363735	98.1980456049281\\
56.0891225547072	97.033545029664\\
65.4586024794234	96.032437748807\\
76.393219280261	95.1725426024222\\
89.1544232683017	94.4344131084257\\
104.047339059545	93.8011054991167\\
121.428061205585	93.2579228439347\\
141.712168532335	92.7921587889332\\
165.384660767468	92.3928541868032\\
193.011555044614	92.0505733579148\\
225.253419560589	91.757202700438\\
262.881167979867	91.5057719875981\\
306.794492235759	91.2902973575633\\
3067.94492235759	90.1290449160574\\
306794.492235759	90.0012904506942\\
306794492.235759	90.0000012904507\\
3067944922357.58	90.000000000129\\
3.0679449223576e+17	90\\
1e+20	90\\
};
\addlegendentry{compensated};

\addplot [color=black,solid,forget plot]
  table[row sep=crcr]{%
0.400001047387089	180\\
0.400001047387089	229.999926139241\\
};
\addplot [color=black,dotted,forget plot]
  table[row sep=crcr]{%
0.400001047387089	180\\
0.400001047387089	362.7\\
};
\addplot [color=black,dotted,forget plot]
  table[row sep=crcr]{%
2.13881627287479	180\\
2.13881627287479	362.7\\
};
\addplot [color=black,dotted,forget plot]
  table[row sep=crcr]{%
0.001	180\\
1000	180\\
};
\addplot [color=mycolor2,solid]
  table[row sep=crcr]{%
1e-20	360\\
2e-18	360\\
2e-13	359.999999999817\\
2e-09	359.999998166535\\
2e-06	359.998166535056\\
0.0002	359.816653677598\\
0.002	358.166707057904\\
0.00233717165820117	357.857713309198\\
0.00273118567994939	357.496671340435\\
0.00319162488223131	357.074832598203\\
0.00372968760918032	356.581989130365\\
0.00435846008706016	356.006236740965\\
0.00509323469443899	355.333703648514\\
0.00595188188820484	354.548242018487\\
0.00695528483103661	353.631081137728\\
0.00812784729090762	352.560443659562\\
0.00949808716524821	351.311131022292\\
0.0110993300648712	349.854091937563\\
0.0129705198263185	348.156000329509\\
0.015157165665104	346.178888370186\\
0.0177124490055704	343.879908783462\\
0.0206985169065763	341.211340669741\\
0.0241879935404239	338.121005299058\\
0.0282657464857158	334.553318224698\\
0.0330309507921571	330.45125668309\\
0.0385995010174335	325.759533418817\\
0.0451068298993264	320.429183255855\\
0.0527112022160033	314.423515661962\\
0.0615975639444767	307.724920694494\\
0.0719820403326324	300.341402826855\\
0.0841171922824609	292.311215861625\\
0.0982981588850128	283.703987688419\\
0.114869835499704	274.617513269482\\
0.134235261956069	265.170676374278\\
0.156865424887466	255.493937938071\\
0.183310712599335	245.718902476226\\
0.214214301065913	235.967959986227\\
0.250327796616312	226.344805701653\\
0.292529515755795	216.92711616751\\
0.341845846705877	207.763129632883\\
0.399476212197379	198.87347874817\\
0.466822240636634	190.258308900665\\
0.545521855116954	181.908190496998\\
0.637489109354333	173.816260230269\\
0.744960739397426	165.988612867988\\
0.870550563296124	158.450214088947\\
1.01731305178338	151.244614722344\\
1.18881761607313	144.42754139993\\
1.38923541952819	138.056522823193\\
1.62344082454525	132.180057608636\\
1.89712994194695	126.829620534566\\
2.21695916612161	122.016214540679\\
2.59070706522436	117.731251952524\\
3.02746356377195	113.950302973245\\
3.53785101874225	110.637989893703\\
4.13428256597126	107.752716289267\\
4.83126402009161	105.250528216065\\
5.64574667052256	103.087901236868\\
6.59753955386447	101.223539630327\\
7.70979122957661	99.6193945177002\\
9.00955277620718	98.2411232672412\\
10.5284357008095	97.058179996095\\
12.3033807625627	96.0436808596074\\
14.3775564091595	95.1741447388524\\
16.801408676838	94.4291757200772\\
19.6338880886805	93.7911289091959\\
22.9438833905788	93.2447841050906\\
26.8118969947666	92.7770406996795\\
31.3320028793888	92.3766401312885\\
36.6141345621924	92.0339179406489\\
42.7867587941599	91.7405849994885\\
50	91.489536140167\\
500	90.1489688725705\\
50000	90.0014896902672\\
50000000	90.0000014896903\\
500000000000	90.000000000149\\
5e+16	90\\
1e+20	90\\
};
\addlegendentry{uncompensated};

\addplot [color=black,solid,forget plot]
  table[row sep=crcr]{%
0.19468266671785	180\\
0.19468266671785	241.942764038024\\
};
\addplot [color=black,dotted,forget plot]
  table[row sep=crcr]{%
0.19468266671785	180\\
0.19468266671785	362.7\\
};
\addplot [color=black,dotted,forget plot]
  table[row sep=crcr]{%
0.565697298192371	180\\
0.565697298192371	362.7\\
};
\addplot [color=black,dotted,forget plot]
  table[row sep=crcr]{%
0.001	180\\
1000	180\\
};
\end{axis}
\end{tikzpicture}%
}
  \caption{The frequency response of the initial, uncontrolled process
    (\texttt{red}) and final, controlled process (\texttt{blue}).}
  \label{fig:bode_1.3}
\end{figure}

\begin{figure}[H]\centering
  \scalebox{0.8}{% This file was created by matlab2tikz.
%
%The latest updates can be retrieved from
%  http://www.mathworks.com/matlabcentral/fileexchange/22022-matlab2tikz-matlab2tikz
%where you can also make suggestions and rate matlab2tikz.
%
\definecolor{mycolor1}{rgb}{0.00000,0.44700,0.74100}%
\definecolor{mycolor2}{rgb}{0.85000,0.32500,0.09800}%
%
\begin{tikzpicture}

\begin{axis}[%
width=4.008in,
height=3.052in,
at={(0.818in,0.44in)},
scale only axis,
unbounded coords=jump,
separate axis lines,
every outer x axis line/.append style={white!40!black},
every x tick label/.append style={font=\color{white!40!black}},
xmin=0,
xmax=100,
xmajorgrids,
xlabel={Time [sec]},
every outer y axis line/.append style={white!40!black},
every y tick label/.append style={font=\color{white!40!black}},
ymin=-0.5,
ymax=3.2,
ymajorgrids,
ylabel={Amplitude},
axis background/.style={fill=white},
legend style={at={(0.711,0.632)},anchor=south west,legend cell align=left,align=left,draw=white!15!black,font=\footnotesize}
]
\addplot [color=mycolor1,solid,forget plot]
  table[row sep=crcr]{%
0	0\\
0.0218018956418615	-0.0337744173799193\\
0.043603791283723	-0.0637965068064521\\
0.0654056869255845	-0.0903942553507392\\
0.087207582567446	-0.113866895373478\\
0.109009478209308	-0.134487433597457\\
0.130811373851169	-0.152504957650852\\
0.152613269493031	-0.168146739661543\\
0.174415165134892	-0.181620154759825\\
0.196217060776754	-0.193114430775673\\
0.218018956418615	-0.20280224398366\\
0.239820852060477	-0.210841174441782\\
0.261622747702338	-0.217375033278443\\
0.2834246433442	-0.222535073194883\\
0.305226538986061	-0.226441092458879\\
0.327028434627923	-0.229202441761414\\
0.348830330269784	-0.230918942483395\\
0.370632225911646	-0.231681724167444\\
0.392434121553507	-0.23157398830391\\
0.414236017195369	-0.230671704914728\\
0.43603791283723	-0.229044247848241\\
0.457839808479092	-0.226754974177844\\
0.479641704120953	-0.223861752622753\\
0.501443599762815	-0.22041744547648\\
0.523245495404676	-0.216470348133891\\
0.545047391046538	-0.21206458994777\\
0.566849286688399	-0.207240499817547\\
0.588651182330261	-0.202034939613427\\
0.610453077972122	-0.196481608266127\\
0.632254973613984	-0.190611319103374\\
0.654056869255845	-0.184452252787233\\
0.675858764897707	-0.178030187999178\\
0.697660660539568	-0.171368711830909\\
0.71946255618143	-0.164489411666666\\
0.741264451823291	-0.157412050185619\\
0.763066347465153	-0.150154724969655\\
0.784868243107014	-0.142734014071159\\
0.806670138748876	-0.135165108776216\\
0.828472034390737	-0.127461934689949\\
0.850273930032599	-0.119637262171556\\
0.87207582567446	-0.111702807056221\\
0.893877721316322	-0.103669322518586\\
0.915679616958183	-0.0955466828572736\\
0.937481512600045	-0.0873439599113762\\
0.959283408241906	-0.0790694927572542\\
0.981085303883768	-0.0707309512769486\\
1.00288719952563	-0.0623353941374829\\
1.02468909516749	-0.0538893216728752\\
1.04649099080935	-0.0453987241174094\\
1.06829288645121	-0.0368691255992398\\
1.09009478209308	-0.0283056242674167\\
1.11189667773494	-0.0197129288925852\\
1.1336985733768	-0.0110953922516727\\
1.15550046901866	-0.00245704157957601\\
1.17730236466052	0.00619839365404605\\
1.19910426030238	0.0148674564074769\\
1.22090615594424	0.0235469389544629\\
1.24270805158611	0.0322338618007562\\
1.26450994722797	0.0409254544511184\\
1.28631184286983	0.049619137863932\\
1.30811373851169	0.0583125084448996\\
1.32991563415355	0.0670033234443786\\
1.35171752979541	0.0756894876348186\\
1.37351942543728	0.0843690411556383\\
1.39532132107914	0.093040148422794\\
1.417123216721	0.101701088009333\\
1.43892511236286	0.110350243411469\\
1.46072700800472	0.118986094622241\\
1.48252890364658	0.127607210441674\\
1.50433079928844	0.136212241458611\\
1.52613269493031	0.144799913645097\\
1.54793459057217	0.153369022509401\\
1.56973648621403	0.161918427758493\\
1.59153838185589	0.170447048425132\\
1.61334027749775	0.17895385841867\\
1.63514217313961	0.18743788246227\\
1.65694406878147	0.195898192382514\\
1.67874596442334	0.204333903720387\\
1.7005478600652	0.212744172635337\\
1.72234975570706	0.221128193076614\\
1.74415165134892	0.229485194198345\\
1.76595354699078	0.237814437996899\\
1.78775544263264	0.246115217150963\\
1.80955733827451	0.254386853046473\\
1.83135923391637	0.262628693970139\\
1.85316112955823	0.270840113456706\\
1.87496302520009	0.279020508776412\\
1.89676492084195	0.287169299550312\\
1.91856681648381	0.29528592648219\\
1.94036871212567	0.303369850196805\\
1.96217060776754	0.311420550175097\\
1.9839725034094	0.31943752377783\\
2.00577439905126	0.327420285349856\\
2.02757629469312	0.335368365397936\\
2.04937819033498	0.343281309835611\\
2.07118008597684	0.351158679289227\\
2.0929819816187	0.35900004845974\\
2.11478387726057	0.366805005535374\\
2.13658577290243	0.374573151650663\\
2.15838766854429	0.382304100387788\\
2.18018956418615	0.389997477316494\\
2.20199145982801	0.397652919569179\\
2.22379335546987	0.405270075448072\\
2.24559525111174	0.412848604061664\\
2.2673971467536	0.420388174987828\\
2.28919904239546	0.427888467961278\\
2.31100093803732	0.435349172583222\\
2.33280283367918	0.442769988051263\\
2.35460472932104	0.450150622907768\\
2.3764066249629	0.457490794805081\\
2.39820852060477	0.464790230286103\\
2.42001041624663	0.47204866457889\\
2.44181231188849	0.47926584140404\\
2.46361420753035	0.486441512793748\\
2.48541610317221	0.493575438921509\\
2.50721799881407	0.500667387941537\\
2.52901989445594	0.507717135837048\\
2.5508217900978	0.514724466276646\\
2.57262368573966	0.521689170478085\\
2.59442558138152	0.52861104707879\\
2.61622747702338	0.535489902012525\\
2.63802937266524	0.542325548391705\\
2.6598312683071	0.549117806394831\\
2.68163316394897	0.555866503158638\\
2.70343505959083	0.562571472674525\\
2.72523695523269	0.569232555688923\\
2.74703885087455	0.575849599607243\\
2.76884074651641	0.582422458401119\\
2.79064264215827	0.588950992518658\\
2.81244453780013	0.595435068797442\\
2.834246433442	0.601874560380058\\
2.85604832908386	0.608269346631944\\
2.87785022472572	0.614619313061361\\
2.89965212036758	0.620924351241308\\
2.92145401600944	0.627184358733243\\
2.9432559116513	0.633399239012439\\
2.96505780729317	0.639568901394872\\
2.98685970293503	0.645693260965495\\
3.00866159857689	0.651772238507812\\
3.03046349421875	0.657805760434641\\
3.05226538986061	0.663793758719984\\
3.07406728550247	0.669736170831913\\
3.09586918114433	0.675632939666411\\
3.1176710767862	0.681484013482091\\
3.13947297242806	0.687289345835736\\
3.16127486806992	0.693048895518605\\
3.18307676371178	0.69876262649346\\
3.20487865935364	0.704430507832251\\
3.2266805549955	0.710052513654448\\
3.24848245063736	0.715628623065948\\
3.27028434627923	0.721158820098554\\
3.29208624192109	0.726643093649977\\
3.31388813756295	0.73208143742434\\
3.33569003320481	0.737473849873156\\
3.35749192884667	0.742820334136768\\
3.37929382448853	0.74812089798621\\
3.4010957201304	0.753375553765499\\
3.42289761577226	0.758584318334314\\
3.44469951141412	0.763747213011068\\
3.46650140705598	0.768864263516348\\
3.48830330269784	0.773935499916714\\
3.5101051983397	0.778960956568858\\
3.53190709398156	0.783940672064084\\
3.55370898962343	0.788874689173143\\
3.57551088526529	0.793763054791375\\
3.59731278090715	0.798605819884176\\
3.61911467654901	0.803403039432779\\
3.64091657219087	0.808154772380342\\
3.66271846783273	0.812861081578334\\
3.6845203634746	0.817522033733226\\
3.70632225911646	0.822137699353476\\
3.72812415475832	0.826708152696807\\
3.74992605040018	0.831233471717773\\
3.77172794604204	0.835713738015615\\
3.7935298416839	0.84014903678241\\
3.81533173732576	0.844539456751492\\
3.83713363296763	0.848885090146169\\
3.85893552860949	0.853186032628719\\
3.88073742425135	0.857442383249666\\
3.90253931989321	0.86165424439734\\
3.92434121553507	0.865821721747724\\
3.94614311117693	0.869944924214571\\
3.96794500681879	0.874023963899814\\
3.98974690246066	0.878058956044252\\
4.01154879810252	0.882050018978517\\
4.03335069374438	0.885997274074329\\
4.05515258938624	0.889900845696025\\
4.0769544850281	0.893760861152379\\
4.09875638066996	0.897577450648702\\
4.12055827631183	0.901350747239226\\
4.14236017195369	0.905080886779774\\
4.16416206759555	0.908768007880715\\
4.18596396323741	0.912412251860207\\
4.20776585887927	0.91601376269772\\
4.22956775452113	0.919572686987858\\
4.25136965016299	0.92308917389446\\
4.27317154580486	0.926563375104992\\
4.29497344144672	0.929995444785234\\
4.31677533708858	0.933385539534251\\
4.33857723273044	0.936733818339658\\
4.3603791283723	0.940040442533179\\
4.38218102401416	0.943305575746497\\
4.40398291965603	0.946529383867396\\
4.42578481529789	0.949712034996202\\
4.44758671093975	0.952853699402515\\
4.46938860658161	0.955954549482235\\
4.49119050222347	0.959014759714892\\
4.51299239786533	0.962034506621264\\
4.53479429350719	0.965013968721299\\
4.55659618914906	0.967953326492332\\
4.57839808479092	0.970852762327601\\
4.60019998043278	0.973712460495065\\
4.62200187607464	0.976532607096515\\
4.6438037717165	0.979313390026996\\
4.66560566735836	0.982054998934517\\
4.68740756300022	0.984757625180073\\
4.70920945864209	0.987421461797963\\
4.73101135428395	0.99004670345641\\
4.75281324992581	0.992633546418486\\
4.77461514556767	0.995182188503337\\
4.79641704120953	0.997692829047709\\
4.81821893685139	1.00016566886779\\
4.84002083249326	1.00260091022132\\
4.86182272813512	1.00499875677006\\
4.88362462377698	1.00735941354254\\
4.90542651941884	1.00968308689704\\
4.9272284150607	1.01196998448503\\
4.94903031070256	1.01422031521475\\
4.97083220634442	1.01643428921522\\
4.99263410198629	1.01861211780047\\
5.01443599762815	1.02075401343411\\
5.03623789327001	1.02286018969423\\
5.05803978891187	1.02493086123854\\
5.07984168455373	1.02696624376983\\
5.10164358019559	1.02896655400183\\
5.12344547583746	1.03093200962522\\
5.14524737147932	1.03286282927404\\
5.16704926712118	1.03475923249241\\
5.18885116276304	1.03662143970147\\
5.2106530584049	1.03844967216674\\
5.23245495404676	1.04024415196567\\
5.25425684968862	1.04200510195556\\
5.27605874533049	1.0437327457418\\
5.29786064097235	1.04542730764629\\
5.31966253661421	1.04708901267632\\
5.34146443225607	1.04871808649368\\
5.36326632789793	1.05031475538399\\
5.38506822353979	1.0518792462265\\
5.40687011918165	1.05341178646402\\
5.42867201482352	1.05491260407329\\
5.45047391046538	1.05638192753553\\
5.47227580610724	1.05781998580736\\
5.4940777017491	1.05922700829201\\
5.51587959739096	1.06060322481078\\
5.53768149303282	1.06194886557487\\
5.55948338867469	1.06326416115744\\
5.58128528431655	1.06454934246598\\
5.60308717995841	1.06580464071502\\
5.62488907560027	1.06703028739904\\
5.64669097124213	1.06822651426577\\
5.66849286688399	1.06939355328975\\
5.69029476252585	1.07053163664612\\
5.71209665816772	1.07164099668478\\
5.73389855380958	1.0727218659048\\
5.75570044945144	1.07377447692916\\
5.7775023450933	1.07479906247969\\
5.79930424073516	1.07579585535238\\
5.82110613637702	1.07676508839296\\
5.84290803201888	1.07770699447272\\
5.86470992766075	1.07862180646465\\
5.88651182330261	1.0795097572199\\
5.90831371894447	1.08037107954442\\
5.93011561458633	1.08120600617598\\
5.95191751022819	1.0820147697614\\
5.97371940587005	1.08279760283414\\
5.99552130151192	1.08355473779205\\
6.01732319715378	1.08428640687551\\
6.03912509279564	1.0849928421458\\
6.0609269884375	1.0856742754637\\
6.08272888407936	1.08633093846847\\
6.10453077972122	1.08696306255698\\
6.12633267536308	1.08757087886323\\
6.14813457100495	1.08815461823801\\
6.16993646664681	1.08871451122898\\
6.19173836228867	1.08925078806087\\
6.21354025793053	1.08976367861605\\
6.23534215357239	1.09025341241534\\
6.25714404921425	1.09072021859903\\
6.27894594485612	1.09116432590826\\
6.30074784049798	1.09158596266655\\
6.32254973613984	1.09198535676173\\
6.3443516317817	1.09236273562797\\
6.36615352742356	1.09271832622819\\
6.38795542306542	1.09305235503667\\
6.40975731870728	1.09336504802193\\
6.43155921434915	1.09365663062988\\
6.45336110999101	1.09392732776718\\
6.47516300563287	1.09417736378489\\
6.49696490127473	1.09440696246237\\
6.51876679691659	1.0946163469914\\
6.54056869255845	1.09480573996057\\
6.56237058820031	1.09497536333992\\
6.58417248384218	1.09512543846578\\
6.60597437948404	1.09525618602594\\
6.6277762751259	1.09536782604499\\
6.64957817076776	1.09546057786991\\
6.67138006640962	1.09553466015591\\
6.69318196205148	1.09559029085254\\
6.71498385769334	1.09562768718996\\
6.73678575333521	1.09564706566554\\
6.75858764897707	1.0956486420306\\
6.78038954461893	1.09563263127746\\
6.80219144026079	1.09559924762667\\
6.82399333590265	1.09554870451446\\
6.84579523154451	1.09548121458051\\
6.86759712718638	1.09539698965583\\
6.88939902282824	1.09529624075093\\
6.9112009184701	1.09517917804419\\
6.93300281411196	1.09504601087047\\
6.95480470975382	1.09489694770995\\
6.97660660539568	1.09473219617713\\
6.99840850103754	1.09455196301011\\
7.02021039667941	1.09435645406007\\
7.04201229232127	1.09414587428095\\
7.06381418796313	1.09392042771936\\
7.08561608360499	1.09368031750467\\
7.10741797924685	1.09342574583936\\
7.12921987488871	1.09315691398956\\
7.15102177053058	1.09287402227573\\
7.17282366617244	1.09257727006367\\
7.1946255618143	1.09226685575566\\
7.21642745745616	1.09194297678175\\
7.23822935309802	1.0916058295914\\
7.26003124873988	1.09125560964515\\
7.28183314438174	1.09089251140663\\
7.30363504002361	1.09051672833468\\
7.32543693566547	1.09012845287568\\
7.34723883130733	1.0897278764561\\
7.36904072694919	1.08931518947521\\
7.39084262259105	1.08889058129802\\
7.41264451823291	1.08845424024836\\
7.43444641387477	1.08800635360218\\
7.45624830951664	1.08754710758104\\
7.4780502051585	1.08707668734573\\
7.49985210080036	1.08659527699016\\
7.52165399644222	1.08610305953536\\
7.54345589208408	1.08560021692369\\
7.56525778772594	1.08508693001322\\
7.58705968336781	1.08456337857227\\
7.60886157900967	1.08402974127418\\
7.63066347465153	1.08348619569218\\
7.65246537029339	1.08293291829451\\
7.67426726593525	1.08237008443961\\
7.69606916157711	1.08179786837158\\
7.71787105721897	1.08121644321574\\
7.73967295286084	1.08062598097441\\
7.7614748485027	1.08002665252279\\
7.78327674414456	1.07941862760504\\
7.80507863978642	1.07880207483054\\
7.82688053542828	1.07817716167026\\
7.84868243107014	1.07754405445332\\
7.87048432671201	1.07690291836371\\
7.89228622235387	1.07625391743717\\
7.91408811799573	1.07559721455815\\
7.93589001363759	1.07493297145704\\
7.95769190927945	1.07426134870747\\
7.97949380492131	1.07358250572377\\
8.00129570056317	1.07289660075861\\
8.02309759620504	1.07220379090074\\
8.0448994918469	1.07150423207288\\
8.06670138748876	1.07079807902985\\
8.08850328313062	1.07008548535665\\
8.11030517877248	1.06936660346687\\
8.13210707441434	1.06864158460115\\
8.15390897005621	1.06791057882575\\
8.17571086569807	1.06717373503132\\
8.19751276133993	1.06643120093177\\
8.21931465698179	1.06568312306327\\
8.24111655262365	1.0649296467834\\
8.26291844826551	1.0641709162704\\
8.28472034390737	1.06340707452261\\
8.30652223954923	1.06263826335792\\
8.3283241351911	1.06186462341348\\
8.35012603083296	1.06108629414546\\
8.37192792647482	1.06030341382891\\
8.39372982211668	1.05951611955781\\
8.41553171775854	1.05872454724519\\
8.4373336134004	1.05792883162338\\
8.45913550904227	1.05712910624439\\
8.48093740468413	1.05632550348037\\
8.50273930032599	1.05551815452423\\
8.52454119596785	1.05470718939034\\
8.54634309160971	1.05389273691535\\
8.56814498725157	1.05307492475911\\
8.58994688289343	1.05225387940573\\
8.6117487785353	1.05142972616469\\
8.63355067417716	1.05060258917213\\
8.65535256981902	1.04977259139217\\
8.67715446546088	1.04893985461839\\
8.69895636110274	1.04810449947537\\
8.7207582567446	1.04726664542033\\
8.74256015238647	1.04642641074495\\
8.76436204802833	1.04558391257715\\
8.78616394367019	1.04473926688308\\
8.80796583931205	1.04389258846913\\
8.82976773495391	1.04304399098412\\
8.85156963059577	1.04219358692146\\
8.87337152623764	1.04134148762153\\
8.8951734218795	1.04048780327403\\
8.91697531752136	1.03963264292054\\
8.93877721316322	1.03877611445703\\
8.96057910880508	1.03791832463657\\
8.98238100444694	1.0370593790721\\
9.0041829000888	1.03619938223922\\
9.02598479573066	1.0353384374791\\
9.04778669137253	1.03447664700153\\
9.06958858701439	1.03361411188796\\
9.09139048265625	1.03275093209463\\
9.11319237829811	1.03188720645585\\
9.13499427393997	1.03102303268727\\
9.15679616958183	1.03015850738924\\
9.1785980652237	1.02929372605033\\
9.20039996086556	1.02842878305077\\
9.22220185650742	1.0275637716661\\
9.24400375214928	1.02669878407079\\
9.26580564779114	1.02583391134203\\
9.287607543433	1.02496924346346\\
9.30940943907486	1.02410486932906\\
9.33121133471673	1.02324087674709\\
9.35301323035859	1.02237735244405\\
9.37481512600045	1.02151438206878\\
9.39661702164231	1.02065205019651\\
9.41841891728417	1.01979044033308\\
9.44022081292603	1.01892963491916\\
9.4620227085679	1.01806971533453\\
9.48382460420976	1.01721076190243\\
9.50562649985162	1.01635285389398\\
9.52742839549348	1.0154960695326\\
9.54923029113534	1.01464048599854\\
9.5710321867772	1.01378617943345\\
9.59283408241907	1.01293322494497\\
9.61463597806093	1.0120816966114\\
9.63643787370279	1.01123166748643\\
9.65823976934465	1.01038320960385\\
9.68004166498651	1.00953639398241\\
9.70184356062837	1.00869129063064\\
9.72364545627023	1.00784796855175\\
9.74544735191209	1.00700649574858\\
9.76724924755396	1.00616693922855\\
9.78905114319582	1.00532936500873\\
9.81085303883768	1.00449383812088\\
9.83265493447954	1.00366042261654\\
9.8544568301214	1.0028291815722\\
9.87625872576326	1.00200017709446\\
9.89806062140513	1.00117347032526\\
9.91986251704699	1.00034912144712\\
9.94166441268885	0.999527189688439\\
9.96346630833071	0.998707733328817\\
9.98526820397257	0.997890809704398\\
10.0070700996144	0.99707647521326\\
10.0288719952563	0.99626478532084\\
10.0506738908982	0.995455794565371\\
10.07247578654	0.994649556563367\\
10.0942776821819	0.993846124015122\\
10.1160795778237	0.993045548710245\\
10.1378814734656	0.992247881533221\\
10.1596833691075	0.991453172468989\\
10.1814852647493	0.990661470608559\\
10.2032871603912	0.98987282415464\\
10.225089056033	0.989087280427298\\
10.2468909516749	0.988304885869636\\
10.2686928473168	0.987525686053493\\
10.2904947429586	0.986749725685163\\
10.3122966386005	0.985977048611139\\
10.3340985342424	0.985207697823867\\
10.3559004298842	0.984441715467529\\
10.3777023255261	0.983679142843836\\
10.3995042211679	0.982920020417836\\
10.4213061168098	0.982164387823746\\
10.4431080124517	0.98141228387079\\
10.4649099080935	0.980663746549058\\
10.4867118037354	0.979918813035374\\
10.5085136993772	0.979177519699178\\
10.5303155950191	0.978439902108423\\
10.552117490661	0.977705995035477\\
10.5739193863028	0.976975832463041\\
10.5957212819447	0.976249447590076\\
10.6175231775866	0.975526872837734\\
10.6393250732284	0.974808139855303\\
10.6611269688703	0.974093279526161\\
10.6829288645121	0.973382321973727\\
10.704730760154	0.972675296567431\\
10.7265326557959	0.97197223192868\\
10.7483345514377	0.971273155936834\\
10.7701364470796	0.970578095735184\\
10.7919383427214	0.969887077736934\\
10.8137402383633	0.969200127631186\\
10.8355421340052	0.96851727038893\\
10.857344029647	0.96783853026903\\
10.8791459252889	0.967163930824215\\
10.9009478209308	0.96649349490707\\
10.9227497165726	0.965827244676028\\
10.9445516122145	0.965165201601355\\
10.9663535078563	0.964507386471145\\
10.9881554034982	0.963853819397299\\
11.0099572991401	0.963204519821515\\
11.0317591947819	0.962559506521263\\
11.0535610904238	0.961918797615768\\
11.0753629860656	0.961282410571976\\
11.0971648817075	0.960650362210528\\
11.1189667773494	0.960022668711716\\
11.1407686729912	0.959399345621446\\
11.1625705686331	0.958780407857183\\
11.184372464275	0.958165869713898\\
11.2061743599168	0.957555744869999\\
11.2279762555587	0.956950046393263\\
11.2497781512005	0.956348786746754\\
11.2715800468424	0.955751977794731\\
11.2933819424843	0.955159630808552\\
11.3151838381261	0.954571756472562\\
11.336985733768	0.953988364889973\\
11.3587876294098	0.953409465588736\\
11.3805895250517	0.952835067527396\\
11.4023914206936	0.952265179100942\\
11.4241933163354	0.951699808146634\\
11.4459952119773	0.951138961949831\\
11.4677971076192	0.950582647249796\\
11.489599003261	0.950030870245489\\
11.5114008989029	0.949483636601349\\
11.5332027945447	0.94894095145306\\
11.5550046901866	0.9484028194133\\
11.5768065858285	0.947869244577477\\
11.5986084814703	0.947340230529452\\
11.6204103771122	0.946815780347238\\
11.642212272754	0.946295896608693\\
11.6640141683959	0.945780581397184\\
11.6858160640378	0.945269836307247\\
11.7076179596796	0.944763662450218\\
11.7294198553215	0.944262060459854\\
11.7512217509634	0.94376503049793\\
11.7730236466052	0.943272572259823\\
11.7948255422471	0.942784684980071\\
11.8166274378889	0.942301367437917\\
11.8384293335308	0.941822617962832\\
11.8602312291727	0.941348434440019\\
11.8820331248145	0.940878814315894\\
11.9038350204564	0.940413754603549\\
11.9256369160982	0.939953251888193\\
11.9474388117401	0.939497302332572\\
11.969240707382	0.939045901682367\\
11.9910426030238	0.938599045271573\\
12.0128444986657	0.93815672802785\\
12.0346463943076	0.937718944477858\\
12.0564482899494	0.937285688752563\\
12.0782501855913	0.936856954592531\\
12.1000520812331	0.936432735353181\\
12.121853976875	0.936013024010034\\
12.1436558725169	0.935597813163925\\
12.1654577681587	0.935187095046195\\
12.1872596638006	0.934780861523861\\
12.2090615594424	0.93437910410476\\
12.2308634550843	0.933981813942666\\
12.2526653507262	0.933588981842386\\
12.274467246368	0.933200598264827\\
12.2962691420099	0.932816653332045\\
12.3180710376518	0.932437136832256\\
12.3398729332936	0.932062038224834\\
12.3616748289355	0.931691346645275\\
12.3834767245773	0.93132505091014\\
12.4052786202192	0.930963139521967\\
12.4270805158611	0.930605600674159\\
12.4488824115029	0.930252422255846\\
12.4706843071448	0.929903591856718\\
12.4924862027866	0.929559096771835\\
12.5142880984285	0.929218924006402\\
12.5360899940704	0.928883060280528\\
12.5578918897122	0.928551492033945\\
12.5796937853541	0.928224205430709\\
12.601495680996	0.927901186363869\\
12.6232975766378	0.927582420460108\\
12.6450994722797	0.92726789308436\\
12.6669013679215	0.92695758934439\\
12.6887032635634	0.926651494095354\\
12.7105051592053	0.926349591944329\\
12.7323070548471	0.926051867254809\\
12.754108950489	0.925758304151181\\
12.7759108461308	0.925468886523163\\
12.7977127417727	0.925183598030223\\
12.8195146374146	0.924902422105958\\
12.8413165330564	0.92462534196245\\
12.8631184286983	0.924352340594598\\
12.8849203243402	0.924083400784406\\
12.906722219982	0.923818505105258\\
12.9285241156239	0.923557635926152\\
12.9503260112657	0.92330077541591\\
12.9721279069076	0.923047905547355\\
12.9939298025495	0.92279900810146\\
13.0157316981913	0.922554064671471\\
13.0375335938332	0.922313056666991\\
13.059335489475	0.922075965318044\\
13.0811373851169	0.9218427716791\\
13.1029392807588	0.921613456633079\\
13.1247411764006	0.921388000895318\\
13.1465430720425	0.92116638501751\\
13.1683449676844	0.920948589391615\\
13.1901468633262	0.920734594253736\\
13.2119487589681	0.92052437968797\\
13.2337506546099	0.920317925630228\\
13.2555525502518	0.920115211872018\\
13.2773544458937	0.91991621806421\\
13.2991563415355	0.919720923720754\\
13.3209582371774	0.919529308222389\\
13.3427601328192	0.9193413508203\\
13.3645620284611	0.919157030639758\\
13.386363924103	0.918976326683728\\
13.4081658197448	0.918799217836443\\
13.4299677153867	0.918625682866949\\
13.4517696110286	0.918455700432618\\
13.4735715066704	0.918289249082638\\
13.4953734023123	0.918126307261464\\
13.5171752979541	0.917966853312239\\
13.538977193596	0.917810865480194\\
13.5607790892379	0.917658321916006\\
13.5825809848797	0.917509200679131\\
13.6043828805216	0.91736347974111\\
13.6261847761634	0.917221136988839\\
13.6479866718053	0.917082150227812\\
13.6697885674472	0.916946497185331\\
13.691590463089	0.916814155513689\\
13.7133923587309	0.916685102793322\\
13.7351942543728	0.916559316535931\\
13.7569961500146	0.91643677418757\\
13.7787980456565	0.916317453131711\\
13.8005999412983	0.916201330692276\\
13.8224018369402	0.916088384136633\\
13.8442037325821	0.915978590678574\\
13.8660056282239	0.915871927481254\\
13.8878075238658	0.9157683716601\\
13.9096094195076	0.915667900285702\\
13.9314113151495	0.915570490386656\\
13.9532132107914	0.915476118952392\\
13.9750151064332	0.915384762935968\\
13.9968170020751	0.915296399256833\\
14.018618897717	0.915211004803562\\
14.0404207933588	0.91512855643656\\
14.0622226890007	0.915049030990743\\
14.0840245846425	0.914972405278179\\
14.1058264802844	0.914898656090709\\
14.1276283759263	0.914827760202539\\
14.1494302715681	0.914759694372793\\
14.17123216721	0.914694435348051\\
14.1930340628518	0.914631959864849\\
14.2148359584937	0.914572244652152\\
14.2366378541356	0.9145152664338\\
14.2584397497774	0.914461001930926\\
14.2802416454193	0.914409427864343\\
14.3020435410612	0.914360520956904\\
14.323845436703	0.914314257935837\\
14.3456473323449	0.914270615535043\\
14.3674492279867	0.914229570497379\\
14.3892511236286	0.914191099576901\\
14.4110530192705	0.914155179541085\\
14.4328549149123	0.914121787173022\\
14.4546568105542	0.914090899273579\\
14.476458706196	0.91406249266354\\
14.4982606018379	0.914036544185716\\
14.5200624974798	0.914013030707024\\
14.5418643931216	0.913991929120548\\
14.5636662887635	0.913973216347564\\
14.5854681844054	0.913956869339542\\
14.6072700800472	0.913942865080125\\
14.6290719756891	0.91393118058707\\
14.6508738713309	0.913921792914179\\
14.6726757669728	0.913914679153185\\
14.6944776626147	0.913909816435629\\
14.7162795582565	0.913907181934697\\
14.7380814538984	0.913906752867041\\
14.7598833495402	0.913908506494568\\
14.7816852451821	0.913912420126202\\
14.803487140824	0.913918471119629\\
14.8252890364658	0.91392663688301\\
14.8470909321077	0.913936894876664\\
14.8688928277495	0.913949222614736\\
14.8906947233914	0.913963597666833\\
14.9124966190333	0.91397999765964\\
14.9342985146751	0.913998400278501\\
14.956100410317	0.914018783268991\\
14.9779023059589	0.914041124438446\\
14.9997042016007	0.914065401657485\\
15.0215060972426	0.914091592861493\\
15.0433079928844	0.91411967605209\\
15.0651098885263	0.914149629298574\\
15.0869117841682	0.914181430739333\\
15.10871367981	0.914215058583245\\
15.1305155754519	0.914250491111044\\
15.1523174710937	0.914287706676665\\
15.1741193667356	0.914326683708575\\
15.1959212623775	0.914367400711062\\
15.2177231580193	0.914409836265522\\
15.2395250536612	0.914453969031705\\
15.2613269493031	0.914499777748951\\
15.2831288449449	0.914547241237396\\
15.3049307405868	0.91459633839916\\
15.3267326362286	0.914647048219508\\
15.3485345318705	0.914699349767994\\
15.3703364275124	0.914753222199578\\
15.3921383231542	0.914808644755723\\
15.4139402187961	0.914865596765475\\
15.4357421144379	0.914924057646511\\
15.4575440100798	0.914984006906178\\
15.4793459057217	0.915045424142497\\
15.5011478013635	0.915108289045159\\
15.5229496970054	0.915172581396491\\
15.5447515926473	0.915238281072405\\
15.5665534882891	0.915305368043327\\
15.588355383931	0.9153738223751\\
15.6101572795728	0.915443624229874\\
15.6319591752147	0.915514753866971\\
15.6537610708566	0.915587191643732\\
15.6755629664984	0.915660918016341\\
15.6973648621403	0.915735913540633\\
15.7191667577821	0.91581215887288\\
15.740968653424	0.915889634770563\\
15.7627705490659	0.915968322093109\\
15.7845724447077	0.916048201802631\\
15.8063743403496	0.916129254964632\\
15.8281762359915	0.916211462748696\\
15.8499781316333	0.916294806429163\\
15.8717800272752	0.916379267385778\\
15.893581922917	0.916464827104331\\
15.9153838185589	0.916551467177269\\
15.9371857142008	0.9166391693043\\
15.9589876098426	0.916727915292969\\
15.9807895054845	0.916817687059223\\
16.0025914011263	0.916908466627958\\
16.0243932967682	0.917000236133543\\
16.0461951924101	0.917092977820335\\
16.0679970880519	0.917186674043168\\
16.0897989836938	0.917281307267832\\
16.1116008793357	0.917376860071534\\
16.1334027749775	0.917473315143334\\
16.1552046706194	0.91757065528458\\
16.1770065662612	0.917668863409311\\
16.1988084619031	0.917767922544654\\
16.220610357545	0.917867815831198\\
16.2424122531868	0.91796852652336\\
16.2642141488287	0.918070037989728\\
16.2860160444705	0.91817233371339\\
16.3078179401124	0.91827539729225\\
16.3296198357543	0.918379212439328\\
16.3514217313961	0.918483762983039\\
16.373223627038	0.91858903286747\\
16.3950255226799	0.918695006152626\\
16.4168274183217	0.918801667014675\\
16.4386293139636	0.918908999746169\\
16.4604312096054	0.919016988756256\\
16.4822331052473	0.919125618570875\\
16.5040350008892	0.919234873832938\\
16.525836896531	0.9193447393025\\
16.5476387921729	0.919455199856907\\
16.5694406878147	0.919566240490945\\
16.5912425834566	0.919677846316957\\
16.6130444790985	0.919790002564965\\
16.6348463747403	0.919902694582766\\
16.6566482703822	0.920015907836015\\
16.6784501660241	0.920129627908307\\
16.7002520616659	0.920243840501233\\
16.7220539573078	0.920358531434431\\
16.7438558529496	0.920473686645619\\
16.7656577485915	0.920589292190622\\
16.7874596442334	0.920705334243381\\
16.8092615398752	0.920821799095952\\
16.8310634355171	0.920938673158496\\
16.8528653311589	0.921055942959253\\
16.8746672268008	0.921173595144505\\
16.8964691224427	0.92129161647853\\
16.9182710180845	0.921409993843542\\
16.9400729137264	0.921528714239621\\
16.9618748093683	0.921647764784631\\
16.9836767050101	0.92176713271413\\
17.005478600652	0.921886805381264\\
17.0272804962938	0.922006770256655\\
17.0490823919357	0.922127014928276\\
17.0708842875776	0.922247527101317\\
17.0926861832194	0.92236829459804\\
17.1144880788613	0.922489305357622\\
17.1362899745031	0.922610547435993\\
17.158091870145	0.922732009005658\\
17.1798937657869	0.922853678355515\\
17.2016956614287	0.922975543890658\\
17.2234975570706	0.923097594132176\\
17.2452994527125	0.923219817716939\\
17.2671013483543	0.923342203397374\\
17.2889032439962	0.923464740041238\\
17.310705139638	0.923587416631372\\
17.3325070352799	0.923710222265456\\
17.3543089309218	0.923833146155751\\
17.3761108265636	0.923956177628832\\
17.3979127222055	0.924079306125313\\
17.4197146178473	0.924202521199565\\
17.4415165134892	0.924325812519425\\
17.4633184091311	0.924449169865896\\
17.4851203047729	0.924572583132839\\
17.5069222004148	0.924696042326663\\
17.5287240960567	0.924819537565996\\
17.5505259916985	0.924943059081359\\
17.5723278873404	0.92506659721483\\
17.5941297829822	0.925190142419696\\
17.6159316786241	0.925313685260101\\
17.637733574266	0.925437216410691\\
17.6595354699078	0.925560726656242\\
17.6813373655497	0.925684206891296\\
17.7031392611915	0.925807648119774\\
17.7249411568334	0.925931041454593\\
17.7467430524753	0.926054378117274\\
17.7685449481171	0.926177649437543\\
17.790346843759	0.926300846852926\\
17.8121487394009	0.926423961908339\\
17.8339506350427	0.926546986255668\\
17.8557525306846	0.926669911653351\\
17.8775544263264	0.926792729965944\\
17.8993563219683	0.926915433163692\\
17.9211582176102	0.927038013322084\\
17.942960113252	0.927160462621411\\
17.9647620088939	0.927282773346317\\
17.9865639045357	0.927404937885337\\
18.0083658001776	0.927526948730443\\
18.0301676958195	0.927648798476573\\
18.0519695914613	0.927770479821164\\
18.0737714871032	0.927891985563673\\
18.0955733827451	0.928013308605099\\
18.1173752783869	0.928134441947499\\
18.1391771740288	0.928255378693494\\
18.1609790696706	0.928376112045781\\
18.1827809653125	0.928496635306632\\
18.2045828609544	0.92861694187739\\
18.2263847565962	0.928737025257969\\
18.2481866522381	0.928856879046337\\
18.2699885478799	0.928976496938005\\
18.2917904435218	0.929095872725511\\
18.3135923391637	0.929215000297892\\
18.3353942348055	0.929333873640169\\
18.3571961304474	0.929452486832808\\
18.3789980260893	0.929570834051196\\
18.4007999217311	0.9296889095651\\
18.422601817373	0.929806707738133\\
18.4444037130148	0.929924223027208\\
18.4662056086567	0.930041449981998\\
18.4880075042986	0.930158383244385\\
18.5098093999404	0.930275017547912\\
18.5316112955823	0.930391347717227\\
18.5534131912241	0.93050736866753\\
18.575215086866	0.930623075404013\\
18.5970169825079	0.930738463021301\\
18.6188188781497	0.930853526702886\\
18.6406207737916	0.930968261720563\\
18.6624226694335	0.931082663433863\\
18.6842245650753	0.931196727289482\\
18.7060264607172	0.931310448820709\\
18.727828356359	0.931423823646851\\
18.7496302520009	0.93153684747266\\
18.7714321476428	0.931649516087753\\
18.7932340432846	0.931761825366034\\
18.8150359389265	0.93187377126511\\
18.8368378345683	0.931985349825712\\
18.8586397302102	0.93209655717111\\
18.8804416258521	0.932207389506525\\
18.9022435214939	0.932317843118545\\
18.9240454171358	0.932427914374534\\
18.9458473127777	0.932537599722046\\
18.9676492084195	0.93264689568823\\
18.9894511040614	0.932755798879244\\
19.0112529997032	0.932864305979657\\
19.0330548953451	0.932972413751859\\
19.054856790987	0.933080119035465\\
19.0766586866288	0.933187418746721\\
19.0984605822707	0.933294309877907\\
19.1202624779125	0.933400789496742\\
19.1420643735544	0.933506854745786\\
19.1638662691963	0.933612502841842\\
19.1856681648381	0.93371773107536\\
19.20747006048	0.933822536809838\\
19.2292719561219	0.93392691748122\\
19.2510738517637	0.934030870597306\\
19.2728757474056	0.934134393737141\\
19.2946776430474	0.934237484550426\\
19.3164795386893	0.934340140756914\\
19.3382814343312	0.934442360145813\\
19.360083329973	0.934544140575185\\
19.3818852256149	0.934645479971347\\
19.4036871212567	0.934746376328276\\
19.4254890168986	0.934846827707005\\
19.4472909125405	0.934946832235031\\
19.4690928081823	0.935046388105712\\
19.4908947038242	0.935145493577671\\
19.5126965994661	0.935244146974202\\
19.5344984951079	0.935342346682669\\
19.5563003907498	0.935440091153915\\
19.5781022863916	0.935537378901664\\
19.5999041820335	0.935634208501925\\
19.6217060776754	0.935730578592404\\
19.6435079733172	0.935826487871905\\
19.6653098689591	0.935921935099742\\
19.6871117646009	0.936016919095145\\
19.7089136602428	0.936111438736675\\
19.7307155558847	0.936205492961625\\
19.7525174515265	0.936299080765445\\
19.7743193471684	0.936392201201142\\
19.7961212428103	0.936484853378703\\
19.8179231384521	0.936577036464507\\
19.839725034094	0.936668749680742\\
19.8615269297358	0.93675999230482\\
19.8833288253777	0.936850763668802\\
19.9051307210196	0.936941063158814\\
19.9269326166614	0.937030890214468\\
19.9487345123033	0.937120244328289\\
19.9705364079451	0.937209125045137\\
19.992338303587	0.937297531961636\\
20.0141401992289	0.937385464725597\\
20.0359420948707	0.937472923035453\\
20.0577439905126	0.937559906639687\\
20.0795458861545	0.937646415336264\\
20.1013477817963	0.937732448972068\\
20.1231496774382	0.937818007442338\\
20.14495157308	0.937903090690104\\
20.1667534687219	0.937987698705627\\
20.1885553643638	0.938071831525843\\
20.2103572600056	0.938155489233807\\
20.2321591556475	0.938238671958134\\
20.2539610512893	0.938321379872451\\
20.2757629469312	0.938403613194844\\
20.2975648425731	0.938485372187312\\
20.3193667382149	0.938566657155219\\
20.3411686338568	0.938647468446748\\
20.3629705294987	0.938727806452363\\
20.3847724251405	0.938807671604264\\
20.4065743207824	0.938887064375854\\
20.4283762164242	0.938965985281201\\
20.4501781120661	0.939044434874503\\
20.471980007708	0.93912241374956\\
20.4937819033498	0.939199922539244\\
20.5155837989917	0.939276961914971\\
20.5373856946335	0.939353532586178\\
20.5591875902754	0.939429635299803\\
20.5809894859173	0.939505270839762\\
20.6027913815591	0.939580440026434\\
20.624593277201	0.939655143716146\\
20.6463951728429	0.939729382800661\\
20.6681970684847	0.93980315820667\\
20.6899989641266	0.939876470895281\\
20.7118008597684	0.939949321861518\\
20.7336027554103	0.940021712133817\\
20.7554046510522	0.940093642773527\\
20.777206546694	0.940165114874416\\
20.7990084423359	0.940236129562169\\
20.8208103379777	0.940306687993906\\
20.8426122336196	0.940376791357687\\
20.8644141292615	0.940446440872026\\
20.8862160249033	0.940515637785409\\
20.9080179205452	0.940584383375814\\
20.929819816187	0.940652678950229\\
20.9516217118289	0.940720525844181\\
20.9734236074708	0.94078792542126\\
20.9952255031126	0.94085487907265\\
21.0170273987545	0.940921388216662\\
21.0388292943964	0.940987454298271\\
21.0606311900382	0.941053078788651\\
21.0824330856801	0.94111826318472\\
21.1042349813219	0.941183009008681\\
21.1260368769638	0.941247317807571\\
21.1478387726057	0.94131119115281\\
21.1696406682475	0.941374630639754\\
21.1914425638894	0.941437637887248\\
21.2132444595312	0.94150021453719\\
21.2350463551731	0.941562362254086\\
21.256848250815	0.94162408272462\\
21.2786501464568	0.941685377657216\\
21.3004520420987	0.941746248781611\\
21.3222539377406	0.941806697848429\\
21.3440558333824	0.941866726628754\\
21.3658577290243	0.941926336913711\\
21.3876596246661	0.941985530514046\\
21.409461520308	0.942044309259715\\
21.4312634159499	0.942102674999468\\
21.4530653115917	0.942160629600439\\
21.4748672072336	0.942218174947743\\
21.4966691028754	0.942275312944074\\
21.5184709985173	0.942332045509298\\
21.5402728941592	0.942388374580064\\
21.562074789801	0.942444302109404\\
21.5838766854429	0.942499830066345\\
21.6056785810848	0.942554960435519\\
21.6274804767266	0.94260969521678\\
21.6492823723685	0.942664036424819\\
21.6710842680103	0.942717986088789\\
21.6928861636522	0.942771546251924\\
21.7146880592941	0.942824718971171\\
21.7364899549359	0.942877506316815\\
21.7582918505778	0.942929910372119\\
21.7800937462196	0.942981933232951\\
21.8018956418615	0.943033577007433\\
21.8236975375034	0.943084843815573\\
21.8454994331452	0.943135735788922\\
21.8673013287871	0.943186255070209\\
21.889103224429	0.943236403813005\\
21.9109051200708	0.943286184181369\\
21.9327070157127	0.943335598349507\\
21.9545089113545	0.943384648501434\\
21.9763108069964	0.943433336830639\\
21.9981127026383	0.943481665539747\\
22.0199145982801	0.943529636840192\\
22.041716493922	0.943577252951887\\
22.0635183895638	0.943624516102903\\
22.0853202852057	0.943671428529143\\
22.1071221808476	0.943717992474031\\
22.1289240764894	0.943764210188185\\
22.1507259721313	0.943810083929118\\
22.1725278677732	0.943855615960919\\
22.194329763415	0.943900808553949\\
22.2161316590569	0.94394566398454\\
22.2379335546987	0.943990184534694\\
22.2597354503406	0.944034372491782\\
22.2815373459825	0.944078230148255\\
22.3033392416243	0.944121759801348\\
22.3251411372662	0.944164963752794\\
22.346943032908	0.944207844308538\\
22.3687449285499	0.944250403778456\\
22.3905468241918	0.944292644476071\\
22.4123487198336	0.944334568718282\\
22.4341506154755	0.944376178825087\\
22.4559525111174	0.94441747711931\\
22.4777544067592	0.94445846592634\\
22.4995563024011	0.944499147573861\\
22.5213581980429	0.944539524391589\\
22.5431600936848	0.944579598711017\\
22.5649619893267	0.944619372865157\\
22.5867638849685	0.944658849188285\\
22.6085657806104	0.944698030015694\\
22.6303676762522	0.944736917683441\\
22.6521695718941	0.94477551452811\\
22.673971467536	0.944813822886563\\
22.6957733631778	0.944851845095704\\
22.7175752588197	0.944889583492243\\
22.7393771544616	0.944927040412463\\
22.7611790501034	0.944964218191986\\
22.7829809457453	0.945001119165552\\
22.8047828413871	0.945037745666786\\
22.826584737029	0.945074100027981\\
22.8483866326709	0.945110184579877\\
22.8701885283127	0.945146001651445\\
22.8919904239546	0.945181553569672\\
22.9137923195964	0.94521684265935\\
22.9355942152383	0.945251871242869\\
22.9573961108802	0.945286641640009\\
22.979198006522	0.945321156167739\\
23.0009999021639	0.945355417140016\\
23.0228017978058	0.945389426867584\\
23.0446036934476	0.945423187657786\\
23.0664055890895	0.945456701814366\\
23.0882074847313	0.94548997163728\\
23.1100093803732	0.945522999422511\\
23.1318112760151	0.945555787461883\\
23.1536131716569	0.94558833804288\\
23.1754150672988	0.945620653448465\\
23.1972169629406	0.945652735956907\\
23.2190188585825	0.945684587841602\\
23.2408207542244	0.945716211370905\\
23.2626226498662	0.945747608807961\\
23.2844245455081	0.945778782410535\\
23.30622644115	0.945809734430852\\
23.3280283367918	0.945840467115435\\
23.3498302324337	0.945870982704944\\
23.3716321280755	0.945901283434021\\
23.3934340237174	0.945931371531133\\
23.4152359193593	0.945961249218429\\
23.4370378150011	0.945990918711578\\
23.458839710643	0.946020382219634\\
23.4806416062848	0.946049641944885\\
23.5024435019267	0.946078700082715\\
23.5242453975686	0.946107558821458\\
23.5460472932104	0.94613622034227\\
23.5678491888523	0.946164686818987\\
23.5896510844942	0.946192960417996\\
23.611452980136	0.946221043298104\\
23.6332548757779	0.946248937610409\\
23.6550567714197	0.946276645498175\\
23.6768586670616	0.946304169096713\\
23.6986605627035	0.946331510533252\\
23.7204624583453	0.946358671926827\\
23.7422643539872	0.946385655388158\\
23.764066249629	0.946412463019541\\
23.7858681452709	0.94643909691473\\
23.8076700409128	0.946465559158829\\
23.8294719365546	0.946491851828189\\
23.8512738321965	0.946517976990295\\
23.8730757278384	0.946543936703668\\
23.8948776234802	0.946569733017761\\
23.9166795191221	0.94659536797286\\
23.9384814147639	0.946620843599991\\
23.9602833104058	0.946646161920818\\
23.9820852060477	0.946671324947557\\
24.0038871016895	0.946696334682879\\
24.0256889973314	0.946721193119827\\
24.0474908929732	0.946745902241725\\
24.0692927886151	0.946770464022095\\
24.091094684257	0.946794880424576\\
24.1128965798988	0.946819153402838\\
24.1346984755407	0.946843284900509\\
24.1565003711826	0.946867276851094\\
24.1783022668244	0.946891131177904\\
24.2001041624663	0.946914849793979\\
24.2219060581081	0.946938434602019\\
24.24370795375	0.946961887494314\\
24.2655098493919	0.946985210352679\\
24.2873117450337	0.947008405048383\\
24.3091136406756	0.947031473442094\\
24.3309155363174	0.947054417383808\\
24.3527174319593	0.947077238712796\\
24.3745193276012	0.947099939257541\\
24.396321223243	0.947122520835684\\
24.4181231188849	0.947144985253972\\
24.4399250145268	0.947167334308198\\
24.4617269101686	0.947189569783156\\
24.4835288058105	0.94721169345259\\
24.5053307014523	0.947233707079144\\
24.5271325970942	0.94725561241432\\
24.5489344927361	0.947277411198429\\
24.5707363883779	0.947299105160552\\
24.5925382840198	0.947320696018499\\
24.6143401796616	0.947342185478765\\
24.6361420753035	0.947363575236497\\
24.6579439709454	0.947384866975456\\
24.6797458665872	0.947406062367982\\
24.7015477622291	0.947427163074962\\
24.723349657871	0.947448170745798\\
24.7451515535128	0.947469087018376\\
24.7669534491547	0.947489913519037\\
24.7887553447965	0.947510651862554\\
24.8105572404384	0.947531303652103\\
24.8323591360803	0.947551870479239\\
24.8541610317221	0.947572353923874\\
24.875962927364	0.947592755554258\\
24.8977648230058	0.947613076926955\\
24.9195667186477	0.947633319586829\\
24.9413686142896	0.947653485067026\\
24.9631705099314	0.947673574888955\\
24.9849724055733	0.947693590562279\\
25.0067743012152	0.947713533584901\\
25.028576196857	0.947733405442949\\
25.0503780924989	0.947753207610769\\
25.0721799881407	0.947772941550918\\
25.0939818837826	0.947792608714152\\
25.1157837794245	0.947812210539422\\
25.1375856750663	0.947831748453872\\
25.1593875707082	0.947851223872828\\
25.18118946635	0.947870638199805\\
25.2029913619919	0.947889992826497\\
25.2247932576338	0.947909289132783\\
25.2465951532756	0.947928528486724\\
25.2683970489175	0.94794771224457\\
25.2901989445594	0.94796684175076\\
25.3120008402012	0.947985918337926\\
25.3338027358431	0.948004943326903\\
25.3556046314849	0.948023918026731\\
25.3774065271268	0.948042843734669\\
25.3992084227687	0.948061721736196\\
25.4210103184105	0.948080553305031\\
25.4428122140524	0.948099339703135\\
25.4646141096942	0.94811808218073\\
25.4864160053361	0.948136781976309\\
25.508217900978	0.948155440316652\\
25.5300197966198	0.948174058416841\\
25.5518216922617	0.948192637480277\\
25.5736235879036	0.948211178698694\\
25.5954254835454	0.948229683252184\\
25.6172273791873	0.94824815230921\\
25.6390292748291	0.94826658702663\\
25.660831170471	0.948284988549717\\
25.6826330661129	0.948303358012182\\
25.7044349617547	0.948321696536195\\
25.7262368573966	0.94834000523241\\
25.7480387530384	0.948358285199994\\
25.7698406486803	0.948376537526645\\
25.7916425443222	0.948394763288624\\
25.813444439964	0.948412963550781\\
25.8352463356059	0.948431139366583\\
25.8570482312478	0.948449291778144\\
25.8788501268896	0.948467421816253\\
25.9006520225315	0.948485530500407\\
25.9224539181733	0.948503618838839\\
25.9442558138152	0.948521687828556\\
25.9660577094571	0.948539738455366\\
25.9878596050989	0.948557771693913\\
26.0096615007408	0.948575788507716\\
26.0314633963826	0.948593789849198\\
26.0532652920245	0.948611776659725\\
26.0750671876664	0.948629749869643\\
26.0968690833082	0.948647710398313\\
26.1186709789501	0.948665659154152\\
26.1404728745919	0.948683597034667\\
26.1622747702338	0.948701524926499\\
26.1840766658757	0.948719443705461\\
26.2058785615175	0.948737354236576\\
26.2276804571594	0.948755257374124\\
26.2494823528013	0.948773153961677\\
26.2712842484431	0.948791044832144\\
26.293086144085	0.948808930807817\\
26.3148880397268	0.948826812700408\\
26.3366899353687	0.948844691311098\\
26.3584918310106	0.948862567430582\\
26.3802937266524	0.948880441839107\\
26.4020956222943	0.948898315306528\\
26.4238975179362	0.948916188592345\\
26.445699413578	0.948934062445753\\
26.4675013092199	0.948951937605693\\
26.4893032048617	0.948969814800891\\
26.5111051005036	0.948987694749914\\
26.5329069961455	0.949005578161213\\
26.5547088917873	0.949023465733176\\
26.5765107874292	0.949041358154175\\
26.598312683071	0.949059256102616\\
26.6201145787129	0.949077160246989\\
26.6419164743548	0.94909507124592\\
26.6637183699966	0.949112989748221\\
26.6855202656385	0.94913091639294\\
26.7073221612803	0.949148851809417\\
26.7291240569222	0.94916679661733\\
26.7509259525641	0.949184751426755\\
26.7727278482059	0.94920271683821\\
26.7945297438478	0.949220693442716\\
26.8163316394897	0.949238681821848\\
26.8381335351315	0.949256682547786\\
26.8599354307734	0.949274696183371\\
26.8817373264152	0.949292723282164\\
26.9035392220571	0.949310764388492\\
26.925341117699	0.949328820037511\\
26.9471430133408	0.949346890755254\\
26.9689449089827	0.949364977058696\\
26.9907468046245	0.949383079455799\\
27.0125487002664	0.949401198445577\\
27.0343505959083	0.949419334518147\\
27.0561524915501	0.949437488154787\\
27.077954387192	0.949455659827996\\
27.0997562828339	0.949473850001544\\
27.1215581784757	0.949492059130536\\
27.1433600741176	0.949510287661465\\
27.1651619697594	0.949528536032273\\
27.1869638654013	0.949546804672406\\
27.2087657610432	0.949565094002872\\
27.230567656685	0.949583404436301\\
27.2523695523269	0.949601736377004\\
27.2741714479687	0.949620090221026\\
27.2959733436106	0.949638466356211\\
27.3177752392525	0.949656865162258\\
27.3395771348943	0.949675287010779\\
27.3613790305362	0.949693732265358\\
27.3831809261781	0.949712201281612\\
27.4049828218199	0.949730694407248\\
27.4267847174618	0.949749211982125\\
27.4485866131036	0.94976775433831\\
27.4703885087455	0.94978632180014\\
27.4921904043874	0.949804914684279\\
27.5139923000292	0.94982353329978\\
27.5357941956711	0.949842177948145\\
27.5575960913129	0.949860848923382\\
27.5793979869548	0.949879546512065\\
27.6011998825967	0.949898270993399\\
27.6230017782385	0.949917022639271\\
27.6448036738804	0.949935801714319\\
27.6666055695223	0.949954608475984\\
27.6884074651641	0.949973443174576\\
27.710209360806	0.949992306053331\\
27.7320112564478	0.95001119734847\\
27.7538131520897	0.950030117289262\\
27.7756150477316	0.950049066098081\\
27.7974169433734	0.950068043990468\\
27.8192188390153	0.950087051175188\\
27.8410207346571	0.950106087854293\\
27.862822630299	0.950125154223182\\
27.8846245259409	0.950144250470658\\
27.9064264215827	0.950163376778989\\
27.9282283172246	0.950182533323968\\
27.9500302128665	0.950201720274974\\
27.9718321085083	0.950220937795026\\
27.9936340041502	0.950240186040852\\
28.015435899792	0.95025946516294\\
28.0372377954339	0.9502787753056\\
28.0590396910758	0.950298116607025\\
28.0808415867176	0.950317489199348\\
28.1026434823595	0.950336893208704\\
28.1244453780013	0.950356328755285\\
28.1462472736432	0.950375795953402\\
28.1680491692851	0.950395294911543\\
28.1898510649269	0.950414825732431\\
28.2116529605688	0.950434388513084\\
28.2334548562107	0.950453983344871\\
28.2552567518525	0.950473610313575\\
28.2770586474944	0.950493269499445\\
28.2988605431362	0.950512960977261\\
28.3206624387781	0.950532684816384\\
28.34246433442	0.950552441080823\\
28.3642662300618	0.950572229829285\\
28.3860681257037	0.950592051115235\\
28.4078700213455	0.950611904986955\\
28.4296719169874	0.950631791487602\\
28.4514738126293	0.950651710655258\\
28.4732757082711	0.950671662522998\\
28.495077603913	0.950691647118936\\
28.5168794995549	0.950711664466289\\
28.5386813951967	0.95073171458343\\
28.5604832908386	0.950751797483945\\
28.5822851864804	0.950771913176689\\
28.6040870821223	0.950792061665843\\
28.6258889777642	0.950812242950965\\
28.647690873406	0.950832457027054\\
28.6694927690479	0.950852703884596\\
28.6912946646897	0.950872983509625\\
28.7130965603316	0.950893295883777\\
28.7348984559735	0.950913640984342\\
28.7567003516153	0.950934018784322\\
28.7785022472572	0.950954429252482\\
28.8003041428991	0.950974872353407\\
28.8221060385409	0.950995348047553\\
28.8439079341828	0.951015856291303\\
28.8657098298246	0.951036397037019\\
28.8875117254665	0.951056970233095\\
28.9093136211084	0.95107757582401\\
28.9311155167502	0.951098213750383\\
28.9529174123921	0.951118883949022\\
28.9747193080339	0.951139586352976\\
28.9965212036758	0.951160320891592\\
29.0183230993177	0.951181087490562\\
29.0401249949595	0.951201886071975\\
29.0619268906014	0.95122271655437\\
29.0837287862433	0.951243578852785\\
29.1055306818851	0.95126447287881\\
29.127332577527	0.951285398540636\\
29.1491344731688	0.951306355743105\\
29.1709363688107	0.95132734438776\\
29.1927382644526	0.951348364372898\\
29.2145401600944	0.951369415593612\\
29.2363420557363	0.951390497941849\\
29.2581439513781	0.951411611306453\\
29.27994584702	0.951432755573215\\
29.3017477426619	0.951453930624922\\
29.3235496383037	0.951475136341406\\
29.3453515339456	0.951496372599589\\
29.3671534295875	0.951517639273534\\
29.3889553252293	0.951538936234488\\
29.4107572208712	0.951560263350935\\
29.432559116513	0.951581620488638\\
29.4543610121549	0.951603007510686\\
29.4761629077968	0.951624424277541\\
29.4979648034386	0.951645870647085\\
29.5197666990805	0.951667346474664\\
29.5415685947223	0.951688851613134\\
29.5633704903642	0.951710385912906\\
29.5851723860061	0.951731949221989\\
29.6069742816479	0.95175354138604\\
29.6287761772898	0.9517751622484\\
29.6505780729317	0.951796811650145\\
29.6723799685735	0.951818489430125\\
29.6941818642154	0.951840195425011\\
29.7159837598572	0.951861929469335\\
29.7377856554991	0.951883691395534\\
29.759587551141	0.951905481033991\\
29.7813894467828	0.951927298213081\\
29.8031913424247	0.951949142759209\\
29.8249932380665	0.951971014496852\\
29.8467951337084	0.951992913248602\\
29.8685970293503	0.952014838835206\\
29.8903989249921	0.952036791075606\\
29.912200820634	0.952058769786979\\
29.9340027162759	0.952080774784781\\
29.9558046119177	0.952102805882781\\
29.9776065075596	0.952124862893103\\
29.9994084032014	0.952146945626268\\
30.0212102988433	0.952169053891228\\
30.0430121944852	0.952191187495407\\
30.064814090127	0.952213346244739\\
30.0866159857689	0.952235529943705\\
30.1084178814107	0.952257738395373\\
30.1302197770526	0.952279971401434\\
30.1520216726945	0.952302228762236\\
30.1738235683363	0.952324510276826\\
30.1956254639782	0.952346815742984\\
30.2174273596201	0.952369144957258\\
30.2392292552619	0.952391497715\\
30.2610311509038	0.952413873810405\\
30.2828330465456	0.952436273036541\\
30.3046349421875	0.952458695185387\\
30.3264368378294	0.952481140047868\\
30.3482387334712	0.952503607413888\\
30.3700406291131	0.952526097072363\\
30.3918425247549	0.952548608811257\\
30.4136444203968	0.952571142417614\\
30.4354463160387	0.952593697677593\\
30.4572482116805	0.952616274376496\\
30.4790501073224	0.952638872298805\\
30.5008520029643	0.952661491228215\\
30.5226538986061	0.95268413094766\\
30.544455794248	0.952706791239351\\
30.5662576898898	0.952729471884803\\
30.5880595855317	0.952752172664867\\
30.6098614811736	0.952774893359764\\
30.6316633768154	0.952797633749108\\
30.6534652724573	0.952820393611944\\
30.6752671680991	0.952843172726773\\
30.697069063741	0.952865970871583\\
30.7188709593829	0.952888787823877\\
30.7406728550247	0.952911623360704\\
30.7624747506666	0.952934477258686\\
30.7842766463085	0.952957349294046\\
30.8060785419503	0.952980239242637\\
30.8278804375922	0.953003146879968\\
30.849682333234	0.953026071981236\\
30.8714842288759	0.953049014321347\\
30.8932861245178	0.953071973674945\\
30.9150880201596	0.953094949816441\\
30.9368899158015	0.953117942520037\\
30.9586918114433	0.953140951559751\\
30.9804937070852	0.953163976709445\\
31.0022956027271	0.953187017742849\\
31.0240974983689	0.953210074433585\\
31.0458993940108	0.953233146555196\\
31.0677012896527	0.953256233881164\\
31.0895031852945	0.953279336184941\\
31.1113050809364	0.953302453239967\\
31.1331069765782	0.953325584819699\\
31.1549088722201	0.953348730697629\\
31.176710767862	0.953371890647311\\
31.1985126635038	0.953395064442384\\
31.2203145591457	0.95341825185659\\
31.2421164547875	0.953441452663802\\
31.2639183504294	0.953464666638043\\
31.2857202460713	0.953487893553506\\
31.3075221417131	0.95351113318458\\
31.329324037355	0.953534385305866\\
31.3511259329969	0.953557649692204\\
31.3729278286387	0.953580926118685\\
31.3947297242806	0.953604214360682\\
31.4165316199224	0.953627514193859\\
31.4383335155643	0.953650825394201\\
31.4601354112062	0.953674147738025\\
31.481937306848	0.953697481002004\\
31.5037392024899	0.953720824963187\\
31.5255410981317	0.953744179399012\\
31.5473429937736	0.953767544087332\\
31.5691448894155	0.953790918806426\\
31.5909467850573	0.953814303335024\\
31.6127486806992	0.953837697452316\\
31.6345505763411	0.953861100937979\\
31.6563524719829	0.953884513572188\\
31.6781543676248	0.953907935135633\\
31.6999562632666	0.95393136540954\\
31.7217581589085	0.953954804175682\\
31.7435600545504	0.953978251216399\\
31.7653619501922	0.954001706314613\\
31.7871638458341	0.954025169253843\\
31.8089657414759	0.954048639818219\\
31.8307676371178	0.954072117792502\\
31.8525695327597	0.954095602962095\\
31.8743714284015	0.954119095113056\\
31.8961733240434	0.954142594032118\\
31.9179752196852	0.9541660995067\\
31.9397771153271	0.954189611324918\\
31.961579010969	0.954213129275607\\
31.9833809066108	0.954236653148324\\
32.0051828022527	0.954260182733371\\
32.0269846978946	0.954283717821801\\
32.0487865935364	0.954307258205436\\
32.0705884891783	0.954330803676874\\
32.0923903848201	0.954354354029506\\
32.114192280462	0.954377909057528\\
32.1359941761039	0.954401468555949\\
32.1577960717457	0.954425032320605\\
32.1795979673876	0.954448600148174\\
32.2013998630295	0.954472171836179\\
32.2232017586713	0.954495747183008\\
32.2450036543132	0.954519325987917\\
32.266805549955	0.954542908051047\\
32.2886074455969	0.954566493173429\\
32.3104093412388	0.954590081156999\\
32.3322112368806	0.954613671804604\\
32.3540131325225	0.954637264920013\\
32.3758150281643	0.954660860307928\\
32.3976169238062	0.954684457773991\\
32.4194188194481	0.954708057124796\\
32.4412207150899	0.954731658167893\\
32.4630226107318	0.954755260711803\\
32.4848245063736	0.95477886456602\\
32.5066264020155	0.954802469541026\\
32.5284282976574	0.954826075448293\\
32.5502301932992	0.954849682100294\\
32.5720320889411	0.954873289310511\\
32.593833984583	0.954896896893442\\
32.6156358802248	0.954920504664608\\
32.6374377758667	0.954944112440557\\
32.6592396715085	0.954967720038879\\
32.6810415671504	0.954991327278204\\
32.7028434627923	0.955014933978215\\
32.7246453584341	0.955038539959649\\
32.746447254076	0.95506214504431\\
32.7682491497179	0.955085749055067\\
32.7900510453597	0.955109351815866\\
32.8118529410016	0.955132953151733\\
32.8336548366434	0.955156552888779\\
32.8554567322853	0.955180150854208\\
32.8772586279272	0.955203746876319\\
32.899060523569	0.955227340784513\\
32.9208624192109	0.955250932409298\\
32.9426643148527	0.955274521582289\\
32.9644662104946	0.955298108136221\\
32.9862681061365	0.955321691904946\\
33.0080700017783	0.95534527272344\\
33.0298718974202	0.955368850427806\\
33.0516737930621	0.955392424855279\\
33.0734756887039	0.955415995844228\\
33.0952775843458	0.955439563234163\\
33.1170794799876	0.955463126865733\\
33.1388813756295	0.955486686580733\\
33.1606832712714	0.955510242222106\\
33.1824851669132	0.955533793633945\\
33.2042870625551	0.955557340661499\\
33.2260889581969	0.95558088315117\\
33.2478908538388	0.95560442095052\\
33.2696927494807	0.955627953908271\\
33.2914946451225	0.95565148187431\\
33.3132965407644	0.955675004699687\\
33.3350984364062	0.955698522236618\\
33.3569003320481	0.955722034338489\\
33.37870222769	0.955745540859855\\
33.4005041233318	0.955769041656443\\
33.4223060189737	0.955792536585153\\
33.4441079146156	0.955816025504057\\
33.4659098102574	0.955839508272404\\
33.4877117058993	0.955862984750617\\
33.5095136015411	0.955886454800297\\
33.531315497183	0.955909918284221\\
33.5531173928249	0.955933375066344\\
33.5749192884667	0.955956825011798\\
33.5967211841086	0.955980267986896\\
33.6185230797504	0.956003703859126\\
33.6403249753923	0.956027132497155\\
33.6621268710342	0.956050553770831\\
33.683928766676	0.956073967551178\\
33.7057306623179	0.956097373710397\\
33.7275325579598	0.956120772121867\\
33.7493344536016	0.956144162660143\\
33.7711363492435	0.956167545200958\\
33.7929382448853	0.956190919621219\\
33.8147401405272	0.956214285799006\\
33.8365420361691	0.956237643613573\\
33.8583439318109	0.956260992945348\\
33.8801458274528	0.956284333675927\\
33.9019477230947	0.956307665688077\\
33.9237496187365	0.956330988865733\\
33.9455515143784	0.956354303093997\\
33.9673534100202	0.956377608259136\\
33.9891553056621	0.956400904248578\\
34.010957201304	0.956424190950915\\
34.0327590969458	0.956447468255897\\
34.0545609925877	0.956470736054433\\
34.0763628882295	0.956493994238585\\
34.0981647838714	0.956517242701569\\
34.1199666795133	0.956540481337754\\
34.1417685751551	0.956563710042653\\
34.163570470797	0.956586928712929\\
34.1853723664388	0.956610137246387\\
34.2071742620807	0.956633335541971\\
34.2289761577226	0.956656523499765\\
34.2507780533644	0.956679701020987\\
34.2725799490063	0.956702868007989\\
34.2943818446482	0.956726024364251\\
34.31618374029	0.956749169994377\\
34.3379856359319	0.956772304804099\\
34.3597875315737	0.956795428700266\\
34.3815894272156	0.956818541590841\\
34.4033913228575	0.956841643384906\\
34.4251932184993	0.956864733992648\\
34.4469951141412	0.956887813325363\\
34.468797009783	0.956910881295448\\
34.4905989054249	0.956933937816399\\
34.5124008010668	0.95695698280281\\
34.5342026967086	0.956980016170362\\
34.5560045923505	0.957003037835828\\
34.5778064879924	0.957026047717062\\
34.5996083836342	0.957049045732999\\
34.6214102792761	0.957072031803649\\
34.6432121749179	0.957095005850095\\
34.6650140705598	0.957117967794486\\
34.6868159662017	0.957140917560034\\
34.7086178618435	0.957163855071013\\
34.7304197574854	0.957186780252747\\
34.7522216531273	0.957209693031614\\
34.7740235487691	0.957232593335037\\
34.795825444411	0.957255481091478\\
34.8176273400528	0.957278356230439\\
34.8394292356947	0.957301218682452\\
34.8612311313366	0.957324068379078\\
34.8830330269784	0.957346905252899\\
34.9048349226203	0.957369729237517\\
34.9266368182621	0.957392540267545\\
34.948438713904	0.957415338278607\\
34.9702406095459	0.957438123207329\\
34.9920425051877	0.957460894991335\\
35.0138444008296	0.957483653569244\\
35.0356462964714	0.957506398880663\\
35.0574481921133	0.957529130866183\\
35.0792500877552	0.957551849467372\\
35.101051983397	0.957574554626774\\
35.1228538790389	0.957597246287897\\
35.1446557746808	0.957619924395216\\
35.1664576703226	0.957642588894163\\
35.1882595659645	0.957665239731119\\
35.2100614616063	0.957687876853416\\
35.2318633572482	0.957710500209326\\
35.2536652528901	0.957733109748059\\
35.2754671485319	0.957755705419752\\
35.2972690441738	0.957778287175472\\
35.3190709398156	0.957800854967203\\
35.3408728354575	0.957823408747845\\
35.3626747310994	0.957845948471208\\
35.3844766267412	0.957868474092002\\
35.4062785223831	0.957890985565838\\
35.428080418025	0.957913482849217\\
35.4498823136668	0.95793596589953\\
35.4716842093087	0.957958434675045\\
35.4934861049505	0.957980889134908\\
35.5152880005924	0.958003329239133\\
35.5370898962343	0.958025754948599\\
35.5588917918761	0.958048166225043\\
35.580693687518	0.958070563031054\\
35.6024955831598	0.958092945330068\\
35.6242974788017	0.95811531308636\\
35.6460993744436	0.958137666265044\\
35.6679012700854	0.958160004832059\\
35.6897031657273	0.958182328754171\\
35.7115050613692	0.95820463799896\\
35.733306957011	0.95822693253482\\
35.7551088526529	0.95824921233095\\
35.7769107482947	0.958271477357348\\
35.7987126439366	0.958293727584809\\
35.8205145395785	0.958315962984912\\
35.8423164352203	0.958338183530022\\
35.8641183308622	0.958360389193275\\
35.885920226504	0.958382579948584\\
35.9077221221459	0.958404755770621\\
35.9295240177878	0.958426916634818\\
35.9513259134296	0.958449062517361\\
35.9731278090715	0.958471193395178\\
35.9949297047133	0.958493309245943\\
36.0167316003552	0.958515410048061\\
36.0385334959971	0.958537495780666\\
36.0603353916389	0.958559566423614\\
36.0821372872808	0.95858162195748\\
36.1039391829227	0.958603662363546\\
36.1257410785645	0.958625687623801\\
36.1475429742064	0.958647697720932\\
36.1693448698482	0.958669692638318\\
36.1911467654901	0.958691672360025\\
36.212948661132	0.9587136368708\\
36.2347505567738	0.958735586156062\\
36.2565524524157	0.958757520201903\\
36.2783543480576	0.958779438995073\\
36.3001562436994	0.958801342522983\\
36.3219581393413	0.95882323077369\\
36.3437600349831	0.958845103735901\\
36.365561930625	0.958866961398956\\
36.3873638262669	0.958888803752832\\
36.4091657219087	0.958910630788132\\
36.4309676175506	0.958932442496078\\
36.4527695131924	0.958954238868509\\
36.4745714088343	0.958976019897871\\
36.4963733044762	0.958997785577216\\
36.518175200118	0.959019535900191\\
36.5399770957599	0.959041270861034\\
36.5617789914018	0.95906299045457\\
36.5835808870436	0.959084694676202\\
36.6053827826855	0.959106383521907\\
36.6271846783273	0.95912805698823\\
36.6489865739692	0.959149715072279\\
36.6707884696111	0.959171357771715\\
36.6925903652529	0.959192985084752\\
36.7143922608948	0.959214597010148\\
36.7361941565366	0.959236193547198\\
36.7579960521785	0.959257774695733\\
36.7797979478204	0.959279340456108\\
36.8015998434622	0.9593008908292\\
36.8234017391041	0.959322425816403\\
36.8452036347459	0.959343945419619\\
36.8670055303878	0.959365449641256\\
36.8888074260297	0.95938693848422\\
36.9106093216715	0.959408411951908\\
36.9324112173134	0.959429870048206\\
36.9542131129553	0.959451312777481\\
36.9760150085971	0.959472740144577\\
36.997816904239	0.959494152154805\\
37.0196187998808	0.959515548813946\\
37.0414206955227	0.959536930128234\\
37.0632225911646	0.959558296104361\\
37.0850244868064	0.959579646749465\\
37.1068263824483	0.959600982071128\\
37.1286282780902	0.959622302077368\\
37.150430173732	0.959643606776634\\
37.1722320693739	0.959664896177803\\
37.1940339650157	0.959686170290171\\
37.2158358606576	0.959707429123451\\
37.2376377562995	0.959728672687764\\
37.2594396519413	0.959749900993637\\
37.2812415475832	0.959771114051995\\
37.303043443225	0.959792311874158\\
37.3248453388669	0.959813494471836\\
37.3466472345088	0.959834661857118\\
37.3684491301506	0.959855814042476\\
37.3902510257925	0.959876951040752\\
37.4120529214344	0.959898072865156\\
37.4338548170762	0.959919179529262\\
37.4556567127181	0.959940271047\\
37.4774586083599	0.959961347432652\\
37.4992605040018	0.959982408700848\\
37.5210623996437	0.960003454866561\\
37.5428642952855	0.960024485945099\\
37.5646661909274	0.960045501952103\\
37.5864680865692	0.960066502903542\\
37.6082699822111	0.960087488815705\\
37.630071877853	0.960108459705199\\
37.6518737734948	0.960129415588944\\
37.6736756691367	0.960150356484167\\
37.6954775647785	0.960171282408396\\
37.7172794604204	0.960192193379458\\
37.7390813560623	0.960213089415473\\
37.7608832517041	0.960233970534849\\
37.782685147346	0.960254836756274\\
37.8044870429879	0.96027568809872\\
37.8262889386297	0.960296524581429\\
37.8480908342716	0.960317346223912\\
37.8698927299134	0.960338153045947\\
37.8916946255553	0.96035894506757\\
37.9134965211972	0.960379722309073\\
37.935298416839	0.960400484790998\\
37.9571003124809	0.960421232534133\\
37.9789022081227	0.96044196555951\\
38.0007041037646	0.960462683888395\\
38.0225059994065	0.960483387542288\\
38.0443078950483	0.960504076542919\\
38.0661097906902	0.960524750912239\\
38.0879116863321	0.960545410672421\\
38.1097135819739	0.960566055845852\\
38.1315154776158	0.96058668645513\\
38.1533173732576	0.960607302523059\\
38.1751192688995	0.960627904072649\\
38.1969211645414	0.960648491127103\\
38.2187230601832	0.960669063709822\\
38.2405249558251	0.960689621844395\\
38.262326851467	0.960710165554596\\
38.2841287471088	0.960730694864384\\
38.3059306427507	0.960751209797892\\
38.3277325383925	0.960771710379429\\
38.3495344340344	0.960792196633471\\
38.3713363296763	0.960812668584663\\
38.3931382253181	0.960833126257808\\
38.41494012096	0.960853569677869\\
38.4367420166018	0.960873998869964\\
38.4585439122437	0.960894413859357\\
38.4803458078856	0.960914814671461\\
38.5021477035274	0.960935201331831\\
38.5239495991693	0.960955573866161\\
38.5457514948111	0.960975932300279\\
38.567553390453	0.960996276660144\\
38.5893552860949	0.961016606971843\\
38.6111571817367	0.961036923261587\\
38.6329590773786	0.961057225555707\\
38.6547609730205	0.96107751388065\\
38.6765628686623	0.961097788262978\\
38.6983647643042	0.961118048729361\\
38.720166659946	0.961138295306574\\
38.7419685555879	0.961158528021498\\
38.7637704512298	0.961178746901111\\
38.7855723468716	0.961198951972487\\
38.8073742425135	0.961219143262791\\
38.8291761381553	0.961239320799281\\
38.8509780337972	0.961259484609297\\
38.8727799294391	0.961279634720264\\
38.8945818250809	0.961299771159685\\
38.9163837207228	0.961319893955138\\
38.9381856163647	0.961340003134277\\
38.9599875120065	0.961360098724822\\
38.9817894076484	0.961380180754563\\
39.0035913032902	0.961400249251352\\
39.0253931989321	0.961420304243101\\
39.047195094574	0.961440345757779\\
39.0689969902158	0.961460373823411\\
39.0907988858577	0.961480388468072\\
39.1126007814996	0.961500389719887\\
39.1344026771414	0.961520377607025\\
39.1562045727833	0.961540352157698\\
39.1780064684251	0.961560313400157\\
39.199808364067	0.961580261362693\\
39.2216102597089	0.961600196073628\\
39.2434121553507	0.961620117561317\\
39.2652140509926	0.961640025854143\\
39.2870159466344	0.961659920980515\\
39.3088178422763	0.961679802968865\\
39.3306197379182	0.961699671847646\\
39.35242163356	0.961719527645331\\
39.3742235292019	0.961739370390404\\
39.3960254248437	0.961759200111365\\
39.4178273204856	0.961779016836724\\
39.4396292161275	0.961798820594998\\
39.4614311117693	0.96181861141471\\
39.4832330074112	0.961838389324385\\
39.5050349030531	0.96185815435255\\
39.5268367986949	0.961877906527727\\
39.5486386943368	0.961897645878437\\
39.5704405899786	0.961917372433193\\
39.5922424856205	0.961937086220499\\
39.6140443812624	0.961956787268848\\
39.6358462769042	0.961976475606719\\
39.6576481725461	0.961996151262576\\
39.6794500681879	0.962015814264864\\
39.7012519638298	0.96203546464201\\
39.7230538594717	0.962055102422416\\
39.7448557551135	0.962074727634461\\
39.7666576507554	0.962094340306499\\
39.7884595463973	0.962113940466853\\
39.8102614420391	0.962133528143817\\
39.832063337681	0.962153103365653\\
39.8538652333228	0.962172666160586\\
39.8756671289647	0.962192216556807\\
39.8974690246066	0.962211754582468\\
39.9192709202484	0.962231280265679\\
39.9410728158903	0.962250793634512\\
39.9628747115321	0.96227029471699\\
39.984676607174	0.962289783541093\\
40.0064785028159	0.962309260134754\\
40.0282803984577	0.962328724525854\\
40.0500822940996	0.962348176742226\\
40.0718841897415	0.962367616811646\\
40.0936860853833	0.962387044761841\\
40.1154879810252	0.962406460620476\\
40.137289876667	0.962425864415163\\
40.1590917723089	0.962445256173451\\
40.1808936679508	0.962464635922831\\
40.2026955635926	0.962484003690728\\
40.2244974592345	0.962503359504506\\
40.2462993548763	0.962522703391461\\
40.2681012505182	0.962542035378824\\
40.2899031461601	0.962561355493755\\
40.3117050418019	0.962580663763345\\
40.3335069374438	0.962599960214614\\
40.3553088330857	0.962619244874509\\
40.3771107287275	0.962638517769901\\
40.3989126243694	0.962657778927588\\
40.4207145200112	0.962677028374289\\
40.4425164156531	0.962696266136646\\
40.464318311295	0.962715492241221\\
40.4861202069368	0.962734706714495\\
40.5079221025787	0.962753909582866\\
40.5297239982205	0.962773100872652\\
40.5515258938624	0.962792280610082\\
40.5733277895043	0.962811448821304\\
40.5951296851461	0.962830605532376\\
40.616931580788	0.962849750769269\\
40.6387334764299	0.962868884557866\\
40.6605353720717	0.962888006923959\\
40.6823372677136	0.962907117893249\\
40.7041391633554	0.962926217491345\\
40.7259410589973	0.962945305743763\\
40.7477429546392	0.962964382675925\\
40.769544850281	0.962983448313157\\
40.7913467459229	0.963002502680689\\
40.8131486415647	0.963021545803656\\
40.8349505372066	0.963040577707092\\
40.8567524328485	0.963059598415935\\
40.8785543284903	0.963078607955022\\
40.9003562241322	0.963097606349089\\
40.9221581197741	0.963116593622772\\
40.9439600154159	0.963135569800604\\
40.9657619110578	0.963154534907015\\
40.9875638066996	0.963173488966332\\
41.0093657023415	0.963192432002776\\
41.0311675979834	0.963211364040465\\
41.0529694936252	0.963230285103409\\
41.0747713892671	0.963249195215513\\
41.0965732849089	0.963268094400574\\
41.1183751805508	0.963286982682281\\
41.1401770761927	0.963305860084215\\
41.1619789718345	0.963324726629847\\
41.1837808674764	0.963343582342538\\
41.2055827631183	0.963362427245541\\
41.2273846587601	0.963381261361994\\
41.249186554402	0.963400084714928\\
41.2709884500438	0.96341889732726\\
41.2927903456857	0.963437699221792\\
41.3145922413276	0.963456490421218\\
41.3363941369694	0.963475270948113\\
41.3581960326113	0.963494040824944\\
41.3799979282531	0.963512800074058\\
41.401799823895	0.963531548717691\\
41.4236017195369	0.963550286777962\\
41.4454036151787	0.963569014276874\\
41.4672055108206	0.963587731236317\\
41.4890074064625	0.96360643767806\\
41.5108093021043	0.963625133623759\\
41.5326111977462	0.963643819094951\\
41.554413093388	0.963662494113056\\
41.5762149890299	0.963681158699377\\
41.5980168846718	0.963699812875099\\
41.6198187803136	0.963718456661288\\
41.6416206759555	0.963737090078892\\
41.6634225715973	0.96375571314874\\
41.6852244672392	0.963774325891542\\
41.7070263628811	0.96379292832789\\
41.7288282585229	0.963811520478254\\
41.7506301541648	0.963830102362988\\
41.7724320498067	0.963848674002324\\
41.7942339454485	0.963867235416374\\
41.8160358410904	0.963885786625132\\
41.8378377367322	0.963904327648468\\
41.8596396323741	0.963922858506136\\
41.881441528016	0.963941379217768\\
41.9032434236578	0.963959889802874\\
41.9250453192997	0.963978390280847\\
41.9468472149415	0.963996880670955\\
41.9686491105834	0.96401536099235\\
41.9904510062253	0.964033831264059\\
42.0122529018671	0.964052291504991\\
42.034054797509	0.964070741733935\\
42.0558566931509	0.964089181969556\\
42.0776585887927	0.964107612230401\\
42.0994604844346	0.964126032534897\\
42.1212623800764	0.964144442901346\\
42.1430642757183	0.964162843347936\\
42.1648661713602	0.964181233892728\\
42.186668067002	0.964199614553666\\
42.2084699626439	0.964217985348575\\
42.2302718582857	0.964236346295155\\
42.2520737539276	0.96425469741099\\
42.2738756495695	0.964273038713543\\
42.2956775452113	0.964291370220156\\
42.3174794408532	0.964309691948052\\
42.3392813364951	0.964328003914333\\
42.3610832321369	0.964346306135984\\
42.3828851277788	0.96436459862987\\
42.4046870234206	0.964382881412734\\
42.4264889190625	0.964401154501203\\
42.4482908147044	0.964419417911784\\
42.4700927103462	0.964437671660867\\
42.4918946059881	0.964455915764721\\
42.5136965016299	0.964474150239498\\
42.5354983972718	0.964492375101234\\
42.5573002929137	0.964510590365844\\
42.5791021885555	0.964528796049128\\
42.6009040841974	0.964546992166768\\
42.6227059798393	0.964565178734329\\
42.6445078754811	0.964583355767259\\
42.666309771123	0.96460152328089\\
42.6881116667648	0.964619681290439\\
42.7099135624067	0.964637829811005\\
42.7317154580486	0.964655968857575\\
42.7535173536904	0.964674098445016\\
42.7753192493323	0.964692218588085\\
42.7971211449741	0.964710329301421\\
42.818923040616	0.96472843059955\\
42.8407249362579	0.964746522496886\\
42.8625268318997	0.964764605007728\\
42.8843287275416	0.96478267814626\\
42.9061306231835	0.964800741926556\\
42.9279325188253	0.964818796362578\\
42.9497344144672	0.964836841468173\\
42.971536310109	0.964854877257079\\
42.9933382057509	0.964872903742922\\
43.0151401013928	0.964890920939217\\
43.0369419970346	0.96490892885937\\
43.0587438926765	0.964926927516674\\
43.0805457883183	0.964944916924316\\
43.1023476839602	0.964962897095371\\
43.1241495796021	0.964980868042806\\
43.1459514752439	0.964998829779482\\
43.1677533708858	0.965016782318149\\
43.1895552665276	0.965034725671452\\
43.2113571621695	0.965052659851926\\
43.2331590578114	0.965070584872003\\
43.2549609534532	0.965088500744008\\
43.2767628490951	0.965106407480159\\
43.298564744737	0.965124305092569\\
43.3203666403788	0.965142193593249\\
43.3421685360207	0.965160072994103\\
43.3639704316625	0.965177943306932\\
43.3857723273044	0.965195804543436\\
43.4075742229463	0.96521365671521\\
43.4293761185881	0.965231499833746\\
43.45117801423	0.965249333910438\\
43.4729799098719	0.965267158956575\\
43.4947818055137	0.965284974983348\\
43.5165837011556	0.965302782001846\\
43.5383855967974	0.96532058002306\\
43.5601874924393	0.965338369057882\\
43.5819893880812	0.965356149117103\\
43.603791283723	0.965373920211419\\
43.6255931793649	0.965391682351426\\
43.6473950750067	0.965409435547625\\
43.6691969706486	0.965427179810419\\
43.6909988662905	0.965444915150117\\
43.7128007619323	0.965462641576931\\
43.7346026575742	0.965480359100978\\
43.7564045532161	0.965498067732283\\
43.7782064488579	0.965515767480775\\
43.8000083444998	0.96553345835629\\
43.8218102401416	0.965551140368574\\
43.8436121357835	0.965568813527278\\
43.8654140314254	0.965586477841963\\
43.8872159270672	0.965604133322098\\
43.9090178227091	0.965621779977064\\
43.9308197183509	0.96563941781615\\
43.9526216139928	0.965657046848557\\
43.9744235096347	0.965674667083397\\
43.9962254052765	0.965692278529694\\
44.0180273009184	0.965709881196384\\
44.0398291965602	0.965727475092318\\
44.0616310922021	0.965745060226259\\
44.083432987844	0.965762636606886\\
44.1052348834858	0.965780204242789\\
44.1270367791277	0.965797763142479\\
44.1488386747696	0.96581531331438\\
44.1706405704114	0.965832854766832\\
44.1924424660533	0.965850387508093\\
44.2142443616951	0.965867911546341\\
44.236046257337	0.965885426889669\\
44.2578481529789	0.965902933546091\\
44.2796500486207	0.965920431523541\\
44.3014519442626	0.965937920829871\\
44.3232538399045	0.965955401472857\\
44.3450557355463	0.965972873460194\\
44.3668576311882	0.965990336799499\\
44.38865952683	0.966007791498313\\
44.4104614224719	0.966025237564101\\
44.4322633181138	0.966042675004248\\
44.4540652137556	0.966060103826067\\
44.4758671093975	0.966077524036795\\
44.4976690050393	0.966094935643595\\
44.5194709006812	0.966112338653554\\
44.5412727963231	0.966129733073689\\
44.5630746919649	0.966147118910942\\
44.5848765876068	0.966164496172185\\
44.6066784832487	0.966181864864217\\
44.6284803788905	0.966199224993766\\
44.6502822745324	0.966216576567492\\
44.6720841701742	0.966233919591983\\
44.6938860658161	0.96625125407376\\
44.715687961458	0.966268580019273\\
44.7374898570998	0.966285897434908\\
44.7592917527417	0.966303206326981\\
44.7810936483835	0.966320506701741\\
44.8028955440254	0.966337798565374\\
44.8246974396673	0.966355081923997\\
44.8464993353091	0.966372356783665\\
44.868301230951	0.966389623150367\\
44.8901031265928	0.966406881030029\\
44.9119050222347	0.966424130428516\\
44.9337069178766	0.966441371351627\\
44.9555088135184	0.966458603805101\\
44.9773107091603	0.966475827794616\\
44.9991126048022	0.966493043325788\\
45.020914500444	0.966510250404176\\
45.0427163960859	0.966527449035275\\
45.0645182917277	0.966544639224526\\
45.0863201873696	0.966561820977308\\
45.1081220830115	0.966578994298943\\
45.1299239786533	0.966596159194697\\
45.1517258742952	0.96661331566978\\
45.173527769937	0.966630463729343\\
45.1953296655789	0.966647603378485\\
45.2171315612208	0.966664734622247\\
45.2389334568626	0.96668185746562\\
45.2607353525045	0.966698971913536\\
45.2825372481464	0.966716077970879\\
45.3043391437882	0.966733175642477\\
45.3261410394301	0.966750264933107\\
45.3479429350719	0.966767345847496\\
45.3697448307138	0.966784418390317\\
45.3915467263557	0.966801482566197\\
45.4133486219975	0.96681853837971\\
45.4351505176394	0.966835585835381\\
45.4569524132813	0.966852624937688\\
45.4787543089231	0.966869655691061\\
45.500556204565	0.966886678099882\\
45.5223581002068	0.966903692168484\\
45.5441599958487	0.966920697901157\\
45.5659618914906	0.966937695302144\\
45.5877637871324	0.966954684375641\\
45.6095656827743	0.966971665125802\\
45.6313675784161	0.966988637556736\\
45.653169474058	0.967005601672506\\
45.6749713696999	0.967022557477135\\
45.6967732653417	0.967039504974603\\
45.7185751609836	0.967056444168845\\
45.7403770566254	0.967073375063758\\
45.7621789522673	0.967090297663197\\
45.7839808479092	0.967107211970976\\
45.805782743551	0.967124117990869\\
45.8275846391929	0.96714101572661\\
45.8493865348348	0.967157905181897\\
45.8711884304766	0.967174786360386\\
45.8929903261185	0.967191659265697\\
45.9147922217603	0.967208523901412\\
45.9365941174022	0.967225380271078\\
45.9583960130441	0.967242228378203\\
45.9801979086859	0.967259068226261\\
46.0019998043278	0.967275899818689\\
46.0238016999696	0.967292723158891\\
46.0456035956115	0.967309538250236\\
46.0674054912534	0.967326345096059\\
46.0892073868952	0.967343143699661\\
46.1110092825371	0.967359934064312\\
46.132811178179	0.967376716193247\\
46.1546130738208	0.967393490089672\\
46.1764149694627	0.967410255756759\\
46.1982168651045	0.96742701319765\\
46.2200187607464	0.967443762415458\\
46.2418206563883	0.967460503413264\\
46.2636225520301	0.967477236194121\\
46.285424447672	0.967493960761051\\
46.3072263433139	0.967510677117048\\
46.3290282389557	0.96752738526508\\
46.3508301345976	0.967544085208086\\
46.3726320302394	0.967560776948976\\
46.3944339258813	0.967577460490635\\
46.4162358215232	0.967594135835922\\
46.438037717165	0.96761080298767\\
46.4598396128069	0.967627461948685\\
46.4816415084487	0.96764411272175\\
46.5034434040906	0.967660755309623\\
46.5252452997325	0.967677389715036\\
46.5470471953743	0.9676940159407\\
46.5688490910162	0.967710633989301\\
46.590650986658	0.967727243863503\\
46.6124528822999	0.967743845565947\\
46.6342547779418	0.967760439099252\\
46.6560566735836	0.967777024466018\\
46.6778585692255	0.967793601668819\\
46.6996604648674	0.967810170710212\\
46.7214623605092	0.967826731592732\\
46.7432642561511	0.967843284318895\\
46.7650661517929	0.967859828891197\\
46.7868680474348	0.967876365312115\\
46.8086699430767	0.967892893584107\\
46.8304718387185	0.967909413709613\\
46.8522737343604	0.967925925691054\\
46.8740756300022	0.967942429530836\\
46.8958775256441	0.967958925231345\\
46.917679421286	0.967975412794952\\
46.9394813169278	0.967991892224011\\
46.9612832125697	0.968008363520859\\
46.9830851082116	0.968024826687819\\
47.0048870038534	0.968041281727197\\
47.0266888994953	0.968057728641286\\
47.0484907951371	0.968074167432362\\
47.070292690779	0.96809059810269\\
47.0920945864209	0.968107020654518\\
47.1138964820627	0.968123435090083\\
47.1356983777046	0.968139841411606\\
47.1575002733464	0.968156239621298\\
47.1793021689883	0.968172629721356\\
47.2011040646302	0.968189011713965\\
47.222905960272	0.968205385601299\\
47.2447078559139	0.96822175138552\\
47.2665097515558	0.968238109068778\\
47.2883116471976	0.968254458653214\\
47.3101135428395	0.968270800140957\\
47.3319154384813	0.968287133534127\\
47.3537173341232	0.968303458834833\\
47.3755192297651	0.968319776045176\\
47.3973211254069	0.968336085167246\\
47.4191230210488	0.968352386203126\\
47.4409249166906	0.968368679154889\\
47.4627268123325	0.968384964024601\\
47.4845287079744	0.968401240814318\\
47.5063306036162	0.968417509526091\\
47.5281324992581	0.968433770161961\\
47.5499343949	0.968450022723964\\
47.5717362905418	0.968466267214128\\
47.5935381861837	0.968482503634476\\
47.6153400818255	0.968498731987022\\
47.6371419774674	0.968514952273777\\
47.6589438731093	0.968531164496744\\
47.6807457687511	0.968547368657923\\
47.702547664393	0.968563564759307\\
47.7243495600348	0.968579752802885\\
47.7461514556767	0.968595932790641\\
47.7679533513186	0.968612104724555\\
47.7897552469604	0.968628268606603\\
47.8115571426023	0.968644424438757\\
47.8333590382442	0.968660572222986\\
47.855160933886	0.968676711961254\\
47.8769628295279	0.968692843655525\\
47.8987647251697	0.968708967307758\\
47.9205666208116	0.968725082919908\\
47.9423685164535	0.968741190493933\\
47.9641704120953	0.968757290031783\\
47.9859723077372	0.968773381535409\\
48.007774203379	0.968789465006762\\
48.0295760990209	0.968805540447788\\
48.0513779946628	0.968821607860434\\
48.0731798903046	0.968837667246646\\
48.0949817859465	0.968853718608369\\
48.1167836815884	0.968869761947549\\
48.1385855772302	0.968885797266127\\
48.1603874728721	0.96890182456605\\
48.1821893685139	0.968917843849262\\
48.2039912641558	0.968933855117707\\
48.2257931597977	0.968949858373332\\
48.2475950554395	0.968965853618081\\
48.2693969510814	0.968981840853903\\
48.2911988467232	0.968997820082745\\
48.3130007423651	0.969013791306558\\
48.334802638007	0.969029754527292\\
48.3566045336488	0.969045709746901\\
48.3784064292907	0.96906165696734\\
48.4002083249325	0.969077596190565\\
48.4220102205744	0.969093527418536\\
48.4438121162163	0.969109450653215\\
48.4656140118581	0.969125365896567\\
48.4874159075	0.969141273150558\\
48.5092178031419	0.96915717241716\\
48.5310196987837	0.969173063698345\\
48.5528215944256	0.969188946996091\\
48.5746234900674	0.969204822312377\\
48.5964253857093	0.969220689649189\\
48.6182272813512	0.969236549008513\\
48.640029176993	0.969252400392341\\
48.6618310726349	0.96926824380267\\
48.6836329682768	0.969284079241499\\
48.7054348639186	0.969299906710833\\
48.7272367595605	0.969315726212681\\
48.7490386552023	0.969331537749058\\
48.7708405508442	0.969347341321981\\
48.7926424464861	0.969363136933475\\
48.8144443421279	0.969378924585568\\
48.8362462377698	0.969394704280294\\
48.8580481334116	0.969410476019694\\
48.8798500290535	0.969426239805811\\
48.9016519246954	0.969441995640697\\
48.9234538203372	0.969457743526409\\
48.9452557159791	0.969473483465007\\
48.967057611621	0.969489215458562\\
48.9888595072628	0.969504939509147\\
49.0106614029047	0.969520655618843\\
49.0324632985465	0.969536363789737\\
49.0542651941884	0.969552064023923\\
49.0760670898303	0.969567756323501\\
49.0978689854721	0.969583440690578\\
49.119670881114	0.969599117127268\\
49.1414727767558	0.96961478563569\\
49.1632746723977	0.969630446217973\\
49.1850765680396	0.969646098876252\\
49.2068784636814	0.969661743612667\\
49.2286803593233	0.969677380429369\\
49.2504822549651	0.969693009328513\\
49.272284150607	0.969708630312265\\
49.2940860462489	0.969724243382795\\
49.3158879418907	0.969739848542282\\
49.3376898375326	0.969755445792915\\
49.3594917331745	0.969771035136886\\
49.3812936288163	0.9697866165764\\
49.4030955244582	0.969802190113666\\
49.4248974201	0.969817755750904\\
49.4466993157419	0.96983331349034\\
49.4685012113838	0.969848863334209\\
49.4903031070256	0.969864405284755\\
49.5121050026675	0.969879939344229\\
49.5339068983094	0.96989546551489\\
49.5557087939512	0.969910983799009\\
49.5775106895931	0.969926494198861\\
49.5993125852349	0.969941996716732\\
49.6211144808768	0.969957491354916\\
49.6429163765187	0.969972978115717\\
49.6647182721605	0.969988457001446\\
49.6865201678024	0.970003928014424\\
49.7083220634442	0.97001939115698\\
49.7301239590861	0.970034846431452\\
49.751925854728	0.970050293840188\\
49.7737277503698	0.970065733385544\\
49.7955296460117	0.970081165069885\\
49.8173315416536	0.970096588895586\\
49.8391334372954	0.970112004865031\\
49.8609353329373	0.970127412980613\\
49.8827372285791	0.970142813244733\\
49.904539124221	0.970158205659803\\
49.9263410198629	0.970173590228243\\
49.9481429155047	0.970188966952485\\
49.9699448111466	0.970204335834966\\
49.9917467067884	0.970219696878136\\
50.0135486024303	0.970235050084454\\
50.0353504980722	0.970250395456386\\
50.057152393714	0.97026573299641\\
50.0789542893559	0.970281062707012\\
50.1007561849977	0.97029638459069\\
50.1225580806396	0.970311698649948\\
50.1443599762815	0.970327004887302\\
50.1661618719233	0.970342303305277\\
50.1879637675652	0.970357593906407\\
50.2097656632071	0.970372876693238\\
50.2315675588489	0.970388151668322\\
50.2533694544908	0.970403418834223\\
50.2751713501326	0.970418678193515\\
50.2969732457745	0.970433929748781\\
50.3187751414164	0.970449173502612\\
50.3405770370582	0.970464409457612\\
50.3623789327001	0.970479637616392\\
50.3841808283419	0.970494857981574\\
50.4059827239838	0.970510070555791\\
50.4277846196257	0.970525275341682\\
50.4495865152675	0.970540472341899\\
50.4713884109094	0.970555661559103\\
50.4931903065513	0.970570842995965\\
50.5149922021931	0.970586016655163\\
50.536794097835	0.970601182539389\\
50.5585959934768	0.970616340651342\\
50.5803978891187	0.970631490993731\\
50.6021997847606	0.970646633569275\\
50.6240016804024	0.970661768380704\\
50.6458035760443	0.970676895430755\\
50.6676054716862	0.970692014722177\\
50.689407367328	0.970707126257728\\
50.7112092629699	0.970722230040175\\
50.7330111586117	0.970737326072296\\
50.7548130542536	0.970752414356878\\
50.7766149498955	0.970767494896716\\
50.7984168455373	0.970782567694618\\
50.8202187411792	0.970797632753398\\
50.842020636821	0.970812690075883\\
50.8638225324629	0.970827739664906\\
50.8856244281048	0.970842781523313\\
50.9074263237466	0.970857815653958\\
50.9292282193885	0.970872842059704\\
50.9510301150303	0.970887860743423\\
50.9728320106722	0.970902871707999\\
50.9946339063141	0.970917874956323\\
51.0164358019559	0.970932870491297\\
51.0382376975978	0.970947858315831\\
51.0600395932397	0.970962838432845\\
51.0818414888815	0.970977810845268\\
51.1036433845234	0.97099277555604\\
51.1254452801652	0.971007732568108\\
51.1472471758071	0.971022681884429\\
51.169049071449	0.971037623507971\\
51.1908509670908	0.971052557441708\\
51.2126528627327	0.971067483688625\\
51.2344547583745	0.971082402251716\\
51.2562566540164	0.971097313133985\\
51.2780585496583	0.971112216338443\\
51.2998604453001	0.971127111868111\\
51.321662340942	0.97114199972602\\
51.3434642365839	0.971156879915208\\
51.3652661322257	0.971171752438724\\
51.3870680278676	0.971186617299623\\
51.4088699235094	0.971201474500973\\
51.4306718191513	0.971216324045847\\
51.4524737147932	0.971231165937328\\
51.474275610435	0.971246000178509\\
51.4960775060769	0.97126082677249\\
51.5178794017188	0.97127564572238\\
51.5396812973606	0.971290457031297\\
51.5614831930025	0.971305260702368\\
51.5832850886443	0.971320056738727\\
51.6050869842862	0.971334845143518\\
51.6268888799281	0.971349625919893\\
51.6486907755699	0.971364399071012\\
51.6704926712118	0.971379164600043\\
51.6922945668536	0.971393922510164\\
51.7140964624955	0.971408672804559\\
51.7358983581374	0.971423415486421\\
51.7577002537792	0.971438150558953\\
51.7795021494211	0.971452878025364\\
51.8013040450629	0.97146759788887\\
51.8231059407048	0.971482310152699\\
51.8449078363467	0.971497014820083\\
51.8667097319885	0.971511711894263\\
51.8885116276304	0.97152640137849\\
51.9103135232723	0.97154108327602\\
51.9321154189141	0.971555757590119\\
51.953917314556	0.971570424324058\\
51.9757192101978	0.971585083481118\\
51.9975211058397	0.971599735064588\\
52.0193230014816	0.971614379077761\\
52.0411248971234	0.971629015523943\\
52.0629267927653	0.971643644406441\\
52.0847286884071	0.971658265728576\\
52.106530584049	0.971672879493671\\
52.1283324796909	0.971687485705059\\
52.1501343753327	0.971702084366079\\
52.1719362709746	0.97171667548008\\
52.1937381666165	0.971731259050414\\
52.2155400622583	0.971745835080442\\
52.2373419579002	0.971760403573534\\
52.259143853542	0.971774964533063\\
52.2809457491839	0.971789517962412\\
52.3027476448258	0.97180406386497\\
52.3245495404676	0.971818602244131\\
52.3463514361095	0.9718331331033\\
52.3681533317513	0.971847656445884\\
52.3899552273932	0.971862172275299\\
52.4117571230351	0.971876680594967\\
52.4335590186769	0.971891181408318\\
52.4553609143188	0.971905674718787\\
52.4771628099607	0.971920160529814\\
52.4989647056025	0.971934638844849\\
52.5207666012444	0.971949109667345\\
52.5425684968862	0.971963573000763\\
52.5643703925281	0.97197802884857\\
52.58617228817	0.971992477214237\\
52.6079741838118	0.972006918101245\\
52.6297760794537	0.972021351513079\\
52.6515779750955	0.972035777453228\\
52.6733798707374	0.972050195925189\\
52.6951817663793	0.972064606932466\\
52.7169836620211	0.972079010478566\\
52.738785557663	0.972093406567003\\
52.7605874533049	0.972107795201297\\
52.7823893489467	0.972122176384973\\
52.8041912445886	0.972136550121562\\
52.8259931402304	0.9721509164146\\
52.8477950358723	0.972165275267629\\
52.8695969315142	0.972179626684195\\
52.891398827156	0.972193970667852\\
52.9132007227979	0.972208307222156\\
52.9350026184397	0.972222636350671\\
52.9568045140816	0.972236958056963\\
52.9786064097235	0.972251272344607\\
53.0004083053653	0.97226557921718\\
53.0222102010072	0.972279878678264\\
53.0440120966491	0.972294170731448\\
53.0658139922909	0.972308455380324\\
53.0876158879328	0.97232273262849\\
53.1094177835746	0.972337002479547\\
53.1312196792165	0.972351264937102\\
53.1530215748584	0.972365520004766\\
53.1748234705002	0.972379767686156\\
53.1966253661421	0.972394007984891\\
53.2184272617839	0.972408240904596\\
53.2402291574258	0.972422466448901\\
53.2620310530677	0.972436684621438\\
53.2838329487095	0.972450895425846\\
53.3056348443514	0.972465098865765\\
53.3274367399933	0.972479294944842\\
53.3492386356351	0.972493483666727\\
53.371040531277	0.972507665035074\\
53.3928424269188	0.97252183905354\\
53.4146443225607	0.972536005725788\\
53.4364462182026	0.972550165055484\\
53.4582481138444	0.972564317046295\\
53.4800500094863	0.972578461701897\\
53.5018519051281	0.972592599025965\\
53.52365380077	0.97260672902218\\
53.5454556964119	0.972620851694225\\
53.5672575920537	0.972634967045789\\
53.5890594876956	0.972649075080562\\
53.6108613833375	0.972663175802239\\
53.6326632789793	0.972677269214516\\
53.6544651746212	0.972691355321096\\
53.676267070263	0.972705434125681\\
53.6980689659049	0.972719505631979\\
53.7198708615468	0.9727335698437\\
53.7416727571886	0.972747626764558\\
53.7634746528305	0.972761676398268\\
53.7852765484723	0.972775718748551\\
53.8070784441142	0.972789753819128\\
53.8288803397561	0.972803781613724\\
53.8506822353979	0.972817802136067\\
53.8724841310398	0.972831815389887\\
53.8942860266817	0.972845821378918\\
53.9160879223235	0.972859820106894\\
53.9378898179654	0.972873811577554\\
53.9596917136072	0.972887795794639\\
53.9814936092491	0.972901772761891\\
54.003295504891	0.972915742483057\\
54.0250974005328	0.972929704961884\\
54.0468992961747	0.972943660202122\\
54.0687011918165	0.972957608207524\\
54.0905030874584	0.972971548981843\\
54.1123049831003	0.972985482528838\\
54.1341068787421	0.972999408852265\\
54.155908774384	0.973013327955887\\
54.1777106700259	0.973027239843465\\
54.1995125656677	0.973041144518765\\
54.2213144613096	0.973055041985552\\
54.2431163569514	0.973068932247594\\
54.2649182525933	0.973082815308663\\
54.2867201482352	0.97309669117253\\
54.308522043877	0.973110559842967\\
54.3303239395189	0.97312442132375\\
54.3521258351607	0.973138275618656\\
54.3739277308026	0.973152122731462\\
54.3957296264445	0.973165962665948\\
54.4175315220863	0.973179795425894\\
54.4393334177282	0.973193621015084\\
54.4611353133701	0.973207439437301\\
54.4829372090119	0.973221250696329\\
54.5047391046538	0.973235054795955\\
54.5265410002956	0.973248851739965\\
54.5483428959375	0.973262641532148\\
54.5701447915794	0.973276424176294\\
54.5919466872212	0.973290199676194\\
54.6137485828631	0.973303968035637\\
54.6355504785049	0.973317729258417\\
54.6573523741468	0.973331483348327\\
54.6791542697887	0.973345230309162\\
54.7009561654305	0.973358970144715\\
54.7227580610724	0.973372702858783\\
54.7445599567143	0.973386428455162\\
54.7663618523561	0.973400146937648\\
54.788163747998	0.973413858310041\\
54.8099656436398	0.973427562576137\\
54.8317675392817	0.973441259739735\\
54.8535694349236	0.973454949804635\\
54.8753713305654	0.973468632774637\\
54.8971732262073	0.97348230865354\\
54.9189751218491	0.973495977445144\\
54.940777017491	0.973509639153252\\
54.9625789131329	0.973523293781662\\
54.9843808087747	0.973536941334177\\
55.0061827044166	0.973550581814599\\
55.0279846000585	0.973564215226727\\
55.0497864957003	0.973577841574366\\
55.0715883913422	0.973591460861315\\
55.093390286984	0.973605073091377\\
55.1151921826259	0.973618678268353\\
55.1369940782678	0.973632276396046\\
55.1587959739096	0.973645867478256\\
55.1805978695515	0.973659451518785\\
55.2023997651933	0.973673028521435\\
55.2242016608352	0.973686598490007\\
55.2460035564771	0.9737001614283\\
55.2678054521189	0.973713717340117\\
55.2896073477608	0.973727266229257\\
55.3114092434027	0.97374080809952\\
55.3332111390445	0.973754342954705\\
55.3550130346864	0.973767870798611\\
55.3768149303282	0.973781391635037\\
55.3986168259701	0.973794905467782\\
55.420418721612	0.973808412300642\\
55.4422206172538	0.973821912137414\\
55.4640225128957	0.973835404981895\\
55.4858244085375	0.97384889083788\\
55.5076263041794	0.973862369709164\\
55.5294281998213	0.973875841599542\\
55.5512300954631	0.973889306512807\\
55.573031991105	0.973902764452751\\
55.5948338867468	0.973916215423167\\
55.6166357823887	0.973929659427846\\
55.6384376780306	0.973943096470577\\
55.6602395736724	0.973956526555151\\
55.6820414693143	0.973969949685354\\
55.7038433649562	0.973983365864975\\
55.725645260598	0.9739967750978\\
55.7474471562399	0.974010177387615\\
55.7692490518817	0.974023572738202\\
55.7910509475236	0.974036961153345\\
55.8128528431655	0.974050342636827\\
55.8346547388073	0.974063717192427\\
55.8564566344492	0.974077084823927\\
55.8782585300911	0.974090445535102\\
55.9000604257329	0.974103799329732\\
55.9218623213748	0.974117146211591\\
55.9436642170166	0.974130486184455\\
55.9654661126585	0.974143819252095\\
55.9872680083004	0.974157145418285\\
56.0090699039422	0.974170464686794\\
56.0308717995841	0.974183777061392\\
56.0526736952259	0.974197082545845\\
56.0744755908678	0.97421038114392\\
56.0962774865097	0.974223672859382\\
56.1180793821515	0.974236957695993\\
56.1398812777934	0.974250235657514\\
56.1616831734353	0.974263506747707\\
56.1834850690771	0.974276770970327\\
56.205286964719	0.974290028329134\\
56.2270888603608	0.97430327882788\\
56.2488907560027	0.974316522470321\\
56.2706926516446	0.974329759260206\\
56.2924945472864	0.974342989201286\\
56.3142964429283	0.974356212297309\\
56.3360983385701	0.974369428552022\\
56.357900234212	0.974382637969168\\
56.3797021298539	0.974395840552491\\
56.4015040254957	0.974409036305731\\
56.4233059211376	0.974422225232628\\
56.4451078167794	0.974435407336918\\
56.4669097124213	0.974448582622336\\
56.4887116080632	0.974461751092616\\
56.510513503705	0.974474912751488\\
56.5323153993469	0.974488067602683\\
56.5541172949888	0.974501215649928\\
56.5759191906306	0.974514356896947\\
56.5977210862725	0.974527491347464\\
56.6195229819143	0.974540619005201\\
56.6413248775562	0.974553739873876\\
56.6631267731981	0.974566853957206\\
56.6849286688399	0.974579961258907\\
56.7067305644818	0.974593061782691\\
56.7285324601237	0.974606155532268\\
56.7503343557655	0.974619242511347\\
56.7721362514074	0.974632322723635\\
56.7939381470492	0.974645396172836\\
56.8157400426911	0.974658462862651\\
56.837541938333	0.974671522796779\\
56.8593438339748	0.974684575978919\\
56.8811457296167	0.974697622412766\\
56.9029476252585	0.974710662102011\\
56.9247495209004	0.974723695050347\\
56.9465514165423	0.97473672126146\\
56.9683533121841	0.974749740739037\\
56.990155207826	0.974762753486761\\
57.0119571034679	0.974775759508314\\
57.0337589991097	0.974788758807374\\
57.0555608947516	0.974801751387617\\
57.0773627903934	0.974814737252717\\
57.0991646860353	0.974827716406347\\
57.1209665816772	0.974840688852174\\
57.142768477319	0.974853654593866\\
57.1645703729609	0.974866613635087\\
57.1863722686027	0.974879565979499\\
57.2081741642446	0.97489251163076\\
57.2299760598865	0.974905450592528\\
57.2517779555283	0.974918382868456\\
57.2735798511702	0.974931308462196\\
57.295381746812	0.974944227377398\\
57.3171836424539	0.974957139617709\\
57.3389855380958	0.974970045186771\\
57.3607874337376	0.974982944088226\\
57.3825893293795	0.974995836325714\\
57.4043912250214	0.97500872190287\\
57.4261931206632	0.975021600823329\\
57.4479950163051	0.975034473090721\\
57.4697969119469	0.975047338708674\\
57.4915988075888	0.975060197680815\\
57.5134007032307	0.975073050010767\\
57.5352025988725	0.97508589570215\\
57.5570044945144	0.975098734758581\\
57.5788063901562	0.975111567183677\\
57.6006082857981	0.975124392981049\\
57.62241018144	0.975137212154307\\
57.6442120770818	0.975150024707059\\
57.6660139727237	0.975162830642908\\
57.6878158683656	0.975175629965457\\
57.7096177640074	0.975188422678304\\
57.7314196596493	0.975201208785045\\
57.7532215552911	0.975213988289275\\
57.775023450933	0.975226761194583\\
57.7968253465749	0.975239527504557\\
57.8186272422167	0.975252287222784\\
57.8404291378586	0.975265040352844\\
57.8622310335004	0.975277786898319\\
57.8840329291423	0.975290526862784\\
57.9058348247842	0.975303260249814\\
57.927636720426	0.97531598706298\\
57.9494386160679	0.975328707305851\\
57.9712405117098	0.975341420981992\\
57.9930424073516	0.975354128094967\\
58.0148443029935	0.975366828648334\\
58.0366461986353	0.975379522645653\\
58.0584480942772	0.975392210090476\\
58.0802499899191	0.975404890986357\\
58.1020518855609	0.975417565336842\\
58.1238537812028	0.975430233145479\\
58.1456556768446	0.975442894415811\\
58.1674575724865	0.975455549151377\\
58.1892594681284	0.975468197355716\\
58.2110613637702	0.975480839032361\\
58.2328632594121	0.975493474184845\\
58.254665155054	0.975506102816697\\
58.2764670506958	0.975518724931441\\
58.2982689463377	0.975531340532602\\
58.3200708419795	0.9755439496237\\
58.3418727376214	0.975556552208253\\
58.3636746332633	0.975569148289774\\
58.3854765289051	0.975581737871775\\
58.407278424547	0.975594320957766\\
58.4290803201888	0.975606897551251\\
58.4508822158307	0.975619467655735\\
58.4726841114726	0.975632031274716\\
58.4944860071144	0.975644588411694\\
58.5162879027563	0.975657139070161\\
58.5380897983982	0.975669683253609\\
58.55989169404	0.975682220965528\\
58.5816935896819	0.975694752209402\\
58.6034954853237	0.975707276988715\\
58.6252973809656	0.975719795306946\\
58.6470992766075	0.975732307167572\\
58.6689011722493	0.975744812574068\\
58.6907030678912	0.975757311529905\\
58.712504963533	0.975769804038551\\
58.7343068591749	0.975782290103473\\
58.7561087548168	0.975794769728131\\
58.7779106504586	0.975807242915986\\
58.7997125461005	0.975819709670495\\
58.8215144417424	0.975832169995112\\
58.8433163373842	0.975844623893288\\
58.8651182330261	0.975857071368471\\
58.8869201286679	0.975869512424106\\
58.9087220243098	0.975881947063636\\
58.9305239199517	0.9758943752905\\
58.9523258155935	0.975906797108134\\
58.9741277112354	0.975919212519973\\
58.9959296068772	0.975931621529447\\
59.0177315025191	0.975944024139984\\
59.039533398161	0.97595642035501\\
59.0613352938028	0.975968810177946\\
59.0831371894447	0.975981193612212\\
59.1049390850866	0.975993570661225\\
59.1267409807284	0.976005941328397\\
59.1485428763703	0.97601830561714\\
59.1703447720121	0.976030663530861\\
59.192146667654	0.976043015072966\\
59.2139485632959	0.976055360246856\\
59.2357504589377	0.976067699055931\\
59.2575523545796	0.976080031503588\\
59.2793542502214	0.976092357593219\\
59.3011561458633	0.976104677328216\\
59.3229580415052	0.976116990711966\\
59.344759937147	0.976129297747855\\
59.3665618327889	0.976141598439264\\
59.3883637284308	0.976153892789574\\
59.4101656240726	0.97616618080216\\
59.4319675197145	0.976178462480396\\
59.4537694153563	0.976190737827652\\
59.4755713109982	0.976203006847299\\
59.4973732066401	0.976215269542699\\
59.5191751022819	0.976227525917216\\
59.5409769979238	0.976239775974209\\
59.5627788935656	0.976252019717035\\
59.5845807892075	0.976264257149048\\
59.6063826848494	0.976276488273599\\
59.6281845804912	0.976288713094037\\
59.6499864761331	0.976300931613706\\
59.671788371775	0.976313143835951\\
59.6935902674168	0.97632534976411\\
59.7153921630587	0.976337549401522\\
59.7371940587005	0.97634974275152\\
59.7589959543424	0.976361929817436\\
59.7807978499843	0.976374110602599\\
59.8025997456261	0.976386285110336\\
59.824401641268	0.976398453343969\\
59.8462035369098	0.97641061530682\\
59.8680054325517	0.976422771002206\\
59.8898073281936	0.976434920433442\\
59.9116092238354	0.976447063603841\\
59.9334111194773	0.976459200516713\\
59.9552130151192	0.976471331175364\\
59.977014910761	0.976483455583098\\
59.9988168064029	0.976495573743218\\
60.0206187020447	0.976507685659022\\
60.0424205976866	0.976519791333807\\
60.0642224933285	0.976531890770865\\
60.0860243889703	0.976543983973487\\
60.1078262846122	0.976556070944961\\
60.129628180254	0.976568151688574\\
60.1514300758959	0.976580226207607\\
60.1732319715378	0.97659229450534\\
60.1950338671796	0.97660435658505\\
60.2168357628215	0.976616412450014\\
60.2386376584634	0.976628462103501\\
60.2604395541052	0.976640505548782\\
60.2822414497471	0.976652542789124\\
60.3040433453889	0.976664573827791\\
60.3258452410308	0.976676598668043\\
60.3476471366727	0.976688617313141\\
60.3694490323145	0.976700629766339\\
60.3912509279564	0.976712636030893\\
60.4130528235982	0.976724636110052\\
60.4348547192401	0.976736630007065\\
60.456656614882	0.976748617725178\\
60.4784585105238	0.976760599267634\\
60.5002604061657	0.976772574637674\\
60.5220623018076	0.976784543838536\\
60.5438641974494	0.976796506873455\\
60.5656660930913	0.976808463745664\\
60.5874679887331	0.976820414458394\\
60.609269884375	0.976832359014872\\
60.6310717800169	0.976844297418324\\
60.6528736756587	0.976856229671972\\
60.6746755713006	0.976868155779037\\
60.6964774669424	0.976880075742737\\
60.7182793625843	0.976891989566285\\
60.7400812582262	0.976903897252896\\
60.761883153868	0.976915798805779\\
60.7836850495099	0.976927694228141\\
60.8054869451517	0.976939583523189\\
60.8272888407936	0.976951466694124\\
60.8490907364355	0.976963343744147\\
60.8708926320773	0.976975214676456\\
60.8926945277192	0.976987079494245\\
60.9144964233611	0.976998938200709\\
60.9362983190029	0.977010790799036\\
60.9581002146448	0.977022637292414\\
60.9799021102866	0.977034477684031\\
61.0017040059285	0.977046311977067\\
61.0235059015704	0.977058140174704\\
61.0453077972122	0.97706996228012\\
61.0671096928541	0.977081778296491\\
61.088911588496	0.97709358822699\\
61.1107134841378	0.977105392074789\\
61.1325153797797	0.977117189843055\\
61.1543172754215	0.977128981534954\\
61.1761191710634	0.977140767153652\\
61.1979210667053	0.977152546702308\\
61.2197229623471	0.977164320184083\\
61.241524857989	0.977176087602132\\
61.2633267536308	0.977187848959611\\
61.2851286492727	0.977199604259671\\
61.3069305449146	0.977211353505462\\
61.3287324405564	0.977223096700131\\
61.3505343361983	0.977234833846823\\
61.3723362318402	0.977246564948681\\
61.394138127482	0.977258290008846\\
61.4159400231239	0.977270009030455\\
61.4377419187657	0.977281722016644\\
61.4595438144076	0.977293428970547\\
61.4813457100495	0.977305129895295\\
61.5031476056913	0.977316824794017\\
61.5249495013332	0.977328513669839\\
61.546751396975	0.977340196525887\\
61.5685532926169	0.977351873365282\\
61.5903551882588	0.977363544191143\\
61.6121570839006	0.97737520900659\\
61.6339589795425	0.977386867814737\\
61.6557608751843	0.977398520618697\\
61.6775627708262	0.977410167421581\\
61.6993646664681	0.977421808226498\\
61.7211665621099	0.977433443036556\\
61.7429684577518	0.977445071854857\\
61.7647703533937	0.977456694684505\\
61.7865722490355	0.977468311528598\\
61.8083741446774	0.977479922390236\\
61.8301760403192	0.977491527272513\\
61.8519779359611	0.977503126178524\\
61.873779831603	0.977514719111359\\
61.8955817272448	0.977526306074107\\
61.9173836228867	0.977537887069856\\
61.9391855185286	0.97754946210169\\
61.9609874141704	0.977561031172692\\
61.9827893098123	0.977572594285942\\
62.0045912054541	0.97758415144452\\
62.026393101096	0.977595702651501\\
62.0481949967379	0.97760724790996\\
62.0699968923797	0.977618787222969\\
62.0917987880216	0.977630320593598\\
62.1136006836634	0.977641848024914\\
62.1354025793053	0.977653369519985\\
62.1572044749472	0.977664885081873\\
62.179006370589	0.97767639471364\\
62.2008082662309	0.977687898418346\\
62.2226101618728	0.977699396199048\\
62.2444120575146	0.977710888058803\\
62.2662139531565	0.977722374000663\\
62.2880158487983	0.97773385402768\\
62.3098177444402	0.977745328142904\\
62.3316196400821	0.977756796349382\\
62.3534215357239	0.977768258650159\\
62.3752234313658	0.977779715048279\\
62.3970253270076	0.977791165546784\\
62.4188272226495	0.977802610148711\\
62.4406291182914	0.9778140488571\\
62.4624310139332	0.977825481674985\\
62.4842329095751	0.9778369086054\\
62.5060348052169	0.977848329651377\\
62.5278367008588	0.977859744815944\\
62.5496385965007	0.977871154102129\\
62.5714404921425	0.977882557512959\\
62.5932423877844	0.977893955051456\\
62.6150442834263	0.977905346720642\\
62.6368461790681	0.977916732523537\\
62.65864807471	0.977928112463159\\
62.6804499703518	0.977939486542524\\
62.7022518659937	0.977950854764645\\
62.7240537616356	0.977962217132535\\
62.7458556572774	0.977973573649204\\
62.7676575529193	0.97798492431766\\
62.7894594485611	0.977996269140909\\
62.811261344203	0.978007608121956\\
62.8330632398449	0.978018941263804\\
62.8548651354867	0.978030268569453\\
62.8766670311286	0.978041590041902\\
62.8984689267705	0.978052905684148\\
62.9202708224123	0.978064215499186\\
62.9420727180542	0.978075519490009\\
62.963874613696	0.978086817659609\\
62.9856765093379	0.978098110010975\\
63.0074784049798	0.978109396547095\\
63.0292803006216	0.978120677270955\\
63.0510821962635	0.978131952185538\\
63.0728840919054	0.978143221293828\\
63.0946859875472	0.978154484598804\\
63.1164878831891	0.978165742103445\\
63.1382897788309	0.978176993810729\\
63.1600916744728	0.978188239723628\\
63.1818935701147	0.978199479845119\\
63.2036954657565	0.97821071417817\\
63.2254973613984	0.978221942725753\\
63.2472992570402	0.978233165490834\\
63.2691011526821	0.978244382476381\\
63.290903048324	0.978255593685356\\
63.3127049439658	0.978266799120724\\
63.3345068396077	0.978277998785444\\
63.3563087352495	0.978289192682475\\
63.3781106308914	0.978300380814776\\
63.3999125265333	0.978311563185301\\
63.4217144221751	0.978322739797004\\
63.443516317817	0.978333910652837\\
63.4653182134589	0.978345075755751\\
63.4871201091007	0.978356235108694\\
63.5089220047426	0.978367388714614\\
63.5307239003844	0.978378536576454\\
63.5525257960263	0.978389678697159\\
63.5743276916682	0.978400815079671\\
63.59612958731	0.978411945726929\\
63.6179314829519	0.978423070641872\\
63.6397333785937	0.978434189827437\\
63.6615352742356	0.978445303286559\\
63.6833371698775	0.97845641102217\\
63.7051390655193	0.978467513037202\\
63.7269409611612	0.978478609334587\\
63.7487428568031	0.978489699917251\\
63.7705447524449	0.978500784788122\\
63.7923466480868	0.978511863950124\\
63.8141485437286	0.978522937406181\\
63.8359504393705	0.978534005159214\\
63.8577523350124	0.978545067212144\\
63.8795542306542	0.978556123567889\\
63.9013561262961	0.978567174229365\\
63.923158021938	0.978578219199488\\
63.9449599175798	0.978589258481172\\
63.9667618132217	0.978600292077328\\
63.9885637088635	0.978611319990866\\
64.0103656045054	0.978622342224696\\
64.0321675001473	0.978633358781724\\
64.0539693957891	0.978644369664855\\
64.075771291431	0.978655374876994\\
64.0975731870728	0.978666374421043\\
64.1193750827147	0.978677368299902\\
64.1411769783566	0.97868835651647\\
64.1629788739984	0.978699339073646\\
64.1847807696403	0.978710315974324\\
64.2065826652821	0.978721287221399\\
64.228384560924	0.978732252817765\\
64.2501864565659	0.978743212766311\\
64.2719883522077	0.978754167069928\\
64.2937902478496	0.978765115731504\\
64.3155921434915	0.978776058753925\\
64.3373940391333	0.978786996140076\\
64.3591959347752	0.978797927892841\\
64.380997830417	0.978808854015101\\
64.4027997260589	0.978819774509737\\
64.4246016217008	0.978830689379628\\
64.4464035173426	0.97884159862765\\
64.4682054129845	0.978852502256679\\
64.4900073086263	0.978863400269591\\
64.5118092042682	0.978874292669256\\
64.5336110999101	0.978885179458547\\
64.5554129955519	0.978896060640332\\
64.5772148911938	0.978906936217481\\
64.5990167868356	0.978917806192859\\
64.6208186824775	0.978928670569332\\
64.6426205781194	0.978939529349763\\
64.6644224737612	0.978950382537015\\
64.6862243694031	0.978961230133947\\
64.708026265045	0.978972072143419\\
64.7298281606868	0.978982908568289\\
64.7516300563287	0.978993739411412\\
64.7734319519706	0.979004564675644\\
64.7952338476124	0.979015384363836\\
64.8170357432543	0.979026198478842\\
64.8388376388961	0.979037007023511\\
64.860639534538	0.979047810000692\\
64.8824414301799	0.979058607413232\\
64.9042433258217	0.979069399263976\\
64.9260452214636	0.97908018555577\\
64.9478471171054	0.979090966291456\\
64.9696490127473	0.979101741473875\\
64.9914509083892	0.979112511105867\\
65.013252804031	0.979123275190271\\
65.0350546996729	0.979134033729923\\
65.0568565953147	0.97914478672766\\
65.0786584909566	0.979155534186316\\
65.1004603865985	0.979166276108722\\
65.1222622822403	0.979177012497711\\
65.1440641778822	0.979187743356112\\
65.1658660735241	0.979198468686754\\
65.1876679691659	0.979209188492464\\
65.2094698648078	0.979219902776066\\
65.2312717604496	0.979230611540385\\
65.2530736560915	0.979241314788245\\
65.2748755517334	0.979252012522465\\
65.2966774473752	0.979262704745867\\
65.3184793430171	0.979273391461267\\
65.3402812386589	0.979284072671485\\
65.3620831343008	0.979294748379334\\
65.3838850299427	0.979305418587629\\
65.4056869255845	0.979316083299184\\
65.4274888212264	0.979326742516808\\
65.4492907168682	0.979337396243313\\
65.4710926125101	0.979348044481507\\
65.492894508152	0.979358687234197\\
65.5146964037938	0.979369324504189\\
65.5364982994357	0.979379956294287\\
65.5583001950776	0.979390582607295\\
65.5801020907194	0.979401203446014\\
65.6019039863613	0.979411818813243\\
65.6237058820032	0.979422428711783\\
65.645507777645	0.979433033144431\\
65.6673096732869	0.979443632113982\\
65.6891115689287	0.979454225623232\\
65.7109134645706	0.979464813674973\\
65.7327153602125	0.979475396271998\\
65.7545172558543	0.979485973417098\\
65.7763191514962	0.979496545113061\\
65.798121047138	0.979507111362675\\
65.8199229427799	0.979517672168727\\
65.8417248384218	0.979528227534002\\
65.8635267340636	0.979538777461285\\
65.8853286297055	0.979549321953356\\
65.9071305253473	0.979559861012999\\
65.9289324209892	0.979570394642991\\
65.9507343166311	0.979580922846111\\
65.9725362122729	0.979591445625137\\
65.9943381079148	0.979601962982844\\
66.0161400035567	0.979612474922007\\
66.0379418991985	0.979622981445398\\
66.0597437948404	0.979633482555788\\
66.0815456904822	0.97964397825595\\
66.1033475861241	0.97965446854865\\
66.125149481766	0.979664953436657\\
66.1469513774078	0.979675432922737\\
66.1687532730497	0.979685907009655\\
66.1905551686915	0.979696375700174\\
66.2123570643334	0.979706838997058\\
66.2341589599753	0.979717296903066\\
66.2559608556171	0.979727749420958\\
66.277762751259	0.979738196553493\\
66.2995646469008	0.979748638303427\\
66.3213665425427	0.979759074673516\\
66.3431684381846	0.979769505666514\\
66.3649703338264	0.979779931285175\\
66.3867722294683	0.97979035153225\\
66.4085741251102	0.979800766410489\\
66.430376020752	0.979811175922641\\
66.4521779163939	0.979821580071453\\
66.4739798120358	0.979831978859673\\
66.4957817076776	0.979842372290045\\
66.5175836033195	0.979852760365313\\
66.5393854989613	0.97986314308822\\
66.5611873946032	0.979873520461506\\
66.5829892902451	0.979883892487911\\
66.6047911858869	0.979894259170173\\
66.6265930815288	0.979904620511031\\
66.6483949771706	0.979914976513219\\
66.6701968728125	0.979925327179473\\
66.6919987684544	0.979935672512524\\
66.7138006640962	0.979946012515107\\
66.7356025597381	0.97995634718995\\
66.7574044553799	0.979966676539784\\
66.7792063510218	0.979977000567336\\
66.8010082466637	0.979987319275334\\
66.8228101423055	0.979997632666502\\
66.8446120379474	0.980007940743565\\
66.8664139335893	0.980018243509246\\
66.8882158292311	0.980028540966265\\
66.910017724873	0.980038833117344\\
66.9318196205148	0.980049119965202\\
66.9536215161567	0.980059401512555\\
66.9754234117986	0.980069677762121\\
66.9972253074404	0.980079948716614\\
67.0190272030823	0.980090214378749\\
67.0408290987241	0.980100474751237\\
67.062630994366	0.980110729836791\\
67.0844328900079	0.980120979638119\\
67.1062347856497	0.980131224157931\\
67.1280366812916	0.980141463398934\\
67.1498385769334	0.980151697363834\\
67.1716404725753	0.980161926055335\\
67.1934423682172	0.980172149476142\\
67.215244263859	0.980182367628957\\
67.2370461595009	0.98019258051648\\
67.2588480551428	0.980202788141411\\
67.2806499507846	0.980212990506449\\
67.3024518464265	0.98022318761429\\
67.3242537420683	0.980233379467631\\
67.3460556377102	0.980243566069165\\
67.3678575333521	0.980253747421587\\
67.3896594289939	0.980263923527588\\
67.4114613246358	0.980274094389859\\
67.4332632202777	0.98028426001109\\
67.4550651159195	0.980294420393968\\
67.4768670115614	0.98030457554118\\
67.4986689072032	0.980314725455413\\
67.5204708028451	0.98032487013935\\
67.542272698487	0.980335009595674\\
67.5640745941288	0.980345143827068\\
67.5858764897707	0.980355272836211\\
67.6076783854125	0.980365396625783\\
67.6294802810544	0.980375515198463\\
67.6512821766963	0.980385628556926\\
67.6730840723381	0.980395736703848\\
67.69488596798	0.980405839641904\\
67.7166878636219	0.980415937373766\\
67.7384897592637	0.980426029902106\\
67.7602916549056	0.980436117229595\\
67.7820935505474	0.9804461993589\\
67.8038954461893	0.980456276292691\\
67.8256973418312	0.980466348033634\\
67.847499237473	0.980476414584394\\
67.8693011331149	0.980486475947635\\
67.8911030287567	0.98049653212602\\
67.9129049243986	0.98050658312221\\
67.9347068200405	0.980516628938867\\
67.9565087156823	0.980526669578648\\
67.9783106113242	0.980536705044212\\
68.000112506966	0.980546735338214\\
68.0219144026079	0.980556760463311\\
68.0437162982498	0.980566780422156\\
68.0655181938916	0.980576795217402\\
68.0873200895335	0.980586804851699\\
68.1091219851754	0.9805968093277\\
68.1309238808172	0.980606808648051\\
68.1527257764591	0.980616802815402\\
68.1745276721009	0.980626791832398\\
68.1963295677428	0.980636775701684\\
68.2181314633847	0.980646754425904\\
68.2399333590265	0.980656728007702\\
68.2617352546684	0.980666696449717\\
68.2835371503103	0.980676659754592\\
68.3053390459521	0.980686617924963\\
68.327140941594	0.980696570963469\\
68.3489428372358	0.980706518872747\\
68.3707447328777	0.98071646165543\\
68.3925466285196	0.980726399314154\\
68.4143485241614	0.98073633185155\\
68.4361504198033	0.980746259270249\\
68.4579523154451	0.980756181572883\\
68.479754211087	0.98076609876208\\
68.5015561067289	0.980776010840467\\
68.5233580023707	0.98078591781067\\
68.5451598980126	0.980795819675314\\
68.5669617936544	0.980805716437024\\
68.5887636892963	0.980815608098422\\
68.6105655849382	0.980825494662129\\
68.63236748058	0.980835376130765\\
68.6541693762219	0.980845252506949\\
68.6759712718638	0.980855123793298\\
68.6977731675056	0.980864989992428\\
68.7195750631475	0.980874851106956\\
68.7413769587893	0.980884707139493\\
68.7631788544312	0.980894558092654\\
68.7849807500731	0.980904403969048\\
68.8067826457149	0.980914244771287\\
68.8285845413568	0.980924080501978\\
68.8503864369986	0.980933911163729\\
68.8721883326405	0.980943736759147\\
68.8939902282824	0.980953557290837\\
68.9157921239242	0.980963372761401\\
68.9375940195661	0.980973183173444\\
68.959395915208	0.980982988529565\\
68.9811978108498	0.980992788832365\\
69.0029997064917	0.981002584084442\\
69.0248016021335	0.981012374288395\\
69.0466034977754	0.981022159446819\\
69.0684053934173	0.981031939562309\\
69.0902072890591	0.981041714637459\\
69.112009184701	0.981051484674861\\
69.1338110803429	0.981061249677108\\
69.1556129759847	0.981071009646788\\
69.1774148716266	0.98108076458649\\
69.1992167672684	0.981090514498802\\
69.2210186629103	0.98110025938631\\
69.2428205585522	0.981109999251599\\
69.264622454194	0.981119734097253\\
69.2864243498359	0.981129463925855\\
69.3082262454777	0.981139188739984\\
69.3300281411196	0.981148908542222\\
69.3518300367615	0.981158623335148\\
69.3736319324033	0.981168333121338\\
69.3954338280452	0.981178037903369\\
69.417235723687	0.981187737683815\\
69.4390376193289	0.981197432465251\\
69.4608395149708	0.98120712225025\\
69.4826414106126	0.981216807041381\\
69.5044433062545	0.981226486841216\\
69.5262452018964	0.981236161652323\\
69.5480470975382	0.98124583147727\\
69.5698489931801	0.981255496318622\\
69.5916508888219	0.981265156178945\\
69.6134527844638	0.981274811060803\\
69.6352546801057	0.981284460966758\\
69.6570565757475	0.981294105899371\\
69.6788584713894	0.981303745861203\\
69.7006603670312	0.981313380854811\\
69.7224622626731	0.981323010882754\\
69.744264158315	0.981332635947588\\
69.7660660539568	0.981342256051868\\
69.7878679495987	0.981351871198148\\
69.8096698452405	0.981361481388979\\
69.8314717408824	0.981371086626914\\
69.8532736365243	0.981380686914503\\
69.8750755321661	0.981390282254293\\
69.896877427808	0.981399872648834\\
69.9186793234499	0.981409458100671\\
69.9404812190917	0.981419038612349\\
69.9622831147336	0.981428614186412\\
69.9840850103755	0.981438184825402\\
70.0058869060173	0.981447750531861\\
70.0276888016592	0.981457311308329\\
70.049490697301	0.981466867157345\\
70.0712925929429	0.981476418081445\\
70.0930944885848	0.981485964083168\\
70.1148963842266	0.981495505165047\\
70.1366982798685	0.981505041329616\\
70.1585001755103	0.981514572579409\\
70.1803020711522	0.981524098916956\\
70.2021039667941	0.981533620344788\\
70.2239058624359	0.981543136865433\\
70.2457077580778	0.981552648481418\\
70.2675096537196	0.981562155195271\\
70.2893115493615	0.981571657009517\\
70.3111134450034	0.981581153926678\\
70.3329153406452	0.981590645949279\\
70.3547172362871	0.98160013307984\\
70.376519131929	0.981609615320881\\
70.3983210275708	0.981619092674921\\
70.4201229232127	0.981628565144478\\
70.4419248188545	0.981638032732069\\
70.4637267144964	0.981647495440208\\
70.4855286101383	0.98165695327141\\
70.5073305057801	0.981666406228187\\
70.529132401422	0.98167585431305\\
70.5509342970638	0.98168529752851\\
70.5727361927057	0.981694735877075\\
70.5945380883476	0.981704169361254\\
70.6163399839894	0.981713597983552\\
70.6381418796313	0.981723021746475\\
70.6599437752731	0.981732440652528\\
70.681745670915	0.981741854704211\\
70.7035475665569	0.981751263904028\\
70.7253494621987	0.981760668254478\\
70.7471513578406	0.981770067758061\\
70.7689532534825	0.981779462417273\\
70.7907551491243	0.981788852234612\\
70.8125570447662	0.981798237212572\\
70.8343589404081	0.981807617353649\\
70.8561608360499	0.981816992660334\\
70.8779627316918	0.981826363135119\\
70.8997646273336	0.981835728780494\\
70.9215665229755	0.981845089598949\\
70.9433684186174	0.981854445592971\\
70.9651703142592	0.981863796765047\\
70.9869722099011	0.981873143117661\\
71.0087741055429	0.981882484653299\\
71.0305760011848	0.981891821374443\\
71.0523778968267	0.981901153283575\\
71.0741797924685	0.981910480383174\\
71.0959816881104	0.981919802675721\\
71.1177835837522	0.981929120163692\\
71.1395854793941	0.981938432849565\\
71.161387375036	0.981947740735815\\
71.1831892706778	0.981957043824916\\
71.2049911663197	0.981966342119341\\
71.2267930619616	0.981975635621561\\
71.2485949576034	0.981984924334048\\
71.2703968532453	0.98199420825927\\
71.2921987488871	0.982003487399695\\
71.314000644529	0.98201276175779\\
71.3358025401709	0.98202203133602\\
71.3576044358127	0.98203129613685\\
71.3794063314546	0.982040556162742\\
71.4012082270964	0.982049811416159\\
71.4230101227383	0.982059061899561\\
71.4448120183802	0.982068307615407\\
71.466613914022	0.982077548566155\\
71.4884158096639	0.982086784754262\\
71.5102177053057	0.982096016182183\\
71.5320196009476	0.982105242852374\\
71.5538214965895	0.982114464767286\\
71.5756233922313	0.982123681929372\\
71.5974252878732	0.982132894341082\\
71.6192271835151	0.982142102004866\\
71.6410290791569	0.982151304923172\\
71.6628309747988	0.982160503098446\\
71.6846328704407	0.982169696533135\\
71.7064347660825	0.982178885229683\\
71.7282366617244	0.982188069190533\\
71.7500385573662	0.982197248418127\\
71.7718404530081	0.982206422914905\\
71.79364234865	0.982215592683308\\
71.8154442442918	0.982224757725773\\
71.8372461399337	0.982233918044738\\
71.8590480355755	0.982243073642638\\
71.8808499312174	0.982252224521908\\
71.9026518268593	0.98226137068498\\
71.9244537225011	0.982270512134288\\
71.946255618143	0.982279648872262\\
71.9680575137848	0.982288780901331\\
71.9898594094267	0.982297908223924\\
72.0116613050686	0.982307030842469\\
72.0334632007104	0.98231614875939\\
72.0552650963523	0.982325261977113\\
72.0770669919942	0.982334370498061\\
72.098868887636	0.982343474324656\\
72.1206707832779	0.98235257345932\\
72.1424726789197	0.982361667904472\\
72.1642745745616	0.98237075766253\\
72.1860764702035	0.982379842735912\\
72.2078783658453	0.982388923127034\\
72.2296802614872	0.982397998838311\\
72.251482157129	0.982407069872156\\
72.2732840527709	0.982416136230981\\
72.2950859484128	0.982425197917199\\
72.3168878440546	0.982434254933218\\
72.3386897396965	0.982443307281447\\
72.3604916353383	0.982452354964294\\
72.3822935309802	0.982461397984166\\
72.4040954266221	0.982470436343466\\
72.4258973222639	0.982479470044599\\
72.4476992179058	0.982488499089967\\
72.4695011135477	0.982497523481972\\
72.4913030091895	0.982506543223014\\
72.5131049048314	0.982515558315491\\
72.5349068004732	0.982524568761801\\
72.5567086961151	0.982533574564341\\
72.578510591757	0.982542575725506\\
72.6003124873988	0.982551572247689\\
72.6221143830407	0.982560564133284\\
72.6439162786826	0.982569551384681\\
72.6657181743244	0.982578534004272\\
72.6875200699663	0.982587511994445\\
72.7093219656081	0.982596485357588\\
72.73112386125	0.982605454096087\\
72.7529257568919	0.982614418212329\\
72.7747276525337	0.982623377708696\\
72.7965295481756	0.982632332587573\\
72.8183314438174	0.982641282851341\\
72.8401333394593	0.98265022850238\\
72.8619352351012	0.982659169543069\\
72.883737130743	0.982668105975787\\
72.9055390263849	0.982677037802911\\
72.9273409220268	0.982685965026815\\
72.9491428176686	0.982694887649875\\
72.9709447133105	0.982703805674463\\
72.9927466089523	0.982712719102952\\
73.0145485045942	0.982721627937713\\
73.0363504002361	0.982730532181113\\
73.0581522958779	0.982739431835523\\
73.0799541915198	0.982748326903309\\
73.1017560871616	0.982757217386837\\
73.1235579828035	0.982766103288472\\
73.1453598784454	0.982774984610576\\
73.1671617740872	0.982783861355513\\
73.1889636697291	0.982792733525643\\
73.2107655653709	0.982801601123326\\
73.2325674610128	0.98281046415092\\
73.2543693566547	0.982819322610783\\
73.2761712522965	0.982828176505272\\
73.2979731479384	0.98283702583674\\
73.3197750435803	0.982845870607541\\
73.3415769392221	0.982854710820029\\
73.363378834864	0.982863546476554\\
73.3851807305058	0.982872377579467\\
73.4069826261477	0.982881204131115\\
73.4287845217896	0.982890026133848\\
73.4505864174314	0.982898843590011\\
73.4723883130733	0.98290765650195\\
73.4941902087152	0.982916464872007\\
73.515992104357	0.982925268702528\\
73.5377939999989	0.982934067995852\\
73.5595958956407	0.98294286275432\\
73.5813977912826	0.982951652980272\\
73.6031996869245	0.982960438676046\\
73.6250015825663	0.982969219843977\\
73.6468034782082	0.982977996486402\\
73.66860537385	0.982986768605656\\
73.6904072694919	0.98299553620407\\
73.7122091651338	0.983004299283978\\
73.7340110607756	0.983013057847709\\
73.7558129564175	0.983021811897594\\
73.7776148520593	0.98303056143596\\
73.7994167477012	0.983039306465135\\
73.8212186433431	0.983048046987445\\
73.8430205389849	0.983056783005213\\
73.8648224346268	0.983065514520765\\
73.8866243302687	0.983074241536422\\
73.9084262259105	0.983082964054505\\
73.9302281215524	0.983091682077335\\
73.9520300171942	0.983100395607229\\
73.9738319128361	0.983109104646505\\
73.995633808478	0.98311780919748\\
74.0174357041198	0.983126509262468\\
74.0392375997617	0.983135204843784\\
74.0610394954035	0.98314389594374\\
74.0828413910454	0.983152582564648\\
74.1046432866873	0.983161264708818\\
74.1264451823291	0.983169942378559\\
74.148247077971	0.98317861557618\\
74.1700489736129	0.983187284303986\\
74.1918508692547	0.983195948564283\\
74.2136527648966	0.983204608359376\\
74.2354546605384	0.983213263691567\\
74.2572565561803	0.983221914563159\\
74.2790584518222	0.983230560976452\\
74.300860347464	0.983239202933746\\
74.3226622431059	0.983247840437339\\
74.3444641387478	0.983256473489529\\
74.3662660343896	0.983265102092611\\
74.3880679300315	0.98327372624888\\
74.4098698256733	0.98328234596063\\
74.4316717213152	0.983290961230152\\
74.4534736169571	0.98329957205974\\
74.4752755125989	0.983308178451681\\
74.4970774082408	0.983316780408265\\
74.5188793038826	0.98332537793178\\
74.5406811995245	0.983333971024512\\
74.5624830951664	0.983342559688747\\
74.5842849908082	0.983351143926768\\
74.6060868864501	0.983359723740859\\
74.6278887820919	0.983368299133301\\
74.6496906777338	0.983376870106374\\
74.6714925733757	0.983385436662358\\
74.6932944690175	0.983393998803532\\
74.7150963646594	0.983402556532171\\
74.7368982603013	0.983411109850552\\
74.7587001559431	0.98341965876095\\
74.780502051585	0.983428203265637\\
74.8023039472268	0.983436743366887\\
74.8241058428687	0.98344527906697\\
74.8459077385106	0.983453810368155\\
74.8677096341524	0.983462337272713\\
74.8895115297943	0.98347085978291\\
74.9113134254361	0.983479377901012\\
74.933115321078	0.983487891629285\\
74.9549172167199	0.983496400969993\\
74.9767191123617	0.983504905925398\\
74.9985210080036	0.983513406497762\\
75.0203229036455	0.983521902689346\\
75.0421247992873	0.983530394502409\\
75.0639266949292	0.983538881939209\\
75.085728590571	0.983547365002003\\
75.1075304862129	0.983555843693046\\
75.1293323818548	0.983564318014594\\
75.1511342774966	0.983572787968899\\
75.1729361731385	0.983581253558215\\
75.1947380687804	0.983589714784791\\
75.2165399644222	0.983598171650879\\
75.2383418600641	0.983606624158726\\
75.2601437557059	0.98361507231058\\
75.2819456513478	0.983623516108687\\
75.3037475469897	0.983631955555294\\
75.3255494426315	0.983640390652643\\
75.3473513382734	0.983648821402978\\
75.3691532339152	0.98365724780854\\
75.3909551295571	0.983665669871571\\
75.412757025199	0.983674087594308\\
75.4345589208408	0.983682500978992\\
75.4563608164827	0.983690910027858\\
75.4781627121245	0.983699314743142\\
75.4999646077664	0.98370771512708\\
75.5217665034083	0.983716111181905\\
75.5435683990501	0.983724502909849\\
75.565370294692	0.983732890313144\\
75.5871721903339	0.983741273394019\\
75.6089740859757	0.983749652154703\\
75.6307759816176	0.983758026597426\\
75.6525778772594	0.983766396724411\\
75.6743797729013	0.983774762537887\\
75.6961816685432	0.983783124040075\\
75.717983564185	0.983791481233201\\
75.7397854598269	0.983799834119485\\
75.7615873554687	0.983808182701149\\
75.7833892511106	0.983816526980412\\
75.8051911467525	0.983824866959493\\
75.8269930423943	0.983833202640609\\
75.8487949380362	0.983841534025976\\
75.8705968336781	0.983849861117809\\
75.8923987293199	0.983858183918323\\
75.9142006249618	0.98386650242973\\
75.9360025206036	0.983874816654241\\
75.9578044162455	0.983883126594067\\
75.9796063118874	0.983891432251417\\
76.0014082075292	0.9838997336285\\
76.0232101031711	0.983908030727522\\
76.045011998813	0.98391632355069\\
76.0668138944548	0.983924612100207\\
76.0886157900967	0.983932896378278\\
76.1104176857385	0.983941176387105\\
76.1322195813804	0.983949452128889\\
76.1540214770223	0.98395772360583\\
76.1758233726641	0.983965990820127\\
76.197625268306	0.983974253773978\\
76.2194271639478	0.983982512469581\\
76.2412290595897	0.983990766909129\\
76.2630309552316	0.983999017094818\\
76.2848328508734	0.984007263028841\\
76.3066347465153	0.98401550471339\\
76.3284366421571	0.984023742150656\\
76.350238537799	0.984031975342828\\
76.3720404334409	0.984040204292096\\
76.3938423290827	0.984048429000648\\
76.4156442247246	0.984056649470668\\
76.4374461203665	0.984064865704343\\
76.4592480160083	0.984073077703858\\
76.4810499116502	0.984081285471394\\
76.502851807292	0.984089489009133\\
76.5246537029339	0.984097688319257\\
76.5464555985758	0.984105883403946\\
76.5682574942176	0.984114074265376\\
76.5900593898595	0.984122260905727\\
76.6118612855013	0.984130443327173\\
76.6336631811432	0.984138621531891\\
76.6554650767851	0.984146795522054\\
76.6772669724269	0.984154965299834\\
76.6990688680688	0.984163130867404\\
76.7208707637106	0.984171292226935\\
76.7426726593525	0.984179449380594\\
76.7644745549944	0.984187602330551\\
76.7862764506362	0.984195751078974\\
76.8080783462781	0.984203895628027\\
76.82988024192	0.984212035979876\\
76.8516821375618	0.984220172136684\\
76.8734840332037	0.984228304100615\\
76.8952859288456	0.98423643187383\\
76.9170878244874	0.984244555458488\\
76.9388897201293	0.984252674856751\\
76.9606916157711	0.984260790070775\\
76.982493511413	0.984268901102718\\
77.0042954070549	0.984277007954735\\
77.0260973026967	0.984285110628982\\
77.0478991983386	0.984293209127612\\
77.0697010939804	0.984301303452777\\
77.0915029896223	0.98430939360663\\
77.1133048852642	0.98431747959132\\
77.135106780906	0.984325561408996\\
77.1569086765479	0.984333639061807\\
77.1787105721897	0.9843417125519\\
77.2005124678316	0.98434978188142\\
77.2223143634735	0.984357847052513\\
77.2441162591153	0.984365908067321\\
77.2659181547572	0.984373964927988\\
77.2877200503991	0.984382017636655\\
77.3095219460409	0.984390066195461\\
77.3313238416828	0.984398110606547\\
77.3531257373246	0.98440615087205\\
77.3749276329665	0.984414186994107\\
77.3967295286084	0.984422218974854\\
77.4185314242502	0.984430246816427\\
77.4403333198921	0.984438270520957\\
77.4621352155339	0.984446290090578\\
77.4839371111758	0.984454305527422\\
77.5057390068177	0.984462316833618\\
77.5275409024595	0.984470324011296\\
77.5493427981014	0.984478327062584\\
77.5711446937432	0.984486325989609\\
77.5929465893851	0.984494320794496\\
77.614748485027	0.984502311479371\\
77.6365503806688	0.984510298046357\\
77.6583522763107	0.984518280497577\\
77.6801541719526	0.984526258835152\\
77.7019560675944	0.984534233061203\\
77.7237579632363	0.984542203177848\\
77.7455598588781	0.984550169187207\\
77.76736175452	0.984558131091395\\
77.7891636501619	0.984566088892529\\
77.8109655458037	0.984574042592725\\
77.8327674414456	0.984581992194094\\
77.8545693370875	0.984589937698751\\
77.8763712327293	0.984597879108807\\
77.8981731283712	0.984605816426372\\
77.919975024013	0.984613749653556\\
77.9417769196549	0.984621678792467\\
77.9635788152968	0.984629603845212\\
77.9853807109386	0.984637524813897\\
78.0071826065805	0.984645441700628\\
78.0289845022223	0.984653354507507\\
78.0507863978642	0.984661263236639\\
78.0725882935061	0.984669167890125\\
78.0943901891479	0.984677068470065\\
78.1161920847898	0.984684964978558\\
78.1379939804316	0.984692857417704\\
78.1597958760735	0.9847007457896\\
78.1815977717154	0.984708630096342\\
78.2033996673572	0.984716510340025\\
78.2252015629991	0.984724386522743\\
78.247003458641	0.98473225864659\\
78.2688053542828	0.984740126713657\\
78.2906072499247	0.984747990726035\\
78.3124091455665	0.984755850685814\\
78.3342110412084	0.984763706595083\\
78.3560129368503	0.984771558455929\\
78.3778148324921	0.984779406270438\\
78.399616728134	0.984787250040697\\
78.4214186237758	0.984795089768789\\
78.4432205194177	0.984802925456798\\
78.4650224150596	0.984810757106806\\
78.4868243107014	0.984818584720895\\
78.5086262063433	0.984826408301143\\
78.5304281019852	0.98483422784963\\
78.552229997627	0.984842043368434\\
78.5740318932689	0.984849854859632\\
78.5958337889107	0.9848576623253\\
78.6176356845526	0.984865465767512\\
78.6394375801945	0.984873265188341\\
78.6612394758363	0.984881060589861\\
78.6830413714782	0.984888851974142\\
78.7048432671201	0.984896639343256\\
78.7266451627619	0.984904422699271\\
78.7484470584038	0.984912202044255\\
78.7702489540456	0.984919977380276\\
78.7920508496875	0.9849277487094\\
78.8138527453294	0.984935516033692\\
78.8356546409712	0.984943279355215\\
78.8574565366131	0.984951038676033\\
78.8792584322549	0.984958793998207\\
78.9010603278968	0.984966545323799\\
78.9228622235387	0.984974292654867\\
78.9446641191805	0.98498203599347\\
78.9664660148224	0.984989775341666\\
78.9882679104642	0.984997510701512\\
79.0100698061061	0.985005242075062\\
79.031871701748	0.985012969464371\\
79.0536735973898	0.985020692871493\\
79.0754754930317	0.985028412298479\\
79.0972773886736	0.985036127747381\\
79.1190792843154	0.985043839220248\\
79.1408811799573	0.98505154671913\\
79.1626830755991	0.985059250246076\\
79.184484971241	0.98506694980313\\
79.2062868668829	0.985074645392341\\
79.2280887625247	0.985082337015751\\
79.2498906581666	0.985090024675406\\
79.2716925538084	0.985097708373347\\
79.2934944494503	0.985105388111616\\
79.3152963450922	0.985113063892255\\
79.337098240734	0.985120735717301\\
79.3589001363759	0.985128403588795\\
79.3807020320178	0.985136067508772\\
79.4025039276596	0.98514372747927\\
79.4243058233015	0.985151383502324\\
79.4461077189433	0.985159035579968\\
79.4679096145852	0.985166683714235\\
79.4897115102271	0.985174327907157\\
79.5115134058689	0.985181968160765\\
79.5333153015108	0.98518960447709\\
79.5551171971527	0.985197236858161\\
79.5769190927945	0.985204865306005\\
79.5987209884364	0.985212489822649\\
79.6205228840782	0.985220110410119\\
79.6423247797201	0.985227727070439\\
79.664126675362	0.985235339805635\\
79.6859285710038	0.985242948617727\\
79.7077304666457	0.985250553508739\\
79.7295323622875	0.98525815448069\\
79.7513342579294	0.985265751535599\\
79.7731361535713	0.985273344675487\\
79.7949380492131	0.985280933902369\\
79.816739944855	0.985288519218263\\
79.8385418404968	0.985296100625184\\
79.8603437361387	0.985303678125146\\
79.8821456317806	0.985311251720162\\
79.9039475274224	0.985318821412245\\
79.9257494230643	0.985326387203406\\
79.9475513187062	0.985333949095655\\
79.969353214348	0.985341507091001\\
79.9911551099899	0.985349061191453\\
80.0129570056317	0.985356611399016\\
80.0347589012736	0.985364157715699\\
80.0565607969155	0.985371700143505\\
80.0783626925573	0.985379238684438\\
80.1001645881992	0.985386773340501\\
80.121966483841	0.985394304113697\\
80.1437683794829	0.985401831006026\\
80.1655702751248	0.985409354019487\\
80.1873721707666	0.98541687315608\\
80.2091740664085	0.985424388417803\\
80.2309759620504	0.985431899806652\\
80.2527778576922	0.985439407324622\\
80.2745797533341	0.985446910973709\\
80.2963816489759	0.985454410755906\\
80.3181835446178	0.985461906673206\\
80.3399854402597	0.9854693987276\\
80.3617873359015	0.985476886921079\\
80.3835892315434	0.985484371255633\\
80.4053911271853	0.985491851733249\\
80.4271930228271	0.985499328355915\\
80.448994918469	0.985506801125619\\
80.4707968141108	0.985514270044344\\
80.4925987097527	0.985521735114076\\
80.5144006053946	0.985529196336798\\
80.5362025010364	0.985536653714492\\
80.5580043966783	0.985544107249139\\
80.5798062923201	0.98555155694272\\
80.601608187962	0.985559002797215\\
80.6234100836039	0.9855664448146\\
80.6452119792457	0.985573882996855\\
80.6670138748876	0.985581317345954\\
80.6888157705294	0.985588747863873\\
80.7106176661713	0.985596174552586\\
80.7324195618132	0.985603597414067\\
80.754221457455	0.985611016450287\\
80.7760233530969	0.985618431663217\\
80.7978252487388	0.985625843054829\\
80.8196271443806	0.98563325062709\\
80.8414290400225	0.985640654381968\\
80.8632309356643	0.985648054321432\\
80.8850328313062	0.985655450447447\\
80.9068347269481	0.985662842761977\\
80.9286366225899	0.985670231266987\\
80.9504385182318	0.98567761596444\\
80.9722404138736	0.985684996856298\\
80.9940423095155	0.985692373944522\\
81.0158442051574	0.985699747231072\\
81.0376461007992	0.985707116717906\\
81.0594479964411	0.985714482406983\\
81.081249892083	0.98572184430026\\
81.1030517877248	0.985729202399692\\
81.1248536833667	0.985736556707235\\
81.1466555790085	0.985743907224842\\
81.1684574746504	0.985751253954466\\
81.1902593702923	0.985758596898059\\
81.2120612659341	0.985765936057573\\
81.233863161576	0.985773271434956\\
81.2556650572179	0.985780603032158\\
81.2774669528597	0.985787930851126\\
81.2992688485016	0.985795254893808\\
81.3210707441434	0.98580257516215\\
81.3428726397853	0.985809891658095\\
81.3646745354272	0.985817204383589\\
81.386476431069	0.985824513340573\\
81.4082783267109	0.985831818530991\\
81.4300802223527	0.985839119956781\\
81.4518821179946	0.985846417619885\\
81.4736840136365	0.985853711522241\\
81.4954859092783	0.985861001665788\\
81.5172878049202	0.985868288052461\\
81.539089700562	0.985875570684197\\
81.5608915962039	0.985882849562931\\
81.5826934918458	0.985890124690596\\
81.6044953874876	0.985897396069126\\
81.6262972831295	0.985904663700452\\
81.6480991787714	0.985911927586504\\
81.6699010744132	0.985919187729214\\
81.6917029700551	0.98592644413051\\
81.7135048656969	0.985933696792319\\
81.7353067613388	0.985940945716569\\
81.7571086569807	0.985948190905185\\
81.7789105526225	0.985955432360093\\
81.8007124482644	0.985962670083216\\
81.8225143439062	0.985969904076477\\
81.8443162395481	0.985977134341798\\
81.86611813519	0.9859843608811\\
81.8879200308318	0.985991583696304\\
81.9097219264737	0.985998802789327\\
81.9315238221156	0.986006018162088\\
81.9533257177574	0.986013229816504\\
81.9751276133993	0.986020437754491\\
81.9969295090411	0.986027641977964\\
82.018731404683	0.986034842488837\\
82.0405333003249	0.986042039289023\\
82.0623351959667	0.986049232380434\\
82.0841370916086	0.986056421764981\\
82.1059389872505	0.986063607444574\\
82.1277408828923	0.986070789421122\\
82.1495427785342	0.986077967696534\\
82.171344674176	0.986085142272716\\
82.1931465698179	0.986092313151575\\
82.2149484654598	0.986099480335016\\
82.2367503611016	0.986106643824942\\
82.2585522567435	0.986113803623258\\
82.2803541523853	0.986120959731865\\
82.3021560480272	0.986128112152665\\
82.3239579436691	0.986135260887558\\
82.3457598393109	0.986142405938443\\
82.3675617349528	0.986149547307218\\
82.3893636305946	0.986156684995782\\
82.4111655262365	0.986163819006029\\
82.4329674218784	0.986170949339856\\
82.4547693175202	0.986178075999157\\
82.4765712131621	0.986185198985825\\
82.498373108804	0.986192318301753\\
82.5201750044458	0.986199433948833\\
82.5419769000877	0.986206545928954\\
82.5637787957295	0.986213654244006\\
82.5855806913714	0.986220758895879\\
82.6073825870133	0.986227859886458\\
82.6291844826551	0.986234957217632\\
82.650986378297	0.986242050891285\\
82.6727882739388	0.986249140909303\\
82.6945901695807	0.986256227273569\\
82.7163920652226	0.986263309985966\\
82.7381939608644	0.986270389048375\\
82.7599958565063	0.986277464462677\\
82.7817977521482	0.986284536230753\\
82.80359964779	0.98629160435448\\
82.8254015434319	0.986298668835737\\
82.8472034390737	0.986305729676402\\
82.8690053347156	0.986312786878348\\
82.8908072303575	0.986319840443453\\
82.9126091259993	0.986326890373589\\
82.9344110216412	0.98633393667063\\
82.956212917283	0.986340979336447\\
82.9780148129249	0.986348018372913\\
82.9998167085668	0.986355053781897\\
83.0216186042086	0.986362085565268\\
83.0434204998505	0.986369113724895\\
83.0652223954924	0.986376138262644\\
83.0870242911342	0.986383159180383\\
83.1088261867761	0.986390176479976\\
83.1306280824179	0.986397190163288\\
83.1524299780598	0.986404200232183\\
83.1742318737017	0.986411206688522\\
83.1960337693435	0.986418209534167\\
83.2178356649854	0.98642520877098\\
83.2396375606272	0.986432204400819\\
83.2614394562691	0.986439196425543\\
83.283241351911	0.986446184847009\\
83.3050432475528	0.986453169667075\\
83.3268451431947	0.986460150887597\\
83.3486470388365	0.986467128510428\\
83.3704489344784	0.986474102537424\\
83.3922508301203	0.986481072970436\\
83.4140527257621	0.986488039811317\\
83.435854621404	0.986495003061918\\
83.4576565170459	0.986501962724089\\
83.4794584126877	0.986508918799678\\
83.5012603083296	0.986515871290535\\
83.5230622039714	0.986522820198506\\
83.5448640996133	0.986529765525437\\
83.5666659952552	0.986536707273175\\
83.588467890897	0.986543645443562\\
83.6102697865389	0.986550580038443\\
83.6320716821808	0.98655751105966\\
83.6538735778226	0.986564438509054\\
83.6756754734645	0.986571362388466\\
83.6974773691063	0.986578282699736\\
83.7192792647482	0.986585199444702\\
83.7410811603901	0.986592112625202\\
83.7628830560319	0.986599022243072\\
83.7846849516738	0.98660592830015\\
83.8064868473156	0.986612830798268\\
83.8282887429575	0.986619729739262\\
83.8500906385994	0.986626625124964\\
83.8718925342412	0.986633516957207\\
83.8936944298831	0.98664040523782\\
83.915496325525	0.986647289968636\\
83.9372982211668	0.986654171151482\\
83.9591001168087	0.986661048788187\\
83.9809020124505	0.986667922880579\\
84.0027039080924	0.986674793430483\\
84.0245058037343	0.986681660439725\\
84.0463076993761	0.986688523910131\\
84.068109595018	0.986695383843522\\
84.0899114906598	0.986702240241722\\
84.1117133863017	0.986709093106553\\
84.1335152819436	0.986715942439835\\
84.1553171775854	0.986722788243388\\
84.1771190732273	0.986729630519031\\
84.1989209688691	0.986736469268582\\
84.220722864511	0.986743304493858\\
84.2425247601529	0.986750136196675\\
84.2643266557947	0.986756964378848\\
84.2861285514366	0.986763789042192\\
84.3079304470785	0.986770610188519\\
84.3297323427203	0.986777427819642\\
84.3515342383622	0.986784241937372\\
84.373336134004	0.98679105254352\\
84.3951380296459	0.986797859639896\\
84.4169399252878	0.986804663228308\\
84.4387418209296	0.986811463310562\\
84.4605437165715	0.986818259888468\\
84.4823456122134	0.98682505296383\\
84.5041475078552	0.986831842538452\\
84.5259494034971	0.98683862861414\\
84.5477512991389	0.986845411192696\\
84.5695531947808	0.986852190275922\\
84.5913550904227	0.986858965865619\\
84.6131569860645	0.986865737963587\\
84.6349588817064	0.986872506571627\\
84.6567607773482	0.986879271691536\\
84.6785626729901	0.986886033325111\\
84.700364568632	0.98689279147415\\
84.7221664642738	0.986899546140448\\
84.7439683599157	0.986906297325799\\
84.7657702555576	0.986913045031998\\
84.7875721511994	0.986919789260837\\
84.8093740468413	0.986926530014107\\
84.8311759424831	0.986933267293602\\
84.852977838125	0.986940001101109\\
84.8747797337669	0.986946731438418\\
84.8965816294087	0.986953458307318\\
84.9183835250506	0.986960181709596\\
84.9401854206924	0.986966901647038\\
84.9619873163343	0.98697361812143\\
84.9837892119762	0.986980331134556\\
85.005591107618	0.986987040688201\\
85.0273930032599	0.986993746784145\\
85.0491948989017	0.987000449424173\\
85.0709967945436	0.987007148610063\\
85.0927986901855	0.987013844343597\\
85.1146005858273	0.987020536626553\\
85.1364024814692	0.98702722546071\\
85.1582043771111	0.987033910847844\\
85.1800062727529	0.987040592789732\\
85.2018081683948	0.98704727128815\\
85.2236100640366	0.987053946344872\\
85.2454119596785	0.987060617961671\\
85.2672138553204	0.98706728614032\\
85.2890157509622	0.987073950882591\\
85.3108176466041	0.987080612190255\\
85.332619542246	0.987087270065081\\
85.3544214378878	0.987093924508839\\
85.3762233335297	0.987100575523296\\
85.3980252291715	0.98710722311022\\
85.4198271248134	0.987113867271378\\
85.4416290204553	0.987120508008533\\
85.4634309160971	0.987127145323452\\
85.485232811739	0.987133779217897\\
85.5070347073808	0.987140409693631\\
85.5288366030227	0.987147036752417\\
85.5506384986646	0.987153660396013\\
85.5724403943064	0.987160280626182\\
85.5942422899483	0.987166897444681\\
85.6160441855902	0.987173510853268\\
85.637846081232	0.987180120853702\\
85.6596479768739	0.987186727447737\\
85.6814498725157	0.987193330637131\\
85.7032517681576	0.987199930423636\\
85.7250536637995	0.987206526809007\\
85.7468555594413	0.987213119794996\\
85.7686574550832	0.987219709383355\\
85.790459350725	0.987226295575836\\
85.8122612463669	0.987232878374187\\
85.8340631420088	0.987239457780158\\
85.8558650376506	0.987246033795498\\
85.8776669332925	0.987252606421953\\
85.8994688289343	0.98725917566127\\
85.9212707245762	0.987265741515194\\
85.9430726202181	0.98727230398547\\
85.9648745158599	0.987278863073842\\
85.9866764115018	0.987285418782052\\
86.0084783071437	0.987291971111842\\
86.0302802027855	0.987298520064954\\
86.0520820984274	0.987305065643127\\
86.0738839940692	0.9873116078481\\
86.0956858897111	0.987318146681613\\
86.117487785353	0.987324682145401\\
86.1392896809948	0.987331214241202\\
86.1610915766367	0.987337742970752\\
86.1828934722786	0.987344268335785\\
86.2046953679204	0.987350790338034\\
86.2264972635623	0.987357308979234\\
86.2482991592041	0.987363824261115\\
86.270101054846	0.987370336185409\\
86.2919029504879	0.987376844753847\\
86.3137048461297	0.987383349968157\\
86.3355067417716	0.987389851830069\\
86.3573086374134	0.987396350341309\\
86.3791105330553	0.987402845503605\\
86.4009124286972	0.987409337318682\\
86.422714324339	0.987415825788265\\
86.4445162199809	0.987422310914078\\
86.4663181156228	0.987428792697845\\
86.4881200112646	0.987435271141287\\
86.5099219069065	0.987441746246126\\
86.5317238025483	0.987448218014083\\
86.5535256981902	0.987454686446877\\
86.5753275938321	0.987461151546226\\
86.5971294894739	0.987467613313849\\
86.6189313851158	0.987474071751463\\
86.6407332807576	0.987480526860784\\
86.6625351763995	0.987486978643526\\
86.6843370720414	0.987493427101404\\
86.7061389676832	0.987499872236132\\
86.7279408633251	0.987506314049422\\
86.7497427589669	0.987512752542985\\
86.7715446546088	0.987519187718533\\
86.7933465502507	0.987525619577775\\
86.8151484458925	0.987532048122421\\
86.8369503415344	0.987538473354178\\
86.8587522371763	0.987544895274754\\
86.8805541328181	0.987551313885855\\
86.90235602846	0.987557729189186\\
86.9241579241018	0.987564141186452\\
86.9459598197437	0.987570549879358\\
86.9677617153856	0.987576955269605\\
86.9895636110274	0.987583357358896\\
87.0113655066693	0.987589756148932\\
87.0331674023112	0.987596151641412\\
87.054969297953	0.987602543838038\\
87.0767711935949	0.987608932740506\\
87.0985730892367	0.987615318350514\\
87.1203749848786	0.98762170066976\\
87.1421768805205	0.987628079699938\\
87.1639787761623	0.987634455442745\\
87.1857806718042	0.987640827899874\\
87.207582567446	0.987647197073018\\
};
\addplot [color=mycolor1,solid]
  table[row sep=crcr]{%
87.207582567446	0.987647197073018\\
87.2293844630879	0.98765356296387\\
87.2511863587298	0.987659925574121\\
87.2729882543716	0.987666284905462\\
87.2947901500135	0.987672640959582\\
87.3165920456553	0.987678993738171\\
87.3383939412972	0.987685343242916\\
87.3601958369391	0.987691689475505\\
87.3819977325809	0.987698032437624\\
87.4037996282228	0.987704372130958\\
87.4256015238647	0.987710708557191\\
87.4474034195065	0.987717041718009\\
87.4692053151484	0.987723371615092\\
87.4910072107902	0.987729698250123\\
87.5128091064321	0.987736021624784\\
87.534611002074	0.987742341740754\\
87.5564128977158	0.987748658599713\\
87.5782147933577	0.987754972203339\\
87.6000166889995	0.98776128255331\\
87.6218185846414	0.987767589651303\\
87.6436204802833	0.987773893498992\\
87.6654223759251	0.987780194098055\\
87.687224271567	0.987786491450164\\
87.7090261672089	0.987792785556992\\
87.7308280628507	0.987799076420213\\
87.7526299584926	0.987805364041498\\
87.7744318541344	0.987811648422518\\
87.7962337497763	0.987817929564942\\
87.8180356454182	0.987824207470439\\
87.83983754106	0.987830482140678\\
87.8616394367019	0.987836753577325\\
87.8834413323438	0.987843021782048\\
87.9052432279856	0.987849286756511\\
87.9270451236275	0.98785554850238\\
87.9488470192693	0.987861807021318\\
87.9706489149112	0.987868062314988\\
87.9924508105531	0.987874314385053\\
88.0142527061949	0.987880563233173\\
88.0360546018368	0.987886808861009\\
88.0578564974786	0.987893051270221\\
88.0796583931205	0.987899290462466\\
88.1014602887624	0.987905526439404\\
88.1232621844042	0.987911759202691\\
88.1450640800461	0.987917988753984\\
88.1668659756879	0.987924215094936\\
88.1886678713298	0.987930438227203\\
88.2104697669717	0.987936658152439\\
88.2322716626135	0.987942874872296\\
88.2540735582554	0.987949088388427\\
88.2758754538973	0.987955298702481\\
88.2976773495391	0.987961505816109\\
88.319479245181	0.987967709730961\\
88.3412811408228	0.987973910448685\\
88.3630830364647	0.987980107970928\\
88.3848849321066	0.987986302299338\\
88.4066868277484	0.987992493435561\\
88.4284887233903	0.98799868138124\\
88.4502906190321	0.988004866138021\\
88.472092514674	0.988011047707548\\
88.4938944103159	0.988017226091462\\
88.5156963059577	0.988023401291405\\
88.5374982015996	0.988029573309018\\
88.5593000972414	0.988035742145941\\
88.5811019928833	0.988041907803814\\
88.6029038885252	0.988048070284273\\
88.624705784167	0.988054229588958\\
88.6465076798089	0.988060385719505\\
88.6683095754508	0.988066538677549\\
88.6901114710926	0.988072688464725\\
88.7119133667345	0.988078835082667\\
88.7337152623764	0.98808497853301\\
88.7555171580182	0.988091118817384\\
88.7773190536601	0.988097255937422\\
88.7991209493019	0.988103389894754\\
88.8209228449438	0.98810952069101\\
88.8427247405857	0.98811564832782\\
88.8645266362275	0.988121772806811\\
88.8863285318694	0.988127894129611\\
88.9081304275112	0.988134012297847\\
88.9299323231531	0.988140127313143\\
88.951734218795	0.988146239177126\\
88.9735361144368	0.988152347891418\\
88.9953380100787	0.988158453457643\\
89.0171399057205	0.988164555877424\\
89.0389418013624	0.988170655152382\\
89.0607436970043	0.988176751284137\\
89.0825455926461	0.98818284427431\\
89.104347488288	0.988188934124519\\
89.1261493839299	0.988195020836383\\
89.1479512795717	0.988201104411519\\
89.1697531752136	0.988207184851543\\
89.1915550708554	0.988213262158072\\
89.2133569664973	0.98821933633272\\
89.2351588621392	0.9882254073771\\
89.256960757781	0.988231475292827\\
89.2787626534229	0.988237540081512\\
89.3005645490647	0.988243601744767\\
89.3223664447066	0.988249660284202\\
89.3441683403485	0.988255715701429\\
89.3659702359903	0.988261767998054\\
89.3877721316322	0.988267817175688\\
89.409574027274	0.988273863235937\\
89.4313759229159	0.988279906180407\\
89.4531778185578	0.988285946010704\\
89.4749797141996	0.988291982728434\\
89.4967816098415	0.9882980163352\\
89.5185835054834	0.988304046832606\\
89.5403854011252	0.988310074222253\\
89.5621872967671	0.988316098505744\\
89.5839891924089	0.988322119684679\\
89.6057910880508	0.988328137760658\\
89.6275929836927	0.98833415273528\\
89.6493948793345	0.988340164610144\\
89.6711967749764	0.988346173386846\\
89.6929986706183	0.988352179066985\\
89.7148005662601	0.988358181652154\\
89.736602461902	0.98836418114395\\
89.7584043575438	0.988370177543967\\
89.7802062531857	0.988376170853797\\
89.8020081488276	0.988382161075034\\
89.8238100444694	0.988388148209269\\
89.8456119401113	0.988394132258092\\
89.8674138357531	0.988400113223095\\
89.889215731395	0.988406091105866\\
89.9110176270369	0.988412065907994\\
89.9328195226787	0.988418037631065\\
89.9546214183206	0.988424006276668\\
89.9764233139625	0.988429971846388\\
89.9982252096043	0.98843593434181\\
90.0200271052462	0.988441893764518\\
90.041829000888	0.988447850116097\\
90.0636308965299	0.988453803398128\\
90.0854327921718	0.988459753612194\\
90.1072346878136	0.988465700759875\\
90.1290365834555	0.988471644842752\\
90.1508384790973	0.988477585862405\\
90.1726403747392	0.988483523820411\\
90.1944422703811	0.988489458718349\\
90.2162441660229	0.988495390557796\\
90.2380460616648	0.988501319340328\\
90.2598479573066	0.988507245067521\\
90.2816498529485	0.988513167740948\\
90.3034517485904	0.988519087362183\\
90.3252536442322	0.988525003932801\\
90.3470555398741	0.988530917454371\\
90.368857435516	0.988536827928467\\
90.3906593311578	0.988542735356659\\
90.4124612267997	0.988548639740515\\
90.4342631224415	0.988554541081606\\
90.4560650180834	0.988560439381499\\
90.4778669137253	0.988566334641761\\
90.4996688093671	0.988572226863959\\
90.521470705009	0.988578116049659\\
90.5432726006509	0.988584002200425\\
90.5650744962927	0.988589885317822\\
90.5868763919346	0.988595765403412\\
90.6086782875764	0.988601642458758\\
90.6304801832183	0.988607516485423\\
90.6522820788602	0.988613387484965\\
90.674083974502	0.988619255458947\\
90.6958858701439	0.988625120408926\\
90.7176877657857	0.988630982336461\\
90.7394896614276	0.98863684124311\\
90.7612915570695	0.98864269713043\\
90.7830934527113	0.988648549999976\\
90.8048953483532	0.988654399853305\\
90.8266972439951	0.988660246691969\\
90.8484991396369	0.988666090517523\\
90.8703010352788	0.98867193133152\\
90.8921029309206	0.988677769135512\\
90.9139048265625	0.988683603931049\\
90.9357067222044	0.988689435719682\\
90.9575086178462	0.988695264502962\\
90.9793105134881	0.988701090282436\\
91.0011124091299	0.988706913059652\\
91.0229143047718	0.988712732836158\\
91.0447162004137	0.988718549613501\\
91.0665180960555	0.988724363393225\\
91.0883199916974	0.988730174176876\\
91.1101218873392	0.988735981965997\\
91.1319237829811	0.988741786762133\\
91.153725678623	0.988747588566825\\
91.1755275742648	0.988753387381614\\
91.1973294699067	0.988759183208042\\
91.2191313655486	0.98876497604765\\
91.2409332611904	0.988770765901975\\
91.2627351568323	0.988776552772556\\
91.2845370524741	0.988782336660932\\
91.306338948116	0.988788117568639\\
91.3281408437579	0.988793895497213\\
91.3499427393997	0.98879967044819\\
91.3717446350416	0.988805442423103\\
91.3935465306835	0.988811211423487\\
91.4153484263253	0.988816977450874\\
91.4371503219672	0.988822740506797\\
91.458952217609	0.988828500592787\\
91.4807541132509	0.988834257710374\\
91.5025560088928	0.988840011861089\\
91.5243579045346	0.98884576304646\\
91.5461598001765	0.988851511268015\\
91.5679616958183	0.988857256527282\\
91.5897635914602	0.988862998825787\\
91.6115654871021	0.988868738165057\\
91.6333673827439	0.988874474546616\\
91.6551692783858	0.988880207971988\\
91.6769711740277	0.988885938442697\\
91.6987730696695	0.988891665960266\\
91.7205749653114	0.988897390526216\\
91.7423768609532	0.988903112142069\\
91.7641787565951	0.988908830809345\\
91.785980652237	0.988914546529563\\
91.8077825478788	0.988920259304243\\
91.8295844435207	0.988925969134901\\
91.8513863391625	0.988931676023056\\
91.8731882348044	0.988937379970224\\
91.8949901304463	0.98894308097792\\
91.9167920260881	0.98894877904766\\
91.93859392173	0.988954474180957\\
91.9603958173718	0.988960166379324\\
91.9821977130137	0.988965855644274\\
92.0039996086556	0.98897154197732\\
92.0258015042974	0.98897722537997\\
92.0476033999393	0.988982905853737\\
92.0694052955812	0.988988583400129\\
92.091207191223	0.988994258020655\\
92.1130090868649	0.988999929716823\\
92.1348109825067	0.98900559849014\\
92.1566128781486	0.989011264342112\\
92.1784147737905	0.989016927274244\\
92.2002166694323	0.989022587288041\\
92.2220185650742	0.989028244385008\\
92.2438204607161	0.989033898566647\\
92.2656223563579	0.989039549834461\\
92.2874242519998	0.989045198189952\\
92.3092261476416	0.989050843634619\\
92.3310280432835	0.989056486169965\\
92.3528299389254	0.989062125797486\\
92.3746318345672	0.989067762518683\\
92.3964337302091	0.989073396335053\\
92.4182356258509	0.989079027248093\\
92.4400375214928	0.989084655259299\\
92.4618394171347	0.989090280370167\\
92.4836413127765	0.989095902582191\\
92.5054432084184	0.989101521896865\\
92.5272451040602	0.989107138315683\\
92.5490469997021	0.989112751840136\\
92.570848895344	0.989118362471717\\
92.5926507909858	0.989123970211915\\
92.6144526866277	0.989129575062222\\
92.6362545822696	0.989135177024126\\
92.6580564779114	0.989140776099116\\
92.6798583735533	0.98914637228868\\
92.7016602691951	0.989151965594305\\
92.723462164837	0.989157556017476\\
92.7452640604789	0.98916314355968\\
92.7670659561207	0.989168728222402\\
92.7888678517626	0.989174310007123\\
92.8106697474044	0.98917988891533\\
92.8324716430463	0.989185464948502\\
92.8542735386882	0.989191038108123\\
92.87607543433	0.989196608395673\\
92.8978773299719	0.989202175812633\\
92.9196792256138	0.98920774036048\\
92.9414811212556	0.989213302040695\\
92.9632830168975	0.989218860854755\\
92.9850849125393	0.989224416804137\\
93.0068868081812	0.989229969890316\\
93.0286887038231	0.98923552011477\\
93.0504905994649	0.989241067478972\\
93.0722924951068	0.989246611984397\\
93.0940943907487	0.989252153632518\\
93.1158962863905	0.989257692424807\\
93.1376981820324	0.989263228362736\\
93.1595000776742	0.989268761447776\\
93.1813019733161	0.989274291681397\\
93.203103868958	0.989279819065069\\
93.2249057645998	0.98928534360026\\
93.2467076602417	0.989290865288439\\
93.2685095558835	0.989296384131072\\
93.2903114515254	0.989301900129626\\
93.3121133471673	0.989307413285567\\
93.3339152428091	0.989312923600359\\
93.355717138451	0.989318431075468\\
93.3775190340928	0.989323935712355\\
93.3993209297347	0.989329437512484\\
93.4211228253766	0.989334936477317\\
93.4429247210184	0.989340432608314\\
93.4647266166603	0.989345925906937\\
93.4865285123022	0.989351416374645\\
93.508330407944	0.989356904012896\\
93.5301323035859	0.98936238882315\\
93.5519341992277	0.989367870806863\\
93.5737360948696	0.989373349965491\\
93.5955379905115	0.989378826300492\\
93.6173398861533	0.989384299813319\\
93.6391417817952	0.989389770505428\\
93.660943677437	0.989395238378272\\
93.6827455730789	0.989400703433304\\
93.7045474687208	0.989406165671975\\
93.7263493643626	0.989411625095739\\
93.7481512600045	0.989417081706044\\
93.7699531556463	0.989422535504341\\
93.7917550512882	0.98942798649208\\
93.8135569469301	0.989433434670707\\
93.8353588425719	0.989438880041672\\
93.8571607382138	0.989444322606421\\
93.8789626338557	0.9894497623664\\
93.9007645294975	0.989455199323055\\
93.9225664251394	0.98946063347783\\
93.9443683207813	0.98946606483217\\
93.9661702164231	0.989471493387517\\
93.987972112065	0.989476919145313\\
94.0097740077068	0.989482342107001\\
94.0315759033487	0.989487762274022\\
94.0533777989906	0.989493179647816\\
94.0751796946324	0.989498594229821\\
94.0969815902743	0.989504006021478\\
94.1187834859161	0.989509415024223\\
94.140585381558	0.989514821239495\\
94.1623872771999	0.989520224668729\\
94.1841891728417	0.989525625313361\\
94.2059910684836	0.989531023174827\\
94.2277929641254	0.989536418254561\\
94.2495948597673	0.989541810553996\\
94.2713967554092	0.989547200074565\\
94.293198651051	0.9895525868177\\
94.3150005466929	0.989557970784832\\
94.3368024423348	0.989563351977393\\
94.3586043379766	0.989568730396811\\
94.3804062336185	0.989574106044516\\
94.4022081292603	0.989579478921937\\
94.4240100249022	0.989584849030501\\
94.4458119205441	0.989590216371634\\
94.4676138161859	0.989595580946764\\
94.4894157118278	0.989600942757315\\
94.5112176074696	0.989606301804713\\
94.5330195031115	0.989611658090381\\
94.5548213987534	0.989617011615742\\
94.5766232943952	0.989622362382219\\
94.5984251900371	0.989627710391234\\
94.6202270856789	0.989633055644207\\
94.6420289813208	0.98963839814256\\
94.6638308769627	0.989643737887711\\
94.6856327726045	0.989649074881079\\
94.7074346682464	0.989654409124084\\
94.7292365638883	0.989659740618141\\
94.7510384595301	0.989665069364667\\
94.772840355172	0.989670395365079\\
94.7946422508139	0.989675718620792\\
94.8164441464557	0.98968103913322\\
94.8382460420976	0.989686356903777\\
94.8600479377394	0.989691671933875\\
94.8818498333813	0.989696984224928\\
94.9036517290232	0.989702293778346\\
94.925453624665	0.989707600595541\\
94.9472555203069	0.989712904677923\\
94.9690574159487	0.9897182060269\\
94.9908593115906	0.989723504643882\\
95.0126612072325	0.989728800530276\\
95.0344631028743	0.98973409368749\\
95.0562649985162	0.98973938411693\\
95.078066894158	0.989744671820002\\
95.0998687897999	0.989749956798111\\
95.1216706854418	0.989755239052661\\
95.1434725810836	0.989760518585055\\
95.1652744767255	0.989765795396698\\
95.1870763723674	0.989771069488989\\
95.2088782680092	0.989776340863332\\
95.2306801636511	0.989781609521127\\
95.2524820592929	0.989786875463773\\
95.2742839549348	0.98979213869267\\
95.2960858505767	0.989797399209216\\
95.3178877462185	0.98980265701481\\
95.3396896418604	0.989807912110848\\
95.3614915375022	0.989813164498726\\
95.3832934331441	0.989818414179841\\
95.405095328786	0.989823661155587\\
95.4268972244278	0.989828905427358\\
95.4486991200697	0.989834146996547\\
95.4705010157115	0.989839385864549\\
95.4923029113534	0.989844622032753\\
95.5141048069953	0.989849855502553\\
95.5359067026371	0.989855086275338\\
95.557708598279	0.989860314352498\\
95.5795104939209	0.989865539735423\\
95.6013123895627	0.989870762425501\\
95.6231142852046	0.98987598242412\\
95.6449161808465	0.989881199732666\\
95.6667180764883	0.989886414352526\\
95.6885199721302	0.989891626285086\\
95.710321867772	0.98989683553173\\
95.7321237634139	0.989902042093843\\
95.7539256590558	0.989907245972808\\
95.7757275546976	0.989912447170007\\
95.7975294503395	0.989917645686824\\
95.8193313459813	0.989922841524638\\
95.8411332416232	0.989928034684832\\
95.8629351372651	0.989933225168784\\
95.8847370329069	0.989938412977874\\
95.9065389285488	0.989943598113479\\
95.9283408241906	0.989948780576979\\
95.9501427198325	0.98995396036975\\
95.9719446154744	0.989959137493168\\
95.9937465111162	0.98996431194861\\
96.0155484067581	0.989969483737449\\
96.0373503024	0.98997465286106\\
96.0591521980418	0.989979819320817\\
96.0809540936837	0.989984983118092\\
96.1027559893255	0.989990144254258\\
96.1245578849674	0.989995302730686\\
96.1463597806093	0.990000458548746\\
96.1681616762511	0.990005611709808\\
96.189963571893	0.990010762215243\\
96.2117654675348	0.990015910066417\\
96.2335673631767	0.9900210552647\\
96.2553692588186	0.990026197811458\\
96.2771711544604	0.990031337708058\\
96.2989730501023	0.990036474955865\\
96.3207749457441	0.990041609556245\\
96.342576841386	0.990046741510561\\
96.3643787370279	0.990051870820178\\
96.3861806326697	0.990056997486458\\
96.4079825283116	0.990062121510764\\
96.4297844239535	0.990067242894457\\
96.4515863195953	0.990072361638898\\
96.4733882152372	0.990077477745447\\
96.495190110879	0.990082591215463\\
96.5169920065209	0.990087702050305\\
96.5387939021628	0.990092810251331\\
96.5605957978046	0.990097915819899\\
96.5823976934465	0.990103018757365\\
96.6041995890884	0.990108119065084\\
96.6260014847302	0.990113216744413\\
96.6478033803721	0.990118311796706\\
96.6696052760139	0.990123404223316\\
96.6914071716558	0.990128494025596\\
96.7132090672977	0.9901335812049\\
96.7350109629395	0.990138665762578\\
96.7568128585814	0.990143747699982\\
96.7786147542232	0.990148827018463\\
96.8004166498651	0.990153903719369\\
96.822218545507	0.990158977804049\\
96.8440204411488	0.990164049273853\\
96.8658223367907	0.990169118130127\\
96.8876242324326	0.990174184374218\\
96.9094261280744	0.990179248007473\\
96.9312280237163	0.990184309031237\\
96.9530299193581	0.990189367446854\\
96.974831815	0.99019442325567\\
96.9966337106419	0.990199476459026\\
97.0184356062837	0.990204527058267\\
97.0402375019256	0.990209575054733\\
97.0620393975674	0.990214620449766\\
97.0838412932093	0.990219663244707\\
97.1056431888512	0.990224703440896\\
97.127445084493	0.990229741039672\\
97.1492469801349	0.990234776042374\\
97.1710488757767	0.990239808450339\\
97.1928507714186	0.990244838264904\\
97.2146526670605	0.990249865487407\\
97.2364545627023	0.990254890119182\\
97.2582564583442	0.990259912161565\\
97.2800583539861	0.990264931615891\\
97.3018602496279	0.990269948483493\\
97.3236621452698	0.990274962765703\\
97.3454640409116	0.990279974463856\\
97.3672659365535	0.990284983579281\\
97.3890678321954	0.990289990113311\\
97.4108697278372	0.990294994067275\\
97.4326716234791	0.990299995442503\\
97.454473519121	0.990304994240323\\
97.4762754147628	0.990309990462065\\
97.4980773104047	0.990314984109056\\
97.5198792060465	0.990319975182623\\
97.5416811016884	0.990324963684091\\
97.5634829973303	0.990329949614786\\
97.5852848929721	0.990334932976034\\
97.607086788614	0.990339913769158\\
97.6288886842558	0.990344891995481\\
97.6506905798977	0.990349867656327\\
97.6724924755396	0.990354840753018\\
97.6942943711814	0.990359811286875\\
97.7160962668233	0.990364779259218\\
97.7378981624651	0.990369744671368\\
97.759700058107	0.990374707524644\\
97.7815019537489	0.990379667820365\\
97.8033038493907	0.990384625559849\\
97.8251057450326	0.990389580744413\\
97.8469076406745	0.990394533375373\\
97.8687095363163	0.990399483454047\\
97.8905114319582	0.990404430981748\\
97.9123133276	0.990409375959792\\
97.9341152232419	0.990414318389493\\
97.9559171188838	0.990419258272164\\
97.9777190145256	0.990424195609117\\
97.9995209101675	0.990429130401665\\
98.0213228058093	0.990434062651118\\
98.0431247014512	0.990438992358787\\
98.0649265970931	0.990443919525982\\
98.0867284927349	0.990448844154012\\
98.1085303883768	0.990453766244187\\
98.1303322840187	0.990458685797813\\
98.1521341796605	0.990463602816197\\
98.1739360753024	0.990468517300647\\
98.1957379709442	0.990473429252468\\
98.2175398665861	0.990478338672966\\
98.239341762228	0.990483245563444\\
98.2611436578698	0.990488149925207\\
98.2829455535117	0.990493051759557\\
98.3047474491536	0.990497951067798\\
98.3265493447954	0.990502847851231\\
98.3483512404373	0.990507742111157\\
98.3701531360791	0.990512633848877\\
98.391955031721	0.99051752306569\\
98.4137569273629	0.990522409762896\\
98.4355588230047	0.990527293941792\\
98.4573607186466	0.990532175603677\\
98.4791626142884	0.990537054749849\\
98.5009645099303	0.990541931381602\\
98.5227664055722	0.990546805500234\\
98.544568301214	0.990551677107039\\
98.5663701968559	0.990556546203311\\
98.5881720924977	0.990561412790345\\
98.6099739881396	0.990566276869434\\
98.6317758837815	0.990571138441869\\
98.6535777794233	0.990575997508944\\
98.6753796750652	0.990580854071948\\
98.6971815707071	0.990585708132172\\
98.7189834663489	0.990590559690907\\
98.7407853619908	0.990595408749441\\
98.7625872576326	0.990600255309062\\
98.7843891532745	0.990605099371059\\
98.8061910489164	0.990609940936719\\
98.8279929445582	0.990614780007327\\
98.8497948402001	0.99061961658417\\
98.8715967358419	0.990624450668534\\
98.8933986314838	0.990629282261701\\
98.9152005271257	0.990634111364957\\
98.9370024227675	0.990638937979585\\
98.9588043184094	0.990643762106866\\
98.9806062140512	0.990648583748083\\
99.0024081096931	0.990653402904517\\
99.024210005335	0.990658219577448\\
99.0460119009768	0.990663033768156\\
99.0678137966187	0.990667845477921\\
99.0896156922606	0.990672654708021\\
99.1114175879024	0.990677461459734\\
99.1332194835443	0.990682265734336\\
99.1550213791862	0.990687067533105\\
99.176823274828	0.990691866857317\\
99.1986251704699	0.990696663708246\\
99.2204270661117	0.990701458087168\\
99.2422289617536	0.990706249995355\\
99.2640308573955	0.990711039434083\\
99.2858327530373	0.990715826404622\\
99.3076346486792	0.990720610908245\\
99.329436544321	0.990725392946224\\
99.3512384399629	0.990730172519828\\
99.3730403356048	0.990734949630329\\
99.3948422312466	0.990739724278995\\
99.4166441268885	0.990744496467094\\
99.4384460225303	0.990749266195896\\
99.4602479181722	0.990754033466667\\
99.4820498138141	0.990758798280674\\
99.5038517094559	0.990763560639184\\
99.5256536050978	0.990768320543461\\
99.5474555007397	0.99077307799477\\
99.5692573963815	0.990777832994376\\
99.5910592920234	0.990782585543541\\
99.6128611876652	0.990787335643529\\
99.6346630833071	0.990792083295603\\
99.656464978949	0.990796828501022\\
99.6782668745908	0.990801571261049\\
99.7000687702327	0.990806311576943\\
99.7218706658745	0.990811049449964\\
99.7436725615164	0.990815784881371\\
99.7654744571583	0.990820517872422\\
99.7872763528001	0.990825248424375\\
99.809078248442	0.990829976538486\\
99.8308801440838	0.990834702216013\\
99.8526820397257	0.99083942545821\\
99.8744839353676	0.990844146266333\\
99.8962858310094	0.990848864641636\\
99.9180877266513	0.990853580585373\\
99.9398896222932	0.990858294098797\\
99.961691517935	0.99086300518316\\
99.9834934135769	0.990867713839714\\
100.005295309219	0.990872420069711\\
100.027097204861	0.9908771238744\\
100.048899100502	0.990881825255032\\
100.070700996144	0.990886524212856\\
100.092502891786	0.99089122074912\\
100.114304787428	0.990895914865072\\
100.13610668307	0.99090060656196\\
100.157908578712	0.99090529584103\\
100.179710474354	0.990909982703529\\
100.201512369995	0.990914667150701\\
100.223314265637	0.990919349183791\\
100.245116161279	0.990924028804044\\
100.266918056921	0.990928706012703\\
100.288719952563	0.99093338081101\\
100.310521848205	0.990938053200208\\
100.332323743847	0.990942723181538\\
100.354125639489	0.990947390756241\\
100.37592753513	0.990952055925558\\
100.397729430772	0.990956718690728\\
100.419531326414	0.990961379052989\\
100.441333222056	0.990966037013581\\
100.463135117698	0.99097069257374\\
100.48493701334	0.990975345734705\\
100.506738908982	0.99097999649771\\
100.528540804623	0.990984644863993\\
100.550342700265	0.990989290834788\\
100.572144595907	0.99099393441133\\
100.593946491549	0.990998575594852\\
100.615748387191	0.991003214386588\\
100.637550282833	0.991007850787771\\
100.659352178475	0.991012484799631\\
100.681154074116	0.991017116423401\\
100.702955969758	0.991021745660312\\
100.7247578654	0.991026372511593\\
100.746559761042	0.991030996978473\\
100.768361656684	0.991035619062182\\
100.790163552326	0.991040238763948\\
100.811965447968	0.991044856084998\\
100.833767343609	0.991049471026558\\
100.855569239251	0.991054083589856\\
100.877371134893	0.991058693776118\\
100.899173030535	0.991063301586566\\
100.920974926177	0.991067907022427\\
100.942776821819	0.991072510084924\\
100.964578717461	0.99107711077528\\
100.986380613103	0.991081709094717\\
101.008182508744	0.991086305044458\\
101.029984404386	0.991090898625723\\
101.051786300028	0.991095489839734\\
101.07358819567	0.991100078687709\\
101.095390091312	0.991104665170868\\
101.117191986954	0.991109249290431\\
101.138993882596	0.991113831047615\\
101.160795778237	0.991118410443637\\
101.182597673879	0.991122987479714\\
101.204399569521	0.991127562157063\\
101.226201465163	0.991132134476899\\
101.248003360805	0.991136704440436\\
101.269805256447	0.99114127204889\\
101.291607152089	0.991145837303474\\
101.31340904773	0.991150400205401\\
101.335210943372	0.991154960755883\\
101.357012839014	0.991159518956132\\
101.378814734656	0.991164074807359\\
101.400616630298	0.991168628310775\\
101.42241852594	0.99117317946759\\
101.444220421582	0.991177728279013\\
101.466022317223	0.991182274746253\\
101.487824212865	0.991186818870518\\
101.509626108507	0.991191360653014\\
101.531428004149	0.99119590009495\\
101.553229899791	0.99120043719753\\
101.575031795433	0.991204971961962\\
101.596833691075	0.99120950438945\\
101.618635586716	0.991214034481197\\
101.640437482358	0.991218562238408\\
101.662239378	0.991223087662286\\
101.684041273642	0.991227610754032\\
101.705843169284	0.99123213151485\\
101.727645064926	0.991236649945941\\
101.749446960568	0.991241166048504\\
101.77124885621	0.99124567982374\\
101.793050751851	0.991250191272849\\
101.814852647493	0.991254700397028\\
101.836654543135	0.991259207197477\\
101.858456438777	0.991263711675392\\
101.880258334419	0.99126821383197\\
101.902060230061	0.991272713668409\\
101.923862125703	0.991277211185903\\
101.945664021344	0.991281706385647\\
101.967465916986	0.991286199268836\\
101.989267812628	0.991290689836664\\
102.01106970827	0.991295178090324\\
102.032871603912	0.991299664031009\\
102.054673499554	0.991304147659909\\
102.076475395196	0.991308628978218\\
102.098277290837	0.991313107987126\\
102.120079186479	0.991317584687822\\
102.141881082121	0.991322059081496\\
102.163682977763	0.991326531169337\\
102.185484873405	0.991331000952534\\
102.207286769047	0.991335468432274\\
102.229088664689	0.991339933609744\\
102.25089056033	0.991344396486131\\
102.272692455972	0.99134885706262\\
102.294494351614	0.991353315340396\\
102.316296247256	0.991357771320645\\
102.338098142898	0.99136222500455\\
102.35990003854	0.991366676393295\\
102.381701934182	0.991371125488063\\
102.403503829824	0.991375572290035\\
102.425305725465	0.991380016800393\\
102.447107621107	0.991384459020318\\
102.468909516749	0.991388898950991\\
102.490711412391	0.991393336593591\\
102.512513308033	0.991397771949298\\
102.534315203675	0.991402205019289\\
102.556117099317	0.991406635804743\\
102.577918994958	0.991411064306838\\
102.5997208906	0.991415490526749\\
102.621522786242	0.991419914465653\\
102.643324681884	0.991424336124726\\
102.665126577526	0.991428755505142\\
102.686928473168	0.991433172608075\\
102.70873036881	0.9914375874347\\
102.730532264451	0.991441999986189\\
102.752334160093	0.991446410263715\\
102.774136055735	0.99145081826845\\
102.795937951377	0.991455224001564\\
102.817739847019	0.99145962746423\\
102.839541742661	0.991464028657616\\
102.861343638303	0.991468427582892\\
102.883145533944	0.991472824241227\\
102.904947429586	0.991477218633789\\
102.926749325228	0.991481610761746\\
102.94855122087	0.991486000626265\\
102.970353116512	0.991490388228512\\
102.992155012154	0.991494773569653\\
103.013956907796	0.991499156650854\\
103.035758803438	0.991503537473279\\
103.057560699079	0.991507916038092\\
103.079362594721	0.991512292346456\\
103.101164490363	0.991516666399535\\
103.122966386005	0.99152103819849\\
103.144768281647	0.991525407744484\\
103.166570177289	0.991529775038677\\
103.188372072931	0.991534140082229\\
103.210173968572	0.991538502876302\\
103.231975864214	0.991542863422052\\
103.253777759856	0.991547221720641\\
103.275579655498	0.991551577773224\\
103.29738155114	0.991555931580961\\
103.319183446782	0.991560283145008\\
103.340985342424	0.99156463246652\\
103.362787238065	0.991568979546655\\
103.384589133707	0.991573324386566\\
103.406391029349	0.991577666987408\\
103.428192924991	0.991582007350336\\
103.449994820633	0.991586345476502\\
103.471796716275	0.99159068136706\\
103.493598611917	0.99159501502316\\
103.515400507558	0.991599346445955\\
103.5372024032	0.991603675636596\\
103.559004298842	0.991608002596233\\
103.580806194484	0.991612327326016\\
103.602608090126	0.991616649827093\\
103.624409985768	0.991620970100614\\
103.64621188141	0.991625288147726\\
103.668013777051	0.991629603969576\\
103.689815672693	0.991633917567313\\
103.711617568335	0.99163822894208\\
103.733419463977	0.991642538095025\\
103.755221359619	0.991646845027293\\
103.777023255261	0.991651149740026\\
103.798825150903	0.991655452234371\\
103.820627046545	0.991659752511469\\
103.842428942186	0.991664050572463\\
103.864230837828	0.991668346418495\\
103.88603273347	0.991672640050707\\
103.907834629112	0.99167693147024\\
103.929636524754	0.991681220678234\\
103.951438420396	0.991685507675828\\
103.973240316038	0.991689792464162\\
103.995042211679	0.991694075044375\\
104.016844107321	0.991698355417603\\
104.038646002963	0.991702633584985\\
104.060447898605	0.991706909547658\\
104.082249794247	0.991711183306756\\
104.104051689889	0.991715454863417\\
104.125853585531	0.991719724218775\\
104.147655481172	0.991723991373964\\
104.169457376814	0.991728256330119\\
104.191259272456	0.991732519088372\\
104.213061168098	0.991736779649857\\
104.23486306374	0.991741038015704\\
104.256664959382	0.991745294187047\\
104.278466855024	0.991749548165015\\
104.300268750665	0.991753799950739\\
104.322070646307	0.991758049545349\\
104.343872541949	0.991762296949974\\
104.365674437591	0.991766542165743\\
104.387476333233	0.991770785193783\\
104.409278228875	0.991775026035222\\
104.431080124517	0.991779264691187\\
104.452882020158	0.991783501162804\\
104.4746839158	0.991787735451199\\
104.496485811442	0.991791967557497\\
104.518287707084	0.991796197482822\\
104.540089602726	0.991800425228299\\
104.561891498368	0.99180465079505\\
104.58369339401	0.991808874184199\\
104.605495289652	0.991813095396867\\
104.627297185293	0.991817314434176\\
104.649099080935	0.991821531297248\\
104.670900976577	0.991825745987203\\
104.692702872219	0.991829958505161\\
104.714504767861	0.99183416885224\\
104.736306663503	0.99183837702956\\
104.758108559145	0.99184258303824\\
104.779910454786	0.991846786879396\\
104.801712350428	0.991850988554145\\
104.82351424607	0.991855188063604\\
104.845316141712	0.99185938540889\\
104.867118037354	0.991863580591116\\
104.888919932996	0.991867773611398\\
104.910721828638	0.99187196447085\\
104.932523724279	0.991876153170586\\
104.954325619921	0.991880339711718\\
104.976127515563	0.991884524095359\\
104.997929411205	0.991888706322621\\
105.019731306847	0.991892886394615\\
105.041533202489	0.991897064312451\\
105.063335098131	0.991901240077241\\
105.085136993772	0.991905413690092\\
105.106938889414	0.991909585152116\\
105.128740785056	0.991913754464419\\
105.150542680698	0.991917921628109\\
105.17234457634	0.991922086644295\\
105.194146471982	0.991926249514082\\
105.215948367624	0.991930410238577\\
105.237750263266	0.991934568818885\\
105.259552158907	0.991938725256112\\
105.281354054549	0.991942879551361\\
105.303155950191	0.991947031705736\\
105.324957845833	0.991951181720342\\
105.346759741475	0.99195532959628\\
105.368561637117	0.991959475334652\\
105.390363532759	0.991963618936561\\
105.4121654284	0.991967760403107\\
105.433967324042	0.99197189973539\\
105.455769219684	0.991976036934512\\
105.477571115326	0.99198017200157\\
105.499373010968	0.991984304937663\\
105.52117490661	0.99198843574389\\
105.542976802252	0.991992564421349\\
105.564778697893	0.991996690971136\\
105.586580593535	0.992000815394348\\
105.608382489177	0.992004937692081\\
105.630184384819	0.992009057865429\\
105.651986280461	0.992013175915489\\
105.673788176103	0.992017291843354\\
105.695590071745	0.992021405650117\\
105.717391967386	0.992025517336872\\
105.739193863028	0.992029626904712\\
105.76099575867	0.992033734354728\\
105.782797654312	0.992037839688012\\
105.804599549954	0.992041942905655\\
105.826401445596	0.992046044008746\\
105.848203341238	0.992050142998376\\
105.870005236879	0.992054239875633\\
105.891807132521	0.992058334641607\\
105.913609028163	0.992062427297385\\
105.935410923805	0.992066517844055\\
105.957212819447	0.992070606282703\\
105.979014715089	0.992074692614417\\
106.000816610731	0.992078776840282\\
106.022618506373	0.992082858961382\\
106.044420402014	0.992086938978803\\
106.066222297656	0.992091016893629\\
106.088024193298	0.992095092706944\\
106.10982608894	0.99209916641983\\
106.131627984582	0.99210323803337\\
106.153429880224	0.992107307548645\\
106.175231775866	0.992111374966738\\
106.197033671507	0.992115440288728\\
106.218835567149	0.992119503515696\\
106.240637462791	0.992123564648722\\
106.262439358433	0.992127623688884\\
106.284241254075	0.992131680637262\\
106.306043149717	0.992135735494933\\
106.327845045359	0.992139788262975\\
106.349646941	0.992143838942464\\
106.371448836642	0.992147887534477\\
106.393250732284	0.99215193404009\\
106.415052627926	0.992155978460377\\
106.436854523568	0.992160020796414\\
106.45865641921	0.992164061049275\\
106.480458314852	0.992168099220032\\
106.502260210493	0.99217213530976\\
106.524062106135	0.99217616931953\\
106.545864001777	0.992180201250415\\
106.567665897419	0.992184231103486\\
106.589467793061	0.992188258879813\\
106.611269688703	0.992192284580466\\
106.633071584345	0.992196308206516\\
106.654873479987	0.992200329759032\\
106.676675375628	0.992204349239082\\
106.69847727127	0.992208366647734\\
106.720279166912	0.992212381986056\\
106.742081062554	0.992216395255114\\
106.763882958196	0.992220406455975\\
106.785684853838	0.992224415589705\\
106.80748674948	0.992228422657369\\
106.829288645121	0.992232427660032\\
106.851090540763	0.992236430598757\\
106.872892436405	0.992240431474609\\
106.894694332047	0.992244430288651\\
106.916496227689	0.992248427041945\\
106.938298123331	0.992252421735553\\
106.960100018973	0.992256414370537\\
106.981901914614	0.992260404947957\\
107.003703810256	0.992264393468874\\
107.025505705898	0.992268379934348\\
107.04730760154	0.992272364345438\\
107.069109497182	0.992276346703202\\
107.090911392824	0.992280327008699\\
107.112713288466	0.992284305262986\\
107.134515184107	0.992288281467121\\
107.156317079749	0.99229225562216\\
107.178118975391	0.992296227729158\\
107.199920871033	0.992300197789172\\
107.221722766675	0.992304165803257\\
107.243524662317	0.992308131772465\\
107.265326557959	0.992312095697853\\
107.2871284536	0.992316057580472\\
107.308930349242	0.992320017421375\\
107.330732244884	0.992323975221616\\
107.352534140526	0.992327930982244\\
107.374336036168	0.992331884704311\\
107.39613793181	0.992335836388869\\
107.417939827452	0.992339786036966\\
107.439741723094	0.992343733649653\\
107.461543618735	0.992347679227978\\
107.483345514377	0.99235162277299\\
107.505147410019	0.992355564285736\\
107.526949305661	0.992359503767264\\
107.548751201303	0.992363441218621\\
107.570553096945	0.992367376640852\\
107.592354992587	0.992371310035004\\
107.614156888228	0.992375241402121\\
107.63595878387	0.992379170743249\\
107.657760679512	0.992383098059431\\
107.679562575154	0.992387023351711\\
107.701364470796	0.992390946621132\\
107.723166366438	0.992394867868737\\
107.74496826208	0.992398787095566\\
107.766770157721	0.992402704302662\\
107.788572053363	0.992406619491066\\
107.810373949005	0.992410532661817\\
107.832175844647	0.992414443815956\\
107.853977740289	0.992418352954522\\
107.875779635931	0.992422260078554\\
107.897581531573	0.992426165189089\\
107.919383427214	0.992430068287165\\
107.941185322856	0.99243396937382\\
107.962987218498	0.992437868450089\\
107.98478911414	0.99244176551701\\
108.006591009782	0.992445660575618\\
108.028392905424	0.992449553626946\\
108.050194801066	0.992453444672031\\
108.071996696708	0.992457333711906\\
108.093798592349	0.992461220747603\\
108.115600487991	0.992465105780157\\
108.137402383633	0.992468988810599\\
108.159204279275	0.992472869839961\\
108.181006174917	0.992476748869275\\
108.202808070559	0.99248062589957\\
108.224609966201	0.992484500931878\\
108.246411861842	0.992488373967228\\
108.268213757484	0.992492245006648\\
108.290015653126	0.992496114051168\\
108.311817548768	0.992499981101816\\
108.33361944441	0.992503846159618\\
108.355421340052	0.992507709225603\\
108.377223235694	0.992511570300796\\
108.399025131335	0.992515429386224\\
108.420827026977	0.992519286482912\\
108.442628922619	0.992523141591884\\
108.464430818261	0.992526994714166\\
108.486232713903	0.992530845850781\\
108.508034609545	0.992534695002751\\
108.529836505187	0.992538542171101\\
108.551638400828	0.992542387356852\\
108.57344029647	0.992546230561026\\
108.595242192112	0.992550071784644\\
108.617044087754	0.992553911028727\\
108.638845983396	0.992557748294295\\
108.660647879038	0.992561583582367\\
108.68244977468	0.992565416893963\\
108.704251670321	0.992569248230102\\
108.726053565963	0.9925730775918\\
108.747855461605	0.992576904980076\\
108.769657357247	0.992580730395947\\
108.791459252889	0.992584553840429\\
108.813261148531	0.992588375314538\\
108.835063044173	0.99259219481929\\
108.856864939815	0.992596012355699\\
108.878666835456	0.99259982792478\\
108.900468731098	0.992603641527546\\
108.92227062674	0.992607453165012\\
108.944072522382	0.992611262838189\\
108.965874418024	0.99261507054809\\
108.987676313666	0.992618876295727\\
109.009478209308	0.99262268008211\\
109.031280104949	0.992626481908252\\
109.053082000591	0.992630281775161\\
109.074883896233	0.992634079683848\\
109.096685791875	0.992637875635322\\
109.118487687517	0.992641669630591\\
109.140289583159	0.992645461670663\\
109.162091478801	0.992649251756547\\
109.183893374442	0.992653039889248\\
109.205695270084	0.992656826069775\\
109.227497165726	0.992660610299132\\
109.249299061368	0.992664392578325\\
109.27110095701	0.992668172908359\\
109.292902852652	0.992671951290239\\
109.314704748294	0.992675727724969\\
109.336506643935	0.992679502213552\\
109.358308539577	0.992683274756991\\
109.380110435219	0.992687045356288\\
109.401912330861	0.992690814012446\\
109.423714226503	0.992694580726465\\
109.445516122145	0.992698345499347\\
109.467318017787	0.992702108332092\\
109.489119913429	0.9927058692257\\
109.51092180907	0.992709628181169\\
109.532723704712	0.9927133851995\\
109.554525600354	0.992717140281689\\
109.576327495996	0.992720893428736\\
109.598129391638	0.992724644641637\\
109.61993128728	0.992728393921388\\
109.641733182922	0.992732141268987\\
109.663535078563	0.992735886685429\\
109.685336974205	0.992739630171708\\
109.707138869847	0.992743371728821\\
109.728940765489	0.99274711135776\\
109.750742661131	0.99275084905952\\
109.772544556773	0.992754584835093\\
109.794346452415	0.992758318685473\\
109.816148348056	0.992762050611652\\
109.837950243698	0.99276578061462\\
109.85975213934	0.99276950869537\\
109.881554034982	0.992773234854892\\
109.903355930624	0.992776959094175\\
109.925157826266	0.99278068141421\\
109.946959721908	0.992784401815986\\
109.968761617549	0.992788120300491\\
109.990563513191	0.992791836868713\\
110.012365408833	0.992795551521639\\
110.034167304475	0.992799264260257\\
110.055969200117	0.992802975085554\\
110.077771095759	0.992806683998515\\
110.099572991401	0.992810391000125\\
110.121374887042	0.99281409609137\\
110.143176782684	0.992817799273235\\
110.164978678326	0.992821500546703\\
110.186780573968	0.992825199912757\\
110.20858246961	0.992828897372381\\
110.230384365252	0.992832592926557\\
110.252186260894	0.992836286576268\\
110.273988156536	0.992839978322493\\
110.295790052177	0.992843668166216\\
110.317591947819	0.992847356108415\\
110.339393843461	0.99285104215007\\
110.361195739103	0.992854726292163\\
110.382997634745	0.99285840853567\\
110.404799530387	0.992862088881571\\
110.426601426029	0.992865767330843\\
110.44840332167	0.992869443884465\\
110.470205217312	0.992873118543412\\
110.492007112954	0.992876791308661\\
110.513809008596	0.992880462181188\\
110.535610904238	0.992884131161969\\
110.55741279988	0.992887798251978\\
110.579214695522	0.992891463452191\\
110.601016591163	0.992895126763579\\
110.622818486805	0.992898788187118\\
110.644620382447	0.99290244772378\\
110.666422278089	0.992906105374537\\
110.688224173731	0.992909761140361\\
110.710026069373	0.992913415022224\\
110.731827965015	0.992917067021097\\
110.753629860656	0.992920717137949\\
110.775431756298	0.992924365373751\\
110.79723365194	0.992928011729472\\
110.819035547582	0.992931656206081\\
110.840837443224	0.992935298804546\\
110.862639338866	0.992938939525836\\
110.884441234508	0.992942578370917\\
110.90624313015	0.992946215340757\\
110.928045025791	0.992949850436321\\
110.949846921433	0.992953483658577\\
110.971648817075	0.992957115008488\\
110.993450712717	0.992960744487021\\
111.015252608359	0.992964372095139\\
111.037054504001	0.992967997833807\\
111.058856399643	0.992971621703987\\
111.080658295284	0.992975243706643\\
111.102460190926	0.992978863842737\\
111.124262086568	0.992982482113231\\
111.14606398221	0.992986098519087\\
111.167865877852	0.992989713061265\\
111.189667773494	0.992993325740726\\
111.211469669136	0.992996936558429\\
111.233271564777	0.993000545515334\\
111.255073460419	0.993004152612401\\
111.276875356061	0.993007757850587\\
111.298677251703	0.993011361230851\\
111.320479147345	0.993014962754149\\
111.342281042987	0.99301856242144\\
111.364082938629	0.993022160233679\\
111.38588483427	0.993025756191822\\
111.407686729912	0.993029350296825\\
111.429488625554	0.993032942549643\\
111.451290521196	0.99303653295123\\
111.473092416838	0.993040121502541\\
111.49489431248	0.993043708204528\\
111.516696208122	0.993047293058146\\
111.538498103763	0.993050876064346\\
111.560299999405	0.99305445722408\\
111.582101895047	0.993058036538301\\
111.603903790689	0.993061614007959\\
111.625705686331	0.993065189634004\\
111.647507581973	0.993068763417387\\
111.669309477615	0.993072335359058\\
111.691111373257	0.993075905459965\\
111.712913268898	0.993079473721057\\
111.73471516454	0.993083040143283\\
111.756517060182	0.993086604727589\\
111.778318955824	0.993090167474924\\
111.800120851466	0.993093728386233\\
111.821922747108	0.993097287462463\\
111.84372464275	0.993100844704559\\
111.865526538391	0.993104400113468\\
111.887328434033	0.993107953690133\\
111.909130329675	0.993111505435498\\
111.930932225317	0.993115055350508\\
111.952734120959	0.993118603436106\\
111.974536016601	0.993122149693234\\
111.996337912243	0.993125694122835\\
112.018139807884	0.993129236725851\\
112.039941703526	0.993132777503223\\
112.061743599168	0.993136316455891\\
112.08354549481	0.993139853584796\\
112.105347390452	0.993143388890878\\
112.127149286094	0.993146922375077\\
112.148951181736	0.99315045403833\\
112.170753077377	0.993153983881577\\
112.192554973019	0.993157511905756\\
112.214356868661	0.993161038111804\\
112.236158764303	0.993164562500657\\
112.257960659945	0.993168085073253\\
112.279762555587	0.993171605830527\\
112.301564451229	0.993175124773415\\
112.323366346871	0.993178641902852\\
112.345168242512	0.993182157219773\\
112.366970138154	0.993185670725111\\
112.388772033796	0.9931891824198\\
112.410573929438	0.993192692304773\\
112.43237582508	0.993196200380963\\
112.454177720722	0.993199706649301\\
112.475979616364	0.993203211110721\\
112.497781512005	0.993206713766152\\
112.519583407647	0.993210214616526\\
112.541385303289	0.993213713662772\\
112.563187198931	0.993217210905821\\
112.584989094573	0.993220706346601\\
112.606790990215	0.993224199986042\\
112.628592885857	0.993227691825071\\
112.650394781498	0.993231181864618\\
112.67219667714	0.993234670105608\\
112.693998572782	0.993238156548969\\
112.715800468424	0.993241641195627\\
112.737602364066	0.993245124046508\\
112.759404259708	0.993248605102537\\
112.78120615535	0.993252084364641\\
112.803008050991	0.993255561833742\\
112.824809946633	0.993259037510765\\
112.846611842275	0.993262511396634\\
112.868413737917	0.993265983492271\\
112.890215633559	0.993269453798599\\
112.912017529201	0.993272922316541\\
112.933819424843	0.993276389047017\\
112.955621320484	0.99327985399095\\
112.977423216126	0.993283317149259\\
112.999225111768	0.993286778522865\\
113.02102700741	0.993290238112687\\
113.042828903052	0.993293695919646\\
113.064630798694	0.993297151944659\\
113.086432694336	0.993300606188645\\
113.108234589978	0.993304058652522\\
113.130036485619	0.993307509337207\\
113.151838381261	0.993310958243618\\
113.173640276903	0.993314405372669\\
113.195442172545	0.993317850725278\\
113.217244068187	0.99332129430236\\
113.239045963829	0.99332473610483\\
113.260847859471	0.993328176133602\\
113.282649755112	0.993331614389591\\
113.304451650754	0.99333505087371\\
113.326253546396	0.993338485586872\\
113.348055442038	0.99334191852999\\
113.36985733768	0.993345349703975\\
113.391659233322	0.993348779109741\\
113.413461128964	0.993352206748197\\
113.435263024605	0.993355632620255\\
113.457064920247	0.993359056726825\\
113.478866815889	0.993362479068817\\
113.500668711531	0.99336589964714\\
113.522470607173	0.993369318462702\\
113.544272502815	0.993372735516414\\
113.566074398457	0.993376150809182\\
113.587876294098	0.993379564341914\\
113.60967818974	0.993382976115516\\
113.631480085382	0.993386386130897\\
113.653281981024	0.99338979438896\\
113.675083876666	0.993393200890613\\
113.696885772308	0.993396605636761\\
113.71868766795	0.993400008628307\\
113.740489563591	0.993403409866156\\
113.762291459233	0.993406809351213\\
113.784093354875	0.99341020708438\\
113.805895250517	0.993413603066559\\
113.827697146159	0.993416997298655\\
113.849499041801	0.993420389781567\\
113.871300937443	0.993423780516198\\
113.893102833085	0.993427169503449\\
113.914904728726	0.99343055674422\\
113.936706624368	0.993433942239412\\
113.95850852001	0.993437325989923\\
113.980310415652	0.993440707996653\\
114.002112311294	0.993444088260501\\
114.023914206936	0.993447466782364\\
114.045716102578	0.993450843563141\\
114.067517998219	0.993454218603728\\
114.089319893861	0.993457591905023\\
114.111121789503	0.993460963467921\\
114.132923685145	0.99346433329332\\
114.154725580787	0.993467701382113\\
114.176527476429	0.993471067735196\\
114.198329372071	0.993474432353464\\
114.220131267712	0.99347779523781\\
114.241933163354	0.993481156389129\\
114.263735058996	0.993484515808312\\
114.285536954638	0.993487873496253\\
114.30733885028	0.993491229453845\\
114.329140745922	0.993494583681978\\
114.350942641564	0.993497936181543\\
114.372744537205	0.993501286953433\\
114.394546432847	0.993504635998537\\
114.416348328489	0.993507983317744\\
114.438150224131	0.993511328911945\\
114.459952119773	0.993514672782029\\
114.481754015415	0.993518014928883\\
114.503555911057	0.993521355353396\\
114.525357806699	0.993524694056456\\
114.54715970234	0.993528031038949\\
114.568961597982	0.993531366301763\\
114.590763493624	0.993534699845783\\
114.612565389266	0.993538031671895\\
114.634367284908	0.993541361780985\\
114.65616918055	0.993544690173938\\
114.677971076192	0.993548016851637\\
114.699772971833	0.993551341814967\\
114.721574867475	0.993554665064811\\
114.743376763117	0.993557986602053\\
114.765178658759	0.993561306427574\\
114.786980554401	0.993564624542258\\
114.808782450043	0.993567940946985\\
114.830584345685	0.993571255642637\\
114.852386241326	0.993574568630094\\
114.874188136968	0.993577879910237\\
114.89599003261	0.993581189483947\\
114.917791928252	0.993584497352101\\
114.939593823894	0.99358780351558\\
114.961395719536	0.99359110797526\\
114.983197615178	0.993594410732022\\
115.004999510819	0.993597711786742\\
115.026801406461	0.993601011140297\\
115.048603302103	0.993604308793564\\
115.070405197745	0.993607604747419\\
115.092207093387	0.993610899002739\\
115.114008989029	0.993614191560397\\
115.135810884671	0.99361748242127\\
115.157612780312	0.993620771586231\\
115.179414675954	0.993624059056155\\
115.201216571596	0.993627344831916\\
115.223018467238	0.993630628914385\\
115.24482036288	0.993633911304436\\
115.266622258522	0.993637192002941\\
115.288424154164	0.993640471010771\\
115.310226049806	0.993643748328799\\
115.332027945447	0.993647023957894\\
115.353829841089	0.993650297898928\\
115.375631736731	0.99365357015277\\
115.397433632373	0.993656840720289\\
115.419235528015	0.993660109602355\\
115.441037423657	0.993663376799836\\
115.462839319299	0.9936666423136\\
115.48464121494	0.993669906144515\\
115.506443110582	0.993673168293448\\
115.528245006224	0.993676428761267\\
115.550046901866	0.993679687548837\\
115.571848797508	0.993682944657024\\
115.59365069315	0.993686200086694\\
115.615452588792	0.993689453838711\\
115.637254484433	0.993692705913941\\
115.659056380075	0.993695956313247\\
115.680858275717	0.993699205037493\\
115.702660171359	0.993702452087543\\
115.724462067001	0.993705697464258\\
115.746263962643	0.993708941168502\\
115.768065858285	0.993712183201137\\
115.789867753926	0.993715423563023\\
115.811669649568	0.993718662255021\\
115.83347154521	0.993721899277994\\
115.855273440852	0.993725134632799\\
115.877075336494	0.993728368320298\\
115.898877232136	0.993731600341348\\
115.920679127778	0.99373483069681\\
115.94248102342	0.993738059387542\\
115.964282919061	0.9937412864144\\
115.986084814703	0.993744511778244\\
116.007886710345	0.99374773547993\\
116.029688605987	0.993750957520313\\
116.051490501629	0.993754177900252\\
116.073292397271	0.9937573966206\\
116.095094292913	0.993760613682215\\
116.116896188554	0.993763829085949\\
116.138698084196	0.993767042832658\\
116.160499979838	0.993770254923196\\
116.18230187548	0.993773465358416\\
116.204103771122	0.993776674139171\\
116.225905666764	0.993779881266313\\
116.247707562406	0.993783086740696\\
116.269509458047	0.99378629056317\\
116.291311353689	0.993789492734588\\
116.313113249331	0.993792693255799\\
116.334915144973	0.993795892127654\\
116.356717040615	0.993799089351003\\
116.378518936257	0.993802284926697\\
116.400320831899	0.993805478855583\\
116.42212272754	0.99380867113851\\
116.443924623182	0.993811861776328\\
116.465726518824	0.993815050769882\\
116.487528414466	0.993818238120022\\
116.509330310108	0.993821423827594\\
116.53113220575	0.993824607893443\\
116.552934101392	0.993827790318418\\
116.574735997033	0.993830971103361\\
116.596537892675	0.993834150249121\\
116.618339788317	0.99383732775654\\
116.640141683959	0.993840503626463\\
116.661943579601	0.993843677859734\\
116.683745475243	0.993846850457197\\
116.705547370885	0.993850021419694\\
116.727349266527	0.993853190748068\\
116.749151162168	0.993856358443161\\
116.77095305781	0.993859524505815\\
116.792754953452	0.993862688936871\\
116.814556849094	0.99386585173717\\
116.836358744736	0.993869012907553\\
116.858160640378	0.993872172448859\\
116.87996253602	0.993875330361927\\
116.901764431661	0.993878486647597\\
116.923566327303	0.993881641306708\\
116.945368222945	0.993884794340098\\
116.967170118587	0.993887945748604\\
116.988972014229	0.993891095533065\\
117.010773909871	0.993894243694316\\
117.032575805513	0.993897390233194\\
117.054377701154	0.993900535150536\\
117.076179596796	0.993903678447178\\
117.097981492438	0.993906820123953\\
117.11978338808	0.993909960181697\\
117.141585283722	0.993913098621245\\
117.163387179364	0.993916235443431\\
117.185189075006	0.993919370649087\\
117.206990970647	0.993922504239047\\
117.228792866289	0.993925636214144\\
117.250594761931	0.993928766575209\\
117.272396657573	0.993931895323075\\
117.294198553215	0.993935022458573\\
117.316000448857	0.993938147982534\\
117.337802344499	0.993941271895788\\
117.359604240141	0.993944394199165\\
117.381406135782	0.993947514893496\\
117.403208031424	0.993950633979608\\
117.425009927066	0.993953751458332\\
117.446811822708	0.993956867330495\\
117.46861371835	0.993959981596925\\
117.490415613992	0.993963094258449\\
117.512217509634	0.993966205315896\\
117.534019405275	0.993969314770091\\
117.555821300917	0.993972422621861\\
117.577623196559	0.993975528872031\\
117.599425092201	0.993978633521427\\
117.621226987843	0.993981736570874\\
117.643028883485	0.993984838021196\\
117.664830779127	0.993987937873218\\
117.686632674768	0.993991036127762\\
117.70843457041	0.993994132785653\\
117.730236466052	0.993997227847713\\
117.752038361694	0.994000321314765\\
117.773840257336	0.99400341318763\\
117.795642152978	0.99400650346713\\
117.81744404862	0.994009592154086\\
117.839245944261	0.994012679249319\\
117.861047839903	0.99401576475365\\
117.882849735545	0.994018848667897\\
117.904651631187	0.99402193099288\\
117.926453526829	0.994025011729419\\
117.948255422471	0.994028090878332\\
117.970057318113	0.994031168440438\\
117.991859213754	0.994034244416553\\
118.013661109396	0.994037318807496\\
118.035463005038	0.994040391614082\\
118.05726490068	0.99404346283713\\
118.079066796322	0.994046532477453\\
118.100868691964	0.99404960053587\\
118.122670587606	0.994052667013194\\
118.144472483248	0.99405573191024\\
118.166274378889	0.994058795227823\\
118.188076274531	0.994061856966757\\
118.209878170173	0.994064917127855\\
118.231680065815	0.994067975711931\\
118.253481961457	0.994071032719796\\
118.275283857099	0.994074088152264\\
118.297085752741	0.994077142010146\\
118.318887648382	0.994080194294254\\
118.340689544024	0.994083245005398\\
118.362491439666	0.99408629414439\\
118.384293335308	0.994089341712039\\
118.40609523095	0.994092387709155\\
118.427897126592	0.994095432136549\\
118.449699022234	0.994098474995027\\
118.471500917875	0.9941015162854\\
118.493302813517	0.994104556008475\\
118.515104709159	0.99410759416506\\
118.536906604801	0.994110630755962\\
118.558708500443	0.994113665781988\\
118.580510396085	0.994116699243945\\
118.602312291727	0.994119731142638\\
118.624114187368	0.994122761478873\\
118.64591608301	0.994125790253455\\
118.667717978652	0.99412881746719\\
118.689519874294	0.994131843120881\\
118.711321769936	0.994134867215332\\
118.733123665578	0.994137889751348\\
118.75492556122	0.99414091072973\\
118.776727456862	0.994143930151282\\
118.798529352503	0.994146948016807\\
118.820331248145	0.994149964327105\\
118.842133143787	0.994152979082979\\
118.863935039429	0.994155992285229\\
118.885736935071	0.994159003934657\\
118.907538830713	0.994162014032061\\
118.929340726355	0.994165022578243\\
118.951142621996	0.994168029574002\\
118.972944517638	0.994171035020135\\
118.99474641328	0.994174038917444\\
119.016548308922	0.994177041266724\\
119.038350204564	0.994180042068775\\
119.060152100206	0.994183041324393\\
119.081953995848	0.994186039034375\\
119.103755891489	0.994189035199519\\
119.125557787131	0.994192029820619\\
119.147359682773	0.994195022898472\\
119.169161578415	0.994198014433873\\
119.190963474057	0.994201004427617\\
119.212765369699	0.994203992880499\\
119.234567265341	0.994206979793312\\
119.256369160982	0.99420996516685\\
119.278171056624	0.994212949001906\\
119.299972952266	0.994215931299274\\
119.321774847908	0.994218912059745\\
119.34357674355	0.994221891284112\\
119.365378639192	0.994224868973167\\
119.387180534834	0.994227845127699\\
119.408982430475	0.994230819748501\\
119.430784326117	0.994233792836363\\
119.452586221759	0.994236764392074\\
119.474388117401	0.994239734416425\\
119.496190013043	0.994242702910204\\
119.517991908685	0.9942456698742\\
119.539793804327	0.994248635309202\\
119.561595699969	0.994251599215997\\
119.58339759561	0.994254561595373\\
119.605199491252	0.994257522448118\\
119.627001386894	0.994260481775017\\
119.648803282536	0.994263439576858\\
119.670605178178	0.994266395854426\\
119.69240707382	0.994269350608506\\
119.714208969462	0.994272303839884\\
119.736010865103	0.994275255549345\\
119.757812760745	0.994278205737672\\
119.779614656387	0.994281154405651\\
119.801416552029	0.994284101554063\\
119.823218447671	0.994287047183693\\
119.845020343313	0.994289991295323\\
119.866822238955	0.994292933889735\\
119.888624134596	0.994295874967711\\
119.910426030238	0.994298814530033\\
119.93222792588	0.994301752577481\\
119.954029821522	0.994304689110837\\
119.975831717164	0.994307624130881\\
119.997633612806	0.994310557638392\\
120.019435508448	0.99431348963415\\
120.041237404089	0.994316420118934\\
120.063039299731	0.994319349093523\\
120.084841195373	0.994322276558695\\
120.106643091015	0.994325202515227\\
120.128444986657	0.994328126963898\\
120.150246882299	0.994331049905485\\
120.172048777941	0.994333971340763\\
120.193850673583	0.99433689127051\\
120.215652569224	0.994339809695501\\
120.237454464866	0.994342726616511\\
120.259256360508	0.994345642034317\\
120.28105825615	0.994348555949691\\
120.302860151792	0.994351468363409\\
120.324662047434	0.994354379276245\\
120.346463943076	0.994357288688971\\
120.368265838717	0.994360196602362\\
120.390067734359	0.994363103017189\\
120.411869630001	0.994366007934225\\
120.433671525643	0.994368911354242\\
120.455473421285	0.994371813278011\\
120.477275316927	0.994374713706304\\
120.499077212569	0.994377612639891\\
120.52087910821	0.994380510079542\\
120.542681003852	0.994383406026027\\
120.564482899494	0.994386300480115\\
120.586284795136	0.994389193442577\\
120.608086690778	0.99439208491418\\
120.62988858642	0.994394974895693\\
120.651690482062	0.994397863387883\\
120.673492377703	0.994400750391519\\
120.695294273345	0.994403635907367\\
120.717096168987	0.994406519936194\\
120.738898064629	0.994409402478766\\
120.760699960271	0.99441228353585\\
120.782501855913	0.99441516310821\\
120.804303751555	0.994418041196612\\
120.826105647196	0.994420917801821\\
120.847907542838	0.994423792924601\\
120.86970943848	0.994426666565716\\
120.891511334122	0.99442953872593\\
120.913313229764	0.994432409406005\\
120.935115125406	0.994435278606704\\
120.956917021048	0.994438146328791\\
120.97871891669	0.994441012573026\\
121.000520812331	0.994443877340172\\
121.022322707973	0.994446740630989\\
121.044124603615	0.994449602446239\\
121.065926499257	0.994452462786682\\
121.087728394899	0.994455321653077\\
121.109530290541	0.994458179046185\\
121.131332186183	0.994461034966765\\
121.153134081824	0.994463889415576\\
121.174935977466	0.994466742393375\\
121.196737873108	0.994469593900922\\
121.21853976875	0.994472443938974\\
121.240341664392	0.994475292508287\\
121.262143560034	0.99447813960962\\
121.283945455676	0.994480985243728\\
121.305747351317	0.994483829411368\\
121.327549246959	0.994486672113296\\
121.349351142601	0.994489513350266\\
121.371153038243	0.994492353123033\\
121.392954933885	0.994495191432353\\
121.414756829527	0.99449802827898\\
121.436558725169	0.994500863663666\\
121.45836062081	0.994503697587166\\
121.480162516452	0.994506530050233\\
121.501964412094	0.994509361053619\\
121.523766307736	0.994512190598076\\
121.545568203378	0.994515018684357\\
121.56737009902	0.994517845313213\\
121.589171994662	0.994520670485394\\
121.610973890303	0.994523494201651\\
121.632775785945	0.994526316462736\\
121.654577681587	0.994529137269397\\
121.676379577229	0.994531956622384\\
121.698181472871	0.994534774522447\\
121.719983368513	0.994537590970334\\
121.741785264155	0.994540405966793\\
121.763587159797	0.994543219512573\\
121.785389055438	0.994546031608421\\
121.80719095108	0.994548842255084\\
121.828992846722	0.994551651453309\\
121.850794742364	0.994554459203843\\
121.872596638006	0.994557265507431\\
121.894398533648	0.99456007036482\\
121.91620042929	0.994562873776754\\
121.938002324931	0.994565675743978\\
121.959804220573	0.994568476267238\\
121.981606116215	0.994571275347276\\
122.003408011857	0.994574072984837\\
122.025209907499	0.994576869180665\\
122.047011803141	0.994579663935501\\
122.068813698783	0.99458245725009\\
122.090615594424	0.994585249125172\\
122.112417490066	0.99458803956149\\
122.134219385708	0.994590828559786\\
122.15602128135	0.9945936161208\\
122.177823176992	0.994596402245272\\
122.199625072634	0.994599186933945\\
122.221426968276	0.994601970187556\\
122.243228863917	0.994604752006846\\
122.265030759559	0.994607532392554\\
122.286832655201	0.994610311345419\\
122.308634550843	0.994613088866179\\
122.330436446485	0.994615864955573\\
122.352238342127	0.994618639614337\\
122.374040237769	0.994621412843209\\
122.395842133411	0.994624184642926\\
122.417644029052	0.994626955014224\\
122.439445924694	0.99462972395784\\
122.461247820336	0.99463249147451\\
122.483049715978	0.994635257564968\\
122.50485161162	0.99463802222995\\
122.526653507262	0.994640785470189\\
122.548455402904	0.994643547286422\\
122.570257298545	0.994646307679381\\
122.592059194187	0.994649066649799\\
122.613861089829	0.99465182419841\\
122.635662985471	0.994654580325948\\
122.657464881113	0.994657335033143\\
122.679266776755	0.994660088320727\\
122.701068672397	0.994662840189434\\
122.722870568038	0.994665590639993\\
122.74467246368	0.994668339673136\\
122.766474359322	0.994671087289592\\
122.788276254964	0.994673833490093\\
122.810078150606	0.994676578275367\\
122.831880046248	0.994679321646145\\
122.85368194189	0.994682063603154\\
122.875483837531	0.994684804147124\\
122.897285733173	0.994687543278783\\
122.919087628815	0.994690280998858\\
122.940889524457	0.994693017308078\\
122.962691420099	0.994695752207168\\
122.984493315741	0.994698485696857\\
123.006295211383	0.994701217777869\\
123.028097107025	0.994703948450932\\
123.049899002666	0.99470667771677\\
123.071700898308	0.994709405576109\\
123.09350279395	0.994712132029674\\
123.115304689592	0.994714857078189\\
123.137106585234	0.994717580722378\\
123.158908480876	0.994720302962965\\
123.180710376518	0.994723023800673\\
123.202512272159	0.994725743236226\\
123.224314167801	0.994728461270345\\
123.246116063443	0.994731177903753\\
123.267917959085	0.994733893137172\\
123.289719854727	0.994736606971323\\
123.311521750369	0.994739319406928\\
123.333323646011	0.994742030444707\\
123.355125541652	0.994744740085381\\
123.376927437294	0.994747448329669\\
123.398729332936	0.994750155178291\\
123.420531228578	0.994752860631966\\
123.44233312422	0.994755564691413\\
123.464135019862	0.994758267357352\\
123.485936915504	0.994760968630499\\
123.507738811145	0.994763668511572\\
123.529540706787	0.99476636700129\\
123.551342602429	0.994769064100369\\
123.573144498071	0.994771759809525\\
123.594946393713	0.994774454129475\\
123.616748289355	0.994777147060935\\
123.638550184997	0.994779838604621\\
123.660352080638	0.994782528761247\\
123.68215397628	0.994785217531528\\
123.703955871922	0.994787904916179\\
123.725757767564	0.994790590915914\\
123.747559663206	0.994793275531447\\
123.769361558848	0.994795958763491\\
123.79116345449	0.994798640612758\\
123.812965350132	0.994801321079962\\
123.834767245773	0.994804000165815\\
123.856569141415	0.994806677871028\\
123.878371037057	0.994809354196314\\
123.900172932699	0.994812029142382\\
123.921974828341	0.994814702709945\\
123.943776723983	0.994817374899712\\
123.965578619625	0.994820045712393\\
123.987380515266	0.994822715148699\\
124.009182410908	0.994825383209338\\
124.03098430655	0.994828049895019\\
124.052786202192	0.994830715206451\\
124.074588097834	0.994833379144342\\
124.096389993476	0.9948360417094\\
124.118191889118	0.994838702902333\\
124.139993784759	0.994841362723847\\
124.161795680401	0.994844021174649\\
124.183597576043	0.994846678255446\\
124.205399471685	0.994849333966944\\
124.227201367327	0.994851988309847\\
124.249003262969	0.994854641284863\\
124.270805158611	0.994857292892695\\
124.292607054252	0.994859943134048\\
124.314408949894	0.994862592009627\\
124.336210845536	0.994865239520135\\
124.358012741178	0.994867885666275\\
124.37981463682	0.994870530448751\\
124.401616532462	0.994873173868265\\
124.423418428104	0.994875815925521\\
124.445220323746	0.994878456621219\\
124.467022219387	0.994881095956062\\
124.488824115029	0.994883733930751\\
124.510626010671	0.994886370545986\\
124.532427906313	0.994889005802469\\
124.554229801955	0.994891639700899\\
124.576031697597	0.994894272241977\\
124.597833593239	0.994896903426402\\
124.61963548888	0.994899533254873\\
124.641437384522	0.994902161728089\\
124.663239280164	0.994904788846748\\
124.685041175806	0.994907414611549\\
124.706843071448	0.994910039023188\\
124.72864496709	0.994912662082364\\
124.750446862732	0.994915283789773\\
124.772248758373	0.994917904146113\\
124.794050654015	0.994920523152078\\
124.815852549657	0.994923140808366\\
124.837654445299	0.994925757115671\\
124.859456340941	0.994928372074689\\
124.881258236583	0.994930985686115\\
124.903060132225	0.994933597950643\\
124.924862027866	0.994936208868967\\
124.946663923508	0.994938818441782\\
124.96846581915	0.99494142666978\\
124.990267714792	0.994944033553654\\
125.012069610434	0.994946639094097\\
125.033871506076	0.994949243291802\\
125.055673401718	0.99495184614746\\
125.077475297359	0.994954447661764\\
125.099277193001	0.994957047835403\\
125.121079088643	0.99495964666907\\
125.142880984285	0.994962244163454\\
125.164682879927	0.994964840319247\\
125.186484775569	0.994967435137137\\
125.208286671211	0.994970028617814\\
125.230088566853	0.994972620761968\\
125.251890462494	0.994975211570287\\
125.273692358136	0.994977801043459\\
125.295494253778	0.994980389182173\\
125.31729614942	0.994982975987116\\
125.339098045062	0.994985561458976\\
125.360899940704	0.994988145598439\\
125.382701836346	0.994990728406193\\
125.404503731987	0.994993309882923\\
125.426305627629	0.994995890029315\\
125.448107523271	0.994998468846056\\
125.469909418913	0.99500104633383\\
125.491711314555	0.995003622493322\\
125.513513210197	0.995006197325216\\
125.535315105839	0.995008770830198\\
125.55711700148	0.99501134300895\\
125.578918897122	0.995013913862156\\
125.600720792764	0.995016483390499\\
125.622522688406	0.995019051594662\\
125.644324584048	0.995021618475328\\
125.66612647969	0.995024184033177\\
125.687928375332	0.995026748268893\\
125.709730270973	0.995029311183157\\
125.731532166615	0.995031872776649\\
125.753334062257	0.995034433050049\\
125.775135957899	0.995036992004039\\
125.796937853541	0.995039549639299\\
125.818739749183	0.995042105956508\\
125.840541644825	0.995044660956344\\
125.862343540466	0.995047214639488\\
125.884145436108	0.995049767006617\\
125.90594733175	0.99505231805841\\
125.927749227392	0.995054867795545\\
125.949551123034	0.995057416218698\\
125.971353018676	0.995059963328548\\
125.993154914318	0.995062509125771\\
126.01495680996	0.995065053611044\\
126.036758705601	0.995067596785042\\
126.058560601243	0.995070138648442\\
126.080362496885	0.995072679201918\\
126.102164392527	0.995075218446147\\
126.123966288169	0.995077756381801\\
126.145768183811	0.995080293009557\\
126.167570079453	0.995082828330087\\
126.189371975094	0.995085362344067\\
126.211173870736	0.995087895052168\\
126.232975766378	0.995090426455063\\
126.25477766202	0.995092956553427\\
126.276579557662	0.99509548534793\\
126.298381453304	0.995098012839245\\
126.320183348946	0.995100539028043\\
126.341985244587	0.995103063914995\\
126.363787140229	0.995105587500773\\
126.385589035871	0.995108109786047\\
126.407390931513	0.995110630771487\\
126.429192827155	0.995113150457764\\
126.450994722797	0.995115668845545\\
126.472796618439	0.995118185935502\\
126.49459851408	0.995120701728301\\
126.516400409722	0.995123216224613\\
126.538202305364	0.995125729425105\\
126.560004201006	0.995128241330445\\
126.581806096648	0.9951307519413\\
126.60360799229	0.995133261258338\\
126.625409887932	0.995135769282225\\
126.647211783574	0.995138276013627\\
126.669013679215	0.995140781453212\\
126.690815574857	0.995143285601643\\
126.712617470499	0.995145788459588\\
126.734419366141	0.99514829002771\\
126.756221261783	0.995150790306674\\
126.778023157425	0.995153289297146\\
126.799825053067	0.995155786999789\\
126.821626948708	0.995158283415266\\
126.84342884435	0.995160778544241\\
126.865230739992	0.995163272387377\\
126.887032635634	0.995165764945336\\
126.908834531276	0.995168256218781\\
126.930636426918	0.995170746208374\\
126.95243832256	0.995173234914776\\
126.974240218201	0.99517572233865\\
126.996042113843	0.995178208480654\\
127.017844009485	0.995180693341451\\
127.039645905127	0.995183176921701\\
127.061447800769	0.995185659222062\\
127.083249696411	0.995188140243196\\
127.105051592053	0.995190619985761\\
127.126853487694	0.995193098450416\\
127.148655383336	0.99519557563782\\
127.170457278978	0.995198051548631\\
127.19225917462	0.995200526183506\\
127.214061070262	0.995202999543104\\
127.235862965904	0.995205471628081\\
127.257664861546	0.995207942439094\\
127.279466757187	0.995210411976801\\
127.301268652829	0.995212880241856\\
127.323070548471	0.995215347234917\\
127.344872444113	0.995217812956638\\
127.366674339755	0.995220277407674\\
127.388476235397	0.995222740588681\\
127.410278131039	0.995225202500313\\
127.432080026681	0.995227663143224\\
127.453881922322	0.995230122518068\\
127.475683817964	0.995232580625498\\
127.497485713606	0.995235037466168\\
127.519287609248	0.99523749304073\\
127.54108950489	0.995239947349837\\
127.562891400532	0.995242400394141\\
127.584693296174	0.995244852174294\\
127.606495191815	0.995247302690947\\
127.628297087457	0.995249751944751\\
127.650098983099	0.995252199936358\\
127.671900878741	0.995254646666417\\
127.693702774383	0.995257092135579\\
127.715504670025	0.995259536344493\\
127.737306565667	0.99526197929381\\
127.759108461308	0.995264420984178\\
127.78091035695	0.995266861416245\\
127.802712252592	0.995269300590661\\
127.824514148234	0.995271738508074\\
127.846316043876	0.99527417516913\\
127.868117939518	0.995276610574479\\
127.88991983516	0.995279044724766\\
127.911721730801	0.995281477620639\\
127.933523626443	0.995283909262744\\
127.955325522085	0.995286339651727\\
127.977127417727	0.995288768788234\\
127.998929313369	0.995291196672911\\
128.020731209011	0.995293623306402\\
128.042533104653	0.995296048689352\\
128.064335000295	0.995298472822407\\
128.086136895936	0.995300895706209\\
128.107938791578	0.995303317341402\\
128.12974068722	0.995305737728631\\
128.151542582862	0.995308156868538\\
128.173344478504	0.995310574761767\\
128.195146374146	0.995312991408958\\
128.216948269788	0.995315406810756\\
128.238750165429	0.995317820967801\\
128.260552061071	0.995320233880735\\
128.282353956713	0.995322645550199\\
128.304155852355	0.995325055976833\\
128.325957747997	0.99532746516128\\
128.347759643639	0.995329873104177\\
128.369561539281	0.995332279806166\\
128.391363434922	0.995334685267886\\
128.413165330564	0.995337089489976\\
128.434967226206	0.995339492473074\\
128.456769121848	0.99534189421782\\
128.47857101749	0.995344294724852\\
128.500372913132	0.995346693994806\\
128.522174808774	0.995349092028322\\
128.543976704415	0.995351488826036\\
128.565778600057	0.995353884388584\\
128.587580495699	0.995356278716604\\
128.609382391341	0.995358671810731\\
128.631184286983	0.995361063671602\\
128.652986182625	0.995363454299852\\
128.674788078267	0.995365843696117\\
128.696589973908	0.99536823186103\\
128.71839186955	0.995370618795227\\
128.740193765192	0.995373004499342\\
128.761995660834	0.995375388974009\\
128.783797556476	0.995377772219862\\
128.805599452118	0.995380154237533\\
128.82740134776	0.995382535027655\\
128.849203243402	0.995384914590862\\
128.871005139043	0.995387292927786\\
128.892807034685	0.995389670039058\\
128.914608930327	0.99539204592531\\
128.936410825969	0.995394420587173\\
128.958212721611	0.995396794025279\\
128.980014617253	0.995399166240258\\
129.001816512895	0.99540153723274\\
129.023618408536	0.995403907003355\\
129.045420304178	0.995406275552734\\
129.06722219982	0.995408642881504\\
129.089024095462	0.995411008990297\\
129.110825991104	0.995413373879739\\
129.132627886746	0.99541573755046\\
129.154429782388	0.995418100003087\\
129.176231678029	0.995420461238249\\
129.198033573671	0.995422821256572\\
129.219835469313	0.995425180058685\\
129.241637364955	0.995427537645212\\
129.263439260597	0.995429894016782\\
129.285241156239	0.99543224917402\\
129.307043051881	0.995434603117551\\
129.328844947522	0.995436955848002\\
129.350646843164	0.995439307365998\\
129.372448738806	0.995441657672162\\
129.394250634448	0.995444006767121\\
129.41605253009	0.995446354651497\\
129.437854425732	0.995448701325916\\
129.459656321374	0.995451046791\\
129.481458217016	0.995453391047372\\
129.503260112657	0.995455734095656\\
129.525062008299	0.995458075936474\\
129.546863903941	0.995460416570449\\
129.568665799583	0.995462755998202\\
129.590467695225	0.995465094220354\\
129.612269590867	0.995467431237528\\
129.634071486509	0.995469767050344\\
129.65587338215	0.995472101659423\\
129.677675277792	0.995474435065385\\
129.699477173434	0.99547676726885\\
129.721279069076	0.995479098270438\\
129.743080964718	0.995481428070769\\
129.76488286036	0.99548375667046\\
129.786684756002	0.995486084070132\\
129.808486651643	0.995488410270402\\
129.830288547285	0.995490735271888\\
129.852090442927	0.99549305907521\\
129.873892338569	0.995495381680982\\
129.895694234211	0.995497703089824\\
129.917496129853	0.995500023302352\\
129.939298025495	0.995502342319181\\
129.961099921136	0.99550466014093\\
129.982901816778	0.995506976768212\\
130.00470371242	0.995509292201645\\
130.026505608062	0.995511606441842\\
130.048307503704	0.99551391948942\\
130.070109399346	0.995516231344992\\
130.091911294988	0.995518542009173\\
130.113713190629	0.995520851482578\\
130.135515086271	0.995523159765818\\
130.157316981913	0.995525466859509\\
130.179118877555	0.995527772764263\\
130.200920773197	0.995530077480692\\
130.222722668839	0.995532381009409\\
130.244524564481	0.995534683351027\\
130.266326460123	0.995536984506156\\
130.288128355764	0.995539284475409\\
130.309930251406	0.995541583259395\\
130.331732147048	0.995543880858728\\
130.35353404269	0.995546177274016\\
130.375335938332	0.99554847250587\\
130.397137833974	0.995550766554899\\
130.418939729616	0.995553059421714\\
130.440741625257	0.995555351106924\\
130.462543520899	0.995557641611137\\
130.484345416541	0.995559930934963\\
130.506147312183	0.995562219079009\\
130.527949207825	0.995564506043883\\
130.549751103467	0.995566791830193\\
130.571552999109	0.995569076438547\\
130.59335489475	0.995571359869552\\
130.615156790392	0.995573642123813\\
130.636958686034	0.995575923201939\\
130.658760581676	0.995578203104534\\
130.680562477318	0.995580481832204\\
130.70236437296	0.995582759385556\\
130.724166268602	0.995585035765194\\
130.745968164243	0.995587310971723\\
130.767770059885	0.995589585005748\\
130.789571955527	0.995591857867873\\
130.811373851169	0.995594129558701\\
130.833175746811	0.995596400078837\\
130.854977642453	0.995598669428884\\
130.876779538095	0.995600937609444\\
130.898581433736	0.995603204621121\\
130.920383329378	0.995605470464517\\
130.94218522502	0.995607735140233\\
130.963987120662	0.995609998648872\\
130.985789016304	0.995612260991035\\
131.007590911946	0.995614522167324\\
131.029392807588	0.995616782178338\\
131.05119470323	0.995619041024679\\
131.072996598871	0.995621298706946\\
131.094798494513	0.99562355522574\\
131.116600390155	0.995625810581661\\
131.138402285797	0.995628064775306\\
131.160204181439	0.995630317807277\\
131.182006077081	0.99563256967817\\
131.203807972723	0.995634820388585\\
131.225609868364	0.995637069939119\\
131.247411764006	0.99563931833037\\
131.269213659648	0.995641565562936\\
131.29101555529	0.995643811637414\\
131.312817450932	0.9956460565544\\
131.334619346574	0.995648300314492\\
131.356421242216	0.995650542918284\\
131.378223137857	0.995652784366374\\
131.400025033499	0.995655024659356\\
131.421826929141	0.995657263797827\\
131.443628824783	0.99565950178238\\
131.465430720425	0.995661738613611\\
131.487232616067	0.995663974292115\\
131.509034511709	0.995666208818484\\
131.53083640735	0.995668442193312\\
131.552638302992	0.995670674417194\\
131.574440198634	0.995672905490723\\
131.596242094276	0.99567513541449\\
131.618043989918	0.995677364189089\\
131.63984588556	0.995679591815112\\
131.661647781202	0.995681818293151\\
131.683449676844	0.995684043623797\\
131.705251572485	0.995686267807642\\
131.727053468127	0.995688490845276\\
131.748855363769	0.995690712737291\\
131.770657259411	0.995692933484277\\
131.792459155053	0.995695153086823\\
131.814261050695	0.995697371545521\\
131.836062946337	0.995699588860958\\
131.857864841978	0.995701805033724\\
131.87966673762	0.995704020064409\\
131.901468633262	0.9957062339536\\
131.923270528904	0.995708446701886\\
131.945072424546	0.995710658309855\\
131.966874320188	0.995712868778094\\
131.98867621583	0.995715078107191\\
132.010478111471	0.995717286297733\\
132.032280007113	0.995719493350307\\
132.054081902755	0.995721699265499\\
132.075883798397	0.995723904043895\\
132.097685694039	0.995726107686082\\
132.119487589681	0.995728310192643\\
132.141289485323	0.995730511564166\\
132.163091380964	0.995732711801235\\
132.184893276606	0.995734910904433\\
132.206695172248	0.995737108874347\\
132.22849706789	0.995739305711559\\
132.250298963532	0.995741501416654\\
132.272100859174	0.995743695990215\\
132.293902754816	0.995745889432825\\
132.315704650457	0.995748081745066\\
132.337506546099	0.995750272927522\\
132.359308441741	0.995752462980775\\
132.381110337383	0.995754651905406\\
132.402912233025	0.995756839701997\\
132.424714128667	0.99575902637113\\
132.446516024309	0.995761211913385\\
132.468317919951	0.995763396329343\\
132.490119815592	0.995765579619585\\
132.511921711234	0.995767761784691\\
132.533723606876	0.995769942825241\\
132.555525502518	0.995772122741813\\
132.57732739816	0.995774301534988\\
132.599129293802	0.995776479205344\\
132.620931189444	0.99577865575346\\
132.642733085085	0.995780831179915\\
132.664534980727	0.995783005485286\\
132.686336876369	0.99578517867015\\
132.708138772011	0.995787350735086\\
132.729940667653	0.995789521680671\\
132.751742563295	0.995791691507481\\
132.773544458937	0.995793860216094\\
132.795346354578	0.995796027807084\\
132.81714825022	0.995798194281029\\
132.838950145862	0.995800359638504\\
132.860752041504	0.995802523880083\\
132.882553937146	0.995804687006344\\
132.904355832788	0.995806849017859\\
132.92615772843	0.995809009915203\\
132.947959624072	0.995811169698952\\
132.969761519713	0.995813328369678\\
132.991563415355	0.995815485927955\\
133.013365310997	0.995817642374357\\
133.035167206639	0.995819797709456\\
133.056969102281	0.995821951933825\\
133.078770997923	0.995824105048037\\
133.100572893565	0.995826257052664\\
133.122374789206	0.995828407948277\\
133.144176684848	0.995830557735448\\
133.16597858049	0.995832706414749\\
133.187780476132	0.995834853986749\\
133.209582371774	0.995837000452021\\
133.231384267416	0.995839145811133\\
133.253186163058	0.995841290064657\\
133.274988058699	0.995843433213161\\
133.296789954341	0.995845575257216\\
133.318591849983	0.995847716197391\\
133.340393745625	0.995849856034254\\
133.362195641267	0.995851994768374\\
133.383997536909	0.995854132400319\\
133.405799432551	0.995856268930658\\
133.427601328192	0.995858404359958\\
133.449403223834	0.995860538688786\\
133.471205119476	0.995862671917709\\
133.493007015118	0.995864804047295\\
133.51480891076	0.99586693507811\\
133.536610806402	0.99586906501072\\
133.558412702044	0.995871193845691\\
133.580214597685	0.995873321583589\\
133.602016493327	0.995875448224979\\
133.623818388969	0.995877573770426\\
133.645620284611	0.995879698220495\\
133.667422180253	0.99588182157575\\
133.689224075895	0.995883943836756\\
133.711025971537	0.995886065004077\\
133.732827867179	0.995888185078276\\
133.75462976282	0.995890304059916\\
133.776431658462	0.995892421949561\\
133.798233554104	0.995894538747773\\
133.820035449746	0.995896654455115\\
133.841837345388	0.995898769072149\\
133.86363924103	0.995900882599437\\
133.885441136672	0.99590299503754\\
133.907243032313	0.99590510638702\\
133.929044927955	0.995907216648437\\
133.950846823597	0.995909325822353\\
133.972648719239	0.995911433909328\\
133.994450614881	0.995913540909922\\
134.016252510523	0.995915646824694\\
134.038054406165	0.995917751654206\\
134.059856301806	0.995919855399014\\
134.081658197448	0.99592195805968\\
134.10346009309	0.995924059636761\\
134.125261988732	0.995926160130815\\
134.147063884374	0.995928259542402\\
134.168865780016	0.995930357872079\\
134.190667675658	0.995932455120402\\
134.212469571299	0.995934551287931\\
134.234271466941	0.99593664637522\\
134.256073362583	0.995938740382828\\
134.277875258225	0.995940833311311\\
134.299677153867	0.995942925161225\\
134.321479049509	0.995945015933125\\
134.343280945151	0.995947105627567\\
134.365082840792	0.995949194245107\\
134.386884736434	0.995951281786299\\
134.408686632076	0.995953368251698\\
134.430488527718	0.995955453641859\\
134.45229042336	0.995957537957335\\
134.474092319002	0.995959621198681\\
134.495894214644	0.995961703366449\\
134.517696110286	0.995963784461194\\
134.539498005927	0.995965864483468\\
134.561299901569	0.995967943433824\\
134.583101797211	0.995970021312814\\
134.604903692853	0.99597209812099\\
134.626705588495	0.995974173858905\\
134.648507484137	0.995976248527109\\
134.670309379779	0.995978322126154\\
134.69211127542	0.995980394656592\\
134.713913171062	0.995982466118972\\
134.735715066704	0.995984536513845\\
134.757516962346	0.995986605841761\\
134.779318857988	0.995988674103271\\
134.80112075363	0.995990741298923\\
134.822922649272	0.995992807429267\\
134.844724544913	0.995994872494852\\
134.866526440555	0.995996936496227\\
134.888328336197	0.995998999433939\\
134.910130231839	0.996001061308538\\
134.931932127481	0.996003122120571\\
134.953734023123	0.996005181870586\\
134.975535918765	0.996007240559129\\
134.997337814406	0.996009298186749\\
135.019139710048	0.996011354753991\\
135.04094160569	0.996013410261403\\
135.062743501332	0.99601546470953\\
135.084545396974	0.996017518098918\\
135.106347292616	0.996019570430113\\
135.128149188258	0.99602162170366\\
135.1499510839	0.996023671920104\\
135.171752979541	0.996025721079991\\
135.193554875183	0.996027769183863\\
135.215356770825	0.996029816232267\\
135.237158666467	0.996031862225745\\
135.258960562109	0.996033907164841\\
135.280762457751	0.996035951050099\\
135.302564353393	0.996037993882062\\
135.324366249034	0.996040035661272\\
135.346168144676	0.996042076388272\\
135.367970040318	0.996044116063604\\
135.38977193596	0.996046154687811\\
135.411573831602	0.996048192261433\\
135.433375727244	0.996050228785013\\
135.455177622886	0.996052264259092\\
135.476979518527	0.996054298684209\\
135.498781414169	0.996056332060907\\
135.520583309811	0.996058364389725\\
135.542385205453	0.996060395671203\\
135.564187101095	0.996062425905881\\
135.585988996737	0.996064455094298\\
135.607790892379	0.996066483236994\\
135.62959278802	0.996068510334507\\
135.651394683662	0.996070536387376\\
135.673196579304	0.99607256139614\\
135.694998474946	0.996074585361336\\
135.716800370588	0.996076608283503\\
135.73860226623	0.996078630163177\\
135.760404161872	0.996080651000897\\
135.782206057513	0.996082670797199\\
135.804007953155	0.996084689552619\\
135.825809848797	0.996086707267694\\
135.847611744439	0.99608872394296\\
135.869413640081	0.996090739578954\\
135.891215535723	0.99609275417621\\
135.913017431365	0.996094767735265\\
135.934819327007	0.996096780256652\\
135.956621222648	0.996098791740907\\
135.97842311829	0.996100802188564\\
136.000225013932	0.996102811600158\\
136.022026909574	0.996104819976222\\
136.043828805216	0.99610682731729\\
136.065630700858	0.996108833623895\\
136.0874325965	0.996110838896571\\
136.109234492141	0.99611284313585\\
136.131036387783	0.996114846342265\\
136.152838283425	0.996116848516348\\
136.174640179067	0.996118849658631\\
136.196442074709	0.996120849769646\\
136.218243970351	0.996122848849925\\
136.240045865993	0.996124846899998\\
136.261847761634	0.996126843920396\\
136.283649657276	0.99612883991165\\
136.305451552918	0.996130834874291\\
136.32725344856	0.996132828808848\\
136.349055344202	0.996134821715851\\
136.370857239844	0.996136813595831\\
136.392659135486	0.996138804449315\\
136.414461031127	0.996140794276833\\
136.436262926769	0.996142783078915\\
136.458064822411	0.996144770856087\\
136.479866718053	0.996146757608879\\
136.501668613695	0.996148743337818\\
136.523470509337	0.996150728043432\\
136.545272404979	0.996152711726249\\
136.567074300621	0.996154694386795\\
136.588876196262	0.996156676025597\\
136.610678091904	0.996158656643182\\
136.632479987546	0.996160636240076\\
136.654281883188	0.996162614816804\\
136.67608377883	0.996164592373894\\
136.697885674472	0.99616656891187\\
136.719687570114	0.996168544431258\\
136.741489465755	0.996170518932581\\
136.763291361397	0.996172492416366\\
136.785093257039	0.996174464883136\\
136.806895152681	0.996176436333416\\
136.828697048323	0.996178406767729\\
136.850498943965	0.996180376186598\\
136.872300839607	0.996182344590549\\
136.894102735248	0.996184311980102\\
136.91590463089	0.996186278355781\\
136.937706526532	0.996188243718109\\
136.959508422174	0.996190208067607\\
136.981310317816	0.996192171404798\\
137.003112213458	0.996194133730203\\
137.0249141091	0.996196095044344\\
137.046716004741	0.996198055347742\\
137.068517900383	0.996200014640918\\
137.090319796025	0.996201972924392\\
137.112121691667	0.996203930198685\\
137.133923587309	0.996205886464316\\
137.155725482951	0.996207841721806\\
137.177527378593	0.996209795971675\\
137.199329274234	0.99621174921444\\
137.221131169876	0.996213701450622\\
137.242933065518	0.996215652680739\\
137.26473496116	0.99621760290531\\
137.286536856802	0.996219552124852\\
137.308338752444	0.996221500339884\\
137.330140648086	0.996223447550924\\
137.351942543728	0.996225393758488\\
137.373744439369	0.996227338963094\\
137.395546335011	0.996229283165259\\
137.417348230653	0.9962312263655\\
137.439150126295	0.996233168564332\\
137.460952021937	0.996235109762271\\
137.482753917579	0.996237049959835\\
137.504555813221	0.996238989157537\\
137.526357708862	0.996240927355894\\
137.548159604504	0.99624286455542\\
137.569961500146	0.99624480075663\\
137.591763395788	0.996246735960039\\
137.61356529143	0.99624867016616\\
137.635367187072	0.996250603375509\\
137.657169082714	0.996252535588597\\
137.678970978355	0.99625446680594\\
137.700772873997	0.996256397028049\\
137.722574769639	0.996258326255438\\
137.744376665281	0.99626025448862\\
137.766178560923	0.996262181728107\\
137.787980456565	0.99626410797441\\
137.809782352207	0.996266033228042\\
137.831584247848	0.996267957489515\\
137.85338614349	0.996269880759339\\
137.875188039132	0.996271803038025\\
137.896989934774	0.996273724326085\\
137.918791830416	0.996275644624029\\
137.940593726058	0.996277563932367\\
137.9623956217	0.996279482251609\\
137.984197517341	0.996281399582266\\
138.005999412983	0.996283315924845\\
138.027801308625	0.996285231279857\\
138.049603204267	0.99628714564781\\
138.071405099909	0.996289059029214\\
138.093206995551	0.996290971424576\\
138.115008891193	0.996292882834404\\
138.136810786835	0.996294793259208\\
138.158612682476	0.996296702699493\\
138.180414578118	0.996298611155768\\
138.20221647376	0.996300518628539\\
138.224018369402	0.996302425118315\\
138.245820265044	0.9963043306256\\
138.267622160686	0.996306235150901\\
138.289424056328	0.996308138694725\\
138.311225951969	0.996310041257577\\
138.333027847611	0.996311942839963\\
138.354829743253	0.996313843442388\\
138.376631638895	0.996315743065356\\
138.398433534537	0.996317641709374\\
138.420235430179	0.996319539374945\\
138.442037325821	0.996321436062574\\
138.463839221462	0.996323331772764\\
138.485641117104	0.996325226506019\\
138.507443012746	0.996327120262843\\
138.529244908388	0.996329013043738\\
138.55104680403	0.996330904849209\\
138.572848699672	0.996332795679757\\
138.594650595314	0.996334685535885\\
138.616452490955	0.996336574418094\\
138.638254386597	0.996338462326888\\
138.660056282239	0.996340349262767\\
138.681858177881	0.996342235226234\\
138.703660073523	0.996344120217788\\
138.725461969165	0.996346004237931\\
138.747263864807	0.996347887287164\\
138.769065760449	0.996349769365987\\
138.79086765609	0.996351650474899\\
138.812669551732	0.996353530614402\\
138.834471447374	0.996355409784994\\
138.856273343016	0.996357287987174\\
138.878075238658	0.996359165221443\\
138.8998771343	0.996361041488298\\
138.921679029942	0.996362916788238\\
138.943480925583	0.996364791121761\\
138.965282821225	0.996366664489366\\
138.987084716867	0.996368536891551\\
139.008886612509	0.996370408328812\\
139.030688508151	0.996372278801646\\
139.052490403793	0.996374148310552\\
139.074292299435	0.996376016856026\\
139.096094195076	0.996377884438564\\
139.117896090718	0.996379751058662\\
139.13969798636	0.996381616716817\\
139.161499882002	0.996383481413524\\
139.183301777644	0.996385345149279\\
139.205103673286	0.996387207924577\\
139.226905568928	0.996389069739913\\
139.248707464569	0.996390930595781\\
139.270509360211	0.996392790492676\\
139.292311255853	0.996394649431092\\
139.314113151495	0.996396507411524\\
139.335915047137	0.996398364434464\\
139.357716942779	0.996400220500407\\
139.379518838421	0.996402075609845\\
139.401320734062	0.996403929763272\\
139.423122629704	0.996405782961179\\
139.444924525346	0.99640763520406\\
139.466726420988	0.996409486492407\\
139.48852831663	0.996411336826711\\
139.510330212272	0.996413186207464\\
139.532132107914	0.996415034635158\\
139.553934003556	0.996416882110284\\
139.575735899197	0.996418728633332\\
139.597537794839	0.996420574204794\\
139.619339690481	0.996422418825159\\
139.641141586123	0.996424262494918\\
139.662943481765	0.996426105214561\\
139.684745377407	0.996427946984578\\
139.706547273049	0.996429787805457\\
139.72834916869	0.996431627677687\\
139.750151064332	0.996433466601759\\
139.771952959974	0.99643530457816\\
139.793754855616	0.996437141607378\\
139.815556751258	0.996438977689903\\
139.8373586469	0.996440812826221\\
139.859160542542	0.99644264701682\\
139.880962438183	0.996444480262188\\
139.902764333825	0.996446312562812\\
139.924566229467	0.996448143919179\\
139.946368125109	0.996449974331774\\
139.968170020751	0.996451803801086\\
139.989971916393	0.996453632327599\\
140.011773812035	0.9964554599118\\
140.033575707676	0.996457286554174\\
140.055377603318	0.996459112255207\\
140.07717949896	0.996460937015383\\
140.098981394602	0.996462760835188\\
140.120783290244	0.996464583715106\\
140.142585185886	0.996466405655622\\
140.164387081528	0.996468226657219\\
140.18618897717	0.996470046720382\\
140.207990872811	0.996471865845594\\
140.229792768453	0.996473684033338\\
140.251594664095	0.996475501284098\\
140.273396559737	0.996477317598357\\
140.295198455379	0.996479132976596\\
140.317000351021	0.996480947419299\\
140.338802246663	0.996482760926948\\
140.360604142304	0.996484573500024\\
140.382406037946	0.996486385139009\\
140.404207933588	0.996488195844384\\
140.42600982923	0.996490005616631\\
140.447811724872	0.996491814456231\\
140.469613620514	0.996493622363663\\
140.491415516156	0.996495429339409\\
140.513217411797	0.996497235383949\\
140.535019307439	0.996499040497762\\
140.556821203081	0.996500844681328\\
140.578623098723	0.996502647935127\\
140.600424994365	0.996504450259637\\
140.622226890007	0.996506251655338\\
140.644028785649	0.996508052122708\\
140.66583068129	0.996509851662226\\
140.687632576932	0.996511650274369\\
140.709434472574	0.996513447959616\\
140.731236368216	0.996515244718445\\
140.753038263858	0.996517040551332\\
140.7748401595	0.996518835458755\\
140.796642055142	0.996520629441191\\
140.818443950783	0.996522422499117\\
140.840245846425	0.996524214633009\\
140.862047742067	0.996526005843343\\
140.883849637709	0.996527796130595\\
140.905651533351	0.996529585495241\\
140.927453428993	0.996531373937755\\
140.949255324635	0.996533161458615\\
140.971057220277	0.996534948058294\\
140.992859115918	0.996536733737267\\
141.01466101156	0.996538518496009\\
141.036462907202	0.996540302334994\\
141.058264802844	0.996542085254696\\
141.080066698486	0.996543867255588\\
141.101868594128	0.996545648338144\\
141.12367048977	0.996547428502838\\
141.145472385411	0.996549207750142\\
141.167274281053	0.996550986080529\\
141.189076176695	0.996552763494472\\
141.210878072337	0.996554539992443\\
141.232679967979	0.996556315574914\\
141.254481863621	0.996558090242357\\
141.276283759263	0.996559863995243\\
141.298085654904	0.996561636834044\\
141.319887550546	0.99656340875923\\
141.341689446188	0.996565179771273\\
141.36349134183	0.996566949870643\\
141.385293237472	0.996568719057811\\
141.407095133114	0.996570487333246\\
141.428897028756	0.996572254697418\\
141.450698924397	0.996574021150798\\
141.472500820039	0.996575786693853\\
141.494302715681	0.996577551327055\\
141.516104611323	0.99657931505087\\
141.537906506965	0.996581077865769\\
141.559708402607	0.996582839772219\\
141.581510298249	0.996584600770689\\
141.603312193891	0.996586360861646\\
141.625114089532	0.996588120045558\\
141.646915985174	0.996589878322893\\
141.668717880816	0.996591635694118\\
141.690519776458	0.9965933921597\\
141.7123216721	0.996595147720105\\
141.734123567742	0.9965969023758\\
141.755925463384	0.996598656127251\\
141.777727359025	0.996600408974924\\
141.799529254667	0.996602160919285\\
141.821331150309	0.9966039119608\\
141.843133045951	0.996605662099934\\
141.864934941593	0.996607411337151\\
141.886736837235	0.996609159672917\\
141.908538732877	0.996610907107696\\
141.930340628518	0.996612653641952\\
141.95214252416	0.99661439927615\\
141.973944419802	0.996616144010753\\
141.995746315444	0.996617887846225\\
142.017548211086	0.996619630783029\\
142.039350106728	0.996621372821629\\
142.06115200237	0.996623113962487\\
142.082953898011	0.996624854206066\\
142.104755793653	0.996626593552828\\
142.126557689295	0.996628332003236\\
142.148359584937	0.996630069557751\\
142.170161480579	0.996631806216835\\
142.191963376221	0.99663354198095\\
142.213765271863	0.996635276850557\\
142.235567167504	0.996637010826116\\
142.257369063146	0.996638743908089\\
142.279170958788	0.996640476096935\\
142.30097285443	0.996642207393116\\
142.322774750072	0.996643937797092\\
142.344576645714	0.996645667309321\\
142.366378541356	0.996647395930264\\
142.388180436998	0.99664912366038\\
142.409982332639	0.996650850500128\\
142.431784228281	0.996652576449967\\
142.453586123923	0.996654301510355\\
142.475388019565	0.996656025681751\\
142.497189915207	0.996657748964613\\
142.518991810849	0.996659471359399\\
142.540793706491	0.996661192866566\\
142.562595602132	0.996662913486573\\
142.584397497774	0.996664633219875\\
142.606199393416	0.99666635206693\\
142.628001289058	0.996668070028196\\
142.6498031847	0.996669787104127\\
142.671605080342	0.996671503295182\\
142.693406975984	0.996673218601814\\
142.715208871625	0.996674933024481\\
142.737010767267	0.996676646563638\\
142.758812662909	0.99667835921974\\
142.780614558551	0.996680070993242\\
142.802416454193	0.9966817818846\\
142.824218349835	0.996683491894267\\
142.846020245477	0.996685201022698\\
142.867822141118	0.996686909270347\\
142.88962403676	0.996688616637669\\
142.911425932402	0.996690323125116\\
142.933227828044	0.996692028733142\\
142.955029723686	0.996693733462201\\
142.976831619328	0.996695437312745\\
142.99863351497	0.996697140285227\\
143.020435410611	0.9966988423801\\
143.042237306253	0.996700543597816\\
143.064039201895	0.996702243938827\\
143.085841097537	0.996703943403584\\
143.107642993179	0.99670564199254\\
143.129444888821	0.996707339706145\\
143.151246784463	0.996709036544851\\
143.173048680105	0.996710732509108\\
143.194850575746	0.996712427599368\\
143.216652471388	0.99671412181608\\
143.23845436703	0.996715815159696\\
143.260256262672	0.996717507630663\\
143.282058158314	0.996719199229434\\
143.303860053956	0.996720889956456\\
143.325661949598	0.996722579812179\\
143.347463845239	0.996724268797053\\
143.369265740881	0.996725956911526\\
143.391067636523	0.996727644156046\\
143.412869532165	0.996729330531061\\
143.434671427807	0.996731016037021\\
143.456473323449	0.996732700674373\\
143.478275219091	0.996734384443564\\
143.500077114732	0.996736067345042\\
143.521879010374	0.996737749379254\\
143.543680906016	0.996739430546647\\
143.565482801658	0.996741110847667\\
143.5872846973	0.996742790282762\\
143.609086592942	0.996744468852377\\
143.630888488584	0.996746146556958\\
143.652690384225	0.996747823396951\\
143.674492279867	0.996749499372802\\
143.696294175509	0.996751174484956\\
143.718096071151	0.996752848733859\\
143.739897966793	0.996754522119954\\
143.761699862435	0.996756194643687\\
143.783501758077	0.996757866305502\\
143.805303653719	0.996759537105843\\
143.82710554936	0.996761207045155\\
143.848907445002	0.99676287612388\\
143.870709340644	0.996764544342463\\
143.892511236286	0.996766211701346\\
143.914313131928	0.996767878200973\\
143.93611502757	0.996769543841787\\
143.957916923212	0.99677120862423\\
143.979718818853	0.996772872548744\\
144.001520714495	0.996774535615771\\
144.023322610137	0.996776197825755\\
144.045124505779	0.996777859179135\\
144.066926401421	0.996779519676354\\
144.088728297063	0.996781179317852\\
144.110530192705	0.996782838104072\\
144.132332088346	0.996784496035453\\
144.154133983988	0.996786153112436\\
144.17593587963	0.996787809335462\\
144.197737775272	0.996789464704969\\
144.219539670914	0.996791119221399\\
144.241341566556	0.996792772885192\\
144.263143462198	0.996794425696785\\
144.284945357839	0.996796077656619\\
144.306747253481	0.996797728765132\\
144.328549149123	0.996799379022764\\
144.350351044765	0.996801028429952\\
144.372152940407	0.996802676987135\\
144.393954836049	0.996804324694751\\
144.415756731691	0.996805971553238\\
144.437558627332	0.996807617563034\\
144.459360522974	0.996809262724575\\
144.481162418616	0.996810907038299\\
144.502964314258	0.996812550504642\\
144.5247662099	0.996814193124042\\
144.546568105542	0.996815834896935\\
144.568370001184	0.996817475823758\\
144.590171896826	0.996819115904945\\
144.611973792467	0.996820755140934\\
144.633775688109	0.996822393532159\\
144.655577583751	0.996824031079056\\
144.677379479393	0.99682566778206\\
144.699181375035	0.996827303641605\\
144.720983270677	0.996828938658128\\
144.742785166319	0.996830572832061\\
144.76458706196	0.99683220616384\\
144.786388957602	0.996833838653898\\
144.808190853244	0.996835470302668\\
144.829992748886	0.996837101110586\\
144.851794644528	0.996838731078083\\
144.87359654017	0.996840360205593\\
144.895398435812	0.996841988493549\\
144.917200331453	0.996843615942384\\
144.939002227095	0.996845242552529\\
144.960804122737	0.996846868324418\\
144.982606018379	0.996848493258482\\
145.004407914021	0.996850117355152\\
145.026209809663	0.996851740614861\\
145.048011705305	0.99685336303804\\
145.069813600946	0.99685498462512\\
145.091615496588	0.996856605376532\\
145.11341739223	0.996858225292705\\
145.135219287872	0.996859844374072\\
145.157021183514	0.996861462621062\\
145.178823079156	0.996863080034104\\
145.200624974798	0.99686469661363\\
145.22242687044	0.996866312360068\\
145.244228766081	0.996867927273847\\
145.266030661723	0.996869541355397\\
145.287832557365	0.996871154605147\\
145.309634453007	0.996872767023525\\
145.331436348649	0.99687437861096\\
145.353238244291	0.99687598936788\\
145.375040139933	0.996877599294713\\
145.396842035574	0.996879208391886\\
145.418643931216	0.996880816659828\\
145.440445826858	0.996882424098965\\
145.4622477225	0.996884030709726\\
145.484049618142	0.996885636492535\\
145.505851513784	0.996887241447821\\
145.527653409426	0.99688884557601\\
145.549455305067	0.996890448877528\\
145.571257200709	0.996892051352801\\
145.593059096351	0.996893653002254\\
145.614860991993	0.996895253826314\\
145.636662887635	0.996896853825406\\
145.658464783277	0.996898452999955\\
145.680266678919	0.996900051350385\\
145.70206857456	0.996901648877122\\
145.723870470202	0.99690324558059\\
145.745672365844	0.996904841461213\\
145.767474261486	0.996906436519415\\
145.789276157128	0.996908030755621\\
145.81107805277	0.996909624170253\\
145.832879948412	0.996911216763736\\
145.854681844054	0.996912808536491\\
145.876483739695	0.996914399488943\\
145.898285635337	0.996915989621514\\
145.920087530979	0.996917578934626\\
145.941889426621	0.996919167428702\\
145.963691322263	0.996920755104165\\
145.985493217905	0.996922341961434\\
146.007295113547	0.996923928000933\\
146.029097009188	0.996925513223084\\
146.05089890483	0.996927097628305\\
146.072700800472	0.99692868121702\\
146.094502696114	0.996930263989649\\
146.116304591756	0.996931845946612\\
146.138106487398	0.99693342708833\\
146.15990838304	0.996935007415223\\
146.181710278681	0.996936586927711\\
146.203512174323	0.996938165626213\\
146.225314069965	0.996939743511148\\
146.247115965607	0.996941320582937\\
146.268917861249	0.996942896841999\\
146.290719756891	0.996944472288751\\
146.312521652533	0.996946046923613\\
146.334323548174	0.996947620747003\\
146.356125443816	0.99694919375934\\
146.377927339458	0.99695076596104\\
146.3997292351	0.996952337352523\\
146.421531130742	0.996953907934205\\
146.443333026384	0.996955477706504\\
146.465134922026	0.996957046669836\\
146.486936817667	0.99695861482462\\
146.508738713309	0.996960182171271\\
146.530540608951	0.996961748710206\\
146.552342504593	0.996963314441842\\
146.574144400235	0.996964879366594\\
146.595946295877	0.996966443484877\\
146.617748191519	0.996968006797109\\
146.639550087161	0.996969569303703\\
146.661351982802	0.996971131005076\\
146.683153878444	0.996972691901642\\
146.704955774086	0.996974251993816\\
146.726757669728	0.996975811282013\\
146.74855956537	0.996977369766646\\
146.770361461012	0.99697892744813\\
146.792163356654	0.996980484326879\\
146.813965252295	0.996982040403307\\
146.835767147937	0.996983595677826\\
146.857569043579	0.996985150150851\\
146.879370939221	0.996986703822794\\
146.901172834863	0.996988256694068\\
146.922974730505	0.996989808765086\\
146.944776626147	0.99699136003626\\
146.966578521788	0.996992910508002\\
146.98838041743	0.996994460180725\\
147.010182313072	0.996996009054839\\
147.031984208714	0.996997557130758\\
147.053786104356	0.996999104408891\\
147.075587999998	0.99700065088965\\
147.09738989564	0.997002196573447\\
147.119191791281	0.997003741460691\\
147.140993686923	0.997005285551793\\
147.162795582565	0.997006828847164\\
147.184597478207	0.997008371347213\\
147.206399373849	0.99700991305235\\
147.228201269491	0.997011453962986\\
147.250003165133	0.997012994079529\\
147.271805060775	0.997014533402389\\
147.293606956416	0.997016071931975\\
147.315408852058	0.997017609668695\\
147.3372107477	0.997019146612959\\
147.359012643342	0.997020682765174\\
147.380814538984	0.997022218125749\\
147.402616434626	0.997023752695091\\
147.424418330268	0.997025286473609\\
147.446220225909	0.99702681946171\\
147.468022121551	0.997028351659801\\
147.489824017193	0.99702988306829\\
147.511625912835	0.997031413687583\\
147.533427808477	0.997032943518086\\
147.555229704119	0.997034472560208\\
147.577031599761	0.997036000814353\\
147.598833495402	0.997037528280928\\
147.620635391044	0.99703905496034\\
147.642437286686	0.997040580852992\\
147.664239182328	0.997042105959291\\
147.68604107797	0.997043630279643\\
147.707842973612	0.997045153814451\\
147.729644869254	0.997046676564122\\
147.751446764895	0.997048198529059\\
147.773248660537	0.997049719709667\\
147.795050556179	0.99705124010635\\
147.816852451821	0.997052759719513\\
147.838654347463	0.997054278549558\\
147.860456243105	0.99705579659689\\
147.882258138747	0.997057313861912\\
147.904060034388	0.997058830345027\\
147.92586193003	0.997060346046637\\
147.947663825672	0.997061860967147\\
147.969465721314	0.997063375106958\\
147.991267616956	0.997064888466472\\
148.013069512598	0.997066401046092\\
148.03487140824	0.99706791284622\\
148.056673303882	0.997069423867257\\
148.078475199523	0.997070934109605\\
148.100277095165	0.997072443573665\\
148.122078990807	0.997073952259838\\
148.143880886449	0.997075460168525\\
148.165682782091	0.997076967300127\\
148.187484677733	0.997078473655044\\
148.209286573375	0.997079979233676\\
148.231088469016	0.997081484036424\\
148.252890364658	0.997082988063687\\
148.2746922603	0.997084491315866\\
148.296494155942	0.997085993793358\\
148.318296051584	0.997087495496565\\
148.340097947226	0.997088996425884\\
148.361899842868	0.997090496581714\\
148.383701738509	0.997091995964455\\
148.405503634151	0.997093494574504\\
148.427305529793	0.997094992412259\\
148.449107425435	0.99709648947812\\
148.470909321077	0.997097985772483\\
148.492711216719	0.997099481295745\\
148.514513112361	0.997100976048306\\
148.536315008002	0.997102470030561\\
148.558116903644	0.997103963242907\\
148.579918799286	0.997105455685742\\
148.601720694928	0.997106947359461\\
148.62352259057	0.997108438264462\\
148.645324486212	0.997109928401141\\
148.667126381854	0.997111417769893\\
148.688928277496	0.997112906371113\\
148.710730173137	0.997114394205199\\
148.732532068779	0.997115881272544\\
148.754333964421	0.997117367573545\\
148.776135860063	0.997118853108596\\
148.797937755705	0.997120337878091\\
148.819739651347	0.997121821882426\\
148.841541546989	0.997123305121994\\
148.86334344263	0.99712478759719\\
148.885145338272	0.997126269308407\\
148.906947233914	0.99712775025604\\
148.928749129556	0.997129230440481\\
148.950551025198	0.997130709862125\\
148.97235292084	0.997132188521364\\
148.994154816482	0.997133666418591\\
149.015956712123	0.997135143554199\\
149.037758607765	0.99713661992858\\
149.059560503407	0.997138095542127\\
149.081362399049	0.997139570395232\\
149.103164294691	0.997141044488286\\
149.124966190333	0.997142517821681\\
149.146768085975	0.997143990395809\\
149.168569981616	0.997145462211062\\
149.190371877258	0.997146933267829\\
149.2121737729	0.997148403566502\\
149.233975668542	0.997149873107472\\
149.255777564184	0.997151341891129\\
149.277579459826	0.997152809917864\\
149.299381355468	0.997154277188066\\
149.321183251109	0.997155743702125\\
149.342985146751	0.997157209460432\\
149.364787042393	0.997158674463374\\
149.386588938035	0.997160138711343\\
149.408390833677	0.997161602204727\\
149.430192729319	0.997163064943914\\
149.451994624961	0.997164526929293\\
149.473796520603	0.997165988161254\\
149.495598416244	0.997167448640183\\
149.517400311886	0.99716890836647\\
149.539202207528	0.997170367340501\\
149.56100410317	0.997171825562666\\
149.582805998812	0.99717328303335\\
149.604607894454	0.997174739752942\\
149.626409790096	0.997176195721828\\
149.648211685737	0.997177650940396\\
149.670013581379	0.997179105409032\\
149.691815477021	0.997180559128122\\
149.713617372663	0.997182012098053\\
149.735419268305	0.997183464319211\\
149.757221163947	0.997184915791982\\
149.779023059589	0.997186366516751\\
149.80082495523	0.997187816493904\\
149.822626850872	0.997189265723826\\
149.844428746514	0.997190714206903\\
149.866230642156	0.997192161943518\\
149.888032537798	0.997193608934057\\
149.90983443344	0.997195055178905\\
149.931636329082	0.997196500678445\\
149.953438224723	0.997197945433062\\
149.975240120365	0.997199389443139\\
149.997042016007	0.99720083270906\\
150.018843911649	0.997202275231209\\
150.040645807291	0.99720371700997\\
150.062447702933	0.997205158045724\\
150.084249598575	0.997206598338855\\
150.106051494217	0.997208037889746\\
150.127853389858	0.997209476698779\\
150.1496552855	0.997210914766337\\
150.171457181142	0.997212352092802\\
150.193259076784	0.997213788678554\\
150.215060972426	0.997215224523978\\
150.236862868068	0.997216659629452\\
150.25866476371	0.997218093995361\\
150.280466659351	0.997219527622083\\
150.302268554993	0.99722096051\\
150.324070450635	0.997222392659494\\
150.345872346277	0.997223824070944\\
150.367674241919	0.99722525474473\\
150.389476137561	0.997226684681234\\
150.411278033203	0.997228113880834\\
150.433079928844	0.997229542343912\\
150.454881824486	0.997230970070845\\
150.476683720128	0.997232397062014\\
150.49848561577	0.997233823317798\\
150.520287511412	0.997235248838575\\
150.542089407054	0.997236673624725\\
150.563891302696	0.997238097676626\\
150.585693198337	0.997239520994656\\
150.607495093979	0.997240943579194\\
150.629296989621	0.997242365430618\\
150.651098885263	0.997243786549305\\
150.672900780905	0.997245206935633\\
150.694702676547	0.99724662658998\\
150.716504572189	0.997248045512722\\
150.73830646783	0.997249463704237\\
150.760108363472	0.997250881164902\\
150.781910259114	0.997252297895093\\
150.803712154756	0.997253713895186\\
150.825514050398	0.997255129165559\\
150.84731594604	0.997256543706586\\
150.869117841682	0.997257957518644\\
150.890919737324	0.997259370602109\\
150.912721632965	0.997260782957355\\
150.934523528607	0.997262194584759\\
150.956325424249	0.997263605484695\\
150.978127319891	0.997265015657538\\
150.999929215533	0.997266425103663\\
151.021731111175	0.997267833823445\\
151.043533006817	0.997269241817257\\
151.065334902458	0.997270649085474\\
151.0871367981	0.99727205562847\\
151.108938693742	0.997273461446618\\
151.130740589384	0.997274866540292\\
151.152542485026	0.997276270909866\\
151.174344380668	0.997277674555712\\
151.19614627631	0.997279077478204\\
151.217948171951	0.997280479677714\\
151.239750067593	0.997281881154615\\
151.261551963235	0.997283281909279\\
151.283353858877	0.997284681942078\\
151.305155754519	0.997286081253385\\
151.326957650161	0.997287479843572\\
151.348759545803	0.997288877713009\\
151.370561441444	0.997290274862069\\
151.392363337086	0.997291671291122\\
151.414165232728	0.997293067000539\\
151.43596712837	0.997294461990692\\
151.457769024012	0.997295856261952\\
151.479570919654	0.997297249814688\\
151.501372815296	0.997298642649271\\
151.523174710937	0.99730003476607\\
151.544976606579	0.997301426165457\\
151.566778502221	0.997302816847801\\
151.588580397863	0.99730420681347\\
151.610382293505	0.997305596062835\\
151.632184189147	0.997306984596264\\
151.653986084789	0.997308372414127\\
151.675787980431	0.997309759516793\\
151.697589876072	0.997311145904629\\
151.719391771714	0.997312531578004\\
151.741193667356	0.997313916537287\\
151.762995562998	0.997315300782845\\
151.78479745864	0.997316684315047\\
151.806599354282	0.997318067134259\\
151.828401249924	0.99731944924085\\
151.850203145565	0.997320830635186\\
151.872005041207	0.997322211317635\\
151.893806936849	0.997323591288564\\
151.915608832491	0.997324970548338\\
151.937410728133	0.997326349097325\\
151.959212623775	0.997327726935891\\
151.981014519417	0.997329104064402\\
152.002816415058	0.997330480483224\\
152.0246183107	0.997331856192723\\
152.046420206342	0.997333231193264\\
152.068222101984	0.997334605485212\\
152.090023997626	0.997335979068934\\
152.111825893268	0.997337351944793\\
152.13362778891	0.997338724113154\\
152.155429684551	0.997340095574383\\
152.177231580193	0.997341466328843\\
152.199033475835	0.997342836376898\\
152.220835371477	0.997344205718914\\
152.242637267119	0.997345574355253\\
152.264439162761	0.99734694228628\\
152.286241058403	0.997348309512357\\
152.308042954045	0.997349676033849\\
152.329844849686	0.997351041851117\\
152.351646745328	0.997352406964526\\
152.37344864097	0.997353771374438\\
152.395250536612	0.997355135081214\\
152.417052432254	0.997356498085219\\
152.438854327896	0.997357860386813\\
152.460656223538	0.99735922198636\\
152.482458119179	0.99736058288422\\
152.504260014821	0.997361943080755\\
152.526061910463	0.997363302576327\\
152.547863806105	0.997364661371296\\
152.569665701747	0.997366019466025\\
152.591467597389	0.997367376860873\\
152.613269493031	0.997368733556202\\
152.635071388672	0.997370089552372\\
152.656873284314	0.997371444849743\\
152.678675179956	0.997372799448676\\
152.700477075598	0.99737415334953\\
152.72227897124	0.997375506552665\\
152.744080866882	0.997376859058441\\
152.765882762524	0.997378210867217\\
152.787684658165	0.997379561979351\\
152.809486553807	0.997380912395205\\
152.831288449449	0.997382262115135\\
152.853090345091	0.997383611139501\\
152.874892240733	0.997384959468661\\
152.896694136375	0.997386307102973\\
152.918496032017	0.997387654042796\\
152.940297927658	0.997389000288487\\
152.9620998233	0.997390345840405\\
152.983901718942	0.997391690698906\\
153.005703614584	0.997393034864347\\
153.027505510226	0.997394378337088\\
153.049307405868	0.997395721117483\\
153.07110930151	0.99739706320589\\
153.092911197152	0.997398404602665\\
153.114713092793	0.997399745308166\\
153.136514988435	0.997401085322748\\
153.158316884077	0.997402424646767\\
153.180118779719	0.997403763280579\\
153.201920675361	0.99740510122454\\
153.223722571003	0.997406438479005\\
153.245524466645	0.997407775044329\\
153.267326362286	0.997409110920869\\
153.289128257928	0.997410446108978\\
153.31093015357	0.997411780609012\\
153.332732049212	0.997413114421325\\
153.354533944854	0.997414447546271\\
153.376335840496	0.997415779984205\\
153.398137736138	0.997417111735481\\
153.419939631779	0.997418442800453\\
153.441741527421	0.997419773179474\\
153.463543423063	0.997421102872897\\
153.485345318705	0.997422431881077\\
153.507147214347	0.997423760204366\\
153.528949109989	0.997425087843118\\
153.550751005631	0.997426414797684\\
153.572552901272	0.997427741068418\\
153.594354796914	0.997429066655672\\
153.616156692556	0.997430391559798\\
153.637958588198	0.997431715781148\\
153.65976048384	0.997433039320074\\
153.681562379482	0.997434362176929\\
153.703364275124	0.997435684352062\\
153.725166170766	0.997437005845826\\
153.746968066407	0.997438326658572\\
153.768769962049	0.997439646790651\\
153.790571857691	0.997440966242413\\
153.812373753333	0.997442285014209\\
153.834175648975	0.997443603106389\\
153.855977544617	0.997444920519305\\
153.877779440259	0.997446237253305\\
153.8995813359	0.997447553308739\\
153.921383231542	0.997448868685958\\
153.943185127184	0.997450183385311\\
153.964987022826	0.997451497407147\\
153.986788918468	0.997452810751815\\
154.00859081411	0.997454123419665\\
154.030392709752	0.997455435411045\\
154.052194605393	0.997456746726303\\
154.073996501035	0.997458057365789\\
154.095798396677	0.99745936732985\\
154.117600292319	0.997460676618834\\
154.139402187961	0.99746198523309\\
154.161204083603	0.997463293172965\\
154.183005979245	0.997464600438806\\
154.204807874886	0.997465907030962\\
154.226609770528	0.997467212949778\\
154.24841166617	0.997468518195603\\
154.270213561812	0.997469822768783\\
154.292015457454	0.997471126669664\\
154.313817353096	0.997472429898593\\
154.335619248738	0.997473732455917\\
154.357421144379	0.997475034341981\\
154.379223040021	0.997476335557131\\
154.401024935663	0.997477636101714\\
154.422826831305	0.997478935976074\\
154.444628726947	0.997480235180558\\
154.466430622589	0.997481533715509\\
154.488232518231	0.997482831581274\\
154.510034413873	0.997484128778197\\
154.531836309514	0.997485425306623\\
154.553638205156	0.997486721166897\\
154.575440100798	0.997488016359362\\
154.59724199644	0.997489310884363\\
154.619043892082	0.997490604742244\\
154.640845787724	0.997491897933348\\
154.662647683366	0.99749319045802\\
154.684449579007	0.997494482316602\\
154.706251474649	0.997495773509438\\
154.728053370291	0.997497064036871\\
154.749855265933	0.997498353899245\\
154.771657161575	0.9974996430969\\
154.793459057217	0.997500931630181\\
154.815260952859	0.99750221949943\\
154.8370628485	0.997503506704988\\
154.858864744142	0.997504793247198\\
154.880666639784	0.997506079126402\\
154.902468535426	0.997507364342942\\
154.924270431068	0.997508648897158\\
154.94607232671	0.997509932789392\\
154.967874222352	0.997511216019985\\
154.989676117993	0.997512498589279\\
155.011478013635	0.997513780497614\\
155.033279909277	0.997515061745331\\
155.055081804919	0.997516342332769\\
155.076883700561	0.99751762226027\\
155.098685596203	0.997518901528174\\
155.120487491845	0.99752018013682\\
155.142289387486	0.997521458086548\\
155.164091283128	0.997522735377698\\
155.18589317877	0.997524012010609\\
155.207695074412	0.99752528798562\\
155.229496970054	0.997526563303071\\
155.251298865696	0.9975278379633\\
155.273100761338	0.997529111966646\\
155.29490265698	0.997530385313447\\
155.316704552621	0.997531658004042\\
155.338506448263	0.997532930038769\\
155.360308343905	0.997534201417966\\
155.382110239547	0.997535472141971\\
155.403912135189	0.997536742211121\\
155.425714030831	0.997538011625755\\
155.447515926473	0.997539280386208\\
155.469317822114	0.997540548492818\\
155.491119717756	0.997541815945923\\
155.512921613398	0.997543082745859\\
155.53472350904	0.997544348892963\\
155.556525404682	0.99754561438757\\
155.578327300324	0.997546879230018\\
155.600129195966	0.997548143420642\\
155.621931091607	0.997549406959779\\
155.643732987249	0.997550669847764\\
155.665534882891	0.997551932084932\\
155.687336778533	0.99755319367162\\
155.709138674175	0.997554454608161\\
155.730940569817	0.997555714894892\\
155.752742465459	0.997556974532147\\
155.7745443611	0.997558233520261\\
155.796346256742	0.997559491859569\\
155.818148152384	0.997560749550404\\
155.839950048026	0.997562006593101\\
155.861751943668	0.997563262987994\\
155.88355383931	0.997564518735417\\
155.905355734952	0.997565773835703\\
155.927157630594	0.997567028289186\\
155.948959526235	0.997568282096199\\
155.970761421877	0.997569535257076\\
155.992563317519	0.99757078777215\\
156.014365213161	0.997572039641752\\
156.036167108803	0.997573290866216\\
156.057969004445	0.997574541445875\\
156.079770900087	0.99757579138106\\
156.101572795728	0.997577040672104\\
156.12337469137	0.997578289319339\\
156.145176587012	0.997579537323096\\
156.166978482654	0.997580784683707\\
156.188780378296	0.997582031401503\\
156.210582273938	0.997583277476817\\
156.23238416958	0.997584522909978\\
156.254186065221	0.997585767701318\\
156.275987960863	0.997587011851168\\
156.297789856505	0.997588255359857\\
156.319591752147	0.997589498227718\\
156.341393647789	0.997590740455079\\
156.363195543431	0.997591982042271\\
156.384997439073	0.997593222989624\\
156.406799334714	0.997594463297467\\
156.428601230356	0.99759570296613\\
156.450403125998	0.997596941995943\\
156.47220502164	0.997598180387235\\
156.494006917282	0.997599418140334\\
156.515808812924	0.99760065525557\\
156.537610708566	0.997601891733271\\
156.559412604207	0.997603127573766\\
156.581214499849	0.997604362777384\\
156.603016395491	0.997605597344452\\
156.624818291133	0.997606831275298\\
156.646620186775	0.997608064570251\\
156.668422082417	0.997609297229638\\
156.690223978059	0.997610529253787\\
156.712025873701	0.997611760643024\\
156.733827769342	0.997612991397678\\
156.755629664984	0.997614221518075\\
156.777431560626	0.997615451004542\\
156.799233456268	0.997616679857406\\
156.82103535191	0.997617908076993\\
156.842837247552	0.99761913566363\\
156.864639143194	0.997620362617642\\
156.886441038835	0.997621588939357\\
156.908242934477	0.997622814629099\\
156.930044830119	0.997624039687194\\
156.951846725761	0.997625264113968\\
156.973648621403	0.997626487909747\\
156.995450517045	0.997627711074855\\
157.017252412687	0.997628933609617\\
157.039054308328	0.997630155514359\\
157.06085620397	0.997631376789404\\
157.082658099612	0.997632597435078\\
157.104459995254	0.997633817451705\\
157.126261890896	0.997635036839609\\
157.148063786538	0.997636255599113\\
157.16986568218	0.997637473730543\\
157.191667577821	0.997638691234221\\
157.213469473463	0.99763990811047\\
157.235271369105	0.997641124359616\\
157.257073264747	0.997642339981979\\
157.278875160389	0.997643554977884\\
157.300677056031	0.997644769347654\\
157.322478951673	0.99764598309161\\
157.344280847315	0.997647196210076\\
157.366082742956	0.997648408703374\\
157.387884638598	0.997649620571826\\
157.40968653424	0.997650831815754\\
157.431488429882	0.997652042435479\\
157.453290325524	0.997653252431325\\
157.475092221166	0.997654461803611\\
157.496894116808	0.99765567055266\\
157.518696012449	0.997656878678792\\
157.540497908091	0.997658086182329\\
157.562299803733	0.997659293063591\\
157.584101699375	0.997660499322899\\
157.605903595017	0.997661704960574\\
157.627705490659	0.997662909976936\\
157.649507386301	0.997664114372306\\
157.671309281942	0.997665318147002\\
157.693111177584	0.997666521301345\\
157.714913073226	0.997667723835656\\
157.736714968868	0.997668925750252\\
157.75851686451	0.997670127045454\\
157.780318760152	0.997671327721581\\
157.802120655794	0.997672527778952\\
157.823922551435	0.997673727217886\\
157.845724447077	0.997674926038701\\
157.867526342719	0.997676124241716\\
157.889328238361	0.997677321827249\\
157.911130134003	0.997678518795619\\
157.932932029645	0.997679715147144\\
157.954733925287	0.997680910882141\\
157.976535820928	0.997682106000928\\
157.99833771657	0.997683300503823\\
158.020139612212	0.997684494391143\\
158.041941507854	0.997685687663206\\
158.063743403496	0.997686880320329\\
158.085545299138	0.997688072362827\\
158.10734719478	0.997689263791019\\
158.129149090422	0.99769045460522\\
158.150950986063	0.997691644805748\\
158.172752881705	0.997692834392918\\
158.194554777347	0.997694023367046\\
158.216356672989	0.997695211728449\\
158.238158568631	0.997696399477442\\
158.259960464273	0.997697586614341\\
158.281762359915	0.997698773139461\\
158.303564255556	0.997699959053117\\
158.325366151198	0.997701144355625\\
158.34716804684	0.997702329047299\\
158.368969942482	0.997703513128455\\
158.390771838124	0.997704696599406\\
158.412573733766	0.997705879460468\\
158.434375629408	0.997707061711955\\
158.456177525049	0.997708243354181\\
158.477979420691	0.997709424387459\\
158.499781316333	0.997710604812104\\
158.521583211975	0.997711784628429\\
158.543385107617	0.997712963836748\\
158.565187003259	0.997714142437374\\
158.586988898901	0.99771532043062\\
158.608790794542	0.997716497816799\\
158.630592690184	0.997717674596225\\
158.652394585826	0.99771885076921\\
158.674196481468	0.997720026336065\\
158.69599837711	0.997721201297105\\
158.717800272752	0.99772237565264\\
158.739602168394	0.997723549402983\\
158.761404064036	0.997724722548446\\
158.783205959677	0.997725895089341\\
158.805007855319	0.997727067025979\\
158.826809750961	0.997728238358671\\
158.848611646603	0.997729409087728\\
158.870413542245	0.997730579213463\\
158.892215437887	0.997731748736185\\
158.914017333529	0.997732917656206\\
158.93581922917	0.997734085973836\\
158.957621124812	0.997735253689385\\
158.979423020454	0.997736420803164\\
159.001224916096	0.997737587315483\\
159.023026811738	0.997738753226652\\
159.04482870738	0.997739918536981\\
159.066630603022	0.997741083246778\\
159.088432498663	0.997742247356355\\
159.110234394305	0.99774341086602\\
159.132036289947	0.997744573776082\\
159.153838185589	0.99774573608685\\
159.175640081231	0.997746897798633\\
159.197441976873	0.99774805891174\\
159.219243872515	0.997749219426479\\
159.241045768156	0.997750379343158\\
159.262847663798	0.997751538662087\\
159.28464955944	0.997752697383573\\
159.306451455082	0.997753855507923\\
159.328253350724	0.997755013035446\\
159.350055246366	0.997756169966448\\
159.371857142008	0.997757326301239\\
159.39365903765	0.997758482040124\\
159.415460933291	0.99775963718341\\
159.437262828933	0.997760791731406\\
159.459064724575	0.997761945684417\\
159.480866620217	0.99776309904275\\
159.502668515859	0.997764251806712\\
159.524470411501	0.997765403976609\\
159.546272307143	0.997766555552747\\
159.568074202784	0.997767706535432\\
159.589876098426	0.99776885692497\\
159.611677994068	0.997770006721666\\
159.63347988971	0.997771155925826\\
159.655281785352	0.997772304537756\\
159.677083680994	0.997773452557761\\
159.698885576636	0.997774599986145\\
159.720687472277	0.997775746823214\\
159.742489367919	0.997776893069271\\
159.764291263561	0.997778038724623\\
159.786093159203	0.997779183789573\\
159.807895054845	0.997780328264426\\
159.829696950487	0.997781472149485\\
159.851498846129	0.997782615445055\\
159.87330074177	0.997783758151439\\
159.895102637412	0.997784900268941\\
159.916904533054	0.997786041797865\\
159.938706428696	0.997787182738513\\
159.960508324338	0.99778832309119\\
159.98231021998	0.997789462856197\\
160.004112115622	0.997790602033838\\
160.025914011263	0.997791740624416\\
160.047715906905	0.997792878628232\\
160.069517802547	0.99779401604559\\
160.091319698189	0.997795152876792\\
160.113121593831	0.997796289122139\\
160.134923489473	0.997797424781935\\
160.156725385115	0.997798559856479\\
160.178527280757	0.997799694346074\\
160.200329176398	0.997800828251022\\
160.22213107204	0.997801961571623\\
160.243932967682	0.99780309430818\\
160.265734863324	0.997804226460992\\
160.287536758966	0.997805358030361\\
160.309338654608	0.997806489016587\\
160.33114055025	0.997807619419971\\
160.352942445891	0.997808749240813\\
160.374744341533	0.997809878479414\\
160.396546237175	0.997811007136073\\
160.418348132817	0.99781213521109\\
160.440150028459	0.997813262704766\\
160.461951924101	0.997814389617399\\
160.483753819743	0.997815515949289\\
160.505555715384	0.997816641700736\\
160.527357611026	0.997817766872039\\
160.549159506668	0.997818891463496\\
160.57096140231	0.997820015475406\\
160.592763297952	0.997821138908068\\
160.614565193594	0.997822261761781\\
160.636367089236	0.997823384036842\\
160.658168984877	0.997824505733551\\
160.679970880519	0.997825626852205\\
160.701772776161	0.997826747393101\\
160.723574671803	0.997827867356539\\
160.745376567445	0.997828986742814\\
160.767178463087	0.997830105552226\\
160.788980358729	0.99783122378507\\
160.810782254371	0.997832341441645\\
160.832584150012	0.997833458522247\\
160.854386045654	0.997834575027173\\
160.876187941296	0.99783569095672\\
160.897989836938	0.997836806311183\\
160.91979173258	0.997837921090861\\
160.941593628222	0.997839035296047\\
160.963395523864	0.99784014892704\\
160.985197419505	0.997841261984135\\
161.006999315147	0.997842374467626\\
161.028801210789	0.997843486377811\\
161.050603106431	0.997844597714985\\
161.072405002073	0.997845708479442\\
161.094206897715	0.997846818671478\\
161.116008793357	0.997847928291388\\
161.137810688998	0.997849037339467\\
161.15961258464	0.997850145816009\\
161.181414480282	0.997851253721309\\
161.203216375924	0.997852361055662\\
161.225018271566	0.997853467819361\\
161.246820167208	0.997854574012701\\
161.26862206285	0.997855679635976\\
161.290423958491	0.997856784689479\\
161.312225854133	0.997857889173503\\
161.334027749775	0.997858993088344\\
161.355829645417	0.997860096434293\\
161.377631541059	0.997861199211644\\
161.399433436701	0.99786230142069\\
161.421235332343	0.997863403061724\\
161.443037227984	0.997864504135039\\
161.464839123626	0.997865604640927\\
161.486641019268	0.99786670457968\\
161.50844291491	0.997867803951591\\
161.530244810552	0.997868902756953\\
161.552046706194	0.997870000996056\\
161.573848601836	0.997871098669193\\
161.595650497478	0.997872195776655\\
161.617452393119	0.997873292318734\\
161.639254288761	0.997874388295722\\
161.661056184403	0.997875483707909\\
161.682858080045	0.997876578555586\\
161.704659975687	0.997877672839045\\
161.726461871329	0.997878766558577\\
161.748263766971	0.997879859714471\\
161.770065662612	0.997880952307018\\
161.791867558254	0.997882044336509\\
161.813669453896	0.997883135803234\\
161.835471349538	0.997884226707483\\
161.85727324518	0.997885317049545\\
161.879075140822	0.997886406829711\\
161.900877036464	0.99788749604827\\
161.922678932105	0.997888584705511\\
161.944480827747	0.997889672801724\\
161.966282723389	0.997890760337198\\
161.988084619031	0.997891847312221\\
162.009886514673	0.997892933727083\\
162.031688410315	0.997894019582072\\
162.053490305957	0.997895104877477\\
162.075292201598	0.997896189613586\\
162.09709409724	0.997897273790688\\
162.118895992882	0.99789835740907\\
162.140697888524	0.99789944046902\\
162.162499784166	0.997900522970826\\
162.184301679808	0.997901604914777\\
162.20610357545	0.997902686301158\\
162.227905471092	0.997903767130258\\
162.249707366733	0.997904847402364\\
162.271509262375	0.997905927117763\\
162.293311158017	0.997907006276742\\
162.315113053659	0.997908084879587\\
162.336914949301	0.997909162926585\\
162.358716844943	0.997910240418023\\
162.380518740585	0.997911317354186\\
162.402320636226	0.997912393735362\\
162.424122531868	0.997913469561835\\
162.44592442751	0.997914544833893\\
162.467726323152	0.99791561955182\\
162.489528218794	0.997916693715902\\
162.511330114436	0.997917767326424\\
162.533132010078	0.997918840383673\\
162.554933905719	0.997919912887932\\
162.576735801361	0.997920984839488\\
162.598537697003	0.997922056238624\\
162.620339592645	0.997923127085626\\
162.642141488287	0.997924197380777\\
162.663943383929	0.997925267124363\\
162.685745279571	0.997926336316668\\
162.707547175212	0.997927404957975\\
162.729349070854	0.997928473048569\\
162.751150966496	0.997929540588733\\
162.772952862138	0.997930607578752\\
162.79475475778	0.997931674018908\\
162.816556653422	0.997932739909485\\
162.838358549064	0.997933805250767\\
162.860160444705	0.997934870043035\\
162.881962340347	0.997935934286574\\
162.903764235989	0.997936997981666\\
162.925566131631	0.997938061128594\\
162.947368027273	0.99793912372764\\
162.969169922915	0.997940185779086\\
162.990971818557	0.997941247283215\\
163.012773714199	0.997942308240308\\
163.03457560984	0.997943368650648\\
163.056377505482	0.997944428514517\\
163.078179401124	0.997945487832196\\
163.099981296766	0.997946546603966\\
163.121783192408	0.997947604830109\\
163.14358508805	0.997948662510906\\
163.165386983692	0.997949719646638\\
163.187188879333	0.997950776237586\\
163.208990774975	0.997951832284031\\
163.230792670617	0.997952887786253\\
163.252594566259	0.997953942744533\\
163.274396461901	0.997954997159151\\
163.296198357543	0.997956051030387\\
163.318000253185	0.997957104358521\\
163.339802148826	0.997958157143834\\
163.361604044468	0.997959209386604\\
163.38340594011	0.997960261087112\\
163.405207835752	0.997961312245637\\
163.427009731394	0.997962362862458\\
163.448811627036	0.997963412937855\\
163.470613522678	0.997964462472106\\
163.492415418319	0.99796551146549\\
163.514217313961	0.997966559918286\\
163.536019209603	0.997967607830773\\
163.557821105245	0.997968655203229\\
163.579623000887	0.997969702035932\\
163.601424896529	0.997970748329161\\
163.623226792171	0.997971794083193\\
163.645028687812	0.997972839298306\\
163.666830583454	0.997973883974779\\
163.688632479096	0.997974928112888\\
163.710434374738	0.997975971712912\\
163.73223627038	0.997977014775127\\
163.754038166022	0.997978057299811\\
163.775840061664	0.99797909928724\\
163.797641957306	0.997980140737691\\
163.819443852947	0.997981181651442\\
163.841245748589	0.997982222028769\\
163.863047644231	0.997983261869948\\
163.884849539873	0.997984301175255\\
163.906651435515	0.997985339944967\\
163.928453331157	0.99798637817936\\
163.950255226799	0.997987415878709\\
163.97205712244	0.997988453043291\\
163.993859018082	0.99798948967338\\
164.015660913724	0.997990525769253\\
164.037462809366	0.997991561331185\\
164.059264705008	0.99799259635945\\
164.08106660065	0.997993630854324\\
164.102868496292	0.997994664816081\\
164.124670391933	0.997995698244997\\
164.146472287575	0.997996731141346\\
164.168274183217	0.997997763505403\\
164.190076078859	0.997998795337441\\
164.211877974501	0.997999826637735\\
164.233679870143	0.99800085740656\\
164.255481765785	0.998001887644188\\
164.277283661426	0.998002917350894\\
164.299085557068	0.99800394652695\\
164.32088745271	0.998004975172632\\
164.342689348352	0.998006003288211\\
164.364491243994	0.998007030873962\\
164.3862931396	1\\
};
\addlegendentry{compensated};

\addplot [color=mycolor2,solid]
  table[row sep=crcr]{%
0	0\\
0.460517018597787	-0.0197120460601753\\
0.921034037195573	-0.0249219637793949\\
1.38155105579336	-0.0174804236051585\\
1.84206807439115	0.000950606030334783\\
2.30258509298893	0.0288800909480801\\
2.76310211158672	0.0649717792206224\\
3.22361913018451	0.108029520403529\\
3.68413614878229	0.156983924295601\\
4.14465316738008	0.210880239213343\\
4.60517018597787	0.268867340421988\\
5.06568720457565	0.330187729080071\\
5.52620422317344	0.394168450910358\\
5.98672124177123	0.460212851882473\\
6.44723826036901	0.527793095551121\\
6.9077552789668	0.596443373401153\\
7.36827229756459	0.665753745664516\\
7.82878931616237	0.73536455564665\\
8.28930633476016	0.804961365678897\\
8.74982335335795	0.874270367442648\\
9.21034037195573	0.94305422362988\\
9.67085739055352	1.01110830174968\\
10.1313744091513	1.07825726439438\\
10.5918914277491	1.14435198347221\\
11.0524084463469	1.20926674882302\\
11.5129254649447	1.27289674428538\\
11.9734424835425	1.33515576669924\\
12.4339595021402	1.3959741655298\\
12.894476520738	1.45529698280368\\
13.3549935393358	1.5130822748762\\
13.8155105579336	1.56929959921286\\
14.2760275765314	1.62392865088481\\
14.7365445951292	1.67695803485948\\
15.197061613727	1.72838416142574\\
15.6575786323247	1.77821025323909\\
16.1180956509225	1.82644545351611\\
16.5786126695203	1.87310402585783\\
17.0391296881181	1.9182046370475\\
17.4996467067159	1.96176971495628\\
17.9601637253137	2.00382487440781\\
18.4206807439115	2.04439840450647\\
18.8811977625093	2.08352081152825\\
19.341714781107	2.12122441201499\\
19.8022317997048	2.15754297120537\\
20.2627488183026	2.1925113823844\\
20.7232658369004	2.22616538314115\\
21.1837828554982	2.25854130489612\\
21.644299874096	2.28967585239691\\
22.1048168926938	2.3196059101886\\
22.5653339112915	2.3483683733445\\
23.0258509298893	2.375999999997\\
23.4863679484871	2.40253728343972\\
23.9468849670849	2.42801634178192\\
24.4074019856827	2.45247282332722\\
24.8679190042805	2.47594182602257\\
25.3284360228783	2.49845782948057\\
25.7889530414761	2.52005463822225\\
26.2494700600738	2.54076533491694\\
26.7099870786716	2.56062224251462\\
27.1705040972694	2.5796568942732\\
27.6310211158672	2.59790001078062\\
28.091538134465	2.61538148316033\\
28.5520551530628	2.6321303617286\\
29.0125721716606	2.64817484944522\\
29.4730891902583	2.66354229956503\\
29.9336062088561	2.67825921695736\\
30.3941232274539	2.69235126261501\\
30.8546402460517	2.7058432609233\\
31.3151572646495	2.71875920930426\\
31.7756742832473	2.73112228989123\\
32.2361913018451	2.74295488292576\\
32.6967083204429	2.75427858160137\\
33.1572253390406	2.76511420810885\\
33.6177423576384	2.77548183066431\\
34.0782593762362	2.78540078132602\\
34.538776394834	2.79488967442735\\
34.9992934134318	2.80396642547366\\
35.4598104320296	2.81264827036817\\
35.9203274506274	2.82095178484839\\
36.3808444692251	2.82889290402886\\
36.8413614878229	2.83648694195918\\
37.3018785064207	2.84374861111786\\
37.7623955250185	2.850692041773\\
38.2229125436163	2.8573308011505\\
38.6834295622141	2.86367791235855\\
39.1439465808119	2.86974587302493\\
39.6044635994097	2.87554667361052\\
40.0649806180074	2.88109181536804\\
40.5254976366052	2.88639232792088\\
40.986014655203	2.89145878644135\\
41.4465316738008	2.89630132841197\\
41.9070486923986	2.9009296699573\\
42.3675657109964	2.905353121737\\
42.8280827295942	2.90958060439383\\
43.2885997481919	2.91362066355281\\
43.7491167667897	2.91748148437008\\
44.2096337853875	2.92117090563196\\
44.6701508039853	2.92469643340635\\
45.1306678225831	2.9280652542503\\
45.5911848411809	2.93128424797857\\
46.0517018597787	2.93435999999934\\
46.5122188783765	2.93729881322412\\
46.9727358969742	2.94010671955949\\
47.433252915572	2.94278949098932\\
47.8937699341698	2.94535265025618\\
48.3542869527676	2.94780148115166\\
48.8148039713654	2.95014103842498\\
49.2753209899632	2.95237615732023\\
49.735838008561	2.95451146275209\\
50.1963550271587	2.95655137813063\\
50.6568720457565	2.95850013384525\\
51.1173890643543	2.96036177541837\\
51.5779060829521	2.96214017133906\\
52.0384231015499	2.96383902058692\\
52.4989401201477	2.96546185985644\\
52.9594571387455	2.96701207049175\\
53.4199741573433	2.96849288514184\\
53.880491175941	2.96990739414578\\
54.3410081945388	2.97125855165766\\
54.8015252131366	2.97254918152047\\
55.2620422317344	2.97378198289808\\
55.7225592503322	2.97495953567429\\
56.18307626893	2.97608430562747\\
56.6435932875278	2.9771586493895\\
57.1041103061255	2.97818481919694\\
57.5646273247233	2.97916496744263\\
58.0251443433211	2.98010115103531\\
58.4856613619189	2.98099533557488\\
58.9461783805167	2.98184939935044\\
59.4066953991145	2.98266513716822\\
59.8672124177123	2.98344426401622\\
60.3277294363101	2.98418841857201\\
60.7882464549078	2.9848991665601\\
61.2487634735056	2.98557800396507\\
61.7092804921034	2.98622636010614\\
62.1697975107012	2.98684560057917\\
62.630314529299	2.98743703007125\\
63.0908315478968	2.98800189505339\\
63.5513485664946	2.98854138635629\\
64.0118655850924	2.98905664163409\\
64.4723826036901	2.98954874772077\\
64.9328996222879	2.99001874288373\\
65.3934166408857	2.99046761897902\\
65.8539336594835	2.99089632351208\\
66.3144506780813	2.99130576160839\\
66.7749676966791	2.99169679789753\\
67.2354847152769	2.99207025831459\\
67.6960017338746	2.99242693182227\\
68.1565187524724	2.99276757205732\\
68.6170357710702	2.99309289890427\\
69.077552789668	2.99340359999991\\
69.5380698082658	2.99370033217119\\
69.9985868268636	2.99398372280974\\
70.4591038454614	2.99425437118548\\
70.9196208640591	2.99451284970214\\
71.3801378826569	2.99475970509718\\
71.8406549012547	2.99499545958851\\
72.3011719198525	2.99522061197029\\
72.7616889384503	2.9954356386602\\
73.2222059570481	2.99564099470008\\
73.6827229756459	2.99583711471218\\
74.1432399942437	2.99602441381285\\
74.6037570128414	2.99620328848556\\
75.0642740314392	2.99637411741511\\
75.524791050037	2.99653726228463\\
75.9853080686348	2.99669306853713\\
76.4458250872326	2.99684186610298\\
76.9063421058304	2.9969839700951\\
77.3668591244282	2.99711968147303\\
77.8273761430259	2.99724928767735\\
78.2878931616	3\\
100 3\\
};
\addlegendentry{uncompensated};

\end{axis}
\end{tikzpicture}%
}
  \caption{The step response of the initial, uncontrolled process
    (\texttt{red}) and final, controlled process (\texttt{blue}).}
  \label{fig:step_response_1.3}
\end{figure}

The bandwidth and the resonance peak of the closed-loop system, along with the
rise time and the overshoot of a step response are found in table \ref{tbl:ex1.3}.

\begin{table}[H] \centering
    \begin{tabular}{|c|c|c|c|} \hline
      $\omega_B$ $[rad/s]$ & $M_T$ $[dB]$ & $T_r$ $[sec]$ & $M$ $\%$  \\ \hline
      $0.86$               & $1.4757$     & $2.7$       & $9.5649$ \\ \hline
    \end{tabular}
    \caption{Closed loop system characteristics for a phase margin of
      $50^{\circ}$.}
    \label{tbl:ex1.3}
\end{table}
