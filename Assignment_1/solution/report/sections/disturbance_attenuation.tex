This exercise deals with the construction of a controller whose function is
twofold: it tracks the reference signal and attenuates disturbances. The
requirements of the system are such that:

\begin{itemize}
  \item The rise time for a step change in the reference signal is less than
  $0.2$ s
  \item The overshoot is less than $10\%$
  \item For a step in the disturbance, $|y(t)| \leq 1\ \forall t$ and
  $|y(t)| \leq 0.1$ for $t > 0.5$ s
  \item Since the signals are scaled, the control signal obeys $|u(t)| \leq 1\
  \forall t$
\end{itemize}

The transfer functions of the plant $G(s)$ and the disturbance $G_d(s)$ have
been estimated to expressions \ref{eq:421_G} and \ref{eq:421_Gd} respectively.
The target of this exercise is to construct the $F_r(s), F_y(s)$ transfer
functions in such a way that all four of the above requirements are met.

\begin{equation}
  G(s) = \dfrac{20}{(s+1)((\dfrac{s}{20})^2 + \dfrac{s}{20} + 1)}
  \label{eq:421_G}
\end{equation}

\begin{equation}
  G_d(s) = \dfrac{10}{s+1}
  \label{eq:421_Gd}
\end{equation}

\begin{figure}[H]\centering
  \scalebox{0.8}{% Generated with LaTeXDraw 2.0.8
% Tue Apr 05 18:18:31 CEST 2016
% \usepackage[usenames,dvipsnames]{pstricks}
% \usepackage{epsfig}
% \usepackage{pst-grad} % For gradients
% \usepackage{pst-plot} % For axes
\scalebox{1} % Change this value to rescale the drawing.
{
\begin{pspicture}(0,-2.78)(16.41,2.78)
\psframe[linewidth=0.04,dimen=outer](8.79,0.3)(6.67,-0.96)
\psline[linewidth=0.04cm,arrowsize=0.05291667cm 2.0,arrowlength=1.4,arrowinset=0.4]{->}(8.79,-0.36)(10.13,-0.36)
\psline[linewidth=0.04cm,arrowsize=0.05291667cm 2.0,arrowlength=1.4,arrowinset=0.4]{->}(5.25,-0.34)(6.71,-0.34)
\usefont{T1}{ppl}{m}{n}
\rput(7.744531,-0.33){$F_y(s)$}
\usefont{T1}{ppl}{m}{n}
\rput(11.164532,-0.37){$G(s)$}
\usefont{T1}{ppl}{m}{n}
\rput(9.334531,-0.03){$u$}
\pscircle[linewidth=0.04,dimen=outer](13.33,-0.34){0.5}
\psline[linewidth=0.04cm,arrowsize=0.05291667cm 2.0,arrowlength=1.4,arrowinset=0.4]{->}(12.21,-0.32)(12.85,-0.32)
\psline[linewidth=0.04cm,arrowsize=0.05291667cm 2.0,arrowlength=1.4,arrowinset=0.4]{->}(13.31,0.86)(13.31,0.13)
\usefont{T1}{ppl}{m}{n}
\rput(13.344531,-0.33){$\Sigma$}
\usefont{T1}{ppl}{m}{n}
\rput(13.854531,2.57){$d$}
\psline[linewidth=0.04cm,arrowsize=0.05291667cm 2.0,arrowlength=1.4,arrowinset=0.4]{->}(13.79,-0.34)(16.39,-0.36)
\pscircle[linewidth=0.04,dimen=outer](4.75,-0.28){0.5}
\usefont{T1}{ppl}{m}{n}
\rput(4.764531,-0.27){$\Sigma$}
\psline[linewidth=0.04cm,arrowsize=0.05291667cm 2.0,arrowlength=1.4,arrowinset=0.4]{->}(3.19,-0.32)(4.29,-0.32)
\usefont{T1}{ppl}{m}{n}
\rput(15.774531,-0.03){$y$}
\usefont{T1}{ppl}{m}{n}
\rput(4.454531,-1.05){$-$}
\psline[linewidth=0.04cm,arrowsize=0.05291667cm 2.0,arrowlength=1.4,arrowinset=0.4]{<-}(4.71,-0.74)(4.73,-2.76)
\psframe[linewidth=0.04,dimen=outer](12.19,0.3)(10.07,-0.96)
\usefont{T1}{ppl}{m}{n}
\rput(0.24453124,0.03){$r$}
\psframe[linewidth=0.04,dimen=outer](14.37,2.1)(12.25,0.84)
\psline[linewidth=0.04cm,arrowsize=0.05291667cm 2.0,arrowlength=1.4,arrowinset=0.4]{->}(13.31,2.76)(13.31,2.03)
\usefont{T1}{ptm}{m}{n}
\rput(13.324532,1.41){$G_d(s)$}
\psline[linewidth=0.04cm](14.95,-0.34)(14.97,-2.74)
\psline[linewidth=0.04cm](14.93,-2.74)(4.71,-2.74)
\psframe[linewidth=0.04,dimen=outer](3.23,0.3)(1.11,-0.96)
\psline[linewidth=0.04cm,arrowsize=0.05291667cm 2.0,arrowlength=1.4,arrowinset=0.4]{->}(0.03,-0.32)(1.13,-0.32)
\usefont{T1}{ptm}{m}{n}
\rput(2.1745312,-0.31){$F_r(s)$}
\end{pspicture} 
}

}
  \caption{$F_r-$ prefilter, $F_y-$ feedback controller, $G-$ system, $G_d-$
    disturbance dynamics, $r-$ reference signal, $u-$ control signal,
    $d-$ disturbance signal, $y-$ measurement signal.}
  \label{fig:block_2}
\end{figure}

%%%%%%%%%%%%%%%%%%%%%%%%%%%%%%%%%%%%%%%%%%%%%%%%%%%%%%%%%%%%%%%%%%%%%%%%%%%%%%%%
\subsection{Exercice 1}

Control action is needed at least at frequencies where $|G_d(j\omega)| > 1$;
this expression is valid for all $\omega < \omega_c = 9.9473$ rad/s\footnote{
this is the crossover frequency of $G_d$}. Hence, the minimal frequency interval
where control is needed is $[0, 9.9473]$ rad/s. The controller $F_y(s)$ that
shall be designed here will be such that
$$L(s) = F_y(s)G(s) = \dfrac{\omega_c}{s}$$

We will approach the design of $F_y$ in two ways: one where it $F_y$ is
improper, and one where it is proper.

\subsubsection{$F_y$ is improper}

The first approach is the most simplistic one:
$$F_y(s) = \dfrac{\omega_c}{sG(s)} = \dfrac{\omega_c (s+1)((\dfrac{s}{20})^2 + \dfrac{s}{20} + 1)}{20s}$$
This means that $F_y$ is not proper since the number of its poles is less than
the number of its zeros. Although this controller displays good disturbance
attenuation (figure \ref{fig:bode_2.1_dy}), the amplitude of the output for a
step disturbance does not meet the third requirement for $|y(t)| \leq 0.1$ for
$t > 0.5$ s.

\subsubsection{$F_y$ is proper}

In order for $F_y$ to be proper, we add the minimum number of poles required,
which is two, $p_1, p_2$, such that
$$F_y(s) = \dfrac{\omega_c}{sG(s)} \cdot \dfrac{p_1 \cdot p_2}{(s+p_1)(s+p_2)}$$
and in order for $L(s) \approx \dfrac{\omega_c}{s}$ we choose their location
so that the frequency response of the closed-loop transfer function from $d$ to
$y$ matches that of the previous case, where $F_y$ was improper. In this case,
the two poles should be positioned away from $\omega_c$. Such a case was
found to be valid experimentally for $p_1 = p_2 \geq 50\omega_c$. Thus, the
poles were chosen as $p_1 = p_2 = 50\omega_c$. As is evident in figures
\ref{fig:bode_2.1_dy} and \ref{fig:step_response_2.1}, the frequency response of
the closed loop transfer function from $d$ to $y$ and the response of the output
for a unit disturbance are equivalent in the frequency band where control action
is appropriate.


\begin{figure}[H]\centering
  \scalebox{1}{% This file was created by matlab2tikz.
%
%The latest updates can be retrieved from
%  http://www.mathworks.com/matlabcentral/fileexchange/22022-matlab2tikz-matlab2tikz
%where you can also make suggestions and rate matlab2tikz.
%
\definecolor{mycolor1}{rgb}{0.00000,0.44700,0.74100}%
\definecolor{mycolor2}{rgb}{0.85000,0.32500,0.09800}%
%
\begin{tikzpicture}

\begin{axis}[%
width=4.008in,
height=1.551in,
at={(0.818in,1.941in)},
scale only axis,
separate axis lines,
every outer x axis line/.append style={white!40!black},
every x tick label/.append style={font=\color{white!40!black}},
xmode=log,
xmin=0.1,
xmax=100000,
xlabel={Frequency [rad/s]},
xtick={1,100,10000},
xticklabels={\empty},
xminorticks=true,
every outer y axis line/.append style={white!40!black},
every y tick label/.append style={font=\color{white!40!black}},
ymin=-200,
ymax=50,
ylabel={Magnitude [dB]},
axis background/.style={fill=white}
]
\addplot [color=mycolor1,solid,forget plot]
  table[row sep=crcr]{%
1e-20	419.954112362353\\
2e-17	353.933512449074\\
2e-12	253.933512449074\\
2e-08	173.933512449074\\
2e-05	113.933512449074\\
0.002	73.9335124490736\\
0.02	53.9335124490736\\
0.023343078933483	52.5909495906919\\
0.0272449667047409	51.2483867323102\\
0.0317990704164442	49.9058238739285\\
0.0371144105371221	48.5632610155468\\
0.0433182307368868	47.220698157165\\
0.050559043967499	45.8781352987833\\
0.0590101877067383	44.5355724404016\\
0.0688739734759021	43.1930095820199\\
0.08038652996553	41.8504467236382\\
0.0938234557087083	40.5078838652564\\
0.109506416621026	39.1653210068747\\
0.127810846345375	37.822758148493\\
0.149174933739768	36.4801952901113\\
0.174110112659225	35.1376324317296\\
0.203213305146095	33.7950695733478\\
0.237181211117964	32.4525067149661\\
0.276826986633286	31.1099438565844\\
0.323099709994953	29.7673809982027\\
0.377107101689883	28.424818139821\\
0.440142042056197	27.0822552814392\\
0.513713521483111	25.7396924230575\\
0.599582764058889	24.3971295646758\\
0.699805389429129	23.0545667062941\\
0.816780622176049	21.7120038479124\\
0.953308726739745	20.3694409895306\\
1.1126580428132	19.0268781311489\\
1.29864322596817	17.6843152727672\\
1.5157165665104	16.3417524143855\\
1.63241863743818	15.6974814723507\\
1.769074572642	14.9991895560038\\
2.06478236942	13.656626697622\\
2.40991889149176	12.3140638392403\\
2.8127463453692	10.9715009808586\\
2.83889695781488	10.8911197740026\\
3.28290799799096	9.62893812247688\\
3.83165902642329	8.28637526409517\\
4.47213595499958	6.94381240571344\\
5.21967112994113	5.60124954733172\\
6.09215975965192	4.25868668895001\\
6.69649237299735	3.43716479175827\\
7.73010902476924	2.19039997810897\\
8.79534869632983	1.06905112289077\\
9.87826120843701	0.0605022439577563\\
10.9656731306893	-0.846593559871402\\
12.0455797306772	-1.66244176962298\\
13.1074038731644	-2.39622126517912\\
14.142135623731	-3.05618759428656\\
15.2585517265911	-3.71615392339401\\
15.3191476249115	-3.75057966407984\\
16.6036010280725	-4.44993341895015\\
18.2387344229937	-5.26578162870173\\
20.246478178688	-6.17287743253089\\
22.7392917444487	-7.18142631146391\\
25.8728563024337	-8.3027751666821\\
29.8663821087099	-9.5495399803314\\
33.3565292381147	-10.5095047532489\\
38.9322047476173	-11.8520676116306\\
45.4398764239077	-13.1946304700124\\
53.0353311045496	-14.5371933283941\\
61.9003960118454	-15.8797561867758\\
72.2472915059182	-17.2223190451575\\
84.3237114176504	-18.5648819035392\\
98.4187525793227	-19.9074447619209\\
114.869835499703	-21.2500076203027\\
134.07078185729	-22.5925704786844\\
156.481242178425	-23.9351333370661\\
182.637699389021	-25.2776961954478\\
213.166311653384	-26.6202590538295\\
248.79790194422	-27.9628219122113\\
290.385453178444	-29.305384770593\\
338.924527733983	-30.6479476289747\\
395.577100169391	-31.9905104873564\\
461.699373676621	-33.3330733457382\\
538.874246163652	-34.6756362041199\\
628.949203170964	-36.0181990625016\\
734.080544738553	-37.3607619208833\\
856.785004968315	-38.703324779265\\
1000	-40.0458876376467\\
10000	-60.0458876376467\\
1000000	-100.045887637647\\
1000000000	-160.045887637647\\
10000000000000	-240.045887637647\\
1e+18	-340.045887637647\\
1e+20	-380.045887637647\\
};
\addplot [color=mycolor2,solid,forget plot]
  table[row sep=crcr]{%
1e-20	419.954112362353\\
2e-17	353.933512449074\\
2e-12	253.933512449074\\
2e-08	173.933512449074\\
2e-05	113.933512449074\\
0.002	73.9335124489332\\
0.02	53.9335124350286\\
0.0233347844644754	52.5940364670696\\
0.027225608300176	51.2545604972775\\
0.031765185079942	49.9150845249898\\
0.0370616873657307	48.575608549305\\
0.0432413243284549	47.2361325689958\\
0.0504513491581485	45.8966565823915\\
0.0588635679273693	44.5571805872179\\
0.0686784335197584	43.2177045803789\\
0.0801298221770681	41.8782285576601\\
0.0934906064839312	40.5387525133244\\
0.109079157587781	39.1992764395621\\
0.12726693159387	37.859800325742\\
0.148487320909905	36.5203241573919\\
0.173245981457001	35.1808479148112\\
0.202132881831781	33.8413715711818\\
0.235836361536394	32.5018950899969\\
0.275159533266892	31.1624184215603\\
0.321039420106428	29.8229414982218\\
0.374569283639183	28.4834642278903\\
0.437024675036664	27.1439864852043\\
0.509893829881897	25.8045080995115\\
0.594913131002997	24.4650288385067\\
0.694108484352057	23.1255483859577\\
0.809843593865946	21.7860663113848\\
0.944876285639899	20.4465820287855\\
1.10242422355006	19.1070947404491\\
1.28624158224786	17.7676033604726\\
1.50070850454998	16.4281064106467\\
1.75093547488394	15.0886018787316\\
2.04288509588103	13.7490870255364\\
2.38351416990365	12.4095581223135\\
2.78093947113623	11.0700100933018\\
3.2446311583858	9.73043602917091\\
3.78563843738269	8.39082652476181\\
4.41685284983792	7.05116877771508\\
5.15331546311358	5.71144536172524\\
6.01257528046018	4.37163255710469\\
6.69649237299735	3.43559037951551\\
7.73010902476924	2.18830209226585\\
8.79534869632984	1.06633530097453\\
9.87826120843702	0.0570766299114215\\
10.9656731306893	-0.850814683985097\\
12.0455797306772	-1.66753497441374\\
13.1074038731644	-2.40225165873822\\
14.142135623731	-3.06320727815003\\
15.2585517265911	-3.72432511500849\\
16.6036010280725	-4.45960785719649\\
18.2387344229938	-5.27745404096433\\
20.246478178688	-6.18725887525121\\
22.7392917444487	-7.19956322688164\\
25.8728563024337	-8.3262479647623\\
29.8663821087099	-9.58080411828918\\
32.7923323262856	-10.3990098253629\\
38.2601003460662	-11.7520573228718\\
44.6395597582327	-13.1099739352565\\
52.0827252773714	-14.4744900363188\\
60.7669584334972	-15.8479370445757\\
70.8991938803697	-17.2334446221991\\
82.7208703951738	-18.635193010782\\
96.5136840692591	-20.0587279948456\\
112.606300781432	-21.5113399199859\\
131.38218790383	-23.0024948802307\\
153.288751860353	-24.5442816128389\\
178.847999274489	-26.1517980650124\\
208.668975748643	-27.843346981679\\
243.462278675871	-29.6402504687055\\
284.056989906575	-31.5660584775764\\
331.420431754879	-33.6449667763826\\
386.681217106174	-35.8994316898377\\
451.156142881678	-38.3472797655156\\
526.381567698401	-40.998957271506\\
614.150021415742	-43.8557332762273\\
716.552918929463	-46.9094720993696\\
836.030396030488	-50.144057348615\\
975.429454856069	-53.5379684931018\\
1138.07180446835	-57.0672090059067\\
1327.83301311828	-60.7078693125019\\
1549.2348582965	-64.4379307555036\\
1807.55307516004	-68.2382527146824\\
2108.94307084796	-72.0928963647152\\
2460.5866003043	-75.989009220224\\
2870.86289871386	-79.9164738939992\\
3349.54834842734	-83.8674677659023\\
3908.04943819457	-87.8360227570913\\
4559.67456583922	-91.8176305766254\\
5319.95116110043	-95.8089105583931\\
6206.9956852907	-99.8073413266582\\
7241.94532440934	-103.811049892841\\
8449.46166243037	-107.818649015979\\
9858.31833668303	-111.82911345621\\
11502.0866784342	-115.841686720521\\
13419.9356766488	-119.855811271847\\
15657.5653270861	-123.871076559202\\
18268.2956072999	-127.887180459143\\
21314.3370264833	-131.90390074362\\
24868.2730258086	-135.921074008051\\
248682.730258086	-195.917635090371\\
24868273.0258086	-315.917600350356\\
24868273025.8087	-495.917600346882\\
248682730258087	-735.917600346881\\
1e+20	-1072.1788627395\\
};
\end{axis}

\begin{axis}[%
width=4.008in,
height=1.376in,
at={(0.818in,0.44in)},
scale only axis,
separate axis lines,
every outer x axis line/.append style={white!40!black},
every x tick label/.append style={font=\color{white!40!black}},
xmode=log,
xmin=0.1,
xmax=100000,
xminorticks=true,
every outer y axis line/.append style={white!40!black},
every y tick label/.append style={font=\color{white!40!black}},
ymin=-272.7,
ymax=2.7,
ytick={-270, -180,  -90,    0},
ylabel={Phase (deg)},
axis background/.style={fill=white},
legend style={at={(0.079,0.148)},anchor=south west,legend cell align=left,align=left,draw=white!15!black,font=\footnotesize}
]
\addplot [color=mycolor1,solid]
  table[row sep=crcr]{%
1e-20	-90\\
2e-17	-90\\
2e-12	-90\\
2e-08	-90\\
2e-05	-90\\
0.002	-90\\
0.02	-90\\
0.023343078933483	-90\\
0.0272449667047409	-90\\
0.0317990704164442	-90\\
0.0371144105371221	-90\\
0.0433182307368868	-90\\
0.050559043967499	-90\\
0.0590101877067383	-90\\
0.0688739734759021	-90\\
0.08038652996553	-90\\
0.0938234557087083	-90\\
0.109506416621026	-90\\
0.127810846345375	-90\\
0.149174933739768	-90\\
0.174110112659225	-90\\
0.203213305146095	-90\\
0.237181211117964	-90\\
0.276826986633286	-90\\
0.323099709994953	-90\\
0.377107101689883	-90\\
0.440142042056197	-90\\
0.513713521483111	-90\\
0.599582764058889	-90\\
0.699805389429129	-90\\
0.816780622176049	-90\\
0.953308726739745	-90\\
1.1126580428132	-90\\
1.29864322596817	-90\\
1.5157165665104	-90\\
1.63241863743818	-90\\
1.769074572642	-90\\
2.06478236942	-90\\
2.40991889149176	-90\\
2.8127463453692	-90\\
2.83889695781488	-90\\
3.28290799799096	-90\\
3.83165902642329	-90\\
4.47213595499958	-90\\
5.21967112994113	-90\\
6.09215975965192	-90\\
6.69649237299735	-90\\
7.73010902476924	-90\\
8.79534869632983	-90\\
9.87826120843701	-90\\
10.9656731306893	-90\\
12.0455797306772	-90\\
13.1074038731644	-90\\
14.142135623731	-90\\
15.2585517265911	-90\\
15.3191476249115	-90\\
16.6036010280725	-90\\
18.2387344229937	-90\\
20.246478178688	-90\\
22.7392917444487	-90\\
25.8728563024337	-90\\
29.8663821087099	-90\\
33.3565292381147	-90\\
38.9322047476173	-90\\
45.4398764239077	-90\\
53.0353311045496	-90\\
61.9003960118454	-90\\
72.2472915059182	-90\\
84.3237114176504	-90\\
98.4187525793227	-90\\
114.869835499703	-90\\
134.07078185729	-90\\
156.481242178425	-90\\
182.637699389021	-90\\
213.166311653384	-90\\
248.79790194422	-90\\
290.385453178444	-90\\
338.924527733983	-90\\
395.577100169391	-90\\
461.699373676621	-90\\
538.874246163652	-90\\
628.949203170964	-90\\
734.080544738553	-90\\
856.785004968315	-90\\
1000	-90\\
10000	-90\\
1000000	-90\\
1000000000	-90\\
10000000000000	-90\\
1e+18	-90\\
1e+20	-90\\
};
\addlegendentry{$F_y$ improper};

\addplot [color=mycolor2,solid]
  table[row sep=crcr]{%
1e-20	-90\\
2e-17	-90\\
2e-12	-90.0000000000005\\
2e-08	-90.000000004608\\
2e-05	-90.000004607942\\
0.002	-90.0004607941971\\
0.02	-90.0046079419687\\
0.0233347844644754	-90.0053762666321\\
0.027225608300176	-90.0062727011525\\
0.031765185079942	-90.0073186064677\\
0.0370616873657307	-90.0085389052209\\
0.0432413243284549	-90.009962675638\\
0.0504513491581485	-90.0116238444245\\
0.0588635679273693	-90.0135619951979\\
0.0686784335197584	-90.0158233117159\\
0.0801298221770681	-90.0184616783779\\
0.0934906064839312	-90.0215399642227\\
0.109079157587781	-90.0251315210185\\
0.12726693159387	-90.0293219311417\\
0.148487320909905	-90.034211046894\\
0.173245981457001	-90.0399153698503\\
0.202132881831781	-90.0465708269333\\
0.235836361536394	-90.0543360093603\\
0.275159533266892	-90.0633959516376\\
0.321039420106428	-90.0739665406428\\
0.374569283639183	-90.0862996598441\\
0.437024675036664	-90.1006891912137\\
0.509893829881897	-90.1174780178207\\
0.594913131002997	-90.1370661939095\\
0.694108484352057	-90.1599204770583\\
0.809843593865946	-90.186585449418\\
0.944876285639899	-90.2176964928124\\
1.10242422355006	-90.2539949265285\\
1.28624158224786	-90.2963456679468\\
1.50070850454998	-90.3457578359509\\
1.75093547488394	-90.403408786652\\
2.04288509588103	-90.4706721519136\\
2.38351416990365	-90.5491505452252\\
2.78093947113623	-90.6407137085767\\
3.2446311583858	-90.7475430003037\\
3.78563843738269	-90.8721832696549\\
4.41685284983792	-91.0176033314266\\
5.15331546311358	-91.187266445551\\
6.01257528046018	-91.3852124236881\\
6.69649237299735	-91.542759195338\\
7.73010902476924	-91.7808513075963\\
8.79534869632984	-92.0262116261343\\
9.87826120843702	-92.2756235338604\\
10.9656731306893	-92.5260500277006\\
12.0455797306772	-92.7747242014934\\
13.1074038731644	-93.0192089886539\\
14.142135623731	-93.2574293296882\\
15.2585517265911	-93.514423747317\\
16.6036010280725	-93.8240013953474\\
18.2387344229938	-94.2002694110846\\
20.246478178688	-94.6621557517745\\
22.7392917444487	-95.2354210557908\\
25.8728563024337	-95.9556627666365\\
29.8663821087099	-96.8728747386034\\
32.7923323262856	-97.5443390253804\\
38.2601003460662	-98.7976898587584\\
44.6395597582327	-100.25734154348\\
52.0827252773714	-101.956133417127\\
60.7669584334972	-103.931484498928\\
70.8991938803697	-106.22565414654\\
82.7208703951738	-108.885777163666\\
96.5136840692591	-111.9635066246\\
112.606300781432	-115.514023983226\\
131.38218790383	-119.59409140569\\
153.288751860353	-124.258746907749\\
178.847999274489	-129.556227168972\\
208.668975748643	-135.520832750602\\
243.462278675871	-142.163846833457\\
284.056989906575	-149.463386140199\\
331.420431754879	-157.355162098268\\
386.681217106174	-165.727198021939\\
451.156142881678	-174.42183020525\\
526.381567698401	-183.247004328754\\
614.150021415742	-191.99587803075\\
716.552918929463	-200.470314106851\\
836.030396030488	-208.502060928747\\
975.429454856069	-215.966492180024\\
1138.07180446835	-222.786970112603\\
1327.83301311828	-228.931187103467\\
1549.2348582965	-234.402644405239\\
1807.55307516004	-239.230502341367\\
2108.94307084796	-243.46009034263\\
2460.5866003043	-247.145221875485\\
2870.86289871386	-250.342590035565\\
3349.54834842734	-253.108032206005\\
3908.04943819457	-255.494264324479\\
4559.67456583922	-257.54967265845\\
5319.95116110043	-259.317815958696\\
6206.9956852907	-260.837375711454\\
7241.94532440934	-262.142369830256\\
8449.46166243037	-263.262506428902\\
9858.31833668303	-264.223598888492\\
11502.0866784342	-265.047994133613\\
13419.9356766488	-265.754986359133\\
15657.5653270861	-266.361201471761\\
18268.2956072999	-266.880945596639\\
21314.3370264833	-267.326515846452\\
24868.2730258086	-267.70847432365\\
248682.730258086	-269.770817187524\\
24868273.0258086	-269.99770816882\\
24868273025.8087	-269.999997708169\\
248682730258087	-269.999999999771\\
1e+20	-270\\
};
\addlegendentry{$F_y$ proper};

\end{axis}
\end{tikzpicture}%
}
  \caption{The frequency response of $L(s) = F_y(s)G(s)$ when $F_y$ is proper
    (\texttt{red}) and when $F_y$ is improper (\texttt{blue}).
    $p_1 = p_2 = 50\omega_c$}
  \label{fig:bode_2.1_L}
\end{figure}

\begin{figure}[H]\centering
  \scalebox{1}{% This file was created by matlab2tikz.
%
%The latest updates can be retrieved from
%  http://www.mathworks.com/matlabcentral/fileexchange/22022-matlab2tikz-matlab2tikz
%where you can also make suggestions and rate matlab2tikz.
%
\definecolor{mycolor1}{rgb}{0.00000,0.44700,0.74100}%
\definecolor{mycolor2}{rgb}{0.85000,0.32500,0.09800}%
%
\begin{tikzpicture}

\begin{axis}[%
width=4.008in,
height=1.551in,
at={(0.818in,1.941in)},
scale only axis,
separate axis lines,
every outer x axis line/.append style={white!40!black},
every x tick label/.append style={font=\color{white!40!black}},
xmode=log,
xmin=0.01,
xmax=1000,
xtick={0.01,0.1,1,10,100,1000},
xticklabels={\empty},
xminorticks=true,
xlabel={Frequency [rad/s]},
every outer y axis line/.append style={white!40!black},
every y tick label/.append style={font=\color{white!40!black}},
ymin=-50,
ymax=0,
ylabel={Magnitude [dB]},
axis background/.style={fill=white}
]
\addplot [color=mycolor1,solid,forget plot]
  table[row sep=crcr]{%
1e-20	-399.954112362353\\
1.99999992937163e-17	-333.933512755809\\
1.99999992937163e-12	-233.933512755809\\
1.99999992937163e-08	-153.933512755809\\
1.99999992937163e-05	-93.9335127575635\\
0.00199999992937163	-53.9335303031151\\
0.0199999992937163	-33.935267142537\\
0.0233430781209175	-32.5933396320881\\
0.0272449657700967	-31.2516421303951\\
0.0317990693416128	-29.910257832441\\
0.0371144093013522	-28.5692999633943\\
0.0433182293164065	-27.22892256564\\
0.0505590423350863	-25.8893351156124\\
0.0590101858312314	-24.5508222918078\\
0.0688739713216429	-23.2137706384801\\
0.0803865274917319	-21.8787043997614\\
0.0938234528687379	-20.5463334376465\\
0.109506413361588	-19.2176168672241\\
0.127810842605588	-17.8938467571691\\
0.149174929450118	-16.57675675354\\
0.17411010774038	-15.2686603926127\\
0.203213299507564	-13.9726224542815\\
0.237181204656585	-12.6926627859202\\
0.276826979231519	-11.4339838598082\\
0.323099701518951	-10.2031988099671\\
0.37710709198733	-9.00851424007883\\
0.440142030953871	-7.85979284544311\\
0.513713508784149	-6.76839258907365\\
0.599582749539728	-5.74667032244078\\
0.699805372836077	-4.80707693183681\\
0.816780603221459	-3.9608827576378\\
0.953308705097753	-3.21674548180985\\
1.11265801811499	-2.57949439908248\\
1.2986431977967	-2.04953903555543\\
1.51571653439462	-1.62314617242709\\
1.76907453605043	-1.29352965918118\\
2.06478232775366	-1.05243125888923\\
2.4099188440765	-0.891778195148951\\
2.81274629144729	-0.805089331069515\\
3.28290793671198	-0.788464433585065\\
3.83165895683431	-0.841121533632388\\
4.47213587603466	-0.965499483877066\\
5.21967104041018	-1.16693239000807\\
6.09215965822894	-1.45287863320635\\
6.69649237299735	-1.67313004713936\\
7.73010902476924	-2.07794454756478\\
8.79534869632984	-2.51847947470582\\
9.87826120843702	-2.97854653477385\\
10.9656731306893	-3.44427321552306\\
12.0455797306772	-3.90452713698073\\
13.1074038731644	-4.35093791898884\\
14.142135623731	-4.77763109431727\\
15.2585517265911	-5.22701547938158\\
16.6036010280725	-5.751733547999\\
18.2387344229938	-6.36385428144053\\
20.246478178688	-7.07664029141319\\
22.7392917444487	-7.90436973253395\\
25.8728563024337	-8.86209453883802\\
29.8663821087099	-9.96538232486107\\
33.3565288678992	-10.837519124531\\
38.9322043351596	-12.0836899761943\\
45.4398759654299	-13.3541360851854\\
53.0353305961912	-14.643002931216\\
61.9003954497409	-15.9457305354149\\
72.2472908863037	-17.2588215787296\\
84.3237107370051	-18.5796243401303\\
98.4187518345561	-19.9061452600433\\
114.869834688397	-21.2368949741511\\
134.070780978007	-22.5707659467225\\
156.481241231109	-23.9069373929821\\
182.637698375497	-25.2448025665727\\
213.166310577985	-26.5839137744024\\
248.797900814579	-27.9239411319643\\
290.385452006476	-29.2646418057829\\
338.924526537099	-30.6058371735664\\
395.577098972007	-31.947395916174\\
461.699372512011	-33.2892215290048\\
538.874245076229	-34.631243112847\\
628.949202219072	-35.9734085911943\\
734.080543997884	-37.3156797191032\\
856.785004536076	-38.6580284127594\\
1000	-40.0004340515589\\
10000	-60.0000043407261\\
1000000	-100.000000000434\\
1000000000	-160\\
10000000000000	-240\\
1e+18	-340\\
1e+20	-380\\
};
\addplot [color=mycolor2,solid,forget plot]
  table[row sep=crcr]{%
1e-20	-399.954112362353\\
2e-17	-333.933512449074\\
2e-12	-233.933512449074\\
2e-08	-153.933512449074\\
2e-05	-93.933512450827\\
0.002	-53.9335299821957\\
0.02	-33.935265417382\\
0.0233273136756512	-32.5992023611062\\
0.0272081781661113	-31.2633667997849\\
0.0317346853311942	-29.9278406015789\\
0.0370142479559427	-28.5927350361496\\
0.0431721486268304	-27.2582012825595\\
0.0503545126534955	-25.9244446492634\\
0.0587317755826319	-24.5917437807848\\
0.0685027275872005	-23.2604765320685\\
0.0798992307032157	-21.9311547005921\\
0.0931917208528568	-20.6044704197571\\
0.108695625215416	-19.2813577098962\\
0.126778847228552	-17.9630733750048\\
0.147870496836895	-16.6513019384161\\
0.172471073154432	-15.3482892581222\\
0.201164341172481	-14.057008180243\\
0.23463118434431	-12.7813559669269\\
0.273665761764463	-11.5263756356181\\
0.319194353348284	-10.2984797580776\\
0.372297340202604	-9.10563408836851\\
0.434234841775839	-7.95743042992835\\
0.506476618150089	-6.86495021922574\\
0.590736947048508	-5.84030926229369\\
0.689015303179857	-4.89580682802544\\
0.803643805230023	-4.04270365914356\\
0.937342556404736	-3.28981759856171\\
1.0932841917395	-2.64228778686729\\
1.27516916386691	-2.10090858303172\\
1.48731355375207	-1.66229579807987\\
1.73475149012111	-1.31986551413235\\
2.02335460796792	-1.06533645356643\\
2.3599713808571	-0.890355399087474\\
2.75258963334067	-0.787913051649271\\
3.21052608986418	-0.753374125226605\\
3.74464745810619	-0.785081979184523\\
4.36762829299866	-0.884560208078287\\
5.09425176047145	-1.0563279262296\\
5.94176043796281	-1.30731322209433\\
6.69649237299737	-1.56233335385491\\
7.73010902476926	-1.94383514572366\\
8.79534869632986	-2.36181493150899\\
9.87826120843704	-2.80086508096023\\
10.9656731306893	-3.24756942661311\\
12.0455797306772	-3.6909734222495\\
13.1074038731645	-4.1226847093634\\
14.142135623731	-4.53668493918955\\
15.2585517265912	-4.97399444097705\\
16.6036010280725	-5.48612181231578\\
18.2387344229938	-6.08537793845757\\
20.2464781786881	-6.78534215098428\\
22.7392917444487	-7.60067394460485\\
25.8728563024337	-8.54684915960854\\
29.8663821087099	-9.63987222342523\\
32.2921682842295	-10.2497594252922\\
37.6644769416569	-11.4774678486834\\
43.9305534023683	-12.7329713408456\\
51.2390899580996	-14.0099316548056\\
59.763516195375	-15.3034328095214\\
69.7061144324689	-16.6097689073696\\
81.302819823852	-17.9262334150994\\
94.8288190372977	-19.2509297864602\\
110.60508035873	-20.5826073158296\\
129.005970182436	-21.9205153133932\\
150.46813662387	-23.2642615234702\\
175.500871060788	-24.6136572094304\\
204.698193479251	-25.9685341239154\\
238.752948406482	-27.3285326762373\\
278.473245923229	-28.692889961624\\
324.801637896396	-30.0602953343622\\
378.837484478726	-31.4289051499209\\
441.863041626496	-32.7965784046808\\
515.37388868493	-34.1612943925717\\
601.114418079677	-35.5215923202992\\
701.119229275059	-36.87683309096\\
817.761409266507	-38.2271787730093\\
825.831581509538	-38.3131960165612\\
953.80884529012	-39.5733454717096\\
1112.48990603467	-40.9162830726764\\
1297.57004995333	-42.2569178360957\\
1513.44117856958	-43.5960124253168\\
1765.225855107	-44.9341279230168\\
2058.89886152253	-46.2716452067465\\
2401.42897845886	-47.6088079982054\\
2448.81769286874	-47.7785845852021\\
2800.94435251543	-48.9457655707449\\
3266.92537495857	-50.2826064212838\\
3810.42964883015	-51.6193817607172\\
4444.35438286313	-52.956120695776\\
5183.74243874018	-54.2928395509423\\
6046.13929411387	-55.6295473918831\\
7052.00939202376	-56.9662492164653\\
8225.22175656882	-58.3029477650804\\
9593.6163983636	-59.6396445342519\\
11189.664950425	-60.9763403387458\\
13051.2412112002	-62.3130356211225\\
15222.5198795127	-63.6497306213306\\
17755.0248081614	-64.9864254692704\\
20708.8516509476	-66.3231202351729\\
24154.0939162091	-67.6598149569602\\
28172.5062667275	-68.9965096550814\\
281725.062667275	-88.9965096833899\\
28172506.2667275	-128.996509683339\\
28172506266.7274	-188.996509683339\\
281725062667274	-268.996509683339\\
1e+20	-380\\
};
\end{axis}

\begin{axis}[%
width=4.008in,
height=1.376in,
at={(0.818in,0.44in)},
scale only axis,
separate axis lines,
every outer x axis line/.append style={white!40!black},
every x tick label/.append style={font=\color{white!40!black}},
xmode=log,
xmin=0.01,
xmax=1000,
xminorticks=true,
every outer y axis line/.append style={white!40!black},
every y tick label/.append style={font=\color{white!40!black}},
ymin=-137.25,
ymax=92.25,
ytick={-135,  -90,  -45,    0,   45,   90},
ylabel={Phase (deg)},
axis background/.style={fill=white},
legend style={at={(0.079,0.148)},anchor=south west,legend cell align=left,align=left,draw=white!15!black,font=\footnotesize}
]
\addplot [color=mycolor1,solid]
  table[row sep=crcr]{%
1e-20	90\\
1.99999992937163e-17	90\\
1.99999992937163e-12	89.9999999998739\\
1.99999992937163e-08	89.9999987388859\\
1.99999992937163e-05	89.9987388859052\\
0.00199999992937163	89.873888743443\\
0.0199999992937163	88.7390388122939\\
0.0233430781209175	88.5283288004589\\
0.0272449657700967	88.2824358848865\\
0.0317990693416128	87.9955015608421\\
0.0371144093013522	87.6607012711143\\
0.0433182293164065	87.2700906923035\\
0.0505590423350863	86.8144308700913\\
0.0590101858312314	86.2829913033406\\
0.0688739713216429	85.6633314295342\\
0.0803865274917319	84.9410633704894\\
0.0938234528687379	84.0996028550733\\
0.109506413361588	83.1199218168358\\
0.127810842605588	81.9803264854588\\
0.149174929450118	80.6563004294182\\
0.17411010774038	79.1204747517263\\
0.203213299507564	77.3428189362572\\
0.237181204656585	75.2911854113277\\
0.276826979231519	72.932383912866\\
0.323099701518951	70.2339939617657\\
0.37710709198733	67.1671150987816\\
0.440142030953871	63.7101543966723\\
0.513713508784149	59.8534957042585\\
0.599582749539728	55.604445373963\\
0.699805372836077	50.9912666442986\\
0.816780603221459	46.0646503746082\\
0.953308705097753	40.8950497054108\\
1.11265801811499	35.5652791290094\\
1.2986431977967	30.1594868637407\\
1.51571653439462	24.7512151889947\\
1.76907453605043	19.3937510704831\\
2.06478232775366	14.1150165107022\\
2.4099188440765	8.91752886702458\\
2.81274629144729	3.78252053414384\\
3.28290793671198	-1.3232720200048\\
3.83165895683431	-6.43947221036898\\
4.47213587603466	-11.6034653536025\\
5.21967104041018	-16.8419539873011\\
6.09215965822894	-22.1633819354566\\
6.69649237299735	-25.4549709036474\\
7.73010902476924	-30.47988633492\\
8.79534869632984	-34.996440282044\\
9.87826120843702	-39.0199562423904\\
10.9656731306893	-42.5772328079483\\
12.0455797306772	-45.7042105722864\\
13.1074038731644	-48.4420573388443\\
14.142135623731	-50.8334843169518\\
15.2585517265911	-53.1493544539292\\
16.6036010280725	-55.6273006698736\\
18.2387344229938	-58.2538910143984\\
20.246478178688	-61.0070090447276\\
22.7392917444487	-63.8549547260476\\
25.8728563024337	-66.7563881738862\\
29.8663821087099	-69.661439055085\\
33.3565288678992	-71.6776368404299\\
38.9322043351596	-74.1960142724517\\
45.4398759654299	-76.3913743749412\\
53.0353305961912	-78.2968167725098\\
61.9003954497409	-79.9451655360468\\
72.2472908863037	-81.3675663929125\\
84.3237107370051	-82.592711585912\\
98.4187518345561	-83.6464980287815\\
114.869834688397	-84.5519648273654\\
134.070780978007	-85.3293985071621\\
156.481241231109	-85.9965297939327\\
182.637698375497	-86.5687725353025\\
213.166310577985	-87.0594741886382\\
248.797900814579	-87.480159964776\\
290.385452006476	-87.8407609238803\\
338.924526537099	-88.1498214635827\\
395.577098972007	-88.4146847512751\\
461.699372512011	-88.6416564520799\\
538.874245076229	-88.8361480765886\\
628.949202219072	-89.0028017362171\\
734.080543997884	-89.1455982535677\\
856.785004536076	-89.2679505598524\\
1000	-89.3727842014915\\
10000	-89.9372765573398\\
1000000	-89.9993727653853\\
1000000000	-89.9999993727654\\
10000000000000	-89.9999999999373\\
1e+18	-90\\
1e+20	-90\\
};
\addlegendentry{$F_y$ improper};

\addplot [color=mycolor2,solid]
  table[row sep=crcr]{%
1e-20	90\\
2e-17	90\\
2e-12	89.9999999998744\\
2e-08	89.9999987434938\\
2e-05	89.9987434938026\\
0.002	89.8743495331682\\
0.02	88.7436466913024\\
0.0233273136756512	88.5346968627506\\
0.0272081781661113	88.2910226292399\\
0.0317346853311942	88.0068691232206\\
0.0370142479559427	87.6755369929939\\
0.0431721486268304	87.2892328262715\\
0.0503545126534955	86.838899083991\\
0.0587317755826319	86.3140226713186\\
0.0685027275872005	85.7024225662906\\
0.0798992307032157	84.9900192236329\\
0.0931917208528568	84.1605923387206\\
0.108695625215416	83.1955398209431\\
0.126778847228552	82.0736606401661\\
0.147870496836895	80.7709990737561\\
0.172471073154432	79.2608095063735\\
0.201164341172481	77.5137307417176\\
0.23463118434431	75.4982966444778\\
0.273665761764463	73.181951561924\\
0.319194353348284	70.5327714482741\\
0.372297340202604	67.5220872232896\\
0.434234841775839	64.1281184969945\\
0.506476618150089	60.3404926422844\\
0.590736947048508	56.1651042323699\\
0.689015303179857	51.6282108058766\\
0.803643805230023	46.7781860329805\\
0.937342556404736	41.6833604894994\\
1.0932841917395	36.4252320446282\\
1.27516916386691	31.0879343692286\\
1.48731355375207	25.7464646776992\\
1.73475149012111	20.4567931161487\\
2.02335460796792	15.2501926652665\\
2.3599713808571	10.1324964972931\\
2.75258963334067	5.08751930171212\\
3.21052608986418	0.083213345393305\\
3.74464745810619	-4.92076311055176\\
4.36762829299866	-9.96494057469194\\
5.09425176047145	-15.0819235708168\\
5.94176043796281	-20.2887854821262\\
6.69649237299737	-24.3945879891351\\
7.73010902476926	-29.3730864362741\\
8.79534869632986	-33.8662465336739\\
9.87826120843704	-37.8858134382279\\
10.9656731306893	-41.4541744217982\\
12.0455797306772	-44.6029268311449\\
13.1074038731645	-47.3694794959785\\
14.142135623731	-49.7935614644938\\
15.2585517265912	-52.1479376244947\\
16.6036010280725	-54.6747204377024\\
18.2387344229938	-57.3618539398813\\
20.2464781786881	-60.188349111308\\
22.7392917444487	-63.1232433369223\\
25.8728563024337	-66.1254619866422\\
29.8663821087099	-69.144975041529\\
32.2921682842295	-70.6514130581974\\
37.6644769416569	-73.3540107134944\\
43.9305534023683	-75.7285384121618\\
51.2390899580996	-77.8075974325911\\
59.763516195375	-79.6245076649526\\
69.7061144324689	-81.2114806923601\\
81.302819823852	-82.5984642349215\\
94.8288190372977	-83.8124399781689\\
110.60508035873	-84.8769890242852\\
129.005970182436	-85.8119896858324\\
150.46813662387	-86.6333756763792\\
175.500871060788	-87.3529638194199\\
204.698193479251	-87.9784623480011\\
238.752948406482	-88.5138755363126\\
278.473245923229	-88.96056492099\\
324.801637896396	-89.3191036697331\\
378.837484478726	-89.59168827462\\
441.863041626496	-89.7843546219279\\
515.37388868493	-89.9079643459779\\
601.114418079677	-89.9773038204002\\
701.119229275059	-90.008614922033\\
817.761409266507	-90.0167487343077\\
825.831581509538	-90.0167710351011\\
953.80884529012	-90.0131882637172\\
1112.48990603467	-90.0054693864956\\
1297.57004995333	-89.9977223737498\\
1513.44117856958	-89.9917040939324\\
1765.225855107	-89.987799888276\\
2058.89886152253	-89.9857532499068\\
2401.42897845886	-89.9851003344822\\
2448.81769286874	-89.9850931332951\\
2800.94435251543	-89.9853839893812\\
3266.92537495857	-89.9862348410437\\
3810.42964883015	-89.9873851633827\\
4444.35438286313	-89.9886551245997\\
5183.74243874018	-89.9899316578545\\
6046.13929411387	-89.991148661057\\
7052.00939202376	-89.9922713192013\\
8225.22175656882	-89.9932847326255\\
9593.6163983636	-89.9941861100023\\
11189.664950425	-89.9949796161467\\
13051.2412112002	-89.9956730804126\\
15222.5198795127	-89.9962759581967\\
17755.0248081614	-89.9967981123455\\
20708.8516509476	-89.997249118518\\
24154.0939162091	-89.9976378981294\\
28172.5062667275	-89.997972551325\\
281725.062667275	-89.9997966315007\\
28172506.2667275	-89.9999979662519\\
28172506266.7274	-89.9999999979663\\
281725062667274	-89.9999999999998\\
1e+20	-90\\
};
\addlegendentry{$F_y$ proper};

\end{axis}
\end{tikzpicture}%
}
  \caption{The frequency response of the closed-loop transfer function from $d$
    to $y$ when $F_y$ is proper (\texttt{red}) and when $F_y$ is improper
    (\texttt{blue}). $p_1 = p_2 = 50\omega_c$}
  \label{fig:bode_2.1_dy}
\end{figure}

\begin{figure}[H]\centering
  \scalebox{0.8}{% This file was created by matlab2tikz.
%
%The latest updates can be retrieved from
%  http://www.mathworks.com/matlabcentral/fileexchange/22022-matlab2tikz-matlab2tikz
%where you can also make suggestions and rate matlab2tikz.
%
\definecolor{mycolor1}{rgb}{0.00000,0.44700,0.74100}%
\definecolor{mycolor2}{rgb}{0.85000,0.32500,0.09800}%
%
\begin{tikzpicture}

\begin{axis}[%
width=4.008in,
height=3.052in,
at={(0.818in,0.44in)},
scale only axis,
separate axis lines,
every outer x axis line/.append style={white!40!black},
every x tick label/.append style={font=\color{white!40!black}},
xmin=0,
xmax=7,
xlabel={Time [sec]},
every outer y axis line/.append style={white!40!black},
every y tick label/.append style={font=\color{white!40!black}},
ymin=0,
ymax=0.8,
ylabel={Amplitude},
axis background/.style={fill=white},
legend style={legend cell align=left,align=left,draw=white!15!black}
]
\addplot [color=mycolor1,solid]
  table[row sep=crcr]{%
0	0\\
0.00925912744565873	0.08804073389352\\
0.0185182548913175	0.167523419442372\\
0.0277773823369762	0.239208550485529\\
0.0370365097826349	0.303789636723998\\
0.0462956372282937	0.361899098234114\\
0.0555547646739524	0.414113641321717\\
0.0648138921196111	0.46095916135362\\
0.0740730195652698	0.502915214187309\\
0.0833321470109286	0.540419094157604\\
0.0925912744565873	0.573869553239053\\
0.101850401902246	0.603630192956758\\
0.111109529347905	0.630032557840279\\
0.120368656793564	0.653378956681636\\
0.129627784239222	0.673945035547754\\
0.138886911684881	0.691982124390311\\
0.14814603913054	0.707719377174027\\
0.157405166576198	0.721365723691562\\
0.166664294021857	0.733111649634632\\
0.175923421467516	0.743130820032961\\
0.185182548913175	0.75158155984307\\
0.194441676358833	0.758608204256197\\
0.203700803804492	0.764342330188711\\
0.212959931250151	0.768903879409719\\
0.22221905869581	0.772402182840657\\
0.231478186141468	0.774936894722716\\
0.240737313587127	0.776598844582789\\
0.249996441032786	0.777470814230832\\
0.259255568478444	0.777628246385105\\
0.268514695924103	0.777139890941342\\
0.277773823369762	0.776068394372546\\
0.287032950815421	0.774470837263355\\
0.296292078261079	0.772399224542597\\
0.305551205706738	0.769900932576145\\
0.314810333152397	0.767019116915924\\
0.324069460598056	0.76379308416695\\
0.333328588043714	0.760258631129664\\
0.342587715489373	0.756448354097031\\
0.351846842935032	0.752391930932482\\
0.361105970380691	0.748116378323749\\
0.370365097826349	0.743646286396879\\
0.379624225272008	0.739004032682515\\
0.388883352717667	0.734209977251281\\
0.398142480163325	0.729282640675199\\
0.407401607608984	0.724238866326324\\
0.416660735054643	0.71909396839077\\
0.425919862500302	0.713861866855068\\
0.43517898994596	0.708555210611179\\
0.444438117391619	0.703185489725639\\
0.453697244837278	0.697763137826294\\
0.462956372282937	0.692297625476227\\
0.472215499728595	0.686797545327922\\
0.481474627174254	0.681270689780974\\
0.490733754619913	0.675724121802958\\
0.499992882065571	0.670164239515085\\
0.50925200951123	0.664596835091302\\
0.518511136956889	0.659027148471218\\
0.527770264402548	0.65345991634323\\
0.537029391848206	0.647899416814043\\
0.546288519293865	0.642349510144176\\
0.555547646739524	0.636813675895634\\
0.564806774185183	0.631295046807462\\
0.574065901630841	0.625796439687145\\
0.5833250290765	0.620320383580433\\
0.592584156522159	0.614869145459113\\
0.601843283967817	0.609444753645141\\
0.611102411413476	0.604049019170349\\
0.620361538859135	0.598683555253396\\
0.629620666304794	0.593349795059666\\
0.638879793750452	0.588049007895219\\
0.648138921196111	0.58278231397261\\
0.65739804864177	0.577550697874273\\
0.666657176087429	0.572355020828095\\
0.675916303533087	0.567196031899718\\
0.685175430978746	0.562074378196934\\
0.694434558424405	0.556990614173097\\
0.703693685870064	0.551945210108886\\
0.712952813315722	0.546938559844725\\
0.722211940761381	0.541970987829831\\
0.73147106820704	0.537042755548041\\
0.740730195652698	0.53215406737529\\
0.749989323098357	0.527305075918762\\
0.759248450544016	0.52249588688337\\
0.768507577989675	0.517726563507156\\
0.777766705435333	0.51299713060359\\
0.787025832880992	0.508307578245366\\
0.796284960326651	0.50365786512128\\
0.80554408777231	0.499047921594965\\
0.814803215217968	0.49447765249176\\
0.824062342663627	0.489946939637641\\
0.833321470109286	0.485455644172071\\
0.842580597554945	0.481003608654674\\
0.851839725000603	0.476590658983908\\
0.861098852446262	0.472216606144299\\
0.870357979891921	0.467881247797342\\
0.879617107337579	0.463584369729855\\
0.888876234783238	0.459325747172349\\
0.898135362228897	0.455105145998876\\
0.907394489674556	0.4509223238188\\
0.916653617120214	0.446777030970036\\
0.925912744565873	0.44266901142244\\
0.935171872011532	0.438598003599282\\
0.944430999457191	0.434563741124028\\
0.953690126902849	0.430565953499035\\
0.962949254348508	0.426604366722158\\
0.972208381794167	0.422678703846768\\
0.981467509239826	0.418788685490171\\
0.990726636685484	0.414934030294997\\
0.999985764131143	0.41111445534771\\
1.0092448915768	0.40732967655804\\
1.01850401902246	0.403579409002799\\
1.02776314646812	0.399863367237224\\
1.03702227391378	0.39618126557673\\
1.04628140135944	0.392532818351709\\
1.0555405288051	0.38891774013774\\
1.06479965625075	0.385335745963422\\
1.07405878369641	0.3817865514978\\
1.08331791114207	0.378269873219204\\
1.09257703858773	0.37478542856716\\
1.10183616603339	0.371332936078875\\
1.11109529347905	0.367912115511678\\
1.12035442092471	0.364522687952665\\
1.12961354837037	0.361164375916704\\
1.13887267581602	0.357836903433822\\
1.14813180326168	0.354539996126957\\
1.15739093070734	0.351273381280907\\
1.166650058153	0.348036787903295\\
1.17590918559866	0.344829946778257\\
1.18516831304432	0.341652590513514\\
1.19442744048998	0.33850445358143\\
1.20368656793563	0.335385272354593\\
1.21294569538129	0.332294785136437\\
1.22220482282695	0.329232732187334\\
1.23146395027261	0.326198855746596\\
1.24072307771827	0.323192900050736\\
1.24998220516393	0.320214611348369\\
1.25924133260959	0.317263737912024\\
1.26850046005525	0.314340030047194\\
1.2777595875009	0.311443240098851\\
1.28701871494656	0.30857312245569\\
1.29627784239222	0.305729433552299\\
1.30553696983788	0.302911931869465\\
1.31479609728354	0.30012037793279\\
1.3240552247292	0.297354534309778\\
1.33331435217486	0.294614165605557\\
1.34257347962052	0.29189903845735\\
1.35183260706617	0.289208921527838\\
1.36109173451183	0.286543585497518\\
1.37035086195749	0.283902803056159\\
1.37960998940315	0.281286348893459\\
1.38886911684881	0.278693999688973\\
1.39812824429447	0.276125534101401\\
1.40738737174013	0.273580732757309\\
1.41664649918579	0.271059378239329\\
1.42590562663144	0.268561255073927\\
1.4351647540771	0.266086149718756\\
1.44442388152276	0.263633850549677\\
1.45368300896842	0.261204147847463\\
1.46294213641408	0.258796833784253\\
1.47220126385974	0.256411702409766\\
1.4814603913054	0.254048549637327\\
1.49071951875106	0.251707173229732\\
1.49997864619671	0.249387372784964\\
1.50923777364237	0.247088949721814\\
1.51849690108803	0.244811707265397\\
1.52775602853369	0.242555450432607\\
1.53701515597935	0.240319986017521\\
1.54627428342501	0.238105122576763\\
1.55553341087067	0.23591067041486\\
1.56479253831633	0.233736441569578\\
1.57405166576198	0.231582249797279\\
1.58331079320764	0.229447910558291\\
1.5925699206533	0.227333241002301\\
1.60182904809896	0.225238059953797\\
1.61108817554462	0.223162187897544\\
1.62034730299028	0.221105446964122\\
1.62960643043594	0.21906766091552\\
1.6388655578816	0.217048655130794\\
1.64812468532725	0.2150482565918\\
1.65738381277291	0.213066293869001\\
1.66664294021857	0.211102597107349\\
1.67590206766423	0.209156998012262\\
1.68516119510989	0.207229329835675\\
1.69442032255555	0.205319427362192\\
1.70367945000121	0.203427126895321\\
1.71293857744687	0.201552266243811\\
1.72219770489252	0.19969468470808\\
1.73145683233818	0.19785422306675\\
1.74071595978384	0.196030723563268\\
1.7499750872295	0.194224029892641\\
1.75923421467516	0.192433987188269\\
1.76849334212082	0.190660442008875\\
1.77775246956648	0.188903242325545\\
1.78701159701214	0.187162237508876\\
1.79627072445779	0.185437278316214\\
1.80552985190345	0.183728216879012\\
1.81478897934911	0.182034906690285\\
1.82404810679477	0.180357202592169\\
1.83330723424043	0.178694960763594\\
1.84256636168609	0.177048038708046\\
1.85182548913175	0.175416295241452\\
1.8610846165774	0.173799590480155\\
1.87034374402306	0.172197785829\\
1.87960287146872	0.170610743969521\\
1.88886199891438	0.169038328848233\\
1.89812112636004	0.167480405665029\\
1.9073802538057	0.165936840861673\\
1.91663938125136	0.164407502110397\\
1.92589850869702	0.162892258302606\\
1.93515763614267	0.161390979537673\\
1.94441676358833	0.159903537111842\\
1.95367589103399	0.158429803507227\\
1.96293501847965	0.156969652380909\\
1.97219414592531	0.155522958554135\\
1.98145327337097	0.154089598001608\\
1.99071240081663	0.15266944784088\\
1.99997152826229	0.151262386321837\\
2.00923065570794	0.149868292816283\\
2.0184897831536	0.148487047807611\\
2.02774891059926	0.147118532880576\\
2.03700803804492	0.145762630711158\\
2.04626716549058	0.144419225056516\\
2.05552629293624	0.143088200745035\\
2.0647854203819	0.141769443666461\\
2.07404454782756	0.14046284076213\\
2.08330367527321	0.139168280015287\\
2.09256280271887	0.137885650441487\\
2.10182193016453	0.136614842079088\\
2.11108105761019	0.135355745979833\\
2.12034018505585	0.134108254199516\\
2.12959931250151	0.132872259788731\\
2.13885843994717	0.131647656783711\\
2.14811756739283	0.130434340197248\\
2.15737669483848	0.129232206009694\\
2.16663582228414	0.12804115116005\\
2.1758949497298	0.126861073537133\\
2.18515407717546	0.125691871970828\\
2.19441320462112	0.12453344622341\\
2.20367233206678	0.123385696980962\\
2.21293145951244	0.122248525844859\\
2.2221905869581	0.121121835323332\\
2.23144971440375	0.120005528823118\\
2.24070884184941	0.118899510641175\\
2.24996796929507	0.117803685956483\\
2.25922709674073	0.116717960821914\\
2.26848622418639	0.115642242156179\\
2.27774535163205	0.114576437735853\\
2.28700447907771	0.113520456187465\\
2.29626360652337	0.112474206979667\\
2.30552273396902	0.111437600415475\\
2.31478186141468	0.110410547624579\\
2.32404098886034	0.109392960555726\\
2.333300116306	0.108384751969167\\
2.34255924375166	0.107385835429187\\
2.35181837119732	0.106396125296688\\
2.36107749864298	0.10541553672185\\
2.37033662608864	0.104443985636858\\
2.37959575353429	0.103481388748695\\
2.38885488097995	0.102527663532\\
2.39811400842561	0.101582728221996\\
2.40737313587127	0.100646501807477\\
2.41663226331693	0.0997189040238667\\
2.42589139076259	0.0987998553463347\\
2.43515051820825	0.0978892769829806\\
2.4444096456539	0.0969870908680788\\
2.45366877309956	0.0960932196553859\\
2.46292790054522	0.0952075867115102\\
2.47218702799088	0.0943301161093413\\
2.48144615543654	0.0934607326215416\\
2.4907052828822	0.0925993617140965\\
2.49996441032786	0.091745929539925\\
2.50922353777352	0.0909003629325487\\
2.51848266521917	0.090062589399819\\
2.52774179266483	0.0892325371177024\\
2.53700092011049	0.0884101349241233\\
2.54626004755615	0.0875953123128625\\
2.55551917500181	0.0867879994275134\\
2.56477830244747	0.0859881270554928\\
2.57403742989313	0.0851956266221072\\
2.58329655733879	0.0844104301846739\\
2.59255568478444	0.0836324704266961\\
2.6018148122301	0.0828616806520918\\
2.61107393967576	0.0820979947794762\\
2.62033306712142	0.0813413473364958\\
2.62959219456708	0.0805916734542158\\
2.63885132201274	0.0798489088615589\\
2.6481104494584	0.0791129898797947\\
2.65736957690406	0.0783838534170811\\
2.66662870434971	0.0776614369630547\\
2.67588783179537	0.0769456785834721\\
2.68514695924103	0.0762365169149002\\
2.69440608668669	0.0755338911594551\\
2.70366521413235	0.0748377410795898\\
2.71292434157801	0.0741480069929306\\
2.72218346902367	0.0734646297671594\\
2.73144259646933	0.0727875508149448\\
2.74070172391498	0.0721167120889193\\
2.74996085136064	0.0714520560767027\\
2.7592199788063	0.0707935257959713\\
2.76847910625196	0.0701410647895729\\
2.77773823369762	0.0694946171206866\\
2.78699736114328	0.068854127368027\\
2.79625648858894	0.0682195406210934\\
2.8055156160346	0.0675908024754614\\
2.81477474348025	0.0669678590281193\\
2.82403387092591	0.0663506568728469\\
2.83329299837157	0.0657391430956364\\
2.84255212581723	0.0651332652701564\\
2.85181125326289	0.0645329714532573\\
2.86107038070855	0.063938210180518\\
2.87032950815421	0.0633489304618334\\
2.87958863559987	0.0627650817770436\\
2.88884776304552	0.062186614071602\\
2.89810689049118	0.0616134777522849\\
2.90736601793684	0.0610456236829388\\
2.9166251453825	0.0604830031802685\\
2.92588427282816	0.0599255680096631\\
2.93514340027382	0.059373270381061\\
2.94440252771948	0.0588260629448525\\
2.95366165516514	0.0582838987878206\\
2.96292078261079	0.0577467314291188\\
2.97217991005645	0.0572145148162865\\
2.98143903750211	0.0566872033213008\\
2.99069816494777	0.0561647517366643\\
2.99995729239343	0.0556471152715301\\
3.00921641983909	0.055134249547861\\
3.01847554728475	0.0546261105966255\\
3.02773467473041	0.0541226548540278\\
3.03699380217606	0.0536238391577735\\
3.04625292962172	0.0531296207433684\\
3.05551205706738	0.0526399572404532\\
3.06477118451304	0.0521548066691702\\
3.0740303119587	0.0516741274365647\\
3.08328943940436	0.0511978783330192\\
3.09254856685002	0.0507260185287199\\
3.10180769429567	0.0502585075701569\\
3.11106682174133	0.0497953053766557\\
3.12032594918699	0.0493363722369411\\
3.12958507663265	0.0488816688057326\\
3.13884420407831	0.0484311561003714\\
3.14810333152397	0.0479847954974779\\
3.15736245896963	0.0475425487296413\\
3.16662158641529	0.0471043778821379\\
3.17588071386094	0.0466702453896812\\
3.1851398413066	0.046240114033201\\
3.19439896875226	0.0458139469366531\\
3.20365809619792	0.0453917075638569\\
3.21291722364358	0.044973359715364\\
3.22217635108924	0.0445588675253543\\
3.2314354785349	0.0441481954585611\\
3.24069460598056	0.0437413083072249\\
3.24995373342621	0.0433381711880746\\
3.25921286087187	0.0429387495393371\\
3.26847198831753	0.0425430091177743\\
3.27773111576319	0.0421509159957471\\
3.28699024320885	0.0417624365583071\\
3.29624937065451	0.0413775375003143\\
3.30550849810017	0.040996185823582\\
3.31476762554583	0.0406183488340479\\
3.32402675299148	0.0402439941389709\\
3.33328588043714	0.0398730896441545\\
3.3425450078828	0.0395056035511943\\
3.35180413532846	0.0391415043547532\\
3.36106326277412	0.0387807608398593\\
3.37032239021978	0.0384233420792304\\
3.37958151766544	0.0380692174306219\\
3.3888406451111	0.0377183565342006\\
3.39809977255675	0.0373707293099414\\
3.40735890000241	0.0370263059550485\\
3.41661802744807	0.0366850569414003\\
3.42587715489373	0.0363469530130182\\
3.43513628233939	0.036011965183558\\
3.44439540978505	0.0356800647338253\\
3.45365453723071	0.0353512232093129\\
3.46291366467637	0.0350254124177616\\
3.47217279212202	0.0347026044267431\\
3.48143191956768	0.0343827715612656\\
3.49069104701334	0.0340658864014006\\
3.499950174459	0.0337519217799327\\
3.50920930190466	0.0334408507800301\\
3.51846842935032	0.0331326467329373\\
3.52772755679598	0.0328272832156883\\
3.53698668424164	0.0325247340488417\\
3.54624581168729	0.0322249732942361\\
3.55550493913295	0.0319279752527666\\
3.56476406657861	0.0316337144621812\\
3.57402319402427	0.0313421656948978\\
3.58328232146993	0.031053303955842\\
3.59254144891559	0.0307671044803038\\
3.60180057636125	0.0304835427318143\\
3.61105970380691	0.0302025944000427\\
3.62031883125256	0.0299242353987114\\
3.62957795869822	0.0296484418635316\\
3.63883708614388	0.0293751901501573\\
3.64809621358954	0.0291044568321578\\
3.6573553410352	0.02883621869901\\
3.66661446848086	0.0285704527541077\\
3.67587359592652	0.0283071362127908\\
3.68513272337217	0.0280462465003915\\
3.69439185081783	0.0277877612502988\\
3.70365097826349	0.0275316583020417\\
3.71291010570915	0.0272779156993884\\
3.72216923315481	0.0270265116884644\\
3.73142836060047	0.0267774247158876\\
3.74068748804613	0.0265306334269205\\
3.74994661549179	0.0262861166636389\\
3.75920574293745	0.0260438534631187\\
3.7684648703831	0.0258038230556381\\
3.77772399782876	0.0255660048628975\\
3.78698312527442	0.0253303784962547\\
3.79624225272008	0.0250969237549775\\
3.80550138016574	0.0248656206245116\\
3.8147605076114	0.0246364492747645\\
3.82401963505706	0.0244093900584058\\
3.83327876250271	0.0241844235091827\\
3.84253788994837	0.0239615303402508\\
3.85179701739403	0.0237406914425211\\
3.86105614483969	0.0235218878830214\\
3.87031527228535	0.0233051009032732\\
3.87957439973101	0.0230903119176836\\
3.88883352717667	0.0228775025119517\\
3.89809265462233	0.0226666544414902\\
3.90735178206798	0.0224577496298609\\
3.91661090951364	0.0222507701672255\\
3.9258700369593	0.0220456983088095\\
3.93512916440496	0.0218425164733814\\
3.94438829185062	0.0216412072417452\\
3.95364741929628	0.021441753355247\\
3.96290654674194	0.0212441377142956\\
3.9721656741876	0.0210483433768963\\
3.98142480163325	0.0208543535571985\\
3.99068392907891	0.0206621516240565\\
3.99994305652457	0.0204717210996039\\
4.00920218397023	0.0202830456578409\\
4.01846131141589	0.0200961091232342\\
4.02772043886155	0.019910895469331\\
4.03697956630721	0.0197273888173844\\
4.04623869375287	0.0195455734349921\\
4.05549782119852	0.0193654337347482\\
4.06475694864418	0.0191869542729062\\
4.07401607608984	0.0190101197480554\\
4.0832752035355	0.0188349149998087\\
4.09253433098116	0.0186613250075033\\
4.10179345842682	0.0184893348889127\\
4.11105258587248	0.0183189298989708\\
4.12031171331814	0.018150095428508\\
4.12957084076379	0.0179828170029983\\
4.13882996820945	0.0178170802813188\\
4.14808909565511	0.01765287105452\\
4.15734822310077	0.0174901752446077\\
4.16660735054643	0.0173289789033361\\
4.17586647799209	0.0171692682110117\\
4.18512560543775	0.0170110294753089\\
4.1943847328834	0.0168542491300959\\
4.20364386032906	0.0166989137342719\\
4.21290298777472	0.0165450099706144\\
4.22216211522038	0.0163925246446375\\
4.23142124266604	0.0162414446834612\\
4.2406803701117	0.0160917571346903\\
4.24993949755736	0.0159434491653036\\
4.25919862500302	0.0157965080605546\\
4.26845775244867	0.0156509212228804\\
4.27771687989433	0.0155066761708225\\
4.28697600733999	0.0153637605379565\\
4.29623513478565	0.0152221620718316\\
4.30549426223131	0.0150818686329205\\
4.31475338967697	0.0149428681935786\\
4.32401251712263	0.0148051488370129\\
4.33327164456829	0.01466869875626\\
4.34253077201395	0.0145335062531743\\
4.3517898994596	0.014399559737425\\
4.36104902690526	0.0142668477255021\\
4.37030815435092	0.0141353588397324\\
4.37956728179658	0.0140050818073036\\
4.38882640924224	0.0138760054592982\\
4.3980855366879	0.0137481187297359\\
4.40734466413356	0.0136214106546248\\
4.41660379157921	0.0134958703710212\\
4.42586291902487	0.0133714871160989\\
4.43512204647053	0.0132482502262261\\
4.44438117391619	0.013126149136051\\
4.45364030136185	0.0130051733775966\\
4.46289942880751	0.0128853125793625\\
4.47215855625317	0.0127665564654365\\
4.48141768369883	0.0126488948546129\\
4.49067681114448	0.0125323176595204\\
4.49993593859014	0.0124168148857565\\
4.5091950660358	0.0123023766310313\\
4.51845419348146	0.0121889930843181\\
4.52771332092712	0.0120766545250128\\
4.53697244837278	0.0119653513220998\\
4.54623157581844	0.0118550739333271\\
4.5554907032641	0.0117458129043877\\
4.56474983070975	0.0116375588681091\\
4.57400895815541	0.0115303025436506\\
4.58326808560107	0.0114240347357071\\
4.59252721304673	0.0113187463337213\\
4.60178634049239	0.0112144283111023\\
4.61104546793805	0.0111110717244518\\
4.62030459538371	0.0110086677127972\\
4.62956372282937	0.0109072074968325\\
4.63882285027502	0.010806682378165\\
4.64808197772068	0.0107070837385698\\
4.65734110516634	0.010608403039251\\
4.666600232612	0.0105106318201096\\
4.67585936005766	0.0104137616990183\\
4.68511848750332	0.0103177843711025\\
4.69437761494898	0.0102226916080289\\
4.70363674239464	0.0101284752572996\\
4.71289586984029	0.0100351272415535\\
4.72215499728595	0.00994263955787335\\
4.73141412473161	0.00985100427710019\\
4.74067325217727	0.00976021354315319\\
4.74993237962293	0.00967025957235629\\
4.75919150706859	0.00958113465277087\\
4.76845063451425	0.00949283114353453\\
4.77770976195991	0.00940534147420611\\
4.78696888940556	0.00931865814411659\\
4.79622801685122	0.00923277372172605\\
4.80548714429688	0.0091476808439866\\
4.81474627174254	0.0090633722157111\\
4.8240053991882	0.00897984060894771\\
4.83326452663386	0.00889707886236023\\
4.84252365407952	0.00881507988061419\\
4.85178278152518	0.0087338366337685\\
4.86104190897083	0.00865334215667275\\
4.87030103641649	0.00857358954837013\\
4.87956016386215	0.00849457197150576\\
4.88881929130781	0.00841628265174055\\
4.89807841875347	0.00833871487717028\\
4.90733754619913	0.00826186199775037\\
4.91659667364479	0.00818571742472565\\
4.92585580109044	0.00811027463006549\\
4.9351149285361	0.00803552714590421\\
4.94437405598176	0.00796146856398647\\
4.95363318342742	0.00788809253511797\\
4.96289231087308	0.00781539276862107\\
4.97215143831874	0.00774336303179551\\
4.9814105657644	0.007671997149384\\
4.99066969321006	0.00760128900304294\\
4.99992882065571	0.00753123253081769\\
5.00918794810137	0.00746182172662306\\
5.01844707554703	0.00739305063972825\\
5.02770620299269	0.00732491337424675\\
5.03696533043835	0.00725740408863088\\
5.04622445788401	0.00719051699517094\\
5.05548358532967	0.00712424635949908\\
5.06474271277533	0.00705858650009759\\
5.07400184022098	0.00699353178781187\\
5.08326096766664	0.00692907664536786\\
5.0925200951123	0.00686521554689386\\
5.10177922255796	0.00680194301744672\\
5.11103835000362	0.00673925363254258\\
5.12029747744928	0.00667714201769172\\
5.12955660489494	0.00661560284793787\\
5.1388157323406	0.00655463084740162\\
5.14807485978625	0.00649422078882815\\
5.15733398723191	0.00643436749313908\\
5.16659311467757	0.00637506582898848\\
5.17585224212323	0.0063163107123229\\
5.18511136956889	0.00625809710594553\\
5.19437049701455	0.00620042001908434\\
5.20362962446021	0.00614327450696421\\
5.21288875190587	0.0060866556703831\\
5.22214787935152	0.00603055865529182\\
5.23140700679718	0.00597497865237811\\
5.24066613424284	0.00591991089665417\\
5.2499252616885	0.00586535066704822\\
5.25918438913416	0.00581129328599973\\
5.26844351657982	0.00575773411905839\\
5.27770264402548	0.00570466857448687\\
5.28696177147114	0.00565209210286703\\
5.29622089891679	0.00560000019671003\\
5.30548002636245	0.00554838839006975\\
5.31473915380811	0.00549725225816\\
5.32399828125377	0.00544658741697523\\
5.33325740869943	0.00539638952291453\\
5.34251653614509	0.00534665427240935\\
5.35177566359075	0.00529737740155458\\
5.36103479103641	0.00524855468574284\\
5.37029391848206	0.00520018193930247\\
5.37955304592772	0.00515225501513858\\
5.38881217337338	0.00510476980437752\\
5.39807130081904	0.00505772223601469\\
5.4073304282647	0.0050111082765654\\
5.41658955571036	0.00496492392971921\\
5.42584868315602	0.00491916523599723\\
5.43510781060168	0.00487382827241267\\
5.44436693804733	0.00482890915213451\\
5.45362606549299	0.00478440402415434\\
5.46288519293865	0.00474030907295618\\
5.47214432038431	0.00469662051818927\\
5.48140344782997	0.00465333461434414\\
5.49066257527563	0.00461044765043131\\
5.49992170272129	0.00456795594966333\\
5.50918083016694	0.0045258558691394\\
5.5184399576126	0.00448414379953321\\
5.52769908505826	0.00444281616478336\\
5.53695821250392	0.00440186942178683\\
5.54621733994958	0.00436130006009526\\
5.55547646739524	0.004321104601614\\
5.5647355948409	0.00428127960030381\\
5.57399472228656	0.00424182164188558\\
5.58325384973222	0.00420272734354748\\
5.59251297717787	0.004163993353655\\
5.60177210462353	0.00412561635146371\\
5.61103123206919	0.00408759304683438\\
5.62029035951485	0.00404992017995101\\
5.62954948696051	0.00401259452104139\\
5.63880861440617	0.00397561287010009\\
5.64806774185183	0.00393897205661424\\
5.65732686929748	0.00390266893929168\\
5.66658599674314	0.00386670040579161\\
5.6758451241888	0.00383106337245779\\
5.68510425163446	0.00379575478405419\\
5.69436337908012	0.00376077161350297\\
5.70362250652578	0.00372611086162507\\
5.71288163397144	0.00369176955688312\\
5.7221407614171	0.00365774475512647\\
5.73139988886275	0.00362403353933902\\
5.74065901630841	0.00359063301938902\\
5.74991814375407	0.00355754033178131\\
5.75917727119973	0.00352475263941183\\
5.76843639864539	0.00349226713132435\\
5.77769552609105	0.00346008102246958\\
5.78695465353671	0.00342819155346631\\
5.79621378098237	0.00339659599036486\\
5.80547290842802	0.00336529162441274\\
5.81473203587368	0.00333427577182236\\
5.82399116331934	0.003303545773541\\
5.833250290765	0.0032730989950228\\
5.84250941821066	0.0032429328260029\\
5.85176854565632	0.00321304468027369\\
5.86102767310198	0.00318343199546306\\
5.87028680054764	0.00315409223281471\\
5.87954592799329	0.00312502287697053\\
5.88880505543895	0.003096221435755\\
5.89806418288461	0.00306768543996139\\
5.90732331033027	0.00303941244314017\\
5.91658243777593	0.00301140002138927\\
5.92584156522159	0.00298364577314621\\
5.93510069266725	0.00295614731898236\\
5.94435982011291	0.00292890230139876\\
5.95361894755856	0.00290190838462412\\
5.96287807500422	0.00287516325441449\\
5.97213720244988	0.00284866461785495\\
5.98139632989554	0.00282241020316298\\
5.9906554573412	0.00279639775949366\\
5.99991458478686	0.00277062505674682\\
6.00917371223252	0.00274508988537566\\
6.01843283967817	0.00271979005619757\\
6.02769196712383	0.00269472340020622\\
6.03695109456949	0.00266988776838571\\
6.04621022201515	0.00264528103152632\\
6.05546934946081	0.00262090108004199\\
6.06472847690647	0.00259674582378937\\
6.07398760435213	0.00257281319188878\\
6.08324673179779	0.00254910113254651\\
6.09250585924345	0.00252560761287904\\
6.1017649866891	0.00250233061873867\\
6.11102411413476	0.00247926815454091\\
6.12028324158042	0.00245641824309332\\
6.12954236902608	0.00243377892542608\\
6.13880149647174	0.002411348260624\\
6.1480606239174	0.00238912432566012\\
6.15731975136306	0.00236710521523087\\
6.16657887880872	0.00234528904159265\\
6.17583800625437	0.00232367393440014\\
6.18509713370003	0.00230225804054579\\
6.19435626114569	0.00228103952400105\\
6.20361538859135	0.00226001656565891\\
6.21287451603701	0.00223918736317802\\
6.22213364348267	0.00221855013082804\\
6.23139277092833	0.00219810309933668\\
6.24065189837398	0.00217784451573793\\
6.24991102581964	0.00215777264322182\\
6.2591701532653	0.0021378857609855\\
6.26842928071096	0.00211818216408574\\
6.27768840815662	0.0020986601632927\\
6.28694753560228	0.00207931808494513\\
6.29620666304794	0.00206015427080697\\
6.3054657904936	0.00204116707792506\\
6.31472491793925	0.00202235487848837\\
6.32398404538491	0.00200371605968837\\
6.33324317283057	0.00198524902358092\\
6.34250230027623	0.00196695218694905\\
6.35176142772189	0.00194882398116739\\
6.36102055516755	0.00193086285206763\\
6.37027968261321	0.00191306725980529\\
6.37953881005887	0.00189543567872773\\
6.38879793750452	0.00187796659724325\\
6.39805706495018	0.00186065851769164\\
6.40731619239584	0.00184350995621568\\
6.4165753198415	0.00182651944263402\\
6.42583444728716	0.00180968552031504\\
6.43509357473282	0.00179300674605199\\
6.44435270217848	0.00177648168993929\\
6.45361182962414	0.00176010893524998\\
6.46287095706979	0.00174388707831415\\
6.47213008451545	0.00172781472839868\\
6.48138921196111	0.00171189050758801\\
6.49064833940677	0.00169611305066602\\
6.49990746685243	0.00168048100499889\\
6.50916659429809	0.00166499303041927\\
6.51842572174375	0.00164964779911131\\
6.52768484918941	0.0016344439954968\\
6.53694397663506	0.0016193803161224\\
6.54620310408072	0.00160445546954796\\
6.55546223152638	0.00158966817623576\\
6.56472135897204	0.00157501716844077\\
6.5739804864177	0.001560501190102\\
6.58323961386336	0.0015461189967348\\
6.59249874130902	0.00153186935532422\\
6.60175786875468	0.00151775104421923\\
6.61101699620033	0.00150376285302804\\
6.62027612364599	0.00148990358251427\\
6.62953525109165	0.00147617204449425\\
6.63879437853731	0.00146256706173498\\
6.64805350598297	0.00144908746785338\\
6.65731263342863	0.00143573210721618\\
6.66657176087429	0.00142249983484085\\
6.67583088831995	0.00140938951629758\\
6.6850900157656	0.00139640002761176\\
6.69434914321126	0.00138353025516788\\
6.70360827065692	0.0013707790956139\\
6.71286739810258	0.00135814545576676\\
6.72212652554824	0.00134562825251856\\
6.7313856529939	0.00133322641274377\\
6.74064478043956	0.00132093887320722\\
6.74990390788521	0.00130876458047285\\
6.75916303533087	0.00129670249081357\\
6.76842216277653	0.00128475157012166\\
6.77768129022219	0.00127291079382014\\
6.78694041766785	0.00126117914677497\\
6.79619954511351	0.00124955562320794\\
6.80545867255917	0.00123803922661048\\
6.81471780000483	0.0012266289696583\\
6.82397692745048	0.00121532387412662\\
6.83323605489614	0.00120412297080639\\
6.8424951823418	0.00119302529942115\\
6.85175430978746	0.00118202990854477\\
6.86101343723312	0.0011711358555198\\
6.87027256467878	0.00116034220637672\\
6.87953169212444	0.0011496480357538\\
6.8887908195701	0.00113905242681786\\
6.89804994701576	0.00112855447118557\\
6.90730907446141	0.00111815326884562\\
6.91656820190707	0.00110784792808159\\
6.92582732935273	0.00109763756539547\\
6.93508645679839	0.00108752130543187\\
6.94434558424405	0.00107749828090308\\
6.95360471168971	0.00106756763251462\\
6.96286383913537	0.00105772850889159\\
6.97212296658102	0.00104798006650571\\
6.98138209402668	0.00103832146960302\\
6.99064122147234	0.00102875189013217\\
6.999900348918	0.00101927050767342\\
7.00915947636366	0.0010098765093684\\
7.01841860380932	0.00100056908985035\\
7.02767773125498	0.000991347451175107\\
7.03693685870064	0.000982210802752643\\
7.04619598614629	0.000973158361279302\\
7.05545511359195	0.000964189350670676\\
7.06471424103761	0.000955303001995117\\
7.07397336848327	0.000946498553407639\\
7.08323249592893	0.000937775250084851\\
7.09249162337459	0.000929132344160035\\
7.10175075082025	0.000920569094659085\\
7.11100987826591	0.000912084767437048\\
7.12026900571156	0.000903678635115125\\
7.12952813315722	0.000895349977018343\\
7.13878726060288	0.000887098079113757\\
7.14804638804854	0.000878922233949207\\
7.1573055154942	0.000870821740592657\\
7.16656464293986	0.00086279590457222\\
7.17582377038552	0.000854844037816462\\
7.18508289783118	0.000846965458595534\\
7.19434202527683	0.000839159491462652\\
7.20360115272249	0.000831425467196197\\
7.21286028016815	0.000823762722742414\\
7.22211940761381	0\\
};
\addlegendentry{$F_y$ improper};

\addplot [color=mycolor2,solid]
  table[row sep=crcr]{%
0	0\\
0.0088768419129783	0.0868477151767971\\
0.0177536838259566	0.16645258330425\\
0.0266305257389349	0.238324869493185\\
0.0355073676519132	0.303109685078454\\
0.0443842095648915	0.361436303258344\\
0.0532610514778698	0.413879646044459\\
0.0621378933908481	0.460964015724568\\
0.0710147353038264	0.503167523633639\\
0.0798915772168047	0.540926154357298\\
0.088768419129783	0.574637472928272\\
0.0976452610427613	0.604664005839607\\
0.10652210295574	0.631336324561213\\
0.115398944868718	0.654955857735919\\
0.124275786781696	0.67579745592834\\
0.133152628694675	0.694111730699262\\
0.142029470607653	0.710127187862484\\
0.150906312520631	0.72405217303388\\
0.159783154433609	0.736076645988965\\
0.168659996346588	0.746373798891994\\
0.177536838259566	0.755101532134248\\
0.186413680172544	0.762403800310399\\
0.195290522085523	0.768411839759417\\
0.204167363998501	0.773245288091101\\
0.213044205911479	0.777013205202372\\
0.221921047824458	0.779815004451178\\
0.230797889737436	0.781741301893218\\
0.239674731650414	0.782874690791102\\
0.248551573563392	0.783290447971189\\
0.257428415476371	0.783057178024812\\
0.266305257389349	0.78223740082294\\
0.275182099302327	0.780888087332119\\
0.284058941215306	0.77906114828064\\
0.292935783128284	0.776803879823656\\
0.301812625041262	0.77415936999089\\
0.310689466954241	0.771166869367679\\
0.319566308867219	0.767862129156471\\
0.328443150780197	0.764277709488967\\
0.337319992693175	0.760443260606563\\
0.346196834606154	0.756385779296416\\
0.355073676519132	0.752129842760408\\
0.36395051843211	0.747697821902699\\
0.372827360345089	0.743110075846829\\
0.381704202258067	0.738385129334017\\
0.390581044171045	0.733539834508932\\
0.399457886084024	0.728589518466723\\
0.408334727997002	0.723548117814171\\
0.41721156990998	0.718428301387623\\
0.426088411822959	0.713241582169812\\
0.434965253735937	0.707998419355961\\
0.443842095648915	0.702708311435969\\
0.452718937561893	0.697379881083187\\
0.461595779474872	0.692020952570745\\
0.47047262138785	0.686638622372957\\
0.479349463300828	0.681239323551463\\
0.488226305213807	0.675828884473017\\
0.497103147126785	0.670412582357701\\
0.505979989039763	0.664995192112456\\
0.514856830952741	0.659581030864798\\
0.52373367286572	0.654173998575088\\
0.532610514778698	0.64877761507242\\
0.541487356691676	0.643395053828838\\
0.550364198604655	0.6380291727589\\
0.559241040517633	0.632682542306353\\
0.568117882430611	0.62735747105665\\
0.57699472434359	0.622056029093024\\
0.585871566256568	0.616780069294702\\
0.594748408169546	0.611531246758337\\
0.603625250082525	0.606311036507831\\
0.612502091995503	0.60112074964317\\
0.621378933908481	0.595961548065648\\
0.630255775821459	0.59083445790476\\
0.639132617734438	0.585740381761037\\
0.648009459647416	0.580680109869021\\
0.656886301560394	0.575654330275427\\
0.665763143473373	0.570663638119158\\
0.674639985386351	0.565708544092234\\
0.683516827299329	0.560789482153722\\
0.692393669212308	0.555906816562412\\
0.701270511125286	0.551060848288225\\
0.710147353038264	0.546251820857005\\
0.719024194951242	0.541479925678611\\
0.727901036864221	0.536745306903765\\
0.736777878777199	0.532048065851161\\
0.745654720690177	0.52738826504265\\
0.754531562603156	0.522765931881032\\
0.763408404516134	0.518181062001898\\
0.772285246429112	0.513633622328235\\
0.781162088342091	0.509123553853966\\
0.790038930255069	0.504650774180301\\
0.798915772168047	0.50021517982666\\
0.807792614081026	0.495816648336023\\
0.816669455994004	0.491455040192828\\
0.825546297906982	0.487130200569918\\
0.83442313981996	0.482841960919597\\
0.843299981732939	0.478590140422543\\
0.852176823645917	0.474374547307095\\
0.861053665558895	0.470194980050334\\
0.869930507471874	0.466051228471392\\
0.878807349384852	0.461943074726481\\
0.88768419129783	0.457870294214301\\
0.896561033210808	0.453832656399751\\
0.905437875123787	0.449829925563128\\
0.914314717036765	0.445861861481403\\
0.923191558949743	0.44192822004756\\
0.932068400862722	0.438028753833471\\
0.9409452427757	0.434163212601286\\
0.949822084688678	0.430331343767893\\
0.958698926601657	0.42653289282659\\
0.967575768514635	0.42276760372975\\
0.976452610427613	0.41903521923593\\
0.985329452340592	0.415335481224571\\
0.99420629425357	0.411668130981151\\
1.00308313616655	0.408032909455415\\
1.01195997807953	0.404429557495062\\
1.0208368199925	0.400857816057065\\
1.02971366190548	0.397317426398609\\
1.03859050381846	0.39380813024946\\
1.04746734573144	0.390329669967401\\
1.05634418764442	0.386881788678264\\
1.0652210295574	0.383464230401903\\
1.07409787147037	0.380076740165375\\
1.08297471338335	0.376719064104473\\
1.09185155529633	0.373390949554639\\
1.10072839720931	0.370092145132209\\
1.10960523912229	0.366822400806867\\
1.11848208103527	0.363581467966081\\
1.12735892294824	0.360369099472252\\
1.13623576486122	0.357185049713221\\
1.1451126067742	0.354029074646744\\
1.15398944868718	0.350900931839472\\
1.16286629060016	0.347800380500938\\
1.17174313251314	0.344727181512999\\
1.18061997442611	0.341681097455153\\
1.18949681633909	0.338661892626098\\
1.19837365825207	0.335669333061896\\
1.20725050016505	0.332703186551025\\
1.21612734207803	0.329763222646635\\
1.22500418399101	0.32684921267625\\
1.23388102590398	0.323960929749161\\
1.24275786781696	0.321098148761715\\
1.25163470972994	0.318260646400719\\
1.26051155164292	0.315448201145116\\
1.2693883935559	0.312660593266107\\
1.27826523546888	0.309897604825871\\
1.28714207738185	0.307159019675017\\
1.29601891929483	0.304444623448879\\
1.30489576120781	0.301754203562794\\
1.31377260312079	0.299087549206438\\
1.32264944503377	0.296444451337331\\
1.33152628694675	0.293824702673593\\
1.34040312885972	0.291228097686022\\
1.3492799707727	0.288654432589571\\
1.35815681268568	0.286103505334293\\
1.36703365459866	0.283575115595797\\
1.37591049651164	0.281069064765288\\
1.38478733842462	0.278585155939224\\
1.39366418033759	0.276123193908641\\
1.40254102225057	0.27368298514819\\
1.41141786416355	0.27126433780491\\
1.42029470607653	0.268867061686787\\
1.42917154798951	0.266490968251113\\
1.43804838990248	0.264135870592689\\
1.44692523181546	0.261801583431877\\
1.45580207372844	0.259487923102548\\
1.46467891564142	0.257194707539921\\
1.4735557575544	0.254921756268334\\
1.48243259946738	0.252668890388945\\
1.49130944138035	0.250435932567402\\
1.50018628329333	0.248222707021462\\
1.50906312520631	0.246029039508609\\
1.51793996711929	0.243854757313656\\
1.52681680903227	0.241699689236349\\
1.53569365094525	0.239563665578988\\
1.54457049285822	0.237446518134067\\
1.5534473347712	0.235348080171938\\
1.56232417668418	0.233268186428522\\
1.57120101859716	0.231206673093044\\
1.58007786051014	0.229163377795835\\
1.58895470242312	0.227138139596167\\
1.59783154433609	0.225130798970158\\
1.60670838624907	0.223141197798731\\
1.61558522816205	0.221169179355638\\
1.62446207007503	0.219214588295556\\
1.63333891198801	0.217277270642242\\
1.64221575390099	0.215357073776772\\
1.65109259581396	0.213453846425847\\
1.65996943772694	0.211567438650181\\
1.66884627963992	0.209697701832963\\
1.6777231215529	0.207844488668399\\
1.68659996346588	0.20600765315034\\
1.69547680537886	0.204187050560983\\
1.70435364729183	0.202382537459667\\
1.71323048920481	0.200593971671737\\
1.72210733111779	0.198821212277509\\
1.73098417303077	0.19706411960131\\
1.73986101494375	0.195322555200601\\
1.74873785685673	0.193596381855194\\
1.7576146987697	0.191885463556549\\
1.76649154068268	0.190189665497157\\
1.77536838259566	0.18850885406001\\
1.78424522450864	0.186842896808156\\
1.79312206642162	0.18519166247434\\
1.8019989083346	0.183555020950733\\
1.81087575024757	0.181932843278738\\
1.81975259216055	0.18032500163889\\
1.82862943407353	0.178731369340837\\
1.83750627598651	0.177151820813404\\
1.84638311789949	0.175586231594745\\
1.85525995981247	0.174034478322572\\
1.86413680172544	0.172496438724473\\
1.87301364363842	0.170971991608311\\
1.8818904855514	0.169461016852704\\
1.89076732746438	0.167963395397588\\
1.89964416937736	0.166479009234859\\
1.90852101129034	0.165007741399101\\
1.91739785320331	0.163549475958384\\
1.92627469511629	0.162104098005155\\
1.93515153702927	0.160671493647197\\
1.94402837894225	0.159251549998671\\
1.95290522085523	0.157844155171235\\
1.9617820627682	0.156449198265244\\
1.97065890468118	0.155066569361018\\
1.97953574659416	0.153696159510194\\
1.98841258850714	0.152337860727155\\
1.99728943042012	0.150991565980521\\
2.0061662723331	0.149657169184734\\
2.01504311424607	0.148334565191699\\
2.02391995615905	0.147023649782507\\
2.03279679807203	0.145724319659231\\
2.04167363998501	0.144436472436789\\
2.05055048189799	0.143160006634886\\
2.05942732381097	0.141894821670017\\
2.06830416572394	0.140640817847549\\
2.07718100763692	0.139397896353869\\
2.0860578495499	0.138165959248599\\
2.09493469146288	0.136944909456886\\
2.10381153337586	0.135734650761749\\
2.11268837528884	0.134535087796508\\
2.12156521720181	0.133346126037266\\
2.13044205911479	0.132167671795464\\
2.13931890102777	0.130999632210503\\
2.14819574294075	0.129841915242425\\
2.15707258485373	0.128694429664666\\
2.16594942676671	0.127557085056865\\
2.17482626867968	0.126429791797744\\
2.18370311059266	0.125312461058045\\
2.19257995250564	0.124205004793534\\
2.20145679441862	0.12310733573806\\
2.2103336363316	0.122019367396684\\
2.21921047824458	0.120941014038863\\
2.22808732015755	0.119872190691693\\
2.23696416207053	0.118812813133217\\
2.24584100398351	0.117762797885787\\
2.25471784589649	0.116722062209489\\
2.26359468780947	0.115690524095623\\
2.27247152972245	0.114668102260237\\
2.28134837163542	0.11365471613773\\
2.2902252135484	0.112650285874499\\
2.29910205546138	0.111654732322647\\
2.30797889737436	0.110667977033749\\
2.31685573928734	0.109689942252669\\
2.32573258120032	0.108720550911434\\
2.33460942311329	0.107759726623161\\
2.34348626502627	0.106807393676038\\
2.35236310693925	0.105863477027359\\
2.36123994885223	0.104927902297609\\
2.37011679076521	0.104000595764606\\
2.37899363267818	0.103081484357691\\
2.38787047459116	0.102170495651967\\
2.39674731650414	0.101267557862597\\
2.40562415841712	0.100372599839146\\
2.4145010003301	0.0994855510599722\\
2.42337784224308	0.0986063416266737\\
2.43225468415605	0.0977349022585776\\
2.44113152606903	0.096871164287283\\
2.45000836798201	0.0960150596512486\\
2.45888520989499	0.0951665208904308\\
2.46776205180797	0.0943254811409672\\
2.47663889372095	0.0934918741299081\\
2.48551573563392	0.0926656341699946\\
2.4943925775469	0.0918466961544822\\
2.50326941945988	0.0910349955520104\\
2.51214626137286	0.0902304684015185\\
2.52102310328584	0.0894330513072046\\
2.52989994519882	0.0886426814335308\\
2.53877678711179	0.0878592965002717\\
2.54765362902477	0.0870828347776068\\
2.55653047093775	0.0863132350812562\\
2.56540731285073	0.0855504367676595\\
2.57428415476371	0.0847943797291968\\
2.58316099667669	0.0840450043894533\\
2.59203783858966	0.0833022516985236\\
2.60091468050264	0.0825660631283592\\
2.60979152241562	0.0818363806681568\\
2.6186683643286	0.0811131468197866\\
2.62754520624158	0.0803963045932619\\
2.63642204815456	0.0796857975022482\\
2.64529889006753	0.0789815695596124\\
2.65417573198051	0.0782835652730106\\
2.66305257389349	0.0775917296405159\\
2.67192941580647	0.0769060081462842\\
2.68080625771945	0.0762263467562579\\
2.68968309963243	0.0755526919139091\\
2.6985599415454	0.0748849905360182\\
2.70743678345838	0.074223190008492\\
2.71631362537136	0.0735672381822174\\
2.72519046728434	0.0729170833689521\\
2.73406730919732	0.0722726743372515\\
2.7429441511103	0.0716339603084321\\
2.75182099302327	0.0710008909525697\\
2.76069783493625	0.0703734163845341\\
2.76957467684923	0.0697514871600576\\
2.77845151876221	0.0691350542718391\\
2.78732836067519	0.0685240691456827\\
2.79620520258817	0.0679184836366694\\
2.80508204450114	0.0673182500253639\\
2.81395888641412	0.0667233210140544\\
2.8228357283271	0.0661336497230251\\
2.83171257024008	0.0655491896868627\\
2.84058941215306	0.0649698948507945\\
2.84946625406603	0.0643957195670598\\
2.85834309597901	0.0638266185913124\\
2.86721993789199	0.0632625470790559\\
2.87609677980497	0.0627034605821098\\
2.88497362171795	0.0621493150451066\\
2.89385046363093	0.0616000668020213\\
2.9027273055439	0.0610556725727295\\
2.91160414745688	0.0605160894595977\\
2.92048098936986	0.0599812749441027\\
2.92935783128284	0.0594511868834811\\
2.93823467319582	0.0589257835074089\\
2.9471115151088	0.0584050234147098\\
2.95598835702177	0.0578888655700926\\
2.96486519893475	0.0573772693009182\\
2.97374204084773	0.056870194293994\\
2.98261888276071	0.0563676005923979\\
2.99149572467369	0.0558694485923294\\
3.00037256658667	0.0553756990399888\\
3.00924940849964	0.0548863130284844\\
3.01812625041262	0.0544012519947661\\
3.0270030923256	0.0539204777165875\\
3.03587993423858	0.0534439523094931\\
3.04475677615156	0.0529716382238338\\
3.05363361806454	0.0525034982418076\\
3.06251045997751	0.0520394954745271\\
3.07138730189049	0.0515795933591127\\
3.08026414380347	0.0511237556558112\\
3.08914098571645	0.0506719464451404\\
3.09801782762943	0.0502241301250589\\
3.10689466954241	0.0497802714081601\\
3.11577151145538	0.0493403353188922\\
3.12464835336836	0.0489042871908017\\
3.13352519528134	0.048472092663802\\
3.14240203719432	0.0480437176814659\\
3.1512788791073	0.0476191284883418\\
3.16015572102028	0.0471982916272939\\
3.16903256293325	0.0467811739368657\\
3.17790940484623	0.0463677425486674\\
3.18678624675921	0.045957964884785\\
3.19566308867219	0.0455518086552142\\
3.20453993058517	0.0451492418553154\\
3.21341677249815	0.0447502327632917\\
3.22229361441112	0.0443547499376896\\
3.2311704563241	0.0439627622149213\\
3.24004729823708	0.0435742387068091\\
3.24892414015006	0.0431891487981512\\
3.25780098206304	0.0428074621443098\\
3.26667782397602	0.0424291486688192\\
3.27555466588899	0.0420541785610167\\
3.28443150780197	0.0416825222736927\\
3.29330834971495	0.0413141505207629\\
3.30218519162793	0.0409490342749607\\
3.31106203354091	0.0405871447655495\\
3.31993887545388	0.0402284534760558\\
3.32881571736686	0.0398729321420221\\
3.33769255927984	0.0395205527487799\\
3.34656940119282	0.0391712875292418\\
3.3554462431058	0.0388251089617138\\
3.36432308501878	0.0384819897677262\\
3.37319992693175	0.0381419029098848\\
3.38207676884473	0.0378048215897397\\
3.39095361075771	0.037470719245674\\
3.39983045267069	0.0371395695508105\\
3.40870729458367	0.0368113464109377\\
3.41758413649665	0.0364860239624528\\
3.42646097840962	0.0361635765703243\\
3.4353378203226	0.0358439788260718\\
3.44421466223558	0.035527205545764\\
3.45309150414856	0.0352132317680339\\
3.46196834606154	0.034902032752112\\
3.47084518797452	0.0345935839758769\\
3.47972202988749	0.034287861133923\\
3.48859887180047	0.033984840135645\\
3.49747571371345	0.0336844971033398\\
3.50635255562643	0.033386808370325\\
3.51522939753941	0.0330917504790737\\
3.52410623945239	0.0327993001793665\\
3.53298308136536	0.032509434426459\\
3.54185992327834	0.0322221303792663\\
3.55073676519132	0.0319373653985627\\
3.5596136071043	0.0316551170451982\\
3.56849044901728	0.03137536307833\\
3.57736729093026	0.03109808145367\\
3.58624413284323	0.030823250321748\\
3.59512097475621	0.0305508480261896\\
3.60399781666919	0.03028085310201\\
3.61287465858217	0.0300132442739224\\
3.62175150049515	0.0297480004546616\\
3.63062834240813	0.0294851007433224\\
3.6395051843211	0.0292245244237126\\
3.64838202623408	0.0289662509627206\\
3.65725886814706	0.0287102600086972\\
3.66613571006004	0.0284565313898523\\
3.67501255197302	0.0282050451126651\\
3.683889393886	0.0279557813603088\\
3.69276623579897	0.0277087204910888\\
3.70164307771195	0.0274638430368953\\
3.71051991962493	0.0272211297016691\\
3.71939676153791	0.0269805613598808\\
3.72827360345089	0.0267421190550243\\
3.73715044536387	0.0265057839981227\\
3.74602728727684	0.0262715375662476\\
3.75490412918982	0.0260393613010519\\
3.7637809711028	0.0258092369073154\\
3.77265781301578	0.0255811462515028\\
3.78153465492876	0.0253550713603351\\
3.79041149684174	0.0251309944193731\\
3.79928833875471	0.024908897771614\\
3.80816518066769	0.0246887639160994\\
3.81704202258067	0.0244705755065371\\
3.82591886449365	0.0242543153499334\\
3.83479570640663	0.024039966405239\\
3.84367254831961	0.0238275117820057\\
3.85254939023258	0.0236169347390556\\
3.86142623214556	0.0234082186831622\\
3.87030307405854	0.0232013471677423\\
3.87917991597152	0.0229963038915606\\
3.8880567578845	0.0227930726974448\\
3.89693359979747	0.0225916375710126\\
3.90581044171045	0.0223919826394098\\
3.91468728362343	0.0221940921700592\\
3.92356412553641	0.0219979505694217\\
3.93244096744939	0.0218035423817666\\
3.94131780936237	0.0216108522879543\\
3.95019465127534	0.021419865104229\\
3.95907149318832	0.0212305657810221\\
3.9679483351013	0.0210429394017667\\
3.97682517701428	0.0208569711817217\\
3.98570201892726	0.0206726464668071\\
3.99457886084024	0.0204899507324494\\
4.00345570275321	0.0203088695824365\\
4.01233254466619	0.020129388747784\\
4.02120938657917	0.0199514940856102\\
4.03008622849215	0.0197751715780224\\
4.03896307040513	0.0196004073310113\\
4.04783991231811	0.0194271875733571\\
4.05671675423108	0.019255498655544\\
4.06559359614406	0.0190853270486843\\
4.07447043805704	0.0189166593434528\\
4.08334727997002	0.0187494822490302\\
4.092224121883	0.0185837825920552\\
4.10110096379598	0.0184195473155873\\
4.10997780570895	0.018256763478077\\
4.11885464762193	0.018095418252347\\
4.12773148953491	0.0179354989245806\\
4.13660833144789	0.0177769928933204\\
4.14548517336087	0.017619887668475\\
4.15436201527385	0.0174641708703349\\
4.16323885718682	0.0173098302285971\\
4.1721156990998	0.0171568535813981\\
4.18099254101278	0.0170052288743557\\
4.18986938292576	0.0168549441596187\\
4.19874622483874	0.0167059875949261\\
4.20762306675172	0.0165583474426733\\
4.21649990866469	0.0164120120689878\\
4.22537675057767	0.0162669699428117\\
4.23425359249065	0.0161232096349938\\
4.24313043440363	0.0159807198173887\\
4.25200727631661	0.0158394892619642\\
4.26088411822959	0.0156995068399161\\
4.26976096014256	0.0155607615207922\\
4.27863780205554	0.0154232423716222\\
4.28751464396852	0.0152869385560568\\
4.2963914858815	0.0151518393335133\\
4.30526832779448	0.0150179340583299\\
4.31414516970746	0.0148852121789262\\
4.32302201162043	0.0147536632369721\\
4.33189885353341	0.0146232768665636\\
4.34077569544639	0.0144940427934059\\
4.34965253735937	0.0143659508340041\\
4.35852937927235	0.0142389908948604\\
4.36740622118533	0.0141131529716788\\
4.3762830630983	0.0139884271485771\\
4.38515990501128	0.013864803597305\\
4.39403674692426	0.0137422725764701\\
4.40291358883724	0.0136208244307702\\
4.41179043075022	0.0135004495902324\\
4.42066727266319	0.0133811385694588\\
4.42954411457617	0.0132628819668793\\
4.43842095648915	0.013145670464011\\
4.44729779840213	0.0130294948247234\\
4.45617464031511	0.0129143458945109\\
4.46505148222809	0.0128002145997714\\
4.47392832414106	0.0126870919470914\\
4.48280516605404	0.0125749690225369\\
4.49168200796702	0.0124638369909517\\
4.50055884988	0.0123536870952606\\
4.50943569179298	0.0122445106557797\\
4.51831253370596	0.012136299069532\\
4.52718937561893	0.0120290438095703\\
4.53606621753191	0.0119227364243045\\
4.54494305944489	0.0118173685368359\\
4.55381990135787	0.0117129318442973\\
4.56269674327085	0.0116094181171984\\
4.57157358518383	0.0115068191987777\\
4.5804504270968	0.0114051270043595\\
4.58932726900978	0.0113043335207167\\
4.59820411092276	0.0112044308054399\\
4.60708095283574	0.0111054109863109\\
4.61595779474872	0.0110072662606829\\
4.6248346366617	0.0109099888948654\\
4.63371147857467	0.0108135712235149\\
4.64258832048765	0.0107180056490307\\
4.65146516240063	0.0106232846409564\\
4.66034200431361	0.0105294007353864\\
4.66921884622659	0.010436346534378\\
4.67809568813957	0.0103441147053681\\
4.68697253005254	0.0102526979805954\\
4.69584937196552	0.0101620891565283\\
4.7047262138785	0.0100722810932965\\
4.71360305579148	0.00998326671412871\\
4.72247989770446	0.00989503900479526\\
4.73135673961743	0.00980759101305495\\
4.74023358153041	0.00972091584810753\\
4.74911042344339	0.00963500668005055\\
4.75798726535637	0.00954985673934122\\
4.76686410726935	0.009465459316263\\
4.77574094918233	0.00938180776039691\\
4.7846177910953	0.00929889548009739\\
4.79349463300828	0.00921671594197294\\
4.80237147492126	0.0091352626703713\\
4.81124831683424	0.00905452924686916\\
4.82012515874722	0.00897450930976645\\
4.8290020006602	0.00889519655358495\\
4.83787884257317	0.00881658472857151\\
4.84675568448615	0.00873866764020548\\
4.85563252639913	0.00866143914871074\\
4.86450936831211	0.00858489316857174\\
4.87338621022509	0.00850902366805408\\
4.88226305213807	0.00843382466872911\\
4.89113989405104	0.00835929024500296\\
4.90001673596402	0.00828541452364948\\
4.908893577877	0.00821219168334757\\
4.91777041978998	0.00813961595422236\\
4.92664726170296	0.00806768161739064\\
4.93552410361594	0.00799638300451006\\
4.94440094552891	0.00792571449733269\\
4.95327778744189	0.00785567052726212\\
4.96215462935487	0.00778624557491481\\
4.97103147126785	0.00771743416968503\\
4.97990831318083	0.00764923088931394\\
4.98878515509381	0.00758163035946212\\
4.99766199700678	0.00751462725328629\\
5.00653883891976	0.00744821629101941\\
5.01541568083274	0.00738239223955468\\
5.02429252274572	0.00731714991203327\\
5.0331693646587	0.00725248416743542\\
5.04204620657168	0.00718838991017547\\
5.05092304848465	0.00712486208970033\\
5.05979989039763	0.00706189570009148\\
5.06867673231061	0.00699948577967042\\
5.07755357422359	0.00693762741060787\\
5.08643041613657	0.00687631571853611\\
5.09530725804955	0.00681554587216492\\
5.10418409996252	0.00675531308290096\\
5.1130609418755	0.00669561260447028\\
5.12193778378848	0.00663643973254446\\
5.13081462570146	0.00657778980436982\\
5.13969146761444	0.0065196581984001\\
5.14856830952742	0.00646204033393217\\
5.15744515144039	0.00640493167074521\\
5.16632199335337	0.00634832770874277\\
5.17519883526635	0.00629222398759834\\
5.18407567717933	0.00623661608640375\\
5.19295251909231	0.00618149962332094\\
5.20182936100529	0.00612687025523654\\
5.21070620291826	0.00607272367741973\\
5.21958304483124	0.00601905562318299\\
5.22845988674422	0.00596586186354594\\
5.2373367286572	0.00591313820690205\\
5.24621357057018	0.00586088049868832\\
5.25509041248316	0.00580908462105797\\
5.26396725439613	0.00575774649255593\\
5.27284409630911	0.00570686206779726\\
5.28172093822209	0.00565642733714832\\
5.29059778013507	0.00560643832641086\\
5.29947462204805	0.00555689109650881\\
5.30835146396102	0.00550778174317797\\
5.317228305874	0.00545910639665827\\
5.32610514778698	0.00541086122138887\\
5.33498198969996	0.0053630424157059\\
5.34385883161294	0.00531564621154298\\
5.35273567352592	0.00526866887413423\\
5.36161251543889	0.00522210670171995\\
5.37048935735187	0.00517595602525496\\
5.37936619926485	0.00513021320811953\\
5.38824304117783	0.00508487464583272\\
5.39711988309081	0.00503993676576843\\
5.40599672500379	0.00499539602687384\\
5.41487356691676	0.00495124891939042\\
5.42375040882974	0.0049074919645773\\
5.43262725074272	0.00486412171443725\\
5.4415040926557	0.00482113475144486\\
5.45038093456868	0.00477852768827732\\
5.45925777648166	0.00473629716754746\\
5.46813461839463	0.00469443986153927\\
5.47701146030761	0.00465295247194556\\
5.48588830222059	0.00461183172960821\\
5.49476514413357	0.00457107439426043\\
5.50364198604655	0.00453067725427147\\
5.51251882795953	0.00449063712639354\\
5.5213956698725	0.00445095085551104\\
5.53027251178548	0.00441161531439187\\
5.53914935369846	0.00437262740344103\\
5.54802619561144	0.00433398405045637\\
5.55690303752442	0.00429568221038651\\
5.5657798794374	0.00425771886509088\\
5.57465672135037	0.00422009102310194\\
5.58353356326335	0.00418279571938942\\
5.59241040517633	0.0041458300151266\\
5.60128724708931	0.00410919099745892\\
5.61016408900229	0.00407287577927428\\
5.61904093091526	0.0040368814989756\\
5.62791777282824	0.00400120532025539\\
5.63679461474122	0.00396584443187215\\
5.6456714566542	0.00393079604742887\\
5.65454829856718	0.00389605740515356\\
5.66342514048016	0.00386162576768148\\
5.67230198239313	0.00382749842183957\\
5.68117882430611	0.00379367267843258\\
5.69005566621909	0.00376014587203112\\
5.69893250813207	0.0037269153607618\\
5.70780935004505	0.00369397852609885\\
5.71668619195803	0.00366133277265794\\
5.725563033871	0.00362897552799155\\
5.73443987578398	0.00359690424238637\\
5.74331671769696	0.0035651163886623\\
5.75219355960994	0.00353360946197334\\
5.76107040152292	0.0035023809796102\\
5.7699472434359	0.00347142848080473\\
5.77882408534887	0.00344074952653592\\
5.78770092726185	0.00341034169933777\\
5.79657776917483	0.00338020260310882\\
5.80545461108781	0.00335032986292318\\
5.81433145300079	0.00332072112484363\\
5.82320829491377	0.00329137405573596\\
5.83208513682674	0.00326228634308516\\
5.84096197873972	0.00323345569481324\\
5.8498388206527	0.00320487983909856\\
5.85871566256568	0.00317655652419681\\
5.86759250447866	0.00314848351826367\\
5.87646934639164	0.00312065860917881\\
5.88534618830461	0.00309307960437174\\
5.89422303021759	0.00306574433064885\\
5.90309987213057	0.00303865063402227\\
5.91197671404355	0.00301179637954015\\
5.92085355595653	0.00298517945111839\\
5.92973039786951	0.00295879775137392\\
5.93860723978248	0.00293264920145939\\
5.94748408169546	0.00290673174089943\\
5.95636092360844	0.00288104332742819\\
5.96523776552142	0.00285558193682851\\
5.9741146074344	0.00283034556277237\\
5.98299144934738	0.00280533221666276\\
5.99186829126035	0.002780539927477\\
6.00074513317333	0.00275596674161149\\
6.00962197508631	0.00273161072272771\\
6.01849881699929	0.00270746995159957\\
6.02737565891227	0.00268354252596237\\
6.03625250082525	0.00265982656036271\\
6.04512934273822	0.00263632018601004\\
6.0540061846512	0.0026130215506293\\
6.06288302656418	0.00258992881831508\\
6.07175986847716	0.00256704016938685\\
6.08063671039014	0.00254435380024564\\
6.08951355230312	0.0025218679232319\\
6.09839039421609	0.00249958076648459\\
6.10726723612907	0.0024774905738016\\
6.11614407804205	0.00245559560450137\\
6.12502091995503	0.00243389413328578\\
6.13389776186801	0.00241238445010404\\
6.14277460378099	0.00239106486001807\\
6.15165144569396	0.00236993368306891\\
6.16052828760694	0.00234898925414432\\
6.16940512951992	0.00232822992284762\\
6.1782819714329	0.00230765405336757\\
6.18715881334588	0.00228726002434953\\
6.19603565525886	0.0022670462287677\\
6.20491249717183	0.00224701107379842\\
6.21378933908481	0.00222715298069475\\
6.22266618099779	0.00220747038466198\\
6.23154302291077	0.00218796173473437\\
6.24041986482375	0.00216862549365295\\
6.24929670673672	0.00214946013774434\\
6.2581735486497	0.00213046415680075\\
6.26705039056268	0.00211163605396091\\
6.27592723247566	0.00209297434559216\\
6.28480407438864	0.00207447756117356\\
6.29368091630162	0.00205614424317992\\
6.30255775821459	0.00203797294696707\\
6.31143460012757	0.00201996224065794\\
6.32031144204055	0.00200211070502977\\
6.32918828395353	0.00198441693340227\\
6.33806512586651	0.00196687953152677\\
6.34694196777949	0.00194949711747635\\
6.35581880969246	0.00193226832153694\\
6.36469565160544	0.00191519178609942\\
6.37357249351842	0.00189826616555258\\
6.3824493354314	0.00188149012617718\\
6.39132617734438	0.00186486234604078\\
6.40020301925736	0.00184838151489358\\
6.40907986117033	0.00183204633406523\\
6.41795670308331	0.00181585551636243\\
6.42683354499629	0.00179980778596754\\
6.43571038690927	0.00178390187833802\\
6.44458722882225	0.0017681365401068\\
6.45346407073523	0.00175251052898349\\
6.4623409126482	0.00173702261365659\\
6.47121775456118	0.00172167157369632\\
6.48009459647416	0.00170645619945855\\
6.48897143838714	0.00169137529198939\\
6.49784828030012	0.0016764276629309\\
6.5067251222131	0.00166161213442721\\
6.51560196412607	0.00164692753903189\\
6.52447880603905	0.00163237271961585\\
6.53335564795203	0.00161794652927621\\
6.54223248986501	0.00160364783124592\\
6.55110933177799	0.00158947549880413\\
6.55998617369097	0.00157542841518751\\
6.56886301560394	0.00156150547350214\\
6.57773985751692	0.00154770557663641\\
6.5866166994299	0.00153402763717444\\
6.59549354134288	0.00152047057731048\\
6.60437038325586	0.00150703332876392\\
6.61324722516884	0.00149371483269516\\
6.62212406708181	0.00148051403962217\\
6.63100090899479	0.0014674299093378\\
6.63987775090777	0.00145446141082775\\
6.64875459282075	0.00144160752218937\\
6.65763143473373	0.00142886723055115\\
6.6665082766467	0.00141623953199288\\
6.67538511855968	0.00140372343146658\\
6.68426196047266	0.00139131794271806\\
6.69313880238564	0.00137902208820918\\
6.70201564429862	0.00136683489904089\\
6.7108924862116	0.00135475541487684\\
6.71976932812457	0.00134278268386762\\
6.72864617003755	0.00133091576257598\\
6.73752301195053	0.00131915371590228\\
6.74639985386351	0.00130749561701088\\
6.75527669577649	0.00129594054725711\\
6.76415353768947	0.0012844875961149\\
6.77303037960244	0.00127313586110496\\
6.78190722151542	0.00126188444772376\\
6.7907840634284	0.00125073246937298\\
6.79966090534138	0.00123967904728967\\
6.80853774725436	0.00122872331047698\\
6.81741458916734	0.00121786439563557\\
6.82629143108031	0.00120710144709551\\
6.83516827299329	0.00119643361674901\\
6.84404511490627	0.00118586006398339\\
6.85292195681925	0.00117537995561495\\
6.86179879873223	0.00116499246582331\\
6.87067564064521	0.00115469677608634\\
6.87955248255818	0.00114449207511563\\
6.88842932447116	0.00113437755879257\\
6.89730616638414	0.00112435243010502\\
6.90618300829712	0.00111441589908446\\
6.9150598502101	0.0011045671827438\\
6.92393669212308	0\\
};
\addlegendentry{$F_y$ proper};

\end{axis}
\end{tikzpicture}%
}
  \caption{The response of the transfer function from $d$ to $y$ for a step in
    the disturbance when $F_y$ is proper (\texttt{red}) and when $F_y$ is
    (\texttt{blue}). $p_1 = p_2 = 50\omega_c$}
  \label{fig:step_response_2.1}
\end{figure}


%%%%%%%%%%%%%%%%%%%%%%%%%%%%%%%%%%%%%%%%%%%%%%%%%%%%%%%%%%%%%%%%%%%%%%%%%%%%%%%%
\subsection{Exercice 2}

Since the previous controller did not exhibit adequate performance and did not
meet the specified requirements, we now attempt to tune it in accordance to a
desire to include integral action. We will again consider two possible designs.

\subsubsection{$F_y$ is improper}

We take $F_y$ to be of the form $F_y(s) = \frac{s+\omega_I}{s}G^{-1}(s)G_d(s)$
as a starting point. Although $F_y$ is now improper, we can design $\omega_I$ so
that our third requirement is met. Indeed, if we choose
$\omega_I = 0.5 \omega_c$, then $|y(t)| \leq 0.1, \forall t \geq 0.5$s and,
$\forall t: |y(t)| < 1$. This behaviour is illustrated in figure
\ref{fig:step_response_2.2} in blue colour.


\subsubsection{$F_y$ is proper}

Since $F_y$ needs to be proper, we need to add two poles to it; hence $F_y$ is
now formed as
$$F_y(s) = \dfrac{s+\omega_I}{s} \cdot \dfrac{p_1 \cdot p_2}{(s+p_1)(s+p_2)} \cdot G^{-1}(s)G_d(s)$$
The location of these poles should be such that when $G_d \approx 1$, e.g.
for $\omega < \omega_c$, the controller should contain the inverse of the
system. This means that $p_1,p_2$ should be larger than $\omega_c$. A choice of
$p_1 = p_2 = 5\omega_c = 10\omega_I$ is appropriate since the poles are
adequately far from the crossover frequency of the disturbance and the third
requirement is satisfied, as can be seen in figure \ref{fig:step_response_2.2}
in red colour.


\begin{figure}[H]\centering
  \scalebox{0.8}{% This file was created by matlab2tikz.
%
%The latest updates can be retrieved from
%  http://www.mathworks.com/matlabcentral/fileexchange/22022-matlab2tikz-matlab2tikz
%where you can also make suggestions and rate matlab2tikz.
%
\definecolor{mycolor1}{rgb}{0.00000,0.44700,0.74100}%
\definecolor{mycolor2}{rgb}{0.85000,0.32500,0.09800}%
%
\begin{tikzpicture}

\begin{axis}[%
width=4.008in,
height=3.052in,
at={(0.818in,0.44in)},
scale only axis,
separate axis lines,
every outer x axis line/.append style={white!40!black},
every x tick label/.append style={font=\color{white!40!black}},
xmin=0,
xmax=1.4,
xlabel={Time [sec]},
every outer y axis line/.append style={white!40!black},
every y tick label/.append style={font=\color{white!40!black}},
ymin=-0.1,
ymax=0.8,
ylabel={Amplitude},
axis background/.style={fill=white},
legend style={legend cell align=left,align=left,draw=white!15!black}
]
\addplot [color=mycolor1,solid]
  table[row sep=crcr]{%
0	0\\
0.00837303670179619	0.0799390279218704\\
0.0167460734035924	0.152551463074496\\
0.0251191101053886	0.218215979461732\\
0.0334921468071848	0.277303382813636\\
0.041865183508981	0.330175944610135\\
0.0502382202107772	0.377186825965202\\
0.0586112569125734	0.418679585575187\\
0.0669842936143696	0.454987766136996\\
0.0753573303161657	0.486434553846608\\
0.0837303670179619	0.513332505794955\\
0.0921034037197581	0.535983340285496\\
0.100476440421554	0.554677785305162\\
0.108849477123351	0.569695480586864\\
0.117222513825147	0.58130492890683\\
0.125595550526943	0.589763492463073\\
0.133968587228739	0.595317430381573\\
0.142341623930535	0.598201973594012\\
0.150714660632331	0.598641433524476\\
0.159087697334128	0.596849341212173\\
0.167460734035924	0.593028613682513\\
0.17583377073772	0.587371744559539\\
0.184206807439516	0.580061016088524\\
0.192579844141312	0.571268729908187\\
0.200952880843109	0.561157454077438\\
0.209325917544905	0.549880284021544\\
0.217698954246701	0.537581115217075\\
0.226071990948497	0.524394925583888\\
0.234445027650293	0.510448065695589\\
0.24281806435209	0.495858555057471\\
0.251191101053886	0.480736382832756\\
0.259564137755682	0.46518381152415\\
0.267937174457478	0.44929568223825\\
0.276310211159274	0.433159720275335\\
0.284683247861071	0.416856839896475\\
0.293056284562867	0.400461447223982\\
0.301429321264663	0.384041740329871\\
0.309802357966459	0.36766000566051\\
0.318175394668255	0.35137291003401\\
0.326548431370052	0.335231787530265\\
0.334921468071848	0.319282920672155\\
0.343294504773644	0.303567815370211\\
0.35166754147544	0.288123469172349\\
0.360040578177236	0.272982632425129\\
0.368413614879033	0.25817406201358\\
0.376786651580829	0.243722767403133\\
0.385159688282625	0.229650248759704\\
0.393532724984421	0.215974726972718\\
0.401905761686217	0.202711365450922\\
0.410278798388014	0.18987248360243\\
0.41865183508981	0.177467761948708\\
0.427024871791606	0.165504438857259\\
0.435397908493402	0.153987498909823\\
0.443770945195198	0.142919852952061\\
0.452143981896995	0.132302509897133\\
0.460517018598791	0.122134740379387\\
0.468890055300587	0.112414232375794\\
0.477263092002383	0.103137238931821\\
0.485636128704179	0.0942987181453196\\
0.494009165405975	0.085892465576887\\
0.502382202107772	0.0779112392680324\\
0.510755238809568	0.0703468775596442\\
0.519128275511364	0.0631904099126631\\
0.52750131221316	0.0564321609407484\\
0.535874348914956	0.050061847871124\\
0.544247385616753	0.0440686716548446\\
0.552620422318549	0.0384414019515188\\
0.560993459020345	0.0331684562161615\\
0.569366495722141	0.0282379731174228\\
0.577739532423937	0.0236378805170303\\
0.586112569125734	0.0193559582399854\\
0.59448560582753	0.0153798958639359\\
0.602858642529326	0.0116973457542958\\
0.611231679231122	0.00829597156916087\\
0.619604715932918	0.0051634924549524\\
0.627977752634715	0.00228772315006275\\
0.636350789336511	-0.000343389790347613\\
0.644723826038307	-0.00274173543980107\\
0.653096862740103	-0.00491900941251005\\
0.661469899441899	-0.0068866880434735\\
0.669842936143695	-0.00865600564368704\\
0.678215972845492	-0.0102379346432066\\
0.686589009547288	-0.0116431684407041\\
0.694962046249084	-0.012882106784196\\
0.70333508295088	-0.0139648435137825\\
0.711708119652677	-0.0149011565034637\\
0.720081156354473	-0.0157004996453837\\
0.728454193056269	-0.0163719967261523\\
0.736827229758065	-0.0169244370511945\\
0.745200266459861	-0.0173662726793485\\
0.753573303161657	-0.0177056171361576\\
0.761946339863454	-0.0179502454804601\\
0.77031937656525	-0.0181075956049585\\
0.778692413267046	-0.0181847706574266\\
0.787065449968842	-0.0181885424750806\\
0.795438486670638	-0.0181253559303882\\
0.803811523372435	-0.0180013340921981\\
0.812184560074231	-0.0178222841115455\\
0.820557596776027	-0.0175937037468098\\
0.828930633477823	-0.0173207884480658\\
0.837303670179619	-0.0170084389254751\\
0.845676706881416	-0.0166612691314034\\
0.854049743583212	-0.0162836145906206\\
0.862422780285008	-0.0158795410174428\\
0.870795816986804	-0.0154528531630019\\
0.8791688536886	-0.0150071038399866\\
0.887541890390397	-0.0145456030761794\\
0.895914927092193	-0.0140714273519284\\
0.904287963793989	-0.0135874288803287\\
0.912661000495785	-0.0130962448923632\\
0.921034037197581	-0.0126003068925522\\
0.929407073899378	-0.0121018498538039\\
0.937780110601174	-0.0116029213231321\\
0.94615314730297	-0.0111053904127306\\
0.954526184004766	-0.0106109566535548\\
0.962899220706562	-0.0101211586910773\\
0.971272257408359	-0.0096373828052518\\
0.979645294110155	-0.00916087123894238\\
0.988018330811951	-0.0086927303211641\\
0.996391367513747	-0.00823393837343258\\
1.00476440421554	-0.00778535338934513\\
1.01313744091734	-0.00734772047921539\\
1.02151047761914	-0.0069216790731637\\
1.02988351432093	-0.00650776987753099\\
1.03825655102273	-0.00610644158083888\\
1.04662958772452	-0.00571805730676861\\
1.05500262442632	-0.00534290081278053\\
1.06337566112812	-0.00498118243404875\\
1.07174869782991	-0.00463304477334732\\
1.08012173453171	-0.0042985681383988\\
1.08849477123351	-0.00397777572898816\\
1.0968678079353	-0.00367063857685914\\
1.1052408446371	-0.00337708024205004\\
1.11361388133889	-0.00309698126989662\\
1.12198691804069	-0.0028301834134343\\
1.13035995474249	-0.00257649362637474\\
1.13873299144428	-0.0023356878322175\\
1.14710602814608	-0.00210751447538729\\
1.15547906484787	-0.00189169786056841\\
1.16385210154967	-0.00168794128664029\\
1.17222513825147	-0.00149592998180703\\
1.18059817495326	-0.00131533384666208\\
1.18897121165506	-0.00114581001203963\\
1.19734424835686	-0.000987005218579615\\
1.20571728505865	-0.000838558024977201\\
1.21409032176045	-0.00070010085190161\\
1.22246335846224	-0.000571261868556328\\
1.23083639516404	-0.000451666728816108\\
1.23920943186584	-0.000340940163816583\\
1.24758246856763	-0.00023870743779339\\
1.25595550526943	-0.00014459567387045\\
1.26432854197123	-5.82350563838561e-05\\
1.27270157867302	2.07400837991603e-05\\
1.28107461537482	9.26902946467648e-05\\
1.28944765207661	0.00015797013284552\\
1.29782068877841	0.000216927392191028\\
1.30619372548021	0.000269902424644922\\
1.314566762182	0.000317227548391717\\
1.3229397988838	0.000359226537408729\\
1.33131283558559	0.000396214187246644\\
1.33968587228739	0.000428495951905646\\
1.34805890898919	0.000456367646881409\\
1.35643194569098	0.000480115213646794\\
1.36480498239278	0.000500014541026089\\
1.37317801909458	0.0005163313391102\\
1.38155105579637	0.000529321061551857\\
1.38992409249817	0.00053922887226816\\
1.39829712919996	0.000546289652765403\\
1.40667016590176	0.0005507280464849\\
1.41504320260356	0.000552758536749923\\
1.42341623930535	0.000552585555071757\\
1.43178927600715	0.000550403616746686\\
1.44016231270895	0.000546397480846138\\
1.44853534941074	0.00054074233186735\\
1.45690838611254	0.000533603980473578\\
1.46528142281433	0.000525139080908461\\
1.47365445951613	0.00051549536282126\\
1.48202749621793	0.000504811875385661\\
1.49040053291972	0.000493219241736315\\
1.49877356962152	0.000480839921883063\\
1.50714660632331	0.000467788482393875\\
1.51551964302511	0.000454171871263308\\
1.52389267972691	0.000440089696502905\\
1.5322657164287	0.000425634507105858\\
1.5406387531305	0.000410892075147704\\
1.5490117898323	0.000395941677889828\\
1.55738482653409	0.00038085637885198\\
1.56575786323589	0.00036570330691512\\
1.57413089993768	0.000350543932605349\\
1.58250393663948	0.000335434340794968\\
1.59087697334128	0\\
};
\addlegendentry{$F_y$ improper};

\addplot [color=mycolor2,solid]
  table[row sep=crcr]{%
0	0\\
0.0056872305737901	0.0567012111122832\\
0.0113744611475802	0.112960757960161\\
0.0170616917213703	0.168532392331302\\
0.0227489222951604	0.223097506925094\\
0.0284361528689505	0.276314555254751\\
0.0341233834427406	0.327849758637442\\
0.0398106140165307	0.377395368085344\\
0.0454978445903208	0.424679713139813\\
0.0511850751641109	0.469471890125933\\
0.056872305737901	0.511583007819072\\
0.0625595363116911	0.550865275878138\\
0.0682467668854812	0.587209793444965\\
0.0739339974592713	0.620543606120851\\
0.0796212280330614	0.650826404397189\\
0.0853084586068515	0.678047105209599\\
0.0909956891806416	0.702220470034592\\
0.0966829197544317	0.723383853905794\\
0.102370150328222	0.741594140435185\\
0.108057380902012	0.756924891972466\\
0.113744611475802	0.769463727077643\\
0.119431842049592	0.779309926575074\\
0.125119072623382	0.786572262608905\\
0.130806303197172	0.791367040981852\\
0.136493533770962	0.79381634472065\\
0.142180764344753	0.794046465657099\\
0.147867994918543	0.792186510429328\\
0.153555225492333	0.788367167415202\\
0.159242456066123	0.782719621522315\\
0.164929686639913	0.775374604352543\\
0.170616917213703	0.766461567951421\\
0.176304147787493	0.756107971089805\\
0.181991378361283	0.744438667772502\\
0.187678608935073	0.73157538840383\\
0.193365839508863	0.71763630474981\\
0.199053070082654	0.702735670513165\\
0.204740300656444	0.686983529976398\\
0.210427531230234	0.670485487768233\\
0.216114761804024	0.653342533369711\\
0.221801992377814	0.635650914498965\\
0.227489222951604	0.617502053999767\\
0.233176453525394	0.598982505310111\\
0.238863684099184	0.58017394200536\\
0.244550914672974	0.56115317729784\\
0.250238145246764	0.541992209733291\\
0.255925375820555	0.522758291656105\\
0.261612606394345	0.503514017321812\\
0.267299836968135	0.484317427818398\\
0.272987067541925	0.465222130219539\\
0.278674298115715	0.446277428634242\\
0.284361528689505	0.42752846504001\\
0.290048759263295	0.409016367992047\\
0.295735989837085	0.390778407490235\\
0.301423220410875	0.372848154459954\\
0.307110450984665	0.355255643463247\\
0.312797681558456	0.338027537404464\\
0.318484912132246	0.321187293130126\\
0.324172142706036	0.304755326947401\\
0.329859373279826	0.288749179199832\\
0.335546603853616	0.273183677143799\\
0.341233834427406	0.258071095465074\\
0.346921065001196	0.243421313862629\\
0.352608295574986	0.229241971206931\\
0.358295526148776	0.215538615853093\\
0.363982756722566	0.202314851755748\\
0.369669987296357	0.189572480093059\\
0.375357217870147	0.177311636162122\\
0.381044448443937	0.165530921357752\\
0.386731679017727	0.154227530091501\\
0.392418909591517	0.143397371548254\\
0.398106140165307	0.133035186214058\\
0.403793370739097	0.123134657141429\\
0.409480601312887	0.113688515947413\\
0.415167831886677	0.104688643565552\\
0.420855062460467	0.096126165795725\\
0.426542293034258	0.0879915437159961\\
0.432229523608048	0.0802746590382009\\
0.437916754181838	0.0729648945043144\\
0.443603984755628	0.0660512094338285\\
0.449291215329418	0.059522210543621\\
0.454978445903208	0.0533662181712761\\
0.460665676476998	0.0475713280406755\\
0.466352907050788	0.0421254687150661\\
0.472040137624578	0.0370164548878587\\
0.477727368198368	0.0322320366652391\\
0.483414598772159	0.0277599449974086\\
0.489101829345949	0.0235879334170046\\
0.494789059919739	0.0197038162440996\\
0.500476290493529	0.0160955034172187\\
0.506163521067319	0.0127510321091462\\
0.511850751641109	0.00965859528498123\\
0.517537982214899	0.00680656735803547\\
0.523225212788689	0.00418352709679433\\
0.528912443362479	0.001778277933368\\
0.53459967393627	-0.000420134179319326\\
0.54028690451006	-0.00242240521785154\\
0.54597413508385	-0.00423895970617535\\
0.55166136565764	-0.00587993936501277\\
0.55734859623143	-0.00735519366104657\\
0.56303582680522	-0.00867427212843839\\
0.56872305737901	-0.00984641833969946\\
0.5744102879528	-0.0108805654074822\\
0.58009751852659	-0.011785332903468\\
0.58578474910038	-0.0125690250851662\\
0.591471979674171	-0.0132396303260909\\
0.597159210247961	-0.0138048216494225\\
0.602846440821751	-0.0142719582698744\\
0.608533671395541	-0.014648088053051\\
0.614220901969331	-0.0149399508060939\\
0.619908132543121	-0.0151539823178468\\
0.625595363116911	-0.0152963190711249\\
0.631282593690701	-0.0153728035539294\\
0.636969824264491	-0.0153889901006108\\
0.642657054838281	-0.0153501511980291\\
0.648344285412072	-0.015261284195693\\
0.654031515985862	-0.0151271183626783\\
0.659718746559652	-0.0149521222378097\\
0.665405977133442	-0.0147405112231589\\
0.671093207707232	-0.0144962553743421\\
0.676780438281022	-0.0142230873444077\\
0.682467668854812	-0.0139245104412715\\
0.688154899428602	-0.0136038067617034\\
0.693842130002392	-0.0132640453677722\\
0.699529360576182	-0.0129080904744381\\
0.705216591149972	-0.0125386096196296\\
0.710903821723763	-0.0121580817906604\\
0.716591052297553	-0.0117688054832414\\
0.722278282871343	-0.0113729066716112\\
0.727965513445133	-0.0109723466704606\\
0.733652744018923	-0.0105689298713592\\
0.739339974592713	-0.0101643113383105\\
0.745027205166503	-0.00976000424886519\\
0.750714435740293	-0.00935738716892226\\
0.756401666314083	-0.00895771115093642\\
0.762088896887874	-0.00856210664674153\\
0.767776127461664	-0.00817159022759084\\
0.773463358035454	-0.00778707110531181\\
0.779150588609244	-0.00740935744967917\\
0.784837819183034	-0.00703916249822846\\
0.790525049756824	-0.00667711045576715\\
0.796212280330614	-0.00632374218179556\\
0.801899510904404	-0.00597952066492791\\
0.807586741478194	-0.00564483628420949\\
0.813273972051984	-0.00532001185796205\\
0.818961202625775	-0.00500530748146005\\
0.824648433199565	-0.00470092515534726\\
0.830335663773355	-0.00440701320725296\\
0.836022894347145	-0.00412367050955806\\
0.841710124920935	-0.0038509504967023\\
0.847397355494725	-0.00358886498581254\\
0.853084586068515	-0.00333738780477607\\
0.858771816642305	-0.00309645823218128\\
0.864459047216095	-0.00286598425380647\\
0.870146277789886	-0.00264584564055667\\
0.875833508363676	-0.00243589685293236\\
0.881520738937466	-0.00223596977726368\\
0.887207969511256	-0.00204587629906357\\
0.892895200085046	-0.00186541071894292\\
0.898582430658836	-0.00169435201659553\\
0.904269661232626	-0.00153246596839943\\
0.909956891806416	-0.00137950712419867\\
0.915644122380206	-0.00123522064882574\\
0.921331352953996	-0.00109934403390247\\
0.927018583527786	-0.000971608685418214\\
0.932705814101577	-0.0008517413925288\\
0.938393044675367	-0.000739465682951355\\
0.944080275249157	-0.00063450307024873\\
0.949767505822947	-0.000536574198205261\\
0.955454736396737	-0.00044539988739312\\
0.961141966970527	-0.000360702088918246\\
0.966829197544317	-0.000282204750216481\\
0.972516428118107	-0.000209634597646277\\
0.978203658691897	-0.000142721840493769\\
0.983890889265687	-8.12008008725216e-05\\
0.989578119839478	-2.48104738611839e-05\\
0.995265350413268	2.67049779179567e-05\\
1.00095258098706	7.35957912198841e-05\\
1.00663981156085	0.0001161062266821\\
1.01232704213464	0.000154474247144288\\
1.01801427270843	0.000188931246158103\\
1.02370150328222	0.000219701823208032\\
1.02938873385601	0.000247003602308664\\
1.0350759644298	0.000271047090787518\\
1.04076319500359	0.000292035575203585\\
1.04645042557738	0.000310165051492317\\
1.05213765615117	0.000325624186564825\\
1.05782488672496	0.0003385943087245\\
1.06351211729875	0.000349249424396367\\
1.06919934787254	0.000357756258793884\\
1.07488657844633	0.00036427431827412\\
1.08057380902012	0\\
};
\addlegendentry{$F_y$ proper};

\end{axis}
\end{tikzpicture}%
}
  \caption{The response of the transfer function from $d$ to $y$ for a step in
    the disturbance when $F_y$ is proper (\texttt{red}) and improper
    (\texttt{blue}). $\omega_I = 0.5 \omega_c$, $p_1 = p_2 = 5\omega_c$.}
  \label{fig:step_response_2.2}
\end{figure}


\subsection{Exercises 3 and 4}

In this subsection we try to bridge all the gaps and meet all specifications.
In order to do so, first we add a lead component to the controller, so that
$F_y$ is now given by
$$F_y(s) = \dfrac{s+\omega_I}{s} \cdot K \dfrac{\tau_D s + 1}{\beta \tau_D s + 1}
\cdot \dfrac{p_1 \cdot p_2}{(s+p_1)(s+p_2)} \cdot G^{-1}G_d$$
Since there is no specification for the phase margin, we can set it to a
sensible $30^{\circ}$ at a crossover frequency larger than the actual one, so
that $\omega_c$ is well inside the control region, $\omega_c + 6 \approx 15$
rad/s, for which $\beta = 0.6565, \tau_D = 0.0826$ and $K = 1.2557$.

As for the prefilter, it is given by $F_r$:
$$F_r = \dfrac{1}{\tau s + 1}$$
Through a process of trial and error, if $\tau = 0.1$, then all requirements
are met. Specifically, controllers $F_y, F_r$, with coefficients given in
table \ref{tbl:ex34} make the system behave as follows:

\begin{itemize}
  \item The rise time for a step change in the reference signal is
    $0.1561 < 0.2$s
  \item The overshoot is $6.5716 < 10\%$
  \item For a step in the disturbance, $|y(t)| \leq 1\ \forall t$ and
  $|y(t)| \leq 0.1$ for $t > 0.445$s
  \item The control signal obeys $|u(t)| \leq 0.511 \leq 1\ \forall t$
\end{itemize}

\begin{table}\centering
    \begin{tabular}{|c|c|c|c|c|c|c|}
    \hline
    $\omega_c$ & $\omega_I$    & $K$       & $\tau_D$ & $\beta$   & $p_1 = p_2$ & $\tau$ \\ \hline
    $9.9473$   & $0.5\omega_c$ & $1.2557$  & $0.0826$ & $0.6565$  & $5\omega_c$ & $0.1$  \\ \hline
    \end{tabular}
    \caption{The values of coefficients for $F_y(s)$ and $F_r(s)$.}
    \label{tbl:ex34}
\end{table}

Figure \ref{fig:bode_2.3} shows the frequency response of the sensitivity $S$
and complimentary sensitivity $T$ transfer functions,
figure \ref{fig:step_response_2.3_r} illustrates the step response of the
closed-loop system and figure \ref{fig:step_response_2.3_d} shows depicts the
response of the transfer function from $d$ to $y$ for a step disturbance.

\begin{figure}[H]\centering
  \scalebox{0.8}{% This file was created by matlab2tikz.
%
%The latest updates can be retrieved from
%  http://www.mathworks.com/matlabcentral/fileexchange/22022-matlab2tikz-matlab2tikz
%where you can also make suggestions and rate matlab2tikz.
%
\definecolor{mycolor1}{rgb}{0.00000,0.44700,0.74100}%
\definecolor{mycolor2}{rgb}{0.85000,0.32500,0.09800}%
%
\begin{tikzpicture}

\begin{axis}[%
width=4.008in,
height=1.551in,
at={(0.818in,1.941in)},
scale only axis,
separate axis lines,
every outer x axis line/.append style={white!40!black},
every x tick label/.append style={font=\color{white!40!black}},
xmode=log,
xmin=0.01,
xmax=10000,
xlabel={Frequency [rad/s]},
xtick={0.01,1,100,10000},
xticklabels={\empty},
xminorticks=true,
every outer y axis line/.append style={white!40!black},
every y tick label/.append style={font=\color{white!40!black}},
ymin=-150,
ymax=50,
ylabel={Magnitude [dB]},
axis background/.style={fill=white}
]
\addplot [color=mycolor1,solid,forget plot]
  table[row sep=crcr]{%
1e-20	-435.911427473309\\
2e-17	-369.890827560029\\
2e-12	-269.890827560029\\
2e-08	-189.890827560029\\
2e-05	-129.890827558323\\
0.002	-89.8908104973053\\
0.02	-69.88912163157\\
0.0233242623764153	-68.5529491264161\\
0.0272010607701931	-67.2165556734501\\
0.03172223391604	-65.8798618423667\\
0.0369948853511869	-64.5427597077751\\
0.0431439206258243	-63.2051026776148\\
0.0503150062311981	-61.8666917338003\\
0.0586780203403717	-60.5272568560816\\
0.0684310771073732	-59.1864320100946\\
0.0798052198576539	-57.8437215934786\\
0.0930698943483714	-56.4984556497403\\
0.108539331756333	-55.1497305051794\\
0.126579992602275	-53.7963308420901\\
0.147619247953009	-52.4366287790663\\
0.172155503553254	-51.0684556584593\\
0.200770006720999	-49.6889436131828\\
0.234140615703763	-48.2943377074772\\
0.273057857682499	-46.8797871522983\\
0.318443655826424	-45.4391378006585\\
0.37137316903002	-43.9647694630598\\
0.43310026170085	-42.4475498361424\\
0.505087206960238	-40.8770054278602\\
0.589039326905568	-39.2418223061206\\
0.686945390538627	-37.5307593036027\\
0.8011247263596	-35.7339588052397\\
0.934282165692261	-33.8444805436965\\
1.08957211831059	-31.8597251428924\\
1.27067329827514	-29.7823656708695\\
1.48187587018372	-27.6205449320389\\
1.72818308027219	-25.38738116629\\
1.99057971779984	-23.2861913141358\\
2.01542977993751	-23.100098533986\\
2.35042064942517	-20.77920902737\\
2.74109139610685	-18.4480792281251\\
3.19669674602653	-16.1329765444242\\
3.7280296821078	-13.8633662989371\\
4.34767712262732	-11.6718407819922\\
5.0703180968049	-9.59266648763977\\
5.9130714810882	-7.65784816351655\\
6.69649237299738	-6.21422747577275\\
7.73010902476927	-4.69089110399905\\
8.79534869632987	-3.44466563788503\\
9.87826120843706	-2.41058027511599\\
10.9656731306893	-1.53688344758826\\
12.0455797306772	-0.786066476350615\\
13.1074038731645	-0.132699802654344\\
14.142135623731	0.43977585510914\\
15.2585517265912	0.997436610518596\\
16.6036010280725	1.5952489857453\\
18.2387344229938	2.21866627905796\\
20.2464781786881	2.82986817100161\\
22.7392917444487	3.35139047571122\\
25.8728563024338	3.65946503906289\\
29.86638210871	3.62230371174603\\
32.0900413592372	3.47018162171137\\
37.4238272166434	2.94737544899085\\
43.6441582565311	2.31225252158762\\
50.8983899186562	1.70457118926299\\
59.3583700499914	1.19433119577822\\
69.2245098641175	0.801305102283554\\
73.6732414865442	0.674190379587223\\
80.7305315474713	0.517608278388174\\
94.1490049800346	0.323420084926056\\
109.797804731638	0.196350445838812\\
128.047640295757	0.116329097531244\\
149.330837946954	0.0675374955108528\\
174.151582258235	0.0385711485044347\\
203.097859892947	0.021742205408485\\
236.855388611577	0.0121313029268876\\
276.223861462211	0.00671570161170747\\
322.135890968561	0.0036954636793757\\
375.679102090551	0.00202431214842138\\
438.121897324805	0.00110511782556976\\
510.943504302832	0.000601778672168981\\
595.869017694217	0.000327073150775221\\
694.910265533838	0.000177519043279215\\
810.413468068782	9.62487629792363e-05\\
945.114818130848	5.21450510013509e-05\\
1102.2052996961	2.82348775440976e-05\\
1285.40628013936	1.52819241252372e-05\\
1499.05766691312	8.26869700486786e-06\\
1748.22071702293	4.47299206193895e-06\\
2038.79793478638	2.41928513618911e-06\\
2377.67289817256	1.30834693902454e-06\\
2772.87332611343	7.07489094825511e-07\\
3233.76124974165	3.82549696600691e-07\\
3771.25479258294	2.06840154191422e-07\\
37712.5479258294	2.06925387772758e-11\\
3771254.79258294	-9.64327466553287e-16\\
3771254792.58294	-9.64327466553287e-16\\
37712547925829.4	-9.64327466553287e-16\\
3.77125479258294e+18	0\\
1e+20	0\\
};
\addplot [color=mycolor2,solid,forget plot]
  table[row sep=crcr]{%
1e-20	-1.83222218645125e-14\\
9.94730921032347e-17	-1.83222218645125e-14\\
9.94730921032343e-12	-1.44649119982993e-14\\
9.94730921032343e-08	-1.83222218645125e-14\\
9.94730921032343e-05	1.10470075847933e-09\\
0.00994730921032343	1.10471566521544e-05\\
0.0994730921032344	0.0011043261272177\\
0.116155725545145	0.00150560523769458\\
0.135636203637027	0.00205260058306127\\
0.158383752937903	0.00279814306444468\\
0.184946293998505	0.00381414676417893\\
0.215963639131547	0.00519844092168196\\
0.252182903580204	0.00708399534452105\\
0.294476501293836	0.00965133250866126\\
0.343863158775469	0.0131451430403216\\
0.401532453161886	0.0178963718125251\\
0.468873465585417	0.0243512854042526\\
0.547508239991364	0.0331091875970057\\
0.6393308533341	0.0449703503162194\\
0.746553038236201	0.0609950503448479\\
0.871757457024287	0.082572764015307\\
1.01795991035403	0.111496584316566\\
1.18868197884439	0.150030243231791\\
1.3880358474412	0.200941622280327\\
1.6208233556757	0.26745506143198\\
1.8926516596432	0.3530444096414\\
2.21006829165343	0.460954511172856\\
2.58071887073646	0.5933189895742\\
3.0135312627795	0.749778917448309\\
3.51893062616225	0.925670916189262\\
4.10909052269843	1.1102182711802\\
4.79822614239611	1.28570445019708\\
5.60293670494616	1.42903190641814\\
6.54260528536827	1.51665951942624\\
6.69649237299738	1.52381849256605\\
7.73010902476927	1.53001061944968\\
8.79534869632987	1.48001040862061\\
9.87826120843706	1.39472477357558\\
10.9656731306893	1.29084290039485\\
12.0455797306772	1.17871752377352\\
13.1074038731645	1.06310989417367\\
14.142135623731	0.944991260383452\\
15.2585517265912	0.808588987399643\\
16.6036010280725	0.625230629355328\\
18.2387344229938	0.361049566903837\\
20.2464781786881	-0.048289656457335\\
22.7392917444487	-0.716707581618016\\
25.8728563024338	-1.81785806425759\\
29.86638210871	-3.55101368637145\\
30.8390334435813	-4.00896018137076\\
36.0110481036681	-6.52853949400014\\
42.0504614029856	-9.41935236424776\\
49.1027447774803	-12.5336674696748\\
57.3377666793254	-15.7866784238507\\
66.9538842007982	-19.141245745565\\
78.1827209044451	-22.5846965215132\\
91.2947459432009	-26.113206038722\\
106.605788854807	-29.7238219105621\\
124.48464695248	-33.4114093654141\\
145.361968551155	-37.1682957940141\\
169.740625999712	-40.9851256455356\\
198.20782837455	-44.8520056200275\\
231.449265593151	-48.7594607016818\\
270.26562463709	-52.699037072892\\
315.591875711001	-56.6635713564619\\
368.519792883494	-60.6472195887659\\
430.324251665009	-64.64534582997\\
502.493936952779	-68.6543477268063\\
586.767201005594	-72.6714684900883\\
685.173935159925	-76.6946220450967\\
800.084463852062	-80.7222428931321\\
934.266638657279	-84.7531633555925\\
1090.95250757096	-88.7865163838849\\
1273.91616539564	-92.8216602808846\\
1487.56466041743	-96.8581212806706\\
1737.04414704211	-100.895550223284\\
2028.36384129114	-104.933690117927\\
2368.54076487534	-108.97235199399\\
2765.76876429885	-113.011396995236\\
3229.61587615899	-117.050723142288\\
3771.25479258294	-121.09025556738\\
37712.5479258294	-181.088708891623\\
3771254.79258294	-301.088693266806\\
3771254792.58294	-481.088693265243\\
37712547925829.4	-721.088693265243\\
3.77125479258294e+18	-1021.08869326524\\
1e+20	-1106.49954075965\\
};
\end{axis}

\begin{axis}[%
width=4.008in,
height=1.376in,
at={(0.818in,0.44in)},
scale only axis,
separate axis lines,
every outer x axis line/.append style={white!40!black},
every x tick label/.append style={font=\color{white!40!black}},
xmode=log,
xmin=0.01,
xmax=10000,
xminorticks=true,
every outer y axis line/.append style={white!40!black},
every y tick label/.append style={font=\color{white!40!black}},
ymin=-365.4,
ymax=185.4,
ytick={-360, -180,    0,  180},
ylabel={Phase (deg)},
axis background/.style={fill=white},
legend style={at={(0.092,0.143)},anchor=south west,legend cell align=left,align=left,draw=white!15!black}
]
\addplot [color=mycolor1,solid]
  table[row sep=crcr]{%
1e-20	90\\
2e-17	90\\
2e-12	90.0000000000911\\
2e-08	90.0000009107471\\
2e-05	90.0009107471278\\
0.002	90.0910745601116\\
0.02	90.9105944845551\\
0.0233242623764153	91.0618831600141\\
0.0272010607701931	91.2382804643887\\
0.03172223391604	91.4439378062421\\
0.0369948853511869	91.683683760621\\
0.0431439206258243	91.9631291222327\\
0.0503150062311981	92.2887850604153\\
0.0586780203403717	92.6681940408748\\
0.0684310771073732	93.1100716992962\\
0.0798052198576539	93.6244553465426\\
0.0930698943483714	94.2228506888989\\
0.108539331756333	94.9183618370652\\
0.126579992602275	95.7257796084137\\
0.147619247953009	96.661588009848\\
0.172155503553254	97.7438269237183\\
0.200770006720999	98.9917189904216\\
0.234140615703763	100.424930597595\\
0.273057857682499	102.062294941859\\
0.318443655826424	103.919792045928\\
0.37137316903002	106.007583711047\\
0.43310026170085	108.325987271877\\
0.505087206960238	110.860501092728\\
0.589039326905568	113.576409775462\\
0.686945390538627	116.414053426546\\
0.8011247263596	119.286325982492\\
0.934282165692261	122.079979643288\\
1.08957211831059	124.661499183629\\
1.27067329827514	126.886752375458\\
1.48187587018372	128.612062283161\\
1.72818308027219	129.703793310017\\
1.99057971779984	130.047289266551\\
2.01542977993751	130.044451941391\\
2.35042064942517	129.535177807208\\
2.74109139610685	128.096371042077\\
3.19669674602653	125.669385995052\\
3.7280296821078	122.222500235055\\
4.34767712262732	117.763458528795\\
5.0703180968049	112.357872787121\\
5.9130714810882	106.14658150961\\
6.69649237299738	100.678151658785\\
7.73010902476927	94.0729678486575\\
8.79534869632987	88.0250364731056\\
9.87826120843706	82.5936647159248\\
10.9656731306893	77.7395137985827\\
12.0455797306772	73.3775563900208\\
13.1074038731645	69.4143167151602\\
14.142135623731	65.7682878969853\\
15.2585517265912	61.9902393258992\\
16.6036010280725	57.563012767147\\
18.2387344229938	52.2703070788556\\
20.2464781786881	45.8482553244757\\
22.7392917444487	38.0807946744686\\
25.8728563024338	29.0503876689774\\
29.86638210871	19.4791966596973\\
32.0900413592372	15.2325879256011\\
37.4238272166434	7.88241488794308\\
43.6441582565311	2.99606108732283\\
50.8983899186562	0.164881623858011\\
59.3583700499914	-1.21476019231652\\
69.2245098641175	-1.6890876652693\\
73.6732414865442	-1.72185794066438\\
80.7305315474713	-1.66736554668633\\
94.1490049800346	-1.42133575401483\\
109.797804731638	-1.11181164932131\\
128.047640295757	-0.821067892521754\\
149.330837946954	-0.581993908075162\\
174.151582258235	-0.400303668380936\\
203.097859892947	-0.269248029792942\\
236.855388611577	-0.178104705106427\\
276.223861462211	-0.116358743043583\\
322.135890968561	-0.0753187272880941\\
375.679102090551	-0.0484195501296961\\
438.121897324805	-0.0309688286892184\\
510.943504302832	-0.0197328894892348\\
595.869017694217	-0.0125385186018418\\
694.910265533838	-0.00795075811843649\\
810.413468068782	-0.0050339890451588\\
945.114818130848	-0.00318368982795102\\
1102.2052996961	-0.00201183288555705\\
1285.40628013936	-0.00127054491702555\\
1499.05766691312	-0.000802037327132964\\
1748.22071702293	-0.000506123765678268\\
2038.79793478638	-0.000319311179619641\\
2377.67289817256	-0.000201416232517975\\
2772.87332611343	-0.000127033465522461\\
3233.76124974165	-8.01124753910761e-05\\
3771.25479258294	-5.05186240648367e-05\\
37712.5479258294	-5.05458329976923e-08\\
3771254.79258294	1.01777749806833e-13\\
3771254792.58294	-2.54444374517081e-14\\
37712547925829.4	-1.27222187258541e-14\\
3.77125479258294e+18	0\\
1e+20	0\\
};
\addlegendentry{S};

\addplot [color=mycolor2,solid]
  table[row sep=crcr]{%
1e-20	5.08888749034163e-14\\
9.94730921032347e-17	5.08888749034163e-14\\
9.94730921032343e-12	-9.09638638898566e-12\\
9.94730921032343e-08	-9.12550500320539e-08\\
9.94730921032343e-05	-9.12550437091112e-05\\
0.00994730921032343	-0.00912565105246981\\
0.0994730921032344	-0.0914016943938055\\
0.116155725545145	-0.106792904156596\\
0.135636203637027	-0.124802203171204\\
0.158383752937903	-0.145890452885223\\
0.184946293998505	-0.170608670864694\\
0.215963639131547	-0.199620729406661\\
0.252182903580204	-0.233734301893338\\
0.294476501293836	-0.273944015167842\\
0.343863158775469	-0.321492891306712\\
0.401532453161886	-0.377961435638589\\
0.468873465585417	-0.445398727814305\\
0.547508239991364	-0.526517409363155\\
0.6393308533341	-0.624985595505511\\
0.746553038236201	-0.745864691765655\\
0.871757457024287	-0.896263872512592\\
1.01795991035403	-1.08630936063726\\
1.18868197884439	-1.33055592675958\\
1.3880358474412	-1.64998754114354\\
1.6208233556757	-2.07473664239993\\
1.8926516596432	-2.64754391637165\\
2.21006829165343	-3.42769615661417\\
2.58071887073646	-4.49461127315232\\
3.0135312627795	-5.94933022066378\\
3.51893062616225	-7.91110569919414\\
4.10909052269843	-10.505782558316\\
4.79822614239611	-13.8442963057337\\
5.60293670494616	-17.9952340749473\\
6.54260528536827	-22.9647199420507\\
6.69649237299738	-23.7778472702583\\
7.73010902476927	-29.1677093931815\\
8.79534869632987	-34.5344879850737\\
9.87826120843706	-39.782614509585\\
10.9656731306893	-44.8826525862682\\
12.0455797306772	-49.8381761408847\\
13.1074038731645	-54.6609698374534\\
14.142135623731	-59.3576834905975\\
15.2585517265912	-64.4584703374688\\
16.6036010280725	-70.682533086664\\
18.2387344229938	-78.3795689333899\\
20.2464781786881	-87.9734816546957\\
22.7392917444487	-99.8657979878308\\
25.8728563024338	-114.1783658843\\
29.86638210871	-130.396307028351\\
30.8390334435813	-133.931570688555\\
36.0110481036681	-150.062361719432\\
42.0504614029856	-164.220855620722\\
49.1027447774803	-176.47212166451\\
57.3377666793254	-187.17160247774\\
66.9538842007982	-196.66776019874\\
78.1827209044451	-205.20786798095\\
91.2947459432009	-212.940511014585\\
106.605788854807	-219.94618047378\\
124.48464695248	-226.268523824554\\
145.361968551155	-231.937525005624\\
169.740625999712	-236.983222711931\\
198.20782837455	-241.441608301912\\
231.449265593151	-245.35537846274\\
270.26562463709	-248.772067703236\\
315.591875711001	-251.741367427649\\
368.519792883494	-254.312646791042\\
430.324251665009	-256.533089358613\\
502.493936952779	-258.446503803513\\
586.767201005594	-260.09270383069\\
685.173935159925	-261.507306115339\\
800.084463852062	-262.721805127978\\
934.266638657279	-263.763813772838\\
1090.95250757096	-264.657390662693\\
1273.91616539564	-265.423401502655\\
1487.56466041743	-266.079881901757\\
1737.04414704211	-266.642382700988\\
2028.36384129114	-267.124287981225\\
2368.54076487534	-267.537101637182\\
2765.76876429885	-267.89070185125\\
3229.61587615899	-268.193564761913\\
3771.25479258294	-268.452959639874\\
37712.5479258294	-269.845282134855\\
3771254.79258294	-269.998452819952\\
3771254792.58294	-269.99999845282\\
37712547925829.4	-269.999999999845\\
3.77125479258294e+18	-270\\
1e+20	-270\\
};
\addlegendentry{T};

\end{axis}
\end{tikzpicture}%
}
  \caption{The frequency responses of the sensitivity (\texttt{blue}) and
  complimentary sensitivity function (\texttt{red}).}
  \label{fig:bode_2.3}
\end{figure}

\noindent\makebox[\textwidth][c]{%
\begin{minipage}{\linewidth}
  \begin{minipage}{0.45\linewidth}
    \begin{figure}[H]\centering
      \scalebox{0.6}{% This file was created by matlab2tikz.
%
%The latest updates can be retrieved from
%  http://www.mathworks.com/matlabcentral/fileexchange/22022-matlab2tikz-matlab2tikz
%where you can also make suggestions and rate matlab2tikz.
%
\definecolor{mycolor1}{rgb}{0.00000,0.44700,0.74100}%
%
\begin{tikzpicture}

\begin{axis}[%
width=4.008in,
height=3.052in,
at={(0.818in,0.44in)},
scale only axis,
separate axis lines,
every outer x axis line/.append style={white!40!black},
every x tick label/.append style={font=\color{white!40!black}},
xmin=0,
xmax=1.2,
xlabel={Time [sec]},
every outer y axis line/.append style={white!40!black},
every y tick label/.append style={font=\color{white!40!black}},
ymin=0,
ymax=1.2,
ylabel={Amplitude},
axis background/.style={fill=white}
]
\addplot [color=mycolor1,solid,forget plot]
  table[row sep=crcr]{%
0	0\\
0.00609813595485721	2.38270361311675e-05\\
0.0121962719097144	0.000334181884173459\\
0.0182944078645716	0.00148708275241709\\
0.0243925438194289	0.00414146667310685\\
0.0304906797742861	0.00892953537250829\\
0.0365888157291433	0.0163856382861915\\
0.0426869516840005	0.0269122116704824\\
0.0487850876388577	0.0407689398982922\\
0.0548832235937149	0.0580761828440027\\
0.0609813595485721	0.0788268395349568\\
0.0670794955034293	0.102902834962641\\
0.0731776314582866	0.130093729030218\\
0.0792757674131438	0.160115809489419\\
0.085373903368001	0.192630605569283\\
0.0914720393228582	0.22726214762028\\
0.0975701752777154	0.263612564895262\\
0.103668311232573	0.301275799438124\\
0.10976644718743	0.339849345029599\\
0.115864583142287	0.37894401302565\\
0.121962719097144	0.418191792652503\\
0.128060855052001	0.45725191915474\\
0.134158991006859	0.495815294075081\\
0.140257126961716	0.533607421355718\\
0.146355262916573	0.570390033432308\\
0.15245339887143	0.605961584972704\\
0.158551534826288	0.640156789914544\\
0.164649670781145	0.672845371215873\\
0.170747806736002	0.703930183280414\\
0.176845942690859	0.733344855222827\\
0.182944078645716	0.761051089737514\\
0.189042214600574	0.787035737953868\\
0.195140350555431	0.81130775582963\\
0.201238486510288	0.833895132792494\\
0.207336622465145	0.854841868848024\\
0.213434758420002	0.874205062516927\\
0.21953289437486	0.892052158967994\\
0.225631030329717	0.908458395737464\\
0.231729166284574	0.923504472581311\\
0.237827302239431	0.937274462358867\\
0.243925438194288	0.949853971419551\\
0.250023574149146	0.961328550751678\\
0.256121710104003	0.971782353118417\\
0.26221984605886	0.98129702649409\\
0.268317982013717	0.989950830250816\\
0.274416117968575	0.997817957645082\\
0.280514253923432	1.00496804612247\\
0.286612389878289	1.01146585569792\\
0.292710525833146	1.01737109507887\\
0.298808661788003	1.02273837518023\\
0.304906797742861	1.0276172701384\\
0.311004933697718	1.03205246677507\\
0.317103069652575	1.03608398460688\\
0.323201205607432	1.03974744986556\\
0.329299341562289	1.04307440851714\\
0.335397477517147	1.04609266488489\\
0.341495613472004	1.04882663413701\\
0.347593749426861	1.05129769855083\\
0.353691885381718	1.05352455907261\\
0.359790021336576	1.05552357522642\\
0.365888157291433	1.05730908786325\\
0.37198629324629	1.05889372056476\\
0.378084429201147	1.06028865671429\\
0.384182565156004	1.06150389031311\\
0.390280701110862	1.06254844955054\\
0.396378837065719	1.06343059293307\\
0.402476973020576	1.06415797844424\\
0.408575108975433	1.0647378067502\\
0.41467324493029	1.06517693989324\\
0.420771380885148	1.06548199723729\\
0.426869516840005	1.06565943065533\\
0.432967652794862	1.06571558108981\\
0.439065788749719	1.06565671868478\\
0.445163924704576	1.06548906869272\\
0.451262060659434	1.06521882531161\\
0.457360196614291	1.06485215551725\\
0.463458332569148	1.06439519483258\\
0.469556468524005	1.06385403682777\\
0.475654604478863	1.0632347179798\\
0.48175274043372	1.06254319934486\\
0.487850876388577	1.06178534631731\\
0.493949012343434	1.06096690756956\\
0.500047148298291	1.06009349409292\\
0.506145284253149	1.05917055909269\\
0.512243420208006	1.05820337933483\\
0.518341556162863	1.05719703839797\\
0.52443969211772	1.05615641215448\\
0.530537828072578	1.05508615668905\\
0.536635964027435	1.05399069876228\\
0.542734099982292	1.05287422884071\\
0.548832235937149	1.05174069664294\\
0.554930371892006	1.0505938090926\\
0.561028507846864	1.04943703052335\\
0.567126643801721	1.04827358494665\\
0.573224779756578	1.04710646016888\\
0.579322915711435	1.04593841352977\\
0.585421051666292	1.04477197902743\\
0.59151918762115	1.04360947559516\\
0.597617323576007	1.042453016301\\
0.603715459530864	1.04130451825154\\
0.609813595485721	1.04016571299495\\
0.615911731440578	1.03903815723532\\
0.622009867395436	1.0379232436881\\
0.628108003350293	1.03682221192656\\
0.63420613930515	1.03573615908831\\
0.640304275260007	1.03466605033138\\
0.646402411214865	1.03361272894809\\
0.652500547169722	1.03257692606336\\
0.658598683124579	1.03155926986145\\
0.664696819079436	1.03056029430043\\
0.670794955034293	1.02958044728827\\
0.676893090989151	1.02862009830679\\
0.682991226944008	1.02767954548071\\
0.689089362898865	1.02675902209839\\
0.695187498853722	1.02585870259875\\
0.701285634808579	1.02497870804471\\
0.707383770763437	1.02411911110903\\
0.713481906718294	1.02327994060137\\
0.719580042673151	1.02246118556832\\
0.725678178628008	1.02166279899939\\
0.731776314582866	1.02088470117262\\
0.737874450537723	1.02012678267305\\
0.74397258649258	1.01938890711687\\
0.750070722447437	1.0186709136124\\
0.756168858402294	1.01797261898741\\
0.762266994357152	1.01729381981056\\
0.768365130312009	1.0166342942321\\
0.774463266266866	1.01599380366702\\
0.780561402221723	1.01537209434133\\
0.78665953817658	1.01476889871981\\
0.792757674131438	1.01418393683119\\
0.798855810086295	1.0136169175048\\
0.804953946041152	1.01306753953044\\
0.811052081996009	1.01253549275127\\
0.817150217950867	1.01202045909813\\
0.823248353905724	1.01152211357173\\
0.829346489860581	1.01104012517818\\
0.835444625815438	1.01057415782172\\
0.841542761770295	1.01012387115794\\
0.847640897725152	1.00968892140962\\
0.85373903368001	1.00926896214669\\
0.859837169634867	1.0088636450314\\
0.865935305589724	1.00847262052921\\
0.872033441544581	1.00809553858568\\
0.878131577499439	1.00773204926946\\
0.884229713454296	1.00738180338121\\
0.890327849409153	1.00704445302855\\
0.89642598536401	1.00671965216678\\
0.902524121318868	1.00640705710537\\
0.908622257273725	1.0061063269804\\
0.914720393228582	1.00581712419293\\
0.920818529183439	1.00553911481386\\
0.926916665138296	1.00527196895547\\
0.933014801093153	1.00501536111054\\
0.939112937048011	1.00476897045948\\
0.945211073002868	1.00453248114667\\
0.951309208957725	1.00430558252668\\
0.957407344912582	1.00408796938183\\
0.96350548086744	1.00387934211197\\
0.969603616822297	1.00367940689798\\
0.975701752777154	1.00348787584019\\
0.981799888732011	1.00330446707327\\
0.987898024686868	1.00312890485876\\
0.993996160641726	1.00296091965686\\
1.00009429659658	1.00280024817884\\
1.00619243255144	1.00264663342142\\
1.0122905685063	1.00249982468454\\
1.01838870446115	1.00235957757387\\
1.02448684041601	1.00222565398929\\
1.03058497637087	1.0020978221007\\
1.03668311232573	1.00197585631215\\
1.04278124828058	1.00185953721567\\
1.04887938423544	1.00174865153559\\
1.0549775201903	1.00164299206453\\
1.06107565614516	1.00154235759194\\
1.06717379210001	1.00144655282601\\
1.07327192805487	1.00135538830978\\
1.07937006400973	1.00126868033218\\
1.08546819996458	1.00118625083472\\
1.09156633591944	1.00110792731433\\
1.0976644718743	1.00103354272306\\
1.10376260782916	1.00096293536504\\
1.10986074378401	1.00089594879124\\
1.11595887973887	1.00083243169237\\
1.12205701569373	1.00077223779042\\
1.12815515164858	1.00071522572903\\
1.13425328760344	1.00066125896323\\
1.1403514235583	1.00061020564851\\
1.14644955951316	1.00056193852978\\
1.15254769546801	1.00051633483027\\
1.15864583142287	1.00047327614064\\
1.16474396737773	1.00043264830848\\
1.17084210333258	1.00039434132828\\
1.17694023928744	1.00035824923224\\
1.1830383752423	1.00032426998176\\
1.18913651119716	1.00029230536005\\
1.19523464715201	1.00026226086563\\
1.20133278310687	1.00023404560722\\
1.20743091906173	1.00020757219968\\
1.21352905501659	1.00018275666149\\
1.21962719097144	1.00015951831352\\
1.2257253269263	1.00013777967942\\
1.23182346288116	1.00011746638746\\
1.23792159883601	1.00009850707405\\
1.24401973479087	1.00008083328893\\
1.25011787074573	1.00006437940209\\
1.25621600670059	1.00004908251235\\
1.26231414265544	1.00003488235786\\
1.2684122786103	1.00002172122833\\
1.27451041456516	1.00000954387911\\
1.28060855052001	0.999998297447201\\
1.28670668647487	0.999987931369054\\
1.29280482242973	0.999978397300335\\
1.29890295838459	0.999969649037554\\
1.30500109433944	0.999961642441618\\
1.3110992302943	0.999954335363288\\
1.31719736624916	0.999947687570535\\
1.32329550220402	0.999941660677809\\
1.32939363815887	0.999936218077191\\
1.33549177411373	0.999931324871424\\
1.34158991006859	1.00000000000077\\
};
\end{axis}
\end{tikzpicture}%
}
      \caption{The step response of the system.}
      \label{fig:step_response_2.3_r}
    \end{figure}
  \end{minipage}
  \hfill
  \begin{minipage}{0.45\linewidth}
    \begin{figure}[H]\centering
      \scalebox{0.6}{% This file was created by matlab2tikz.
%
%The latest updates can be retrieved from
%  http://www.mathworks.com/matlabcentral/fileexchange/22022-matlab2tikz-matlab2tikz
%where you can also make suggestions and rate matlab2tikz.
%
\definecolor{mycolor1}{rgb}{0.00000,0.44700,0.74100}%
%
\begin{tikzpicture}

\begin{axis}[%
width=4.008in,
height=3.052in,
at={(0.818in,0.44in)},
scale only axis,
separate axis lines,
every outer x axis line/.append style={white!40!black},
every x tick label/.append style={font=\color{white!40!black}},
xmin=0,
xmax=1,
xlabel={Time [sec]},
every outer y axis line/.append style={white!40!black},
every y tick label/.append style={font=\color{white!40!black}},
ymin=-0.01,
ymax=0.7,
ylabel={Amplitude},
axis background/.style={fill=white}
]
\addplot [color=mycolor1,solid,forget plot]
  table[row sep=crcr]{%
0	0\\
0.00609813595485963	0.0607717063778322\\
0.0121962719097193	0.120880152996352\\
0.0182944078645789	0.179741587012516\\
0.0243925438194385	0.236635497100332\\
0.0304906797742981	0.290832723563366\\
0.0365888157291578	0.341670410943436\\
0.0426869516840174	0.388592680004906\\
0.048785087638877	0.43116921491826\\
0.0548832235937366	0.469099707047286\\
0.0609813595485963	0.502209374790753\\
0.0670794955034559	0.530439025699757\\
0.0731776314583155	0.55383198729055\\
0.0792757674131752	0.572519482754464\\
0.0853739033680348	0.586705526141624\\
0.0914720393228944	0.596652069764523\\
0.097570175277754	0.602664898578851\\
0.103668311232614	0.605080596849287\\
0.109766447187473	0.604254789157265\\
0.115864583142333	0.600551766540058\\
0.121962719097193	0.594335540131844\\
0.128060855052052	0.585962313146801\\
0.134158991006912	0.57577432341868\\
0.140257126961771	0.564094980241585\\
0.146355262916631	0.551225198964441\\
0.152453398871491	0.537440823185254\\
0.15855153482635	0.522991016291165\\
0.16464967078121	0.508097500524194\\
0.17074780673607	0.492954521892396\\
0.176845942690929	0.477729422371339\\
0.182944078645789	0.46256370631674\\
0.189042214600648	0.447574495274839\\
0.195140350555508	0.432856273938011\\
0.201238486510368	0.418482839416111\\
0.207336622465227	0.404509375904444\\
0.213434758420087	0.39097458690718\\
0.219532894374947	0.377902827153037\\
0.225631030329806	0.365306185999043\\
0.231729166284666	0.353186483284633\\
0.237827302239525	0.341537147139606\\
0.243925438194385	0.330344951070492\\
0.250023574149245	0.319591594688515\\
0.256121710104104	0.309255118665263\\
0.262219846058964	0.299311149901049\\
0.268317982013824	0.289733977477869\\
0.274416117968683	0.280497463773187\\
0.280514253923543	0.27157579817504\\
0.286612389878402	0.262944103215934\\
0.292710525833262	0.254578904692737\\
0.298808661788122	0.246458478526682\\
0.304906797742981	0.238563087808693\\
0.311004933697841	0.230875123737569\\
0.317103069652701	0.223379164058176\\
0.32320120560756	0.21606196220693\\
0.32929934156242	0.208912379732199\\
0.335397477517279	0.201921273733023\\
0.341495613472139	0.195081350101104\\
0.347593749426999	0.188386992303099\\
0.353691885381858	0.181834074342572\\
0.359790021336718	0.175419765427338\\
0.365888157291578	0.169142332767124\\
0.371986293246437	0.163000947861949\\
0.378084429201297	0.15699550063216\\
0.384182565156157	0.151126424800962\\
0.390280701111016	0.145394537079687\\
0.396378837065876	0.139800891931872\\
0.402476973020735	0.134346653007734\\
0.408575108975595	0.129032981746917\\
0.414673244930455	0.123860943142866\\
0.420771380885314	0.118831428243696\\
0.426869516840174	0.113945092627175\\
0.432967652795034	0.109202309825686\\
0.439065788749893	0.104603138484049\\
0.445163924704753	0.100147301901894\\
0.451262060659612	0.0958341785354255\\
0.457360196614472	0.0916628020035646\\
0.463458332569332	0.0876318691533987\\
0.469556468524191	0.0837397547825354\\
0.475654604479051	0.0799845316849566\\
0.481752740433911	0.0763639947761243\\
0.48785087638877	0.0728756881569369\\
0.49394901234363	0.0695169340897569\\
0.500047148298489	0.0662848629787741\\
0.506145284253349	0.0631764435677383\\
0.512243420208209	0.0601885126874357\\
0.518341556163068	0.0573178040006323\\
0.524439692117928	0.0545609753015139\\
0.530537828072788	0.0519146340283526\\
0.536635964027647	0.0493753607411131\\
0.542734099982507	0.0469397303992442\\
0.548832235937366	0.0446043313486029\\
0.554930371892226	0.0423657819902434\\
0.561028507847086	0.0402207451578309\\
0.567126643801945	0.0381659402750675\\
0.573224779756805	0.0361981534002752\\
0.579322915711665	0.0343142452928024\\
0.585421051666524	0.0325111576559338\\
0.591519187621384	0.0307859177242761\\
0.597617323576244	0.029135641370961\\
0.603715459531103	0.0275575349122714\\
0.609813595485963	0.0260488957852454\\
0.615911731440822	0.0246071122682099\\
0.622009867395682	0.0232296624057562\\
0.628108003350542	0.0219141122890624\\
0.634206139305401	0.0206581138302866\\
0.640304275260261	0.0194594021565473\\
0.646402411215121	0.0183157927352421\\
0.65250054716998	0.0172251783285385\\
0.65859868312484	0.0161855258611446\\
0.664696819079699	0.015194873272216\\
0.670794955034559	0.0142513264096926\\
0.676893090989419	0.0133530560136683\\
0.682991226944278	0.0124982948246809\\
0.689089362899138	0.0116853348431654\\
0.695187498853998	0.010912524757768\\
0.701285634808857	0.0101782675527832\\
0.707383770763717	0.00948101829863669\\
0.713481906718576	0.00881928212404612\\
0.719580042673436	0.00819161236419352\\
0.725678178628296	0.007596608875868\\
0.731776314583155	0.00703291650800139\\
0.737874450538015	0.0064992237142377\\
0.743972586492875	0.00599426129306023\\
0.750070722447734	0.00551680124045776\\
0.756168858402594	0.00506565570005618\\
0.762266994357453	0.00463967599598897\\
0.768365130312313	0.00423775173445171\\
0.774463266267173	0.00385880996080634\\
0.780561402222032	0.00350181436020647\\
0.786659538176892	0.00316576449094366\\
0.792757674131752	0.00284969504101507\\
0.798855810086611	0.00255267509974122\\
0.804953946041471	0.00227380743757776\\
0.81105208199633	0.00201222778854037\\
0.81715021795119	0.00176710413086833\\
0.82324835390605	0.00153763596267374\\
0.829346489860909	0.00132305357034492\\
0.835444625815769	0.00112261728838597\\
0.841542761770629	0.000935616750174556\\
0.847640897725488	0.000761370129805888\\
0.853739033680348	0.000599223375764481\\
0.859837169635207	0.000448549437630262\\
0.865935305590067	0.000308747487388829\\
0.872033441544927	0.000179242137184015\\
0.878131577499786	5.9482655533365e-05\\
0.884229713454646	-5.10578158669367e-05\\
0.890327849409506	-0.00015288304258283\\
0.896425985364365	-0.000246474472982856\\
0.902524121319225	-0.000332292006799977\\
0.908622257274085	-0.000410774754742617\\
0.914720393228944	-0.000482341787320449\\
0.920818529183804	-0.000547392871210694\\
0.926916665138663	-0.000606309191666976\\
0.933014801093523	-0.000659454059658817\\
0.939112937048383	-0.000707173602619411\\
0.945211073003242	-0.000749797437869042\\
0.951309208958102	-0.000787639327965945\\
0.957407344912962	-0.000820997817413169\\
0.963505480867821	-0.000850156850316671\\
0.969603616822681	-0.000875386368743829\\
0.97570175277754	-0.000896942891672245\\
0.9817998887324	-0.000915070074544883\\
0.98789802468726	-0.000929999249558826\\
0.993996160642119	-0.000941949946911897\\
1.00009429659698	-0.000951130397313061\\
1.00619243255184	-0.000957738016131457\\
1.0122905685067	-0.000961959869613697\\
1.01838870446156	-0.000963973123642712\\
1.02448684041642	-0.000963945475543396\\
1.03058497637128	-0.000962035569462433\\
1.03668311232614	-0.00095839339586364\\
1.042781248281	-0.000953160675685342\\
1.04887938423586	-0.00094647122970622\\
1.05497752019072	-0.000938451333659637\\
1.06107565614558	-0.000929220059625973\\
1.06717379210043	-0.000918889604218457\\
1.07327192805529	-0.000907565604061252\\
1.07937006401015	-0.00089534743903989\\
1.08546819996501	-0.000882328523784355\\
1.09156633591987	-0.000868596587824636\\
1.09766447187473	-0.000854233944837665\\
1.10376260782959	-0.000839317751384466\\
1.10986074378445	-0.000823920255516211\\
1.11595887973931	-0.000808109035608838\\
1.12205701569417	-0.000791947229768197\\
1.12815515164903	-0.000775493756130103\\
1.13425328760389	-0.000758803524364927\\
1.14035142355875	-0.000741927638681072\\
1.14644955951361	-0.000724913592609149\\
1.15254769546847	-0.000707805455836477\\
1.15864583142333	0\\
};
\end{axis}
\end{tikzpicture}%
}
      \caption{The response of the transfer function from $d$ to $y$ for a step
        in the disturbance.}
      \label{fig:step_response_2.3_d}
    \end{figure}
  \end{minipage}
\end{minipage}
}
