% ------------------------------------------------------------------------------
\subsection*{Exercise 2.2.1}

Table \ref{tbl:221} presents the experimental and calculated steady-state values
for the height in each tank for both the minimum and non-minimum phase cases.
It is evident that not all real values agree with their theoretically derived
ones. This is reasonable: non-zero steady-state in this setting means that the
amount of flow coming into a tank is equal to that coming out of it, during a
given amount of time. This means that there are infinite stationary points,
depending on the amount of water present in the tanks at time $t=0$, or at later
points of time.

\begin{table}[H]\centering
  \begin{tabular}{c|c|c}
    Height of tank i & minimum phase [cm] / calculated & non-minimum phase [cm] / calculated\\ \hline
    1                & 17.5 / 26.545      & 28 / 27.0478           \\
    2                & 22 / 27.0478       & 14 / 26.5450           \\
    3                & 6.5 / 9.3885       & 12 / 10.8637            \\
    4                & 7.5 / 8.7095       & 9.5 / 10.0779           \\
    \end{tabular}
    \caption{Steady-state water height in each tank, experimentally derived and
      calculated from the system's model. Initially the tanks were
      empty. The inputs were set to $0.5 u_{max} = 7.5 V$ and the system was
      permitted to reach steady-state from these initial conditions.}
    \label{tbl:221}
\end{table}



% ------------------------------------------------------------------------------
\subsection*{Exercise 2.2.2}

Figure \ref{fig:222} shows the step responses from when one input is ON for
the minimum and non-minimum phase cases.

\begin{figure}[ht]
\begin{multicols}{2}
\subfigure[]{
  \scalebox{0.6}{% This file was created by matlab2tikz.
%
%The latest updates can be retrieved from
%  http://www.mathworks.com/matlabcentral/fileexchange/22022-matlab2tikz-matlab2tikz
%where you can also make suggestions and rate matlab2tikz.
%
\definecolor{mycolor1}{rgb}{0.00000,0.44700,0.74100}%
\definecolor{mycolor2}{rgb}{0.85000,0.32500,0.09800}%
%
\begin{tikzpicture}

\begin{axis}[%
width=4.133in,
height=3.26in,
at={(0.693in,0.44in)},
scale only axis,
xmin=0,
xmax=175,
xmajorgrids,
ymin=0,
ymax=55,
ymajorgrids,
axis background/.style={fill=white},
legend style={at={(0.69,0.653)},anchor=south west,legend cell align=left,align=left,draw=white!15!black}
]
\addplot [color=mycolor1,solid]
  table[row sep=crcr]{%
1	0\\
2	6.40089725959079\\
3	8.80371838453147\\
4	9.83893059805061\\
5	10.8045942562659\\
6	11.1455398380296\\
7	11.3783353729629\\
8	12.3035835596562\\
9	13.5586647018507\\
10	14.9092734416732\\
11	16.2139442791076\\
12	17.4069688311017\\
13	18.5559968139215\\
14	19.6518554256773\\
15	20.7141886030145\\
16	21.729712575171\\
17	22.7474363963973\\
18	23.6534862733627\\
19	24.4970149356334\\
20	25.3360626335914\\
21	26.137854655063\\
22	26.934994487285\\
23	27.7255571014376\\
24	28.4390422174315\\
25	29.034256545607\\
26	29.5531677636324\\
27	30.0770730893407\\
28	30.6879380905947\\
29	31.3080591231521\\
30	31.8684695547298\\
31	32.4423795836198\\
32	33.0143110938899\\
33	33.5915883695553\\
34	34.1282043263877\\
35	34.6419729584743\\
36	35.135038135164\\
37	35.610716923574\\
38	36.065893592775\\
39	36.507180065072\\
40	36.9190429601786\\
41	37.307818574798\\
42	37.7077457586808\\
43	38.0669954139075\\
44	38.419128941965\\
45	38.7485927660805\\
46	39.0513055531161\\
47	39.4023051398675\\
48	39.7184507027546\\
49	40.0151748662449\\
50	40.2810808202605\\
51	40.5544183940549\\
52	40.84735963508\\
53	41.109190216246\\
54	41.3223021709602\\
55	41.5781445844059\\
56	41.8162745548539\\
57	42.0506944811583\\
58	42.265464674838\\
59	42.4712494489069\\
60	42.6527699804757\\
61	42.8250880728482\\
62	43.0644475565335\\
63	43.251088765139\\
64	43.4348255101613\\
65	43.8212157418369\\
66	44.0769843598297\\
67	44.1282479886309\\
68	44.1791133544904\\
69	44.2770160841907\\
70	44.4134377367258\\
71	44.5701987821822\\
72	44.7390571305609\\
73	44.8839500689577\\
74	45.2011738187191\\
75	45.2324860928916\\
76	45.3158300888783\\
77	45.4345092295673\\
78	45.5496691624678\\
79	45.6548520539965\\
80	45.8243794396172\\
81	45.9051361471214\\
82	46.1023858974051\\
83	46.1732887647787\\
84	46.2565460171454\\
85	46.3232882895189\\
86	46.4282713367717\\
87	46.5509498954733\\
88	46.6208461905644\\
89	46.7272613405484\\
90	46.7901506615567\\
91	46.917468853572\\
92	46.9679260418999\\
93	47.0023479561661\\
94	47.0866638492715\\
95	47.121979266154\\
96	47.2034052908474\\
97	47.3383177762175\\
98	47.496995944409\\
99	47.6509587573607\\
100	47.750354733037\\
101	47.8829895992861\\
102	47.963244314025\\
103	48.0472311278073\\
104	48.1415265304237\\
105	48.2365744985478\\
106	48.2861507548449\\
107	48.3622805988006\\
108	48.4830394931945\\
109	48.5541081003107\\
110	48.575350982541\\
111	48.6516646023547\\
112	48.6842487053457\\
113	48.7245264518786\\
114	48.800728949099\\
115	48.8254143218991\\
116	48.8935508461602\\
117	48.9118674027886\\
118	48.9719401029249\\
119	49.0328197822648\\
120	49.0551820385216\\
121	49.0717874211943\\
122	49.0985798152239\\
123	49.1975889684233\\
124	49.2508401588783\\
125	49.258197165125\\
126	49.2483500563019\\
127	49.2435494373302\\
128	49.3230546805219\\
129	49.3677047309561\\
130	49.4031591777215\\
131	49.412591301256\\
132	49.4264070714132\\
133	49.4727265236757\\
134	49.4688082855136\\
135	49.5054264892575\\
136	49.4787899210885\\
137	49.5176127375293\\
138	49.540816649438\\
139	49.549129294907\\
140	49.552883237303\\
141	49.5506328788856\\
142	49.6135232504967\\
143	49.6204236284369\\
144	49.6295986868317\\
145	49.6670974094121\\
146	49.6792766283483\\
147	49.6926256680522\\
148	49.6906573279276\\
149	49.6683062695271\\
150	49.6742066138766\\
151	49.713182975495\\
152	49.7189326282054\\
153	49.7164459529475\\
154	49.7105915637951\\
155	49.6931845778726\\
156	49.7167219672778\\
157	49.6989863936572\\
158	49.6950708959825\\
159	49.7080651164283\\
160	49.6921418591578\\
161	49.6819203493622\\
162	49.6720728425721\\
163	49.6665342186004\\
164	49.6220557342134\\
165	49.6500005588391\\
166	49.7000018752042\\
167	49.6895315664025\\
168	49.701835857171\\
169	49.6952397403345\\
170	49.7149595699065\\
171	49.7069590759617\\
172	49.6975595512716\\
173	49.6949210824833\\
174	49.6913243936305\\
175	49.7156675564542\\
176	49.6906377346914\\
177	49.6686198961302\\
178	49.6524075435742\\
179	49.6698570550075\\
180	49.638356529906\\
181	49.6050769849882\\
182	49.6029209292966\\
183	49.608077450887\\
184	49.5771360416351\\
185	49.5508299495527\\
186	49.5406865212808\\
187	49.5356062921776\\
188	49.5847041995738\\
189	49.5568835980753\\
190	49.5147706420033\\
191	49.5172458938402\\
192	49.5073646999957\\
193	49.4363220534994\\
194	49.4581595489\\
195	49.457842544219\\
196	49.4550487450021\\
197	49.4375114171457\\
198	49.4555898180252\\
199	49.425423176441\\
200	49.441643420233\\
201	49.437570880527\\
202	49.4175049508442\\
203	49.4352341693868\\
204	49.4702712788983\\
205	49.5060340743613\\
206	49.500581190729\\
207	49.486570142003\\
208	49.4552459571383\\
209	49.4313530074013\\
210	49.4233749628836\\
211	49.4164174778493\\
212	49.4089034255229\\
213	49.4187315879449\\
};
\addlegendentry{tank 1};

\addplot [color=mycolor2,solid]
  table[row sep=crcr]{%
1	0\\
2	9.16575557294001\\
3	12.5360250730246\\
4	13.7774352604789\\
5	14.2328950480213\\
6	14.4022191162897\\
7	14.464814408186\\
8	14.4900226549448\\
9	14.4986739096217\\
10	14.710294569936\\
11	15.1554724638119\\
12	15.4522816943191\\
13	15.566927275574\\
14	15.5714148921135\\
15	15.5413333891193\\
16	15.4401514202123\\
17	15.3381061513125\\
18	15.2216426090112\\
19	15.0947892872276\\
20	14.9544588958857\\
21	14.900454192083\\
22	14.8607433433744\\
23	14.8470138615502\\
24	14.8893052614662\\
25	14.8815357996347\\
26	14.8975000714475\\
27	14.8504167503741\\
28	14.9376999607878\\
29	14.8850713571271\\
30	14.8894271097477\\
31	14.9068673163855\\
32	14.9058670824973\\
33	14.9415353146598\\
34	14.9377734494257\\
35	14.9693539612827\\
36	14.9224532083485\\
37	14.8706185290402\\
38	14.8580539542589\\
39	14.880290413889\\
40	14.9223530234785\\
41	14.949611276738\\
42	14.9390948925697\\
43	14.9506353129119\\
44	14.9532237334626\\
45	14.9617356330195\\
46	14.9228557038313\\
47	14.8810794047652\\
48	14.8973349455408\\
49	14.8459253341544\\
50	14.8852679461018\\
51	14.8534660853656\\
52	14.9588068448323\\
53	15.010204824756\\
54	14.9674009869441\\
55	14.9616929582728\\
56	15.0540854717445\\
57	15.105028376284\\
58	15.0719860455493\\
59	15.1014927850464\\
60	15.0882709895835\\
61	15.1884392709481\\
62	15.2586599180941\\
63	15.2955354619237\\
64	15.235561176542\\
65	15.2170768079509\\
66	15.2530963691451\\
67	15.1413596925959\\
68	15.1907849844859\\
69	15.2515474096709\\
70	15.381940006489\\
71	15.3606206991918\\
72	15.3277013924466\\
73	15.3472574278013\\
74	15.291530328917\\
75	15.2776591337548\\
76	15.2993172036947\\
77	15.3557177351513\\
78	15.3534380514588\\
79	15.4707075596313\\
80	15.4231801142926\\
81	15.4013736682749\\
82	15.4081377455943\\
83	15.3660947139995\\
84	15.3351836584536\\
85	15.3754510069341\\
86	15.4317020667057\\
87	15.433301775805\\
88	15.4434047991602\\
89	15.3858022351416\\
90	15.4144223622075\\
91	15.5089812231632\\
92	15.4902341790052\\
93	15.4505418127489\\
94	15.4454950302187\\
95	15.4504428111564\\
96	15.4944721457096\\
97	15.5363155794665\\
98	15.5022590856893\\
99	15.4272331842862\\
100	15.5120230054441\\
101	15.5487839973238\\
102	15.5761879778834\\
103	15.5734867571909\\
104	15.6072671535418\\
105	15.651117556115\\
106	15.5823065045711\\
107	15.6359528570435\\
108	15.6565958470998\\
109	15.7711286635956\\
110	15.6457955288384\\
111	15.6289490388706\\
112	15.7381087005332\\
113	15.726053679768\\
114	15.7778701145217\\
115	15.6879138892038\\
116	15.684159768881\\
117	15.7058121837551\\
118	15.6540496451295\\
119	15.7635331042856\\
120	15.7669995017278\\
121	15.8133799720682\\
122	15.8060869174501\\
123	15.8297922252631\\
124	15.9000767755079\\
125	15.8565440864044\\
126	15.8951134197272\\
127	15.8393066448419\\
128	15.8748551787787\\
129	15.9090143165913\\
130	15.9164896121011\\
131	15.9148254053142\\
132	15.9624654680032\\
133	15.9698539196232\\
134	15.9676511246928\\
135	15.9001998271636\\
136	15.9391728474318\\
137	15.9459340002772\\
138	15.9655890394538\\
139	15.9577429631392\\
140	15.9858846927441\\
141	15.9949314302154\\
142	16.0061863459897\\
143	16.027173373435\\
144	16.0672254899692\\
145	16.0306169380096\\
146	15.9998346704515\\
147	16.0490124353436\\
148	16.0391687633345\\
149	16.0379879487937\\
150	16.1262908787363\\
151	16.1156129322096\\
152	16.0672810537132\\
153	16.0092832883719\\
154	16.0602606419747\\
155	16.0841177026741\\
156	16.0662836657559\\
157	16.0921264342866\\
158	16.1097061412913\\
159	16.0947355852677\\
160	16.077884642592\\
161	16.0871851767088\\
162	16.1116100834929\\
163	16.1326977842575\\
164	16.0827762640512\\
165	16.1374138894801\\
166	16.1429783604334\\
167	16.1654035505427\\
168	16.1244568068036\\
169	16.1126264027437\\
170	16.1668808428367\\
171	16.1338783352261\\
172	16.0852067001711\\
173	16.1359604565369\\
174	16.1614520838683\\
175	16.1957263281357\\
176	16.2043279797197\\
177	16.2012666763305\\
178	16.1904807170895\\
179	16.1986791438766\\
180	16.2407153830371\\
181	16.2609332975253\\
182	16.2473035670773\\
183	16.1800567244062\\
184	16.1759539630614\\
185	16.1301629757493\\
186	16.1625957560801\\
187	16.1959212968589\\
188	16.206630659894\\
189	16.2097260199258\\
190	16.1959601736242\\
191	16.2277609618111\\
192	16.2422669613102\\
193	16.2968876967137\\
194	16.3328909727166\\
195	16.3401572182912\\
196	16.3938851251846\\
197	16.3838252313545\\
198	16.3158627998147\\
199	16.2684989049092\\
200	16.3044803720765\\
201	16.3001511702431\\
202	16.2674691233642\\
203	16.2895260148825\\
204	16.2960526786332\\
205	16.2402353667685\\
206	16.1843032717786\\
207	16.2328345528735\\
208	16.2517492424993\\
209	16.2670794089761\\
210	16.2813752671321\\
211	16.3159452564264\\
212	16.2835678549099\\
213	16.2552835992827\\
};
\addlegendentry{tank 2};

\end{axis}
\end{tikzpicture}%}
  \label{fig:}}
\quad
\subfigure[]{
  \scalebox{0.6}{% This file was created by matlab2tikz.
%
%The latest updates can be retrieved from
%  http://www.mathworks.com/matlabcentral/fileexchange/22022-matlab2tikz-matlab2tikz
%where you can also make suggestions and rate matlab2tikz.
%
\definecolor{mycolor1}{rgb}{0.00000,0.44700,0.74100}%
\definecolor{mycolor2}{rgb}{0.85000,0.32500,0.09800}%
%
\begin{tikzpicture}

\begin{axis}[%
width=4.133in,
height=3.26in,
at={(0.693in,0.44in)},
scale only axis,
xmin=0,
xmax=175,
xmajorgrids,
ymin=0,
ymax=55,
ymajorgrids,
axis background/.style={fill=white},
legend style={at={(0.686,0.702)},anchor=south west,legend cell align=left,align=left,draw=white!15!black}
]
\addplot [color=mycolor1,solid]
  table[row sep=crcr]{%
1	0\\
2	6.36760071082487\\
3	8.7062243773326\\
4	9.56852086625784\\
5	9.88287312197602\\
6	10.0007500895833\\
7	10.0440430679399\\
8	10.0584364171409\\
9	10.0673379408051\\
10	10.0746703950887\\
11	10.2792162865091\\
12	10.5628556253617\\
13	10.7806316694446\\
14	10.9525237785629\\
15	11.0372661399537\\
16	11.0658890331447\\
17	11.0800521237289\\
18	11.0617289258014\\
19	11.1027796293703\\
20	11.1434806782852\\
21	11.1669607646484\\
22	11.1795029494408\\
23	11.2000423466858\\
24	11.1959816705967\\
25	11.2272892044688\\
26	11.2674172224859\\
27	11.3115523711227\\
28	11.3215996189883\\
29	11.3394743781555\\
30	11.3130675652674\\
31	11.3040944615721\\
32	11.2734908471338\\
33	11.2432568260487\\
34	11.2519635469107\\
35	11.3193883026228\\
36	11.3927920789243\\
37	11.4429056018563\\
38	11.4348462570696\\
39	11.4251574941463\\
40	11.4272534703845\\
41	11.4405509334522\\
42	11.4532153791153\\
43	11.4501809614627\\
44	11.439478928284\\
45	11.4950577890197\\
46	11.500349623147\\
47	11.4653737096481\\
48	11.4892562950344\\
49	11.4896840770395\\
50	11.4991141405516\\
51	11.5131328050447\\
52	11.5608346265471\\
53	11.5546415531468\\
54	11.5496117651061\\
55	11.5377466593684\\
56	11.5636760152311\\
57	11.5933719679106\\
58	11.574026848648\\
59	11.5913756884229\\
60	11.6387039859934\\
61	11.643381781886\\
62	11.6629280700639\\
63	11.7097446175886\\
64	11.7534153967606\\
65	11.8356508332983\\
66	11.8445602154378\\
67	11.8640132967797\\
68	11.8841101755409\\
69	11.8657466547336\\
70	11.8925219713735\\
71	11.9024800759029\\
72	11.9132381147374\\
73	11.9000274205725\\
74	11.8841804611639\\
75	11.904774679968\\
76	11.9111075717373\\
77	11.9386029323368\\
78	11.8690778183835\\
79	11.8407353748791\\
80	11.8269080009507\\
81	11.8597479077799\\
82	11.9019048531895\\
83	11.9185512775607\\
84	11.9672766874278\\
85	11.9185920464529\\
86	11.9042904494263\\
87	11.9516898572918\\
88	11.9829440513763\\
89	11.9753196548691\\
90	11.8695247768214\\
91	11.833675155222\\
92	11.7725343033886\\
93	11.6813477664953\\
94	11.7030126674276\\
95	11.7374065540083\\
96	11.7300478781498\\
97	11.7504050862628\\
98	11.7816731512046\\
99	11.8011680633838\\
100	11.8013907362454\\
101	11.9084568339136\\
102	11.8644543996026\\
103	11.8406271460358\\
104	11.8791227652519\\
105	11.8189666320036\\
106	11.7627025318159\\
107	11.8231269088852\\
108	11.8609314147594\\
109	11.755051601195\\
110	11.7694682054509\\
111	11.7835723075004\\
112	11.7347733430966\\
113	11.7613781412956\\
114	11.7746981409081\\
115	11.7709077848298\\
116	11.8048249552912\\
117	11.761290626525\\
118	11.8926641216279\\
119	11.8866594201127\\
120	11.8487870474404\\
121	11.8182794204841\\
122	11.7596665520511\\
123	11.767701506611\\
124	11.8297075430876\\
125	11.8216842490534\\
126	11.7957002832749\\
127	11.81412864912\\
128	11.8479012325337\\
129	11.8321817296932\\
130	11.8149719114064\\
131	11.8267106990801\\
132	11.848120354065\\
133	11.8446351157838\\
134	11.8018738289099\\
135	11.805640840706\\
136	11.7176676894491\\
137	11.8082252603621\\
138	11.8094098469308\\
139	11.8396500093811\\
140	11.7988633279932\\
141	11.7842427567846\\
142	11.8268658830275\\
143	11.812170273314\\
144	11.8643605861907\\
145	11.8469130204755\\
146	11.8443644532422\\
147	11.8438600030146\\
148	11.8632247438815\\
149	11.7881752470163\\
150	11.7438105959872\\
151	11.846700524838\\
152	11.8235281806355\\
153	11.8524592310154\\
154	11.8329487824444\\
155	11.8790033595621\\
156	11.8761405571389\\
157	11.8697187784449\\
158	11.8277851764652\\
159	11.854088628379\\
160	11.9175983459468\\
161	11.8866526751438\\
162	11.905186191468\\
163	11.8838268018912\\
164	11.8879249707162\\
165	11.8970845018572\\
166	11.92698598799\\
167	11.8324159560716\\
168	11.9082600007341\\
169	11.8978163322912\\
170	11.8867419487763\\
171	11.8553764647043\\
172	11.8545525056\\
173	11.8783723063097\\
174	11.8653205453362\\
175	11.8944718683367\\
176	11.9034858823577\\
177	11.9179713590986\\
178	11.9293918514772\\
179	11.9465916682113\\
180	11.9193001468135\\
181	11.8931063417988\\
182	11.8994840020463\\
183	11.9031787243755\\
184	11.8863937536239\\
185	11.8879297120448\\
186	11.8876611291669\\
187	11.8794264096501\\
188	11.8903524563999\\
189	11.8632117685169\\
190	11.8560713500809\\
191	11.8915790217929\\
192	11.9163991092988\\
193	11.8980627590209\\
194	11.9047387439119\\
195	11.8841183671231\\
196	11.8841522507773\\
197	11.8973706965818\\
198	11.8862614220858\\
199	11.8924699232655\\
200	11.8856763395284\\
201	11.8420888442083\\
202	11.8816135598973\\
203	11.880906050804\\
204	11.9035423720779\\
205	11.9296955646267\\
206	11.9082790838059\\
207	11.8966579201205\\
208	11.8583177033618\\
209	11.8390421498211\\
210	11.8113416014577\\
211	11.8213555788267\\
212	11.9020510808235\\
213	11.8765701997875\\
214	11.8991889908363\\
215	11.8752968478994\\
216	11.8340855920621\\
217	11.8677917869199\\
218	11.8540089861573\\
219	11.8343449987628\\
220	11.7749586662889\\
221	11.8067631678591\\
222	11.8383794446397\\
223	11.9177199640329\\
224	11.9053664149936\\
225	11.8799841943394\\
226	11.8772792257356\\
227	11.8024183305207\\
228	11.845693726688\\
229	11.8399504748107\\
230	11.8285165004354\\
231	11.8456043765305\\
232	11.812760316122\\
233	11.8102925556827\\
234	11.8233468591925\\
235	11.8243482475947\\
236	11.8499565138745\\
237	11.8258099518401\\
238	11.8691623190383\\
239	11.8000672091311\\
240	11.7707970469095\\
241	11.8448469768155\\
242	11.8613359718168\\
243	11.7930280313876\\
244	11.8478373133686\\
245	11.8256296984927\\
246	11.8865491652283\\
247	11.8862019214644\\
248	11.8753941440571\\
249	11.8776739534803\\
250	11.8682646882111\\
};
\addlegendentry{tank 1};

\addplot [color=mycolor2,solid]
  table[row sep=crcr]{%
1	0\\
2	9.65989420074989\\
3	13.234929556559\\
4	14.5726715188714\\
5	15.0842967387899\\
6	16.2110360806939\\
7	17.8772850898958\\
8	19.3707878786143\\
9	20.8031164207409\\
10	22.2109972140248\\
11	23.5099943851768\\
12	24.7852115101301\\
13	26.0098329951845\\
14	27.135842823859\\
15	28.2211942949301\\
16	29.2625161601505\\
17	30.2536145379236\\
18	31.1947899993711\\
19	32.088459202858\\
20	32.9153513006253\\
21	33.711904330167\\
22	34.4410141055697\\
23	35.1419361655579\\
24	35.8169897659631\\
25	36.4623758401129\\
26	37.0988841910294\\
27	37.6967526216112\\
28	38.2552662265756\\
29	38.8077756370086\\
30	39.3445373685235\\
31	39.8546157342669\\
32	40.3305754870914\\
33	40.7864627677575\\
34	41.2204714350139\\
35	41.6399809195022\\
36	42.0563167225751\\
37	42.4579732625913\\
38	42.8117773252066\\
39	43.1684331466072\\
40	43.5287316920213\\
41	43.8629114533043\\
42	44.1827376824028\\
43	44.5027772633827\\
44	44.8027187668655\\
45	45.0811806449856\\
46	45.3461685401833\\
47	45.6068775820542\\
48	45.8663235528383\\
49	46.1176812105484\\
50	46.3637296501198\\
51	46.598327849658\\
52	46.8385545111965\\
53	47.0406185606287\\
54	47.2629976466319\\
55	47.455613501246\\
56	47.666953230211\\
57	47.8739755576573\\
58	48.0694399686473\\
59	48.2442621347381\\
60	48.4139857643318\\
61	48.5911554915003\\
62	48.7433137777253\\
63	48.9130450207591\\
64	49.060249747713\\
65	49.2103775864945\\
66	49.359083861677\\
67	49.5077905720142\\
68	49.6464648551297\\
69	49.8010033359505\\
70	49.9493073424029\\
71	50.0693547954407\\
72	50.2108612392302\\
73	50.3196722020382\\
74	50.4381249442193\\
75	50.5533974912547\\
76	50.6806871278177\\
77	50.7968346139315\\
78	50.9112526092692\\
79	51.0092291359526\\
80	51.1261112846229\\
81	51.2251923339378\\
82	51.3234283269292\\
83	51.4191701548951\\
84	51.4808771684317\\
85	51.5661946226903\\
86	51.6642142136844\\
87	51.7509803118105\\
88	51.8272649739678\\
89	51.9010466211595\\
90	51.9773192776064\\
91	52.0526680086613\\
92	52.0908192733476\\
93	52.1413547460509\\
94	52.2031581994729\\
95	52.2700101690585\\
96	52.3261818696439\\
97	52.3932981137376\\
98	52.4488632312544\\
99	52.4835817835628\\
100	52.5470261730246\\
101	52.5976716134271\\
102	52.6372561484263\\
103	52.6587814679393\\
104	52.7471797762641\\
105	52.7954256473027\\
106	52.8290242282688\\
107	52.8707046406906\\
108	52.9202239493791\\
109	52.9652977169049\\
110	52.9809092086742\\
111	53.016429592422\\
112	53.0439950582675\\
113	53.0962807207042\\
114	53.1405247935762\\
115	53.1865785938908\\
116	53.2068682957981\\
117	53.2404048740888\\
118	53.2804470023258\\
119	53.2926800395681\\
120	53.3199213626646\\
121	53.3686723633587\\
122	53.3715923915653\\
123	53.3712320190358\\
124	53.4036609627903\\
125	53.4264300914264\\
126	53.4475889160831\\
127	53.4725935728913\\
128	53.484122099088\\
129	53.4978010146269\\
130	53.5690945850013\\
131	53.6098166794111\\
132	53.6298632045406\\
133	53.6318147288471\\
134	53.6729663719101\\
135	53.6882591627626\\
136	53.7060783392278\\
137	53.7276949078819\\
138	53.7576790492027\\
139	53.7824213483676\\
140	53.7934102449923\\
141	53.7931604267523\\
142	53.8087444055808\\
143	53.8445382319241\\
144	53.8510929802508\\
145	53.858147209863\\
146	53.8725380196944\\
147	53.9189232242818\\
148	53.9125824197383\\
149	53.9170120348118\\
150	53.9452185801212\\
151	53.9452963366104\\
152	53.9549415711537\\
153	53.9666815137584\\
154	54.0100430422347\\
155	54.0183840730794\\
156	54.0035155951672\\
157	53.990542513974\\
158	54.0071253540959\\
159	54.0154159902039\\
160	54.0188450511047\\
161	54.0246791672036\\
162	54.0252711958413\\
163	54.0288560783031\\
164	54.0311356086785\\
165	54.0518592083474\\
166	54.0495239562013\\
167	54.0756948571276\\
168	54.0881949935298\\
169	54.0801929928835\\
170	54.1030442760783\\
171	54.1029732532326\\
172	54.1138261346831\\
173	54.1180186196478\\
174	54.1281143963295\\
175	54.1452242207343\\
176	54.1350249217991\\
177	54.1400940877146\\
178	54.1617957940666\\
179	54.1689291803313\\
180	54.1761923955018\\
181	54.1960952192497\\
182	54.2003137711791\\
183	54.1907866536143\\
184	54.1765506207154\\
185	54.1964006418928\\
186	54.2099410933137\\
187	54.1883643108458\\
188	54.193447369706\\
189	54.191233446374\\
190	54.183224056176\\
191	54.1778254973115\\
192	54.1713232309643\\
193	54.1765869066993\\
194	54.1826085720866\\
195	54.1668011745827\\
196	54.1426626449539\\
197	54.1463622645322\\
198	54.1373926090772\\
199	54.1445632589344\\
200	54.1452828530244\\
201	54.150957020111\\
202	54.1355123039129\\
203	54.1266576583702\\
204	54.1218572846409\\
205	54.1195800648811\\
206	54.1251080679744\\
207	54.1088422207259\\
208	54.0785652655788\\
209	54.076506915865\\
210	54.0908941858386\\
211	54.0865723285144\\
212	54.0793901895898\\
213	54.0862939271847\\
214	54.097349882988\\
215	54.0631616906425\\
216	54.0597268372361\\
217	54.059942348342\\
218	54.0699992964529\\
219	54.0695419690024\\
220	54.0539612673069\\
221	54.0493084111528\\
222	54.0338592264748\\
223	54.0601709760941\\
224	54.0784495683766\\
225	54.0871125471689\\
226	54.0845054854985\\
227	54.0748009167583\\
228	54.0642867584507\\
229	54.0626842610033\\
230	54.0757227905939\\
231	54.0741691107038\\
232	54.071807334526\\
233	54.0839958848619\\
234	54.0878789155053\\
235	54.0782992594326\\
236	54.0752034805295\\
237	54.1027872066052\\
238	54.1133458969268\\
239	54.1176746369215\\
240	54.1194589491334\\
241	54.1200738693682\\
242	54.1199801628109\\
243	54.1321818332351\\
244	54.1528395339359\\
245	54.1723078984986\\
246	54.1522305064335\\
247	54.1486010265098\\
248	54.1842533383559\\
249	54.1534615856823\\
250	54.1294285613576\\
};
\addlegendentry{tank 2};

\end{axis}
\end{tikzpicture}%}
  \label{fig:}}

\subfigure[]{
  \scalebox{0.6}{% This file was created by matlab2tikz.
%
%The latest updates can be retrieved from
%  http://www.mathworks.com/matlabcentral/fileexchange/22022-matlab2tikz-matlab2tikz
%where you can also make suggestions and rate matlab2tikz.
%
\definecolor{mycolor1}{rgb}{0.00000,0.44700,0.74100}%
\definecolor{mycolor2}{rgb}{0.85000,0.32500,0.09800}%
%
\begin{tikzpicture}

\begin{axis}[%
width=4.133in,
height=3.26in,
at={(0.693in,0.44in)},
scale only axis,
xmin=0,
xmax=240,
xmajorgrids,
ymin=0,
ymax=45,
ymajorgrids,
axis background/.style={fill=white},
legend style={at={(0.699,0.674)},anchor=south west,legend cell align=left,align=left,draw=white!15!black}
]
\addplot [color=mycolor1,solid]
  table[row sep=crcr]{%
1	0\\
2	3.76633838757454\\
3	5.14234333743396\\
4	5.63668740932473\\
5	5.80745981196086\\
6	5.86206611083089\\
7	5.90077976059068\\
8	6.10764101106519\\
9	6.90815741959318\\
10	7.57350218519009\\
11	8.1038397372039\\
12	8.51758076593842\\
13	8.82590360335112\\
14	9.00327663771283\\
15	9.14481390591998\\
16	9.25245010681415\\
17	9.35975039122627\\
18	9.48131482117437\\
19	9.59852936376167\\
20	9.67924539002966\\
21	9.77577314254525\\
22	9.84605029893269\\
23	9.90175194200402\\
24	10.0159433134991\\
25	10.0752047019452\\
26	10.1602258609479\\
27	10.2426148890111\\
28	10.2989756203402\\
29	10.3316354604606\\
30	10.3775324633013\\
31	10.4377206831629\\
32	10.4834260397042\\
33	10.5036851548038\\
34	10.5192106975306\\
35	10.5545101321762\\
36	10.6230913280349\\
37	10.6912063338247\\
38	10.7643363968783\\
39	10.8276030684226\\
40	10.799662602376\\
41	10.8140893922164\\
42	10.8479377513453\\
43	10.8752287124064\\
44	10.887625901337\\
45	10.9018923677604\\
46	10.8968321639255\\
47	10.886831746613\\
48	10.9014986507462\\
49	10.9567859302747\\
50	10.9705624655425\\
51	10.9997694527905\\
52	11.0055991505711\\
53	11.0430506322698\\
54	11.0500956943329\\
55	11.0987406031253\\
56	11.1066119370491\\
57	11.0501613757597\\
58	11.0304237395509\\
59	11.0092764567395\\
60	11.0505224538626\\
61	11.090871381121\\
62	11.0721814739067\\
63	11.0795633577703\\
64	11.0678088309539\\
65	11.0530306652479\\
66	11.0603026137035\\
67	11.0654307112562\\
68	11.0811096692573\\
69	11.074018218388\\
70	11.061555952316\\
71	11.052418988108\\
72	11.0923255532259\\
73	11.1024009138241\\
74	11.1072093264171\\
75	11.1525710407931\\
76	11.1107641712435\\
77	11.1271767931201\\
78	11.1318544064682\\
79	11.1470563995691\\
80	11.1673759999558\\
81	11.1611672014713\\
82	11.1048671395422\\
83	11.1442647661021\\
84	11.1472613143412\\
85	11.1505377898784\\
86	11.1608636375954\\
87	11.13433106376\\
88	11.1026768121192\\
89	11.0831217906632\\
90	11.0580334084341\\
91	11.1119100291809\\
92	11.1570202268013\\
93	11.1290937255266\\
94	11.1262203384633\\
95	11.1450814442363\\
96	11.1848662095845\\
97	11.2136484688073\\
98	11.182901792858\\
99	11.1743850341833\\
100	11.1770278996182\\
101	11.1791781918698\\
102	11.1515885965855\\
103	11.1720596975491\\
104	11.2002063413836\\
105	11.1693045597236\\
106	11.1716287906276\\
107	11.1593064762604\\
108	11.1611479235359\\
109	11.1649501031862\\
110	11.1680288757737\\
111	11.1967302296223\\
112	11.1723324674143\\
113	11.1548643020898\\
114	11.1875552038073\\
115	11.1656308713644\\
116	11.1390188666809\\
117	11.1640112332621\\
118	11.1525876599567\\
119	11.1568216016794\\
120	11.1992481973952\\
121	11.1854201615096\\
122	11.1618990152024\\
123	11.1290859568232\\
124	11.1636885230578\\
125	11.1622367851611\\
126	11.1206374947646\\
127	11.1488115303569\\
128	11.1333769243049\\
129	11.1786833883625\\
130	11.1861774177614\\
131	11.2059511787034\\
132	11.1781911952377\\
133	11.1720186827593\\
134	11.2121862290706\\
135	11.1698571174463\\
136	11.1734252435369\\
137	11.2026946172107\\
138	11.2163976907577\\
139	11.2200746443974\\
140	11.2286800355437\\
141	11.1867865385419\\
142	11.1863385888442\\
143	11.2371670940413\\
144	11.2589943783501\\
145	11.2696188708184\\
146	11.2056203165936\\
147	11.1891872201572\\
148	11.23106567077\\
149	11.2760948443998\\
150	11.2551254627\\
151	11.2468058848124\\
152	11.1789893606881\\
153	11.2052305970507\\
154	11.2210130920763\\
155	11.2340989223257\\
156	11.2664167783617\\
157	11.21678327139\\
158	11.2205245848888\\
159	11.2266819208756\\
160	11.2273637788089\\
161	11.2291795514236\\
162	11.177367269043\\
163	11.1240048783642\\
164	11.165265509961\\
165	11.2042790370347\\
166	11.2036863979625\\
167	11.173776433503\\
168	11.1672512038382\\
169	11.1765346982693\\
170	11.1707653360927\\
171	11.1412755627999\\
172	11.1140743124064\\
173	11.167012170206\\
174	11.1599294129893\\
175	11.1705591736236\\
176	11.1471717306443\\
177	11.1575345513609\\
178	11.1605267596841\\
179	11.1654570632217\\
180	11.1324488487351\\
181	11.0400751648309\\
182	11.0430847205779\\
183	11.0983940599603\\
184	11.1566213975538\\
185	11.1083350686682\\
186	11.1301899255543\\
187	11.1472388048711\\
188	11.1604487361231\\
189	11.2083138844049\\
190	11.1784982933121\\
191	11.1511123418069\\
192	11.0963342469699\\
193	11.1280709867102\\
194	11.1401721936498\\
195	11.1515171213318\\
196	11.1654555841267\\
197	11.1329157943835\\
198	11.1572737554627\\
199	11.1172844809754\\
200	11.0657199511384\\
201	11.1112695874101\\
202	11.1240357320943\\
203	11.1627825455228\\
204	11.1893124151609\\
205	11.1228323798335\\
206	11.1306065201387\\
207	11.1751724158144\\
208	11.1457613238016\\
209	11.1362402137805\\
210	11.1657833455035\\
211	11.1314481946554\\
212	11.1301480237903\\
213	11.1509282720123\\
214	11.113769004964\\
215	11.1342314752809\\
216	11.1309011979545\\
217	11.1244839051226\\
218	11.1428181061238\\
219	11.1522736418215\\
220	11.1672926471918\\
221	11.1643023681448\\
222	11.1634484203772\\
223	11.1669585148221\\
224	11.2080464144399\\
225	11.1804151905422\\
226	11.1880672077114\\
227	11.2208070092739\\
228	11.257429105862\\
229	11.2380728820909\\
230	11.2397088805115\\
231	11.2396492722994\\
232	11.2966256146214\\
233	11.272329748509\\
234	11.2971392532466\\
235	11.2977313870224\\
236	11.3036247175084\\
237	11.2872633142225\\
238	11.2360645981694\\
239	11.2666193225798\\
240	11.2835628598278\\
241	11.3126905496722\\
242	11.2973871129075\\
243	11.3442354543988\\
244	11.3338177247374\\
};
\addlegendentry{tank 1};

\addplot [color=mycolor2,solid]
  table[row sep=crcr]{%
1	0\\
2	11.1505660446702\\
3	15.2494092238755\\
4	16.7527997270429\\
5	17.3030192718119\\
6	17.5059402822555\\
7	17.5726770427157\\
8	17.7519137090544\\
9	17.9861133100196\\
10	18.0164838530066\\
11	18.1284845815357\\
12	18.139784663922\\
13	17.9744724502155\\
14	18.038585875865\\
15	18.2169531814271\\
16	18.3445756808047\\
17	18.3888631589461\\
18	18.5778215595325\\
19	18.809605046121\\
20	19.0263651174243\\
21	19.2726499154031\\
22	19.4708095418435\\
23	19.6753781423627\\
24	19.8643280508743\\
25	20.1164362869645\\
26	20.4266088843456\\
27	20.6364301860934\\
28	20.8753737416767\\
29	21.1674273561181\\
30	21.4159857589374\\
31	21.7457189961108\\
32	22.0387222715011\\
33	22.3315740398801\\
34	22.5851772230225\\
35	22.8810028803529\\
36	23.1304177918666\\
37	23.3798739120105\\
38	23.7057628780568\\
39	23.944756867253\\
40	24.2263683349418\\
41	24.4894642262098\\
42	24.7484705088183\\
43	25.0321053586228\\
44	25.3801705485972\\
45	25.6714976995239\\
46	25.9322769077876\\
47	26.1876594468054\\
48	26.4184914408994\\
49	26.6489090530148\\
50	26.9676587707303\\
51	27.1177100434492\\
52	27.4023132902227\\
53	27.6530185530078\\
54	27.9774450455797\\
55	28.1383606925278\\
56	28.418803418524\\
57	28.663598233449\\
58	28.844179235775\\
59	29.2028335053298\\
60	29.3505713581648\\
61	29.6025436560193\\
62	29.7752415449803\\
63	30.0440364676799\\
64	30.2923924622946\\
65	30.5990884358256\\
66	30.8198115603084\\
67	30.9703442242346\\
68	31.3155069259871\\
69	31.4422638724665\\
70	31.5850287118074\\
71	31.7287591283612\\
72	31.8600117626579\\
73	32.0284686861481\\
74	32.2344035110965\\
75	32.4616295243449\\
76	32.6484927597902\\
77	32.7220874826323\\
78	33.0056275900757\\
79	33.1298205277022\\
80	33.257221307089\\
81	33.384936864449\\
82	33.6116277281436\\
83	33.8242254977402\\
84	33.9315477513898\\
85	33.9705124566836\\
86	34.1708317452594\\
87	34.2206629652804\\
88	34.4244575450887\\
89	34.6438897437035\\
90	34.8013208995201\\
91	35.0559591356194\\
92	35.0195521234829\\
93	35.1781861881363\\
94	35.2554126794338\\
95	35.3084915066524\\
96	35.4173755135869\\
97	35.5950446526966\\
98	35.7166471184536\\
99	35.8157499724534\\
100	35.8990257913424\\
101	36.0386448590617\\
102	36.1145446646222\\
103	36.3064020597658\\
104	36.4109765960492\\
105	36.4817903869302\\
106	36.4880285715212\\
107	36.6237074301517\\
108	36.827673370748\\
109	36.9146805959821\\
110	36.9519795880555\\
111	37.0104362179583\\
112	37.1515625930911\\
113	37.2618973811939\\
114	37.3138745693838\\
115	37.3675011415621\\
116	37.3925941544703\\
117	37.457473590457\\
118	37.5967185327463\\
119	37.6367703480027\\
120	37.7251378513394\\
121	37.7855750392223\\
122	37.9494972551239\\
123	38.0499140277312\\
124	38.1386588583861\\
125	38.0133560525495\\
126	38.0774183165788\\
127	38.1759269935984\\
128	38.2188899725841\\
129	38.3076276096873\\
130	38.3578124159028\\
131	38.379489035398\\
132	38.5228072120472\\
133	38.4847677644498\\
134	38.5481621405734\\
135	38.6010587079488\\
136	38.6248282630011\\
137	38.6819751912986\\
138	38.7237820282189\\
139	38.7630994655747\\
140	38.7488968186255\\
141	38.7817005510003\\
142	38.8557100509191\\
143	38.8829618776207\\
144	38.954737935238\\
145	38.9614231288744\\
146	39.0317726580459\\
147	39.1107605698752\\
148	39.1658679043164\\
149	39.3492543279478\\
150	39.3088084997523\\
151	39.3228411413885\\
152	39.2927944525557\\
153	39.2614001391926\\
154	39.3554186615469\\
155	39.348438337117\\
156	39.4757597407815\\
157	39.5203352428816\\
158	39.4754759062483\\
159	39.5609293418529\\
160	39.6561261391529\\
161	39.6159039672146\\
162	39.7061743907332\\
163	39.6907664841518\\
164	39.6946917749948\\
165	39.7500898858827\\
166	39.7586264203357\\
167	39.766807422415\\
168	39.721196425741\\
169	39.748512267189\\
170	39.819128685796\\
171	39.8513078133219\\
172	39.8489963502713\\
173	39.9743912375271\\
174	39.9600630057828\\
175	40.0277178525904\\
176	40.0088307172002\\
177	40.0556777047846\\
178	40.1691337244626\\
179	40.140384146553\\
180	40.0704634558113\\
181	40.0750051308122\\
182	40.1152983388487\\
183	40.0649687176762\\
184	40.1007092494332\\
185	40.1168982477766\\
186	40.2582010150749\\
187	40.3506023616517\\
188	40.4177091055688\\
189	40.3840207544833\\
190	40.4112081208426\\
191	40.4700712987472\\
192	40.5183097854357\\
193	40.5538458339805\\
194	40.5015705659713\\
195	40.5099559779099\\
196	40.4779792909883\\
197	40.4533306652016\\
198	40.5008771628315\\
199	40.5554016597492\\
200	40.6560602792693\\
201	40.6415698129757\\
202	40.5808945890927\\
203	40.5488436950634\\
204	40.7081730246016\\
205	40.7263160743604\\
206	40.7540095350317\\
207	40.6726618405329\\
208	40.6845561335397\\
209	40.5839420747951\\
210	40.5708429958501\\
211	40.630941798202\\
212	40.6329257381257\\
213	40.6596800756979\\
214	40.6161620678059\\
215	40.621482747804\\
216	40.6594634350147\\
217	40.7733925221964\\
218	40.8342760880004\\
219	40.7932122146657\\
220	40.7771071690871\\
221	40.7031494587222\\
222	40.6901300039444\\
223	40.8122935365339\\
224	40.8185183233248\\
225	40.7872511004091\\
226	40.7803231850139\\
227	40.7877124561434\\
228	40.8318936799873\\
229	40.7787250102106\\
230	40.7732645502096\\
231	40.7843060581062\\
232	40.7924331488035\\
233	40.8471869041999\\
234	40.8772792783104\\
235	40.9603193344166\\
236	40.8785008990885\\
237	40.8706801224796\\
238	40.8943047434731\\
239	40.9871085877829\\
240	41.0565969856584\\
241	41.0492631081998\\
242	41.0744022307786\\
243	41.1933752438175\\
244	41.0994762051926\\
};
\addlegendentry{tank 2};

\end{axis}
\end{tikzpicture}%
}
  \label{fig:}}
\quad
\subfigure[]{
  \scalebox{0.6}{% This file was created by matlab2tikz.
%
%The latest updates can be retrieved from
%  http://www.mathworks.com/matlabcentral/fileexchange/22022-matlab2tikz-matlab2tikz
%where you can also make suggestions and rate matlab2tikz.
%
\definecolor{mycolor1}{rgb}{0.00000,0.44700,0.74100}%
\definecolor{mycolor2}{rgb}{0.85000,0.32500,0.09800}%
%
\begin{tikzpicture}

\begin{axis}[%
width=4.133in,
height=3.26in,
at={(0.693in,0.44in)},
scale only axis,
xmin=0,
xmax=240,
xmajorgrids,
ymin=0,
ymax=55,
ymajorgrids,
axis background/.style={fill=white},
legend style={at={(0.691,0.658)},anchor=south west,legend cell align=left,align=left,draw=white!15!black}
]
\addplot [color=mycolor1,solid]
  table[row sep=crcr]{%
1	0\\
2	3.63116087879238\\
3	4.96973635197563\\
4	5.46199089039449\\
5	5.64168447186807\\
6	5.7149665684148\\
7	5.82706158731218\\
8	6.3286960070718\\
9	6.91922813040444\\
10	7.40757330661964\\
11	7.81433697370657\\
12	8.32531755440929\\
13	8.84395553646031\\
14	9.34477082552388\\
15	9.89955046123439\\
16	10.4801305105973\\
17	11.072758705107\\
18	11.6823381385993\\
19	12.3488303823138\\
20	12.8738687359413\\
21	13.4506908099127\\
22	14.1033031601578\\
23	14.7181028266807\\
24	15.2655844628944\\
25	15.8968540090452\\
26	16.5276458980112\\
27	17.1924779205287\\
28	17.8252171503053\\
29	18.4988413839422\\
30	19.1151234186592\\
31	19.7757357767171\\
32	20.4226838261379\\
33	21.0486573289695\\
34	21.6234874450378\\
35	22.2614925041241\\
36	22.9126312435761\\
37	23.6112708177909\\
38	24.0408983403539\\
39	24.6057707425986\\
40	25.1936089798978\\
41	25.8191040169238\\
42	26.2764508966357\\
43	26.8058482129216\\
44	27.3517795771818\\
45	27.8803268474071\\
46	28.4931809009651\\
47	28.9846324234619\\
48	29.5152329189153\\
49	30.0089770716511\\
50	30.4493956377957\\
51	30.8786489415131\\
52	31.3644293725102\\
53	31.7938862407145\\
54	32.25516007243\\
55	32.7501072655868\\
56	33.222081067771\\
57	33.58249181335\\
58	33.9479870365025\\
59	34.4207155525252\\
60	34.842323093346\\
61	35.2218972925613\\
62	35.6366389381846\\
63	35.9490877947543\\
64	36.3614626157201\\
65	36.7442340896327\\
66	37.1321869883948\\
67	37.4306579858368\\
68	37.7632017936634\\
69	38.1225977960446\\
70	38.3971132181605\\
71	38.6488474112129\\
72	39.006104755343\\
73	39.3362841256057\\
74	39.6271517393434\\
75	39.9161241570385\\
76	40.2119001492279\\
77	40.5194587399314\\
78	40.802122668815\\
79	41.0581184900425\\
80	41.3188950837293\\
81	41.5659289429649\\
82	41.7439964233387\\
83	41.9742991439031\\
84	42.1660798553164\\
85	42.3504132916587\\
86	42.6169473223889\\
87	42.8670433482282\\
88	43.1281190905789\\
89	43.3487470070778\\
90	43.585213774281\\
91	43.7527759065593\\
92	43.9226254950669\\
93	44.0898234734385\\
94	44.3294156096507\\
95	44.5305046840231\\
96	44.7117343297711\\
97	44.8602535733036\\
98	44.9790804906941\\
99	45.1556364732023\\
100	45.3498087678109\\
101	45.5271234852602\\
102	45.6795557792301\\
103	45.7876524121688\\
104	45.9261842850715\\
105	45.9795557719044\\
106	46.1000070440059\\
107	46.2421840961192\\
108	46.3263349848195\\
109	46.458730989013\\
110	46.6135883382812\\
111	46.6880255720024\\
112	46.7890911255688\\
113	46.8845218230433\\
114	47.0171239430569\\
115	47.113134316619\\
116	47.244620960708\\
117	47.3237654703942\\
118	47.4379183706389\\
119	47.5509203799653\\
120	47.6677059538873\\
121	47.7088316615913\\
122	47.8269857906068\\
123	47.9038129860554\\
124	48.0636089890847\\
125	48.2490027576106\\
126	48.3552138814542\\
127	48.3651551014421\\
128	48.3619042783741\\
129	48.4562148456078\\
130	48.5449557530335\\
131	48.6308331957321\\
132	48.7259446110937\\
133	48.7793980446805\\
134	48.9634010470055\\
135	49.1217271228499\\
136	49.2448596757486\\
137	49.2703557022816\\
138	49.3793207547068\\
139	49.4439133830976\\
140	49.542851794307\\
141	49.6409430424568\\
142	49.6793173372598\\
143	49.7004503607335\\
144	49.7994132168799\\
145	49.8491915700133\\
146	49.9089722328766\\
147	49.922765615738\\
148	49.9640112224557\\
149	50.0439148564463\\
150	50.0676609885333\\
151	50.1298827525931\\
152	50.189734725475\\
153	50.2767836568894\\
154	50.2736017686236\\
155	50.3307911287605\\
156	50.412045602802\\
157	50.440086968687\\
158	50.4399863763517\\
159	50.5092087907992\\
160	50.5570351772524\\
161	50.598468887969\\
162	50.6358735165248\\
163	50.6553386681615\\
164	50.6675228791648\\
165	50.7580754835991\\
166	50.8375083892045\\
167	50.8797722833522\\
168	50.9307925861167\\
169	50.9188168825522\\
170	50.9489250703127\\
171	51.0063517011936\\
172	50.9601646938151\\
173	51.0221912853412\\
174	51.0403955995426\\
175	51.0671046970741\\
176	51.1098086893835\\
177	51.1267285336378\\
178	51.1828729251925\\
179	51.1669844818841\\
180	51.1791070583144\\
181	51.2346015012219\\
182	51.2745652113421\\
183	51.3268463963618\\
184	51.3190222607213\\
185	51.3793470191358\\
186	51.4061071462445\\
187	51.398898767575\\
188	51.3872872111205\\
189	51.41267367231\\
190	51.4309791701341\\
191	51.4611787321754\\
192	51.5525501726606\\
193	51.5564730469372\\
194	51.5493164888538\\
195	51.5237618345481\\
196	51.5265935384152\\
197	51.5095123702692\\
198	51.5403833850393\\
199	51.5478483290726\\
200	51.5675447835311\\
201	51.5778080340932\\
202	51.5678891197534\\
203	51.5874734827876\\
204	51.5969768990554\\
205	51.6228613184686\\
206	51.6152730268232\\
207	51.6240403151134\\
208	51.5719605905085\\
209	51.6231141350285\\
210	51.5934964435285\\
211	51.5718660363986\\
212	51.5983319938196\\
213	51.5963750552611\\
214	51.5850451444972\\
215	51.599972258205\\
216	51.6278483232124\\
217	51.6488384310073\\
218	51.6858108724367\\
219	51.6839808862892\\
220	51.6639846095338\\
221	51.6454273852521\\
222	51.6738567805247\\
223	51.7029354965346\\
224	51.6631655056257\\
225	51.5786660282843\\
226	51.5596922545796\\
227	51.618479341536\\
228	51.6136854654788\\
229	51.6194942812702\\
230	51.643023649265\\
231	51.6437290367765\\
232	51.6525257061824\\
233	51.6795851386675\\
234	51.6902715302554\\
235	51.6786066133105\\
236	51.72754880148\\
237	51.7504306338977\\
238	51.7548444243933\\
239	51.725794661712\\
240	51.7424355867725\\
241	51.7308726682766\\
242	51.7471688736871\\
243	51.7658745738019\\
244	51.7715633382855\\
245	51.7559368683858\\
246	51.7659487620931\\
247	51.7540676709942\\
248	51.7228105923085\\
249	51.7096588463521\\
250	51.7689051914563\\
251	51.7807753892999\\
252	51.7899983224873\\
253	51.8256025555482\\
254	51.8291951317478\\
255	51.8659330911394\\
256	51.8296699914552\\
257	51.8406323068885\\
258	51.8330512878584\\
259	51.835448165543\\
260	51.8236651991413\\
261	51.8291428730445\\
262	51.8114706763354\\
263	51.8536617617664\\
264	51.875472742324\\
265	51.8747072833899\\
266	51.9076402414655\\
267	51.8657138057502\\
268	51.9187992743105\\
269	51.9110575022943\\
270	51.873530600166\\
271	51.8945539489128\\
272	51.8791225401948\\
273	51.8885378531862\\
274	51.9306459452034\\
275	51.9252937468409\\
276	51.9404870507474\\
277	51.9705720850293\\
278	51.9591288707795\\
279	51.9461016609353\\
280	51.9625372170412\\
281	51.9737220555714\\
282	51.9860353984504\\
283	52.0221914380431\\
284	52.0546423948317\\
285	52.0765877162151\\
286	52.0856424430892\\
287	52.0228409456681\\
288	51.9564450040403\\
289	51.949119821029\\
290	51.9396622538493\\
291	51.9581477539362\\
292	52.0624867923732\\
293	52.1100957899362\\
294	52.1352578708593\\
295	52.1737060738375\\
296	52.2018354976863\\
297	52.2119450602601\\
298	52.2438443598994\\
299	52.2217303763627\\
300	52.2233507006274\\
301	52.2168054416964\\
302	52.2094681920216\\
303	52.1756052738009\\
304	52.148768859449\\
305	52.1640608278111\\
306	52.1976465805812\\
307	52.229518257589\\
308	52.2536280456787\\
309	52.2861665932444\\
310	52.287566444998\\
311	52.2775185609343\\
312	52.2852631635205\\
313	52.2583717859571\\
314	52.2576783251643\\
315	52.2340068856786\\
316	52.2485950091166\\
317	52.2682528882137\\
318	52.2381480248414\\
319	52.2220657709836\\
320	52.2224044797715\\
321	52.2801696854136\\
322	52.2911215804366\\
323	52.2844401797584\\
324	52.2717596472629\\
325	52.2870061993753\\
326	52.2994416449586\\
327	52.3053793422807\\
328	52.3196194970723\\
329	52.2966603664795\\
330	52.2813264977102\\
331	52.2656287461455\\
332	52.2261175569035\\
333	52.2509788521447\\
334	52.2599519105326\\
};
\addlegendentry{tank 1};

\addplot [color=mycolor2,solid]
  table[row sep=crcr]{%
1	0\\
2	11.105575983619\\
3	15.2121342940417\\
4	16.7204101575441\\
5	17.2750640683042\\
6	17.4695366195987\\
7	17.5169509986413\\
8	17.5082513675712\\
9	17.6366840533709\\
10	18.0364010604742\\
11	18.5222673366665\\
12	18.9683357348788\\
13	19.2879326430269\\
14	19.4209835269529\\
15	19.463994020685\\
16	19.4946704721206\\
17	19.4923426216576\\
18	19.4858689737496\\
19	19.4503557257572\\
20	19.4620745389308\\
21	19.4071842902253\\
22	19.4203626513122\\
23	19.4020191002292\\
24	19.3433401746487\\
25	19.3770210879218\\
26	19.3539284156878\\
27	19.2798450529038\\
28	19.3010202937266\\
29	19.2982460308702\\
30	19.267528974226\\
31	19.2389836450236\\
32	19.2849156946111\\
33	19.2724861274846\\
34	19.2667211179798\\
35	19.2706653888425\\
36	19.288706007576\\
37	19.2733415034806\\
38	19.2669259399485\\
39	19.2875208850975\\
40	19.2930164870237\\
41	19.2496582293735\\
42	19.197024859754\\
43	19.2383153707493\\
44	19.273544067097\\
45	19.2689350576652\\
46	19.2707501050672\\
47	19.2386347255198\\
48	19.2314430616964\\
49	19.2275680284352\\
50	19.2086753378982\\
51	19.1850407795649\\
52	19.1794480256131\\
53	19.205220540755\\
54	19.1856614196272\\
55	19.1573454456557\\
56	19.1361252838611\\
57	19.1190213465267\\
58	19.1173831057237\\
59	19.1146445639102\\
60	19.0624740607197\\
61	19.0876929424951\\
62	19.1985323716589\\
63	19.1930953480607\\
64	19.1583102132397\\
65	19.2096467834962\\
66	19.2117773577812\\
67	19.243487289818\\
68	19.2101912177807\\
69	19.1365219980476\\
70	19.1489406696737\\
71	19.168841374532\\
72	19.1430186319715\\
73	19.1572199761247\\
74	19.1801956795511\\
75	19.1679633122007\\
76	19.1441517543445\\
77	19.2050468936952\\
78	19.1865183650462\\
79	19.1622371213517\\
80	19.1547973349057\\
81	19.1220920112427\\
82	19.1364341759777\\
83	19.1219733824997\\
84	19.1340004631756\\
85	19.1574738363153\\
86	19.1498482887576\\
87	19.1724427397196\\
88	19.172899034659\\
89	19.1475234144468\\
90	19.1269625567324\\
91	19.0954993589796\\
92	19.1282553937381\\
93	19.1131748096838\\
94	19.1037732436272\\
95	19.1386324274876\\
96	19.1514308496001\\
97	19.1904250857745\\
98	19.0940279463096\\
99	19.0461134137757\\
100	18.9807542687565\\
101	18.9500282678514\\
102	19.0042314995601\\
103	19.0519426009453\\
104	19.1001110723497\\
105	19.121091766329\\
106	19.1172980732345\\
107	19.1677334865748\\
108	19.129959582728\\
109	19.0958616932221\\
110	19.1212464113393\\
111	19.1312653214536\\
112	19.1165437314518\\
113	19.0709070295193\\
114	19.1288743648032\\
115	19.117645006153\\
116	19.0563005536209\\
117	19.0559177460093\\
118	19.0766436744251\\
119	19.1088961096544\\
120	19.1235480287891\\
121	19.1285911500594\\
122	19.0965637387484\\
123	19.0922713232391\\
124	19.1342411811731\\
125	19.1254636400216\\
126	19.1400822934418\\
127	19.155203613465\\
128	19.1585941070463\\
129	19.1624398816837\\
130	19.1268997963623\\
131	19.0695151589617\\
132	19.0330128169272\\
133	19.0763835833041\\
134	19.0498467456071\\
135	19.0316212129083\\
136	19.002527864462\\
137	18.9679419227242\\
138	18.9487689141636\\
139	18.9344651402588\\
140	18.9243632105235\\
141	18.9360576690331\\
142	18.9392922100707\\
143	18.9459803183672\\
144	18.9685628201686\\
145	18.9651752544888\\
146	18.9569022221039\\
147	18.9445345443682\\
148	18.9362783194667\\
149	18.9265092836379\\
150	18.8842723459206\\
151	18.9164485293608\\
152	18.9335736187932\\
153	18.9340501191287\\
154	18.9287732390624\\
155	18.9164455089967\\
156	18.941277007197\\
157	18.9414130999207\\
158	18.9516353448206\\
159	18.9517284100898\\
160	18.9417134298876\\
161	18.9516236400364\\
162	18.9553626751002\\
163	18.9665225731024\\
164	18.9562193727365\\
165	18.9459755033541\\
166	18.9566973102264\\
167	18.9572014471248\\
168	18.9436076770743\\
169	18.9666304376867\\
170	18.9934080875468\\
171	19.0069035724976\\
172	18.959015232729\\
173	18.9287163414929\\
174	18.9146522710363\\
175	18.9295428788643\\
176	18.9309841580942\\
177	18.9611945517596\\
178	18.9325298090098\\
179	18.940598815245\\
180	18.9581964021725\\
181	18.9482581269123\\
182	18.9163390725996\\
183	18.9198675190102\\
184	18.9311542162495\\
185	18.9598221899085\\
186	19.0145011286805\\
187	19.0579912409478\\
188	19.0609603333751\\
189	19.058370643543\\
190	19.015561987923\\
191	18.9690162470203\\
192	18.9527703359935\\
193	18.9776704050532\\
194	19.0278567853202\\
195	19.0591030787649\\
196	19.0819246151698\\
197	19.1040463085311\\
198	19.1068963761578\\
199	19.0494719371099\\
200	19.0154264349455\\
201	19.0772326283731\\
202	19.0945620718557\\
203	19.1441870080019\\
204	19.1472078837876\\
205	19.1409335411125\\
206	19.1159572002434\\
207	19.1066129053857\\
208	19.0891876291023\\
209	19.0570045346749\\
210	19.0619158622278\\
211	19.1204684591652\\
212	19.0967257966711\\
213	19.0370478214721\\
214	19.0264958122601\\
215	19.077698846195\\
216	19.0820299823002\\
217	19.1434485667504\\
218	19.155006557663\\
219	19.1579213843585\\
220	19.1397150366289\\
221	19.145746556312\\
222	19.1399824305354\\
223	19.1451365837211\\
224	19.1006096964731\\
225	19.0471732534803\\
226	19.0271904153692\\
227	19.0273223310928\\
228	18.9812514615784\\
229	18.9634151699115\\
230	18.99367340204\\
231	19.0422918884101\\
232	19.0837005502804\\
233	19.1126249988927\\
234	19.102455664601\\
235	19.1252163523378\\
236	19.0888635371486\\
237	19.0266550299853\\
238	19.0102905892227\\
239	19.0081160050531\\
240	19.0140902764607\\
241	19.013269467275\\
242	19.0516458927188\\
243	19.048663253063\\
244	19.0355912598031\\
245	19.0463529562724\\
246	19.0876874482389\\
247	19.0880696177576\\
248	19.0856155752666\\
249	19.0729466398997\\
250	19.0315209803195\\
251	19.0254107982221\\
252	19.0004611966509\\
253	18.9818366049798\\
254	18.9452204039162\\
255	18.9872080926451\\
256	18.9942524736045\\
257	18.9684402993684\\
258	18.9531432441897\\
259	18.9702267725905\\
260	18.9814909378275\\
261	18.9801188983053\\
262	18.9463942137824\\
263	18.9245830782\\
264	18.9113665889652\\
265	18.8807848287178\\
266	18.8828899373016\\
267	18.8646145043534\\
268	18.8677962654774\\
269	18.8808747117128\\
270	18.9039246086762\\
271	18.8897635248251\\
272	18.8972700724215\\
273	18.9078384426525\\
274	18.8925426057717\\
275	18.8833178168819\\
276	18.9109481360093\\
277	18.913643440208\\
278	18.9478631839998\\
279	18.9524071716366\\
280	18.9431122246997\\
281	18.9157622452446\\
282	18.8793336972133\\
283	18.8914076553859\\
284	18.9096355417391\\
285	18.9015311079447\\
286	18.8726769525389\\
287	18.8602591031553\\
288	18.9051356206539\\
289	18.9182082053927\\
290	18.9216348778712\\
291	18.9138627441331\\
292	18.9093173965851\\
293	18.9081026383023\\
294	18.894307182597\\
295	18.8762931204112\\
296	18.9028373605558\\
297	18.9258495193378\\
298	18.8862553864787\\
299	18.8722163564038\\
300	18.8871871284849\\
301	18.8953993197369\\
302	18.919819955817\\
303	18.9522898712588\\
304	18.920332811783\\
305	18.9042052906359\\
306	18.9572061251658\\
307	18.94910437063\\
308	18.9205190121619\\
309	18.8876692798763\\
310	18.8732739804053\\
311	18.8853500239622\\
312	18.8982088851023\\
313	18.8842878575687\\
314	18.8892932113547\\
315	18.8842551114197\\
316	18.9073822101963\\
317	18.9072579878619\\
318	18.8836180023493\\
319	18.8882745986253\\
320	18.8876840992567\\
321	18.9062533232758\\
322	18.9226880504535\\
323	18.9139839747656\\
324	18.9227773202343\\
325	18.8828369938636\\
326	18.9063051477039\\
327	18.926605287035\\
328	18.9232440733694\\
329	18.9211782003503\\
330	18.9049436138699\\
331	18.902128086544\\
332	18.8960739285532\\
333	18.8951769939014\\
334	18.8891659261569\\
};
\addlegendentry{tank 2};

\end{axis}
\end{tikzpicture}%
}
  \label{fig:}}

\end{multicols}
\caption{Step responses for the minimum phase case (left) and non-minumum phase
  case (right). The input step was set to $0.5 u_{max} = 7.5V$. The vertical
  axes show the height of water in each tank in percentage of its maximum height
  value. Figures (a) and (c) correspond to input 1 being ON, while figures
  (b) and (d) correspond to input 2 being ON.}
\label{fig:222}
\end{figure}

The system is indeed coupled. In accordance to their respective RGA matrices,
we can see that, in the minimum phase case, input $i$ drives tank $i$ but is
coupled to output $j$, while, in the non-minimum phase case, input $i$ drives
tank $j$ but is coupled to output $i$.


% ------------------------------------------------------------------------------
\subsection*{Exercise 2.2.3}

We set the reference levels to $50\%$ of the maximum height for tanks 1 and 2.
In the minimum phase case we found it easier to control the system, and the
transient, in our experiment, was roughly 500 seconds. Contrary to this, the
non-minimum phase case is much harder to control manually, if it is actually
feasible to do so. In the end we could not reach our goal, however, in any case
it is reasonable to assume that its transient would be more than in the
minimum phase case.


% ------------------------------------------------------------------------------
\subsection*{Exercise 2.2.4}

Among the most important differences between the minimum and non-minimum phase
cases are the actual physical setup of the plant for each case, the fact that
the latter is more difficult to control by hand, and more difficult to obtain
an intuitive understanding due to the increased complexity of the intrinsics
of the physical process itself.
