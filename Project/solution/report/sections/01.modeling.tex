%-------------------------------------------------------------------------------
\subsection*{Exercise 2.1.1}

From the physical setup of the process, the following hold:

\begin{align*}
  A_1\dfrac{dh_1}{dt} &= q_{L,1} + q_{out,3} - q_{out_1}  \\
  A_2\dfrac{dh_2}{dt} &= q_{L,2} + q_{out,4} - q_{out_2}  \\
  A_3\dfrac{dh_3}{dt} &= q_{U,1} - q_{out,3} \\
  A_4\dfrac{dh_4}{dt} &= q_{U,2} - q_{out,4}
\end{align*}

where $q_{\{L,U\}, j}$ is the flow directed from pump $j$ to the lower or upper
tank connected to valve $j$, and $q_{out,i}$ is the outflow of tank $i$. Since

\begin{align*}
  q_{L,j} &= \gamma_j k_j u_j  \\
  q_{U,j} &= (1-\gamma_j) k_j u_j  \\
  q_{out,i} &= a_i \sqrt{2gh_i}
\end{align*}

the system of differential equations becomes

\begin{align*}
  \dfrac{dh_1}{dt} &= -\dfrac{a_1}{A_1}\sqrt{2gh_1} + \dfrac{a_3}{A_1}\sqrt{2gh_3} + \dfrac{\gamma_1 k_1}{A_1}u_1 \\
  \dfrac{dh_2}{dt} &= -\dfrac{a_2}{A_2}\sqrt{2gh_2} + \dfrac{a_4}{A_2}\sqrt{2gh_4} + \dfrac{\gamma_2 k_2}{A_2}u_2 \\
  \dfrac{dh_3}{dt} &= -\dfrac{a_3}{A_3}\sqrt{2gh_3} + \dfrac{(1-\gamma_2)k_2}{A_3}u_2 \\
  \dfrac{dh_4}{dt} &= -\dfrac{a_4}{A_4}\sqrt{2gh_4} + \dfrac{(1-\gamma_1)k_1}{A_4}u_1 \\
\end{align*}



%-------------------------------------------------------------------------------
\subsection*{Exercise 2.1.2}

By definition, in an equilibrium all derivatives $\dfrac{dh_i}{dt}$ are zero:

\begin{align*}
  -\dfrac{a_1}{A_1}\sqrt{2gh_1^0} + \dfrac{a_3}{A_1}\sqrt{2gh_3^0} + \dfrac{\gamma_1 k_1}{A_1}u_1^0 &= 0\\
  -\dfrac{a_2}{A_2}\sqrt{2gh_2^0} + \dfrac{a_4}{A_2}\sqrt{2gh_4^0} + \dfrac{\gamma_2 k_2}{A_2}u_2^0 &= 0\\
  -\dfrac{a_3}{A_3}\sqrt{2gh_3^0} + \dfrac{(1-\gamma_2)k_2}{A_3}u_2^0 &= 0\\
  -\dfrac{a_4}{A_4}\sqrt{2gh_4^0} + \dfrac{(1-\gamma_1)k_1}{A_4}u_1^0 &= 0\\
  y_i^0 = k_c h_i^0, i &= \{1,2,3,4\}
\end{align*}

where $h_i^0, u_i^0$ and $y_i^0$ denote steady-state values of their
corresponding variables.



%-------------------------------------------------------------------------------
\subsection*{Exercise 2.1.3}

Denoting the deviations from equilibrium with $\Delta u_i = u_i - u_i^0$,
$\Delta h_i = h_i - h_i^0, \Delta y_i = y_i - y_i^0$, and introducing the
vectors

\[
u=
\begin{bmatrix}
  \Delta u_1  \\
  \Delta u_2
\end{bmatrix},
x =
\begin{bmatrix}
\Delta h_1 \\
\Delta h_2 \\
\Delta h_3 \\
\Delta h_4
\end{bmatrix},
y =
\begin{bmatrix}
  \Delta y_1  \\
  \Delta y_2
\end{bmatrix}
\]

we can linearize around the steady-state operating point by performing a Taylor
series expansion. Neglecting the high-order terms, the system is now expressed by

\begin{align*}
  \dot{x} &= Ax + Bu  \\
   y &= Cx + Du
\end{align*}

where

\[
A =
\begin{bmatrix}
  \dfrac{\partial \Delta h_1}{\partial h_1} \bigg|_{h_i^0, u_i^0} &
  \dfrac{\partial \Delta h_1}{\partial h_2} \bigg|_{h_i^0, u_i^0} &
  \dfrac{\partial \Delta h_1}{\partial h_3} \bigg|_{h_i^0, u_i^0} &
  \dfrac{\partial \Delta h_1}{\partial h_4} \bigg|_{h_i^0, u_i^0} \\

  \dfrac{\partial \Delta h_2}{\partial h_1} \bigg|_{h_i^0, u_i^0} &
  \dfrac{\partial \Delta h_2}{\partial h_2} \bigg|_{h_i^0, u_i^0} &
  \dfrac{\partial \Delta h_2}{\partial h_3} \bigg|_{h_i^0, u_i^0} &
  \dfrac{\partial \Delta h_2}{\partial h_4} \bigg|_{h_i^0, u_i^0} \\

  \dfrac{\partial \Delta h_3}{\partial h_1} \bigg|_{h_i^0, u_i^0} &
  \dfrac{\partial \Delta h_3}{\partial h_2} \bigg|_{h_i^0, u_i^0} &
  \dfrac{\partial \Delta h_3}{\partial h_3} \bigg|_{h_i^0, u_i^0} &
  \dfrac{\partial \Delta h_3}{\partial h_4} \bigg|_{h_i^0, u_i^0} \\

  \dfrac{\partial \Delta h_4}{\partial h_1} \bigg|_{h_i^0, u_i^0} &
  \dfrac{\partial \Delta h_4}{\partial h_2} \bigg|_{h_i^0, u_i^0} &
  \dfrac{\partial \Delta h_4}{\partial h_3} \bigg|_{h_i^0, u_i^0} &
  \dfrac{\partial \Delta h_4}{\partial h_4} \bigg|_{h_i^0, u_i^0} \\
\end{bmatrix}
\]

\[
B =
\begin{bmatrix}
  \dfrac{\partial \Delta h_1}{\partial u_1} \bigg|_{h_i^0, u_i^0} &
  \dfrac{\partial \Delta h_1}{\partial u_2} \bigg|_{h_i^0, u_i^0} \\

  \dfrac{\partial \Delta h_2}{\partial u_1} \bigg|_{h_i^0, u_i^0} &
  \dfrac{\partial \Delta h_2}{\partial u_2} \bigg|_{h_i^0, u_i^0} \\

  \dfrac{\partial \Delta h_3}{\partial u_1} \bigg|_{h_i^0, u_i^0} &
  \dfrac{\partial \Delta h_3}{\partial u_2} \bigg|_{h_i^0, u_i^0} \\

  \dfrac{\partial \Delta h_4}{\partial u_1} \bigg|_{h_i^0, u_i^0} &
  \dfrac{\partial \Delta h_4}{\partial u_2} \bigg|_{h_i^0, u_i^0} \\
\end{bmatrix}
\]

\[
C =
\begin{bmatrix}
  \dfrac{\partial \Delta y_1}{\partial h_1} \bigg|_{h_i^0, u_i^0} &
  \dfrac{\partial \Delta y_1}{\partial h_2} \bigg|_{h_i^0, u_i^0} &
  \dfrac{\partial \Delta y_1}{\partial h_3} \bigg|_{h_i^0, u_i^0} &
  \dfrac{\partial \Delta y_1}{\partial h_4} \bigg|_{h_i^0, u_i^0} \\

  \dfrac{\partial \Delta y_2}{\partial h_1} \bigg|_{h_i^0, u_i^0} &
  \dfrac{\partial \Delta y_2}{\partial h_2} \bigg|_{h_i^0, u_i^0} &
  \dfrac{\partial \Delta y_2}{\partial h_3} \bigg|_{h_i^0, u_i^0} &
  \dfrac{\partial \Delta y_2}{\partial h_4} \bigg|_{h_i^0, u_i^0} \\
\end{bmatrix}
\]

\[
D =
\begin{bmatrix}
  \dfrac{\partial \Delta y_1}{\partial u_1} \bigg|_{h_i^0, u_i^0} &
  \dfrac{\partial \Delta y_1}{\partial u_2} \bigg|_{h_i^0, u_i^0} \\

  \dfrac{\partial \Delta y_2}{\partial u_1} \bigg|_{h_i^0, u_i^0} &
  \dfrac{\partial \Delta y_2}{\partial u_2} \bigg|_{h_i^0, u_i^0} \\
\end{bmatrix}
\]

After performing the necessary operations, we get

\[
A =
\begin{bmatrix}
  -\dfrac{1}{T_1} & 0 & \dfrac{A_3}{A_1 T_3} & 0  \\
  0 & -\dfrac{1}{T_2} & 0 & \dfrac{A_4}{A_2 T_4}  \\
  0 & 0 & -\dfrac{1}{T_3} & 0 \\
  0 & 0 & 0 & -\dfrac{1}{T_4}
\end{bmatrix},
B =
\begin{bmatrix}
  \dfrac{\gamma_1 k_1}{A_1} & 0 \\
  0 & \dfrac{\gamma_2 k_2}{A_2} \\
  0 & \dfrac{(1-\gamma_2) k_2}{A_3} \\
  \dfrac{(1-\gamma_1) k_1}{A_4} & 0
\end{bmatrix},
C =
\begin{bmatrix}
k_c & 0 & 0 & 0 \\
0 & k_c & 0 & 0 \\
\end{bmatrix}
\]

and D = 0, with $T_i = \dfrac{A_i}{a_i}\sqrt{\dfrac{2h_i^0}{g}}$.



%-------------------------------------------------------------------------------
\subsection*{Exercise 2.1.4}

The transfer matrix can be obtained by

\begin{align*}
  G(s) &= C (sI - A)^{-1} B = \\
\end{align*}
\[
\hspace*{-2cm}
\begin{bmatrix}
  k_c & 0 & 0 & 0 \\
  0 & k_c & 0 & 0 \\
\end{bmatrix}
\begin{bmatrix}
  \dfrac{T_1}{1 + s T_1} & 0 & \dfrac{\dfrac{A_3 T_1}{A_1}}{(1 + s T_1)(1 + s T_3)} & 0 \\
  0 & \dfrac{T_2}{1 + s T_2} & 0 & \dfrac{\dfrac{A_4 T_2}{A_2}}{(1+s T_2)(1 + s T_4)} \\
  0 & 0 & \dfrac{T_3}{1 + s T_3} & 0 \\
  0 & 0 & 0 & \dfrac{T_4}{1 + s T_4}
\end{bmatrix}
\begin{bmatrix}
  \dfrac{\gamma_1 k_1}{A_1} & 0 \\
  0 & \dfrac{\gamma_2 k_2}{A_2} \\
  0 & \dfrac{(1-\gamma_2) k_2}{A_3} \\
  \dfrac{(1-\gamma_1) k_1}{A_4} & 0
\end{bmatrix} =
\]
\[
\begin{bmatrix}
  \dfrac{\gamma_1 k_1 c_1}{1 + s T_1} & \dfrac{(1-\gamma_2)k_2 c_1}{(1 + s T_3)(1 + s T_1)} \\\\
  \dfrac{(1-\gamma_1) k_1 c_2}{(1 + s T_4)(1 + s T_2)} & \dfrac{\gamma_2 k_2 c_2}{1 + s T_2}
\end{bmatrix}
\]



%-------------------------------------------------------------------------------
\subsection*{Exercise 2.1.5}

The zeros of $G(s)$ are given by the solution to the quadratic equation

\begin{equation*}
  T_3 T_4 s^2 + (T_3 + T_4) s + \dfrac{\gamma_1 + \gamma_2 - 1}{\gamma_1 \gamma_2} = 0
\end{equation*}

which are $s_1$ and $s_2$:

\begin{align*}
  s_1 &= -\dfrac{T_3 + T_4}{2 T_3 T_4} + \dfrac{1}{2 T_3 T_4} \sqrt{(T_3 + T_4)^2 - 4 \dfrac{\gamma_1 + \gamma_2 - 1}{\gamma_1 \gamma_2}} \\
  s_2 &= -\dfrac{T_3 + T_4}{2 T_3 T_4} - \dfrac{1}{2 T_3 T_4} \sqrt{(T_3 + T_4)^2 - 4 \dfrac{\gamma_1 + \gamma_2 - 1}{\gamma_1 \gamma_2}} \\
\end{align*}

$s_2$ is strictly negative, but $s_1$ might not be. $G(s)$ is non-minimum phase
when $s_1 \geq 0$, which happens when $\gamma_1 + \gamma_2 \leq 1$. But
$0 < \gamma_i \leq 1$, hence $0 < \gamma_1 + \gamma_2 \leq 2$, which means that
$G(s)$ is non-minimum phase when $0 < \gamma_1 + \gamma_2 \leq 1$ and minimum
phase when $1 < \gamma_1 + \gamma_2 \leq 2$.



%-------------------------------------------------------------------------------
\subsection*{Exercise 2.1.6}

$G(0)$ and $G(0)^{-T}$ are given by

\[
G(0) =
\begin{bmatrix}
  \gamma_1 k_1 c_1 & (1-\gamma_2)k_2 c_1 \\
  (1-\gamma_1) k_1 c_2 & \gamma_2 k_2 c_2
\end{bmatrix}
\]

\[
G(0)^{-T} =
\dfrac{1}{\gamma_1 \gamma_2 k_1 k_2 c_1 c_2 - (1-\gamma_1) (1-\gamma_2) k_1 k_2 c_1 c_2}
\begin{bmatrix}
  \gamma_2 k_2 c_2 & -(1-\gamma_1) k_1 c_2 \\
  -(1-\gamma_2)k_2 c_1 & \gamma_1 k_1 c_1
\end{bmatrix}
\]

Hence, the RGA of $G(0)$ is given by

\begin{align*}
  RGA(G(0)) &= G(0) .* G(0)^{-T} = \\
  \dfrac{1}{\gamma_1 \gamma_2 k_1 k_2 c_1 c_2 - (1-\gamma_1) (1-\gamma_2) k_1 k_2 c_1 c_2}
  &\begin{bmatrix}
    \gamma_1 \gamma_2 k_1 k_2 c_1 c_2 & - (1-\gamma_1) (1-\gamma_2) k_1 k_2 c_1 c_2 \\
    - (1-\gamma_1) (1-\gamma_2) k_1 k_2 c_1 c_2 & \gamma_1 \gamma_2 k_1 k_2 c_1 c_2
  \end{bmatrix}
\end{align*}

and

\begin{align*}
  \lambda &= \dfrac{\gamma_1 \gamma_2 k_1 k_2 c_1 c_2}{\gamma_1 \gamma_2 k_1 k_2 c_1 c_2 - (1 - \gamma_1 - \gamma_2 + \gamma_1 \gamma_2) k_1 k_2 c_1 c_2)}
          = \dfrac{\gamma_1 \gamma_2}{\gamma_1 + \gamma_2 - 1}
\end{align*}

In the minimum phase case, where $\gamma_1 = \gamma_2 = 0.625$, $\lambda_{mp} = 1.5625$,
and

\[
RGA(G_{mp}(0) =
\begin{bmatrix}
  1.5625 & -0.5625 \\
  -0.5625 & 1.5625 \\
\end{bmatrix}
\]

In the non-minimum phase case, where $\gamma_1 = \gamma_2 = 0.375$, $\lambda_{nmp} = −0.5625$,
and

\[
RGA(G_{nmp}(0) =
\begin{bmatrix}
  -0.5625 & 1.5625 \\
  1.5625 & -0.5625 \\
\end{bmatrix}
\]



%-------------------------------------------------------------------------------
\subsection*{Exercise 2.1.7}

In order to accurately determine the values of $k_1, k_2$, we turn to tanks
1 and 2, since only there can the height of the water be recorded. The process
we followed for resolving $k_1$ is completely analogous for $k_2$. Shutting the
outflow of tank 1 while having only pump 1 being active, with both of its tubes
feeding water to tank 1, with constant voltage applied results in a situation
where no extraneous flow comes into tank 1 (other than that from pump 1), and no
outflow happens. Hence, we can see that the rate at which a tank is filled in
this setting is linear in time:

\begin{align*}
  \dfrac{dh_1}{dt} &= \cancelto{0}{-\dfrac{a_1}{A_1}\sqrt{2gh_1}} + \cancelto{0}{\dfrac{a_3}{A_1}\sqrt{2gh_3}} + \dfrac{k_1}{A_1}u_1 \\
  \dfrac{dh_2}{dt} &= \cancelto{0}{-\dfrac{a_2}{A_2}\sqrt{2gh_2}} + \cancelto{0}{\dfrac{a_4}{A_2}\sqrt{2gh_4}} + \dfrac{k_2}{A_2}u_2 \\
\end{align*}

According to multiple measurements\footnote{The maximum height of each tank as
perceived by the sensors was not taken to be 25cm, but was calculated for each
tank. Furthermore the pressure sensors at tanks 1 and 2 were found to have an
offset of $9\%$ and $15\%$ respectively the last time they were measured. These
offsets can be seen in figures \ref{fig:mp_411} and \ref{fig:mp_421} as at time
$t = 0$ the levels of the tanks are not zero.} with
different input voltages, the values of $k_1$ and $k_2$ were calculated to be:

\begin{align*}
  k_1 &= 3.8454\ cm^3 / sV \\
  k_2 &= 3.9925\ cm^3 / sV\\
\end{align*}


%-------------------------------------------------------------------------------
\subsection*{Exercise 2.1.8}

In order to calculate the effective outlet areas $a_i$ for the minimum phase case
we focus our attention to the equations describing the equilibrium state and,
at first, to the last two equations of exercise 2.1.2, where we can see that if
we drive tanks 3 and 4 to equilibrium, we can directly calculate $a_3$ and $a_4$.
In this case we are not interested in what happens to the lower tanks.


\begin{align*}
  a_3 = a_4 &= \dfrac{(1-\gamma_2)k_2}{A_3}u_2^0 \dfrac{A_3}{\sqrt{2gh_3^0}} = 0.0827\ cm^2 \\
\end{align*}

As for the effective outlet areas of tanks 1 and 2, since their values are
constant irrespective of the conditions in which the system is running, they can
be calculated in the same manner by driving the tanks to equilibrium without the
intervention of flows from the upper tanks. In this case:

\begin{align*}
  a_1 = a_2 &= \dfrac{\gamma_1k_1}{A_1}u_1^0 \dfrac{A_1}{\sqrt{2gh_1^0}} = 0.1282\ cm^2 \\
\end{align*}

It should be noted at this point that since the process is subject to changes
due to other groups interchanging the outlet nozzles, and since their areas are
approximately equal, as they are different versions of the same thing, we can
regard them in our calculations as being equal, since measurements can be noisy.
